\chapter{PLURALISM AND TOLERATION\\多元主义与宽容}
\begin{paracol}{2}
\hbadness5000

One of the central facts of modern life, which any political
theory must confront, is moral pluralism. Individuals
have different concepts of the meaning of life, of the existence of
God, of the ways to pursue happiness. One response to this reality may be termed ``perfectionism,'' a political philosophy that
seeks an institutional structure that will perfect human nature.
Marx offered such an answer, claiming that socialism would
allow human beings for the first time to achieve their full
human potential. Theocratic religions offer a different answer,
proposing to unite an entire people in a common understanding
of their relationship to God. Communitarian philosophers also
seek to bring about a community whose ``substantive life,'' in
the words of the Harvard University philosopher Michael Walzer, ``is lived in a certain way---that is, faithful to the shared
understandings of its members.'' Even some modern conservatives who believe, as the columnist George F. Will put it, that ``statecraft is soulcraft,'' are trying to use the power of government to \textit{remedy} the fact of moral pluralism.
\switchcolumn
现代生活的一个重要事实是道德多元主义,这是任何一位
政治理论家都必须面对的。个人对生命的意义、上帝的存在、
追求幸福的方式有着不同的理念。对这个事实的回答,其中一
种 被 称 作 “完美主义”,这是一种追求能够使人性更完美的制
度结构的政治哲学。马克思就提供了这样一种回答,他认为社
会主义让人类第一次能够发挥他们的全部潜力。有神论宗教则
提供了一种不同的回答,试图把全体人民统一在对他们与上帝
之间关系的共同理解之上。社群主义理论家也在试图实现一种
社群,其 “本质的生活方式” ,用哈佛大学哲学家迈克尔$\cdot$沃
尔 泽(Michael Walzer)的话说就是, “以某种有确定性的方
式来生活,即忠诚于社群成员的共同认识。”甚至一些现代保
守主义者相信如专栏作家乔治$\cdot$ 威 尔(George Will)所说的
“治理国家就是管理思想”,也是在试图使用国家权力来\textbf{校正}
道德多元主义的事实。
\switchcolumn*
Libertarians and individualist liberals have a different answer. Liberal theory accepts that in modern societies there will
be irresolvable differences over what the good for human beings
is and what their ultimate nature is. Some more Aristotelian
liberals argue that human beings do indeed have one nature but
that each human has an individual set of talents, needs, circumstances, and ambitions; so the good life for one person may differ from the good for another, despite their common nature.
Self-directedness, the ability to choose one's own course in life,
is part of the human good.
\switchcolumn
古典自由主义者和个人自由主义的回答有所不同。自由主
义理论认为,现代社会中,对人类来说什么是善以及人类的终
极本质是什么的认识必然存在不同,这是无法解决的。一些更
注重实践的自由主义者认为人类的确只有一种本性,但是每个
人都有个人化的智力、需求、环境和目标;因此一个人的好生
活也许和另一个人的好生活大不相同,尽管他们的本性是相同
的。自我引导是选择一个人自己生活道路的能力,是人类美德
的一部分。
\switchcolumn*
Thus, on either approach, libertarians believe the role of government is not to impose a particular morality but to establish a
framework of rules that will guarantee each individual the freedom to pursue his own good in his own way---whether individually or in cooperation with others---so long as he does not
infringe the freedom of others. Because no modern government
can assume that its citizens share a complete and exhaustive
moral code, the obligations imposed on people by force should
be minimal. In the libertarian conception, the fundamental
rules of the political system should be essentially negative:
Don't violate the rights of others to pursue their own good in
their own way. If a government tries to allocate resources and
assign duties on the basis of a particular moral conception---according to need or moral desert---it will create social and political conflict. This is not to say that there is no substantive
morality, or that all ways of life are ``equally good,'' but merely
that consensus on the best is unlikely to be reached and that
when such matters are placed in the political realm, conflict is
inevitable.
\switchcolumn
而古典自由主义者相信政府的角色不是强加一种道德,而
是建立一个规则框架,确保每个个人用自己的方式(通过个人,或者通过与他人的合作) 自由追求自己的善,只要他不侵犯别
人的自由。因为没有一个现代政府能够假设其公民分享一个共
同的完全详尽的道德规则,因此要求人们遵从的强制行为应当
控制到最小。在古典自由主义的概念当中,政治体制的基础规
则应当基本上是消极的:不要侵犯其他人以自己的方式追求善
的权利。如果政府试图以某个道德概念为基础来分配资源或分
派责任一根据需要或者道德赏罚 --- 就将制造社会和政治冲
突。这并不是说没有一种本质上的道德,或者说各种生活是
“一样的善”,但是至少对于什么是最善我们基本上无法达成共
识,因而,一旦把这些事务放到政治领域,冲突就会无法避免。

\switchcolumn*[\section{Religious Toleration\\宗教宽容}]
One of the obvious implications of individualism, the idea that
each person is an individual moral agent, is religious toleration.
Libertarianism developed out of the long struggle for toleration, from the early Christians in the Roman Empire, to the Netherlands, to the Anabaptists in Central Europe, to the Dissenters from the Church of England, to the experiences of
Roger Williams and Anne Hutchinson in the American colonies
and beyond.
\switchcolumn
个人主义,即每个人是独立道德主体这一理念的一个显而
易见的推论,就是宗教宽容。古典自由主义就是在争取宽容的
长期斗争中得到发展的,从罗马帝国时期的早期基督教到荷兰
的宗教宽容,从中欧的再洗礼教派到英国国教会的不信国教
者 ,从罗杰 $\cdot$ 威廉姆斯\footnote{罗杰$\cdot$威廉姆斯(Roger Williams),17世纪北美宗教家。威廉姆斯在罗得岛进行了重大的社会改革,实施了宗教自由和政教分离等民主原则,坚持和平协议解决争端。他不仅容忍新教内部的不同派系,尊重其信仰自由的权利,而且把这一权利推广到一般新教徒所不能容忍的天主教、犹太教,乃至异教。}到 安 妮 $\cdot$哈钦森\footnote{安妮$\cdot$哈钦森(Anne Hutchinson,1591 $\sim$ 1643),北美殖民地一个淸教团体的领导人。她以宗教改革形式表现的女权意识影响广泛,成为美国早期女权主义思想的起源。} 在北美殖民地的经验 ,等等。
\switchcolumn*
Self-ownership certainly included the concept of ``a property
in one's conscience,'' as James Madison put it. The Leveller Richard Overton wrote in 1646 that ``every man by nature (is)
Priest and Prophet in his own natural circuit and compass.'' Locke agreed that ``liberty of conscience is every man's natural right.''
\switchcolumn
自我所有权自然包含如麦迪逊所说的“ 内心产权”  (a
property in one’s conscience)概念。平权派的奥弗顿在1646年
写道: “每个人都是他自己自然边界以内的牧师和先知。”洛
克同意:“信仰自由是每个人的自然权利。”
\switchcolumn*
Beyond moral and theological arguments, though, there
were strong practical arguments for religious toleration. As
George H. Smith argues in his 1991 essay ``Philosophies of Toleration,'' one group of advocates of toleration would have preferred to see uniformity of religious belief, ``but they did not
wish to impose uniformity in practice because of its staggering
social costs---massive compulsion, civil wars, and social chaos.''
They recommended toleration as the best way to produce peace
in society. The Jewish philosopher Baruch Spinoza, explaining
the Dutch policy of toleration, wrote, ``It is imperative that
freedom of judgment should be granted, so that men may live
together in harmony, however diverse, or even openly contradictory, their opinions may be.'' Spinoza pointed to the prosperity that the Dutch had achieved by allowing people of every
sect to live peacefully and do business in their cities. As the
English, observing the Dutch example, adopted a policy of relative toleration, Voltaire noted the same effect and recommended it to the French. Although Marx would later denounce
the market for its impersonal nature, Voltaire recognized the
advantages of that impersonality. As George Smith put it, ``The
ability to deal with others impersonally, to deal with them
solely for mutual profit, means that personal characteristics,
such as religious belief, become largely irrelevant.''
\switchcolumn
除了道德和神学的观点之外,对宗教宽容还有很强的实践
上的理由。乔 治 $\cdot$ 史 密 斯(George H. Smith)在 1991年的一
篇 文 章 “宽容的哲学” 当中说,宗教宽容的拥护者也许希望
看到宗教信仰的统一,“但是他们不希望在实践中进行宗教统
一 , 因为其社会代价大得惊人 --- 大规模的强制、内战以及社
会混乱”。他们会把宽容作为在社会中建立和平的最佳办法。
犹太哲学家斯宾诺莎在解释荷兰的宗教宽容政策时说:“不论
他们的观点多么不同,甚至公开对立,意见自由是不可或缺
的,只有这样人们才能和谐地生活在一起。”斯宾诺莎指出了
荷兰通过允许不同教派的人和平生活在一起并在他们城市里做
生意而使自己获得了繁荣。英国在看到荷兰的例子之后,也采
取了一种相对的宗教宽容政策。伏尔泰同样注意到了荷兰宗教
宽容政策产生的影响,并把它介绍给了法国人。尽管马克思后
来对市场非人格化的本质进行了谴责,但伏尔泰承认了市场这
种非人格性的好处。正如乔治$\cdot$ 史密斯所指出的:“能够与其
他人进行非人格化的交往,能 够 把 与 他 人交往仅仅建立在双方
利益的基础之上,意味着个人的一些特征,如 宗教信仰 ,很大
程度上变得无关紧要。”
\switchcolumn*
Other advocates of toleration stressed the benefits of religious pluralism in theory. Out of argument, they said, truth
will emerge. John Milton was the preeminent defender of this
view, but Spinoza and Locke also endorsed it. British libertarians in the nineteenth century used terms like ``free trade in religion'' to oppose the establishment of the Anglican Church.
\switchcolumn
其他的宗教宽容的拥护者们则在理论上强调了宗教多元主
义的好处。他们说,真理将在争论中脱颖而出。弥尔顿是这个
观点的杰出辩护人,斯宾诺莎和洛克也同样支持这个观点。19
世纪的英国古典自由主义者使用了 “宗教的自由贸易”概念来反对英国国教。
\switchcolumn*
Some English Dissenters came to America to find freedom to
practice religion in their own way, but not to grant it to others.
They did not oppose special privileges for one religion; they just
wanted it to be their own. But others among the new Americans not only supported religious toleration, they extended the
argument to call for the separation of church and state, a truly
radical idea at the time. After he was banished from the Massachusetts Bay Colony in 1636 for his heretical opinions, Roger Williams wrote \textit{The Bloudy Tenent of Persecution}, urging separation and seeking to protect Christianity from political control.
Williams's ideas, along with those of John Locke, spread
throughout the American colonies; established churches were
gradually disestablished, and the Constitution adopted in 1787
included no mention of God or religion, except for forbidding
religious tests for public office. In 1791 the First Amendment
was added, guaranteeing the free exercise of religion and forbidding any established church.
\switchcolumn
一些英国的不信国教者来到美国,寻求用他们自己的方式
实践宗教自由,但是不把这种自同授与他人。他们并不反对别
的宗教;他们只想拥有自己的宗教。其他的美国新移民不仅支
持宗教宽容,还把争论扩大到要求政教分离。这在当时是一个
相当激进的理念。1636年因为异端观点而被从马萨诸塞湾殖
民地驱逐出去之后,罗 杰 $\cdot$ 威廉姆斯写了《迫害之血腥信条》(\textit{The Bloudy Tenent of Persecution}) 一书,极力主张政教分离,主张让基督教脱离政治控制。威廉姆斯的理念和洛克的理论在北美殖民地被广泛传播。官方教会逐渐与政治脱钩。1787年实施的美国宪法没有提及上帝或宗教,只是在公立机构进行的
宗教宣誓没有被废除。1791年宪法第一修正案通过,确保了宗教信仰自由,禁止设立任何形式的国教或官方教会。
\switchcolumn*
Members of the religious right today insist that America is---
or at least was---a Christian nation with a Christian government. The Dallas Baptist minister who delivered the
benediction at the Republican National Convention in 1984
says that ``there is no such thing as separation of church and
state,'' and Christian Coalition founder Pat Robertson writes,
``The Constitution was designed to perpetuate a Christian
order.'' But as Isaac Kramnick and R. Laurence Moore note in
\textit{The Godless Constitution}, Robertson's forebears understood the
Constitution better. Some Americans opposed ratification of
the Constitution because it was ``coldly indifferent towards religion'' and would leave ``religion to shift wholly for itself.'' Nevertheless, the revolutionary Constitution was adopted, and
most of us believe that the experience with the separation of
church and state has been a happy one. As Roger Williams
might have predicted, churches are far stronger in the United
States, where they are left to fend for themselves, than in European countries where there is still an established church (such as
England and Sweden) or where churches are supported by government-collected taxes on their adherents (such as Germany).
\switchcolumn
基督教右翼今天仍然坚持美国是或至少曾经是一个基督教
国家,拥有一个基督教政府,或者说曾经是基督教国家。达拉
斯的浸礼会牧师在为1984年共和党全国代表大会举行的祝福
式上说:“根本就没有什么政教分离”;而基督教联盟创始人
罗伯特森(Pat Robertson)也在一篇文章中写道 : “宪法就是
为基督的秩序永存而设计的。”但 是 克 兰 尼 克(Issac Kramnick)和摩尔(R. Laurence Moore)在 《没有神的宪法》(\textit{The Godless Constitution}) 一书中指出,罗伯特森的祖先比他更理解宪法。有的美国人拒绝承认宪法,因 为 宪 法 “对宗教冷酷地中立”,是 放 任 “宗教彻底的转向”。尽管如此,革命性的
宪法仍然被采用了,而今天绝大多数人都认为政教分离的经验是令人愉快的。正如罗杰$\cdot$ 威廉姆斯所预言的那样,在教会不得不依靠自己之后,美国的基督教会就发展得远比欧洲国家的教会更强大,而欧洲国家仍然在设立国教(如英国和瑞典)或者由政府向信徒征税来支持教会(如德国)。

\switchcolumn*[\section{Separation of Conscience and State\\政教分离}]
We might reflect on why the separation of church and state
seems such a wise idea. First, it is wrong for the coercive authority of the state to interfere in matters of individual conscience. If we have rights, if we are individual moral agents, we
must be free to exercise our judgment and define our own relationship with God. That doesn't mean that a free, pluralistic society won't have lots of persuasion and proselytizing---no doubt
it will---but it does mean that such proselytizing must remain
entirely persuasive, entirely voluntary.
\switchcolumn
我们也许应该深思为什么政教分离是明智的做法。首先,
国家强制力介入个人信仰事务是错误的。如果我们拥有权利,
如果我们是个人道德的主体,那么我们就有运用我们的判断界
定自己与上帝关系的自由。这并不是说一个自由的、多元化的
社会里不会有大量的说服皈依和信仰改宗,当然会有的,但是
这样的宗教改宗必须完全是说服性的和自愿的。
\switchcolumn*
Second, social harmony is enhanced by removing religion
from the sphere of politics. Europe suffered through the Wars of
Religion, as churches made alliances with rulers and sought to
impose their theology on everyone in a region. Religious inquisitions, Roger Williams said, put towns ``in an uproar.'' If people
take their faith seriously, and if government is going to make
one faith universal and compulsory, then people must contend
bitterly---even to the death---to make sure that the \textit{true} faith is
established. Enshrine religion in the realm of persuasion, and
there may be vigorous debate in society, but there won't be political conflict. As the experiences of Holland, England, and
later the United States have shown, people can deal with one
another in secular life without endorsing each other's private
opinions.
\switchcolumn
其二,宗教从政治领域脱离会增进社会和谐。欧洲曾经深
陷宗教战争,因为教会与统治者结盟,试图推行所有人必须信
仰同一宗教的神学观点。结果就像罗杰$\cdot$威廉姆斯所说的那
样 :宗教法庭闹得各个城镇鸡犬不宁。如果人们很把自己信仰
当回事,同时政府企图强制推行一个统一的信仰,那么人们就
必然会为了确立\textbf{真正}信仰的正统地位而发生战争。如果将宗教
限定在说服劝导的范围内,也许社会当中会有关于信仰的激烈
争论,但是不会有政治冲突。正如荷兰、英格兰和后来美国的
经验所显示的那样,人们能够在不赞同别人个人观点的情况
下,在世俗生活中和平相处。
\switchcolumn*
Third, competition produces better results than subsidy, protection, and conformity. ``Free trade in religion'' is the best tool
humans have to find the nearest approximation to the truth.
Businesses coddled behind subsidies and tariffs will be weak
and uncompetitive, and so will churches, synagogues, mosques,
and temples. Religions that are protected from political interference but are otherwise on their own are likely to be stronger
and more vigorous than a church that draws its support from
government.
\switchcolumn
其三,让不同宗教进行竞争比资助、保护宗教和设立国教
的结果更好。“宗教的自由贸易”是人类最大限度接近真理的
最佳工具。企业如果被补贴和关税所保护就会变得虚弱和缺乏
竞争力,宗教也是一样,基督教堂、犹太教堂、伊斯兰清真
寺 、佛教寺院都是如此。一个宗教如果远离政府干预,依靠自
己进行发展,极有可能会比得到政府支持的教会更加强大、更
有活力。
\switchcolumn*
This last point reflects the humility that is an essential part of
the libertarian worldview. Libertarians are sometimes criticized
for being too ``extreme,'' for having a ``dogmatic'' view of the
role of government. In fact, their firm commitment to the full
protection of individual rights and a strictly limited government reflects their fundamental humility. One reason to oppose
the establishment of religion or any other morality is that we
recognize the very real possibility that our own views may be
wrong. Libertarians support a free market and widely dispersed
property ownership because they know that the odds of a monopolist finding a great new advance for civilization are slim.
Hayek stressed the crucial significance of human ignorance
throughout his work. In \textit{The Constitution of Liberty}, he wrote, ``The case for individual freedom rests chiefly on the recognition
of the inevitable ignorance of all of us concerning a great many
of the factors on which the achievement of our ends and welfare
depends$\ldots$ Liberty is essential in order to leave room for the
unforeseeable and unpredictable.'' The nineteenth-century
American libertarian Lillian Harman, rejecting state control of
marriage and family, wrote in \textit{Liberty} in 1895, ``If I should be
able to bring the entire world to live exactly as I live at present,
what would that avail me in ten years, when as I hope, I shall
have a broader knowledge of life, and my life therefore probably
changed?'' Ignorance, humility, toleration---not exactly a ringing battle cry, but an important argument for limiting the role
of coercion in society.
\switchcolumn
最后一点理由是:谦虚。这是新古典主义的基本世界观之一。古典自由主义者常常被批评为走极端,对政府作用的观点
太教条主义。但对个人权利彻底保护和对政府严厉限制的观点恰恰表明,古典自由主义者本质上是谦虚的。反对设立国教和
建立其他官方道德体系的一个理由是我们认识到这样一个非常接近真实的可能:我们自己的观点可能是错的。古典自由主义
者支持自由市场和分散的财产所有权是因为他们知道,垄断者推动人类文明重大进步的机会是极小的。哈耶克在他的著作中
强调,无知是人的决定性的显著特征。在《自由宪章》一书中他说:“个人自由的基础主要是,我们认识到我们所有人的
无知是不可避免的,我们对达到目标和福利所依赖的许多东西具有一种不可避免的无知……为了给无法预见和预测的情形留
下空间,自由就是必须的。” 19世纪美国古典自由主义者莉莉安 $\cdot$ 哈 曼 (Lillian  Harman)反对国家对婚姻和家庭的控制,她在1895年写下了《自由》(\textit{Liberty}) 一书,在书中她写道:“假如我有能力让整个世界按照我现在的生活方式来生活,但即便如我所愿又对我有什么好处呢?因为十年内我的生活会因为获得更多的生活知识而发生改变。” 无知、谦虚和宽容 ---并不是一个响亮的战斗口号,却是在我们的社会中对强制进行限制的一种重要辩护。
\switchcolumn*
If these themes are true, they have implications beyond religion. Religion is not the only thing that affects us personally
and spiritually, and it is not the only thing that leads to cultural wars. For example, the family is the institution in which
we learn most of our understanding of the world and our moral
values. Despite Mario Cuomo's vision of America as a great
family, or Hillary Clinton's global village, we each of us care
more for our own children than for any other children, and we
want to instill our own moral values and worldview in our children. That's why government interference into the family is so
offensive and so controversial. We ought to establish a principle of the separation of family and state, a wall of separation as
firm as that between church and state, and for the same reasons: to protect individual consciences, to reduce social conflict, and to lessen the baleful effects of subsidy and regulation
on our families.
\switchcolumn
如果这些是真理的话,它们的应用将超越宗教领域。宗教
并不是影响我们个人精神世界的唯一因素,也不是导致文化战
争的唯一原因。例如,绝大多数人的世界观和道德价值观都是
在家庭当中学到的。尽 管 库 尔 摩(Mario Cuomo) 认为美国是
一个大家庭,希拉里认为地球是一个村庄,我们每个人还是更
关心自己的孩子,而不是别人的孩子,我们仍然希望把我们自
己的道德价值观和世界观灌输给自己的孩子。这就是为什么政府对家庭的干预是令人厌恶和富有争议的。我们应当树立一种
原则叫作:政家分离。就像在政府和教会之间建立的坚实隔离
墙一样,在政府和家庭之间树立一道同样坚实的墙。理由则和
政教分离一样:保护个人的信仰,降低社会冲突,减少对家庭
的补贴和管制所带来的破坏性影响。
\switchcolumn*
The other arena where we formally teach values to our children is education. We expect schools to give our children not
only knowledge but also the moral strength to make wise decisions. Alas, in a pluralistic society we don't all agree on what
those moral values should be. To begin with, some parents want
reverence for God taught in the schools, and others don't. The
First Amendment has correctly been interpreted to ban prayer
in government schools; but to compel religious parents to pay
taxes for schools and then forbid the tax-supported schools to
give their children the education they want is surely unfair. In the Virginia Statute for Religious Freedom, Thomas Jefferson
wrote, ``To compel a man to furnish contributions of money for
the propagation of opinions which he disbelieves is sinful and
tyrannical.'' How much more offensive it is to tax a family to
propagate \textit{to their own children} opinions that they disbelieve.
\switchcolumn
另一个对孩子正式传授价值观的舞台是学校教育。我们希
望学校教给孩子的不仅仅是知识,而且还有道德,好让他们能
够做出明智的决策。不过很不幸,在多元化社会,我们对应该
传授什么样的道德价值观并没有共识。有的家长希望在学校讲
授对上帝的敬畏,而其他人则表示反对。政府援引第一修正案
在公立学校里禁止举行祷告。这是正确的,但是一方面强迫信
教的家长为学校交税,另一方面却不让用税收支持的公立学校
给他们的孩子以他们想要的教育,则是不公平的。杰斐逊在
《弗吉尼亚宗教自由法令》 中说: “强迫人们给他们不相信的
观点的宣传活动捐款是邪恶和专制的。” 那么,向一个家庭收
税 ,用来\textbf{向其子女宣传}他们不相信的观点则更是一种侵犯。
\switchcolumn*
The problems go well beyond religion. Should the schools require uniforms, open with the Pledge of Allegiance, allow gay
teachers, separate boys and girls, teach antibusiness environmentalism, cultivate support for the Persian Gulf War, celebrate Christmas and/or Hanukkah, require drug tests? All of
these decisions involve moral choices, and different parents will
have different preferences. A government-run monopoly system has to make one decision for the whole community on such
issues. Strict separation of education and state would respect
the individual consciences of each family, reduce political conflict over highly charged issues, and strengthen each school in
its sense of mission and the commitment of its students and
their families. Parents could choose private schools for their
children on the basis of the moral values and educational mission the schools offered, and no political conflict over what to
teach would arise.
\switchcolumn
宗教不是学校里的唯一问题,此外还有很多。比如,学校
是否应该统一制服?是否应进行国家忠诚宣誓?是否允许同性
恋当老师?是否应将男女学生隔离?是否可以讲授反商业的环
保主义?是否应该培养对海湾战争的支持?是否应该庆祝圣诞节和光明节\footnote{光明节(Hunukkah),犹太教节日之一。}?是否应当进行吸毒测试?所有这些决策都包含了道德选择,不同的家长会有不同的偏好。对于这些问题,一
个政府经营的垄断体制将不得不为整个社会提供唯一的答案。
而政府和教育的严格分离将会尊重每个家庭的个人信仰,在这
些代价高昂的事务上减少社会冲突,更好地完成对学生和家庭
的责任和任务。家长们也就能够从私立学校当中,根据学校所
能提供的道德价值观和教学目标的能力进行选择,而且不会因
为应该教什么而发生政治冲突。
\switchcolumn*
Like the church, the family, and the school, art also expresses,
transmits, and challenges our deepest values. As the managing
director of Baltimore's Center Stage put it, ``Art has power. It
has the power to sustain, to heal, to humanize $\ldots$ to change
something in you. It's a frightening power, and also a beautiful
power$\ldots$ And it's essential to a civilized society.'' Because
art---by which I include painting, sculpture, drama, literature,
music, film, and more---is so powerful, dealing as it does with
such basic human truths, we dare not entangle it with coercive
government power. That means no censorship or regulation of
art. It also means no tax-funded subsidies for arts and artists,
for when government gets into the arts-funding business, we
get political conflicts: Can the National Endowment for the
Arts fund erotic photography? Can the Public Broadcasting
System broadcast \textit{Tales of the City}, which has gay characters? Can
the Library of Congress display an exhibit on antebellum slave
life? To avoid these political battles over how to spend the tax-payers' money, to keep art and its power in the realm of persuasion, we would be well advised to establish the separation of art
and state.
\switchcolumn
与教会、家庭和学校一样,艺术也对我们最深层的价值观
进行了表达和传播,并对其提出了挑战。就像巴尔的摩中央剧
院总经理说的,“艺术是有力量的。它能支撑你的精神、治疗
你的心灵、让你变得仁爱……改变的是你的内心。它是一种令
人恐惧的力量, 也是一种美的力量……它是文明社会的基本要
素。” 正因为艺术(包括绘画、雕塑、戏剧、文学、音乐、电
影以及其他艺术)是如此有力量,而且关乎人类的基本真理,
我们自然不敢让它与强制性的政府权力纠缠在一起。这就意味
着对艺术不能有审查和管制。另一方面不能用税款对艺术和艺
术家进行资助,因为只要当政府设立艺术基金,我们就会面临
着政治冲突:国家艺术基金能不能资助色情图片?公共广播电
台能不能播放有同性恋角色的《城市故事》(\textit{Tales of the City}) 国家图书馆能不能举办反映南北战争前奴隶生活的展览?为了避免这些关于如何花纳税人的钱的政治争论,为了把艺术及其力量限制在劝说的领域,我们有必要让政府与艺术分离。
\switchcolumn*
And how about the divisive issue of race? Haven't we suffered through enough generations of government-supported
racial discrimination? After the end of slavery---which was far
too odious a violation of individual rights to be categorized as
mere race discrimination---we added three amendments to the
U.S. Constitution, each one designed to make good on the
promises of the Declaration of Independence by guaranteeing
every (male) American equal rights. Specifically, those amendments abolished slavery, promised equal protection of the laws
to all citizens, and guaranteed that the right to vote would not
be denied to anyone on the basis of race. But within a few years,
state governments, with the acquiescence of the federal courts,
began to limit the rights of African Americans to vote, to use
public facilities, and to enter into economic life. The Jim Crow
era lasted into the 1960s. Then, unfortunately, the federal government passed over the libertarian policy of equal rights for all
in a blink of an eye and began to replace old forms of racial discrimination with new ones---quotas and set-asides and mandatory racial preferences. Just as the Jim Crow laws angered
blacks (and all those who believed in equal rights), the new
quota regime angered whites (and all those who believed in
equal rights). The stage was set for more social conflict, and
racial animosity seems in many ways to be increasing even as integration proceeds and incomes of African Americans rise
rapidly relative to those of whites. Surely it would be better to
apply the lesson of the Wars of Religion and keep government
out of this sensitive area: Repeal laws that grant or deny rights
or privileges on the basis of race, and establish separation of race
and state.
\switchcolumn
那么引起分裂的种族问题呢?难道我们对政府支持的延续
多少代人的种族歧视还没受够吗?先不说奴隶制,奴隶制是对
个人权利如此可怕的侵犯,以至于不能仅仅归类于种族歧视。
奴隶制结束之后,美国宪法当中加入了三个修正案,每个修正
案都是根据《独立宣言》 的承诺而设计,以保护每个(男性)
美国人的平等权利。这几个修正案废除了奴隶制,承诺保护所
有公民在法律面前的平等权利,保证不因任何人的种族身份而
否认其选举权。但是若干年之后,一些州政府在联邦法院的默
许之下,又开始限制黑人的选举权,限制他们使用公共设施,限制他们进入经济生活。吉 姆 $\cdot$ 克劳时期一直持续到I960年
代。随后,很不幸的是,联邦政府很快完全拋开古典自由主义
的平等权利政策,眨眼之间把旧形式的种族歧视换成了新的种
族歧视 --- 雇员配额制、职位预留制、强制种族优先权\footnote{雇员配额制度(hiring quotas),指美国法律规定为企业单位等规定的雇员中各人种的比额。职位预留制度(Set-asides),指美国法律规定公立和私立单位为少数族裔预留一定职位。强制种族优先权(mandatory  racial preferences)指美国法律规定在公私立机构中,同一职位,申请人在其他条件相等的情况下,少数民族裔申请人优先。}。吉姆$\cdot$ 克劳法激怒了黑人,而新的种族配额制度则激怒了白人。
同时,这两种制度都激怒了所有相信平等权利的人。尽管这期间种族整合取得进步,同时黑人与白人相比,收入增长更快,
但是种族配额制度为更多的社会冲突搭起了舞台,从很多方面来看,种族间的仇视都在增长。很显然,我们应当吸取宗教战
争的教训,让政府远离敏感区域:废除那些按照种族身份而给部分人授予特权或剥夺权利的法律,让政府与种族分离。
\switchcolumn*
At the same time, we should take a critical look at policies
that have a disproportionately negative impact on those who
have long suffered at the hands of government. Taxes and regulations that impede new businesses and job creation, for instance, especially hurt those who are not already part of the
economic mainstream. Benjamin Hooks, who went on to head
the NAACP, once bought a doughnut shop in Memphis from a
man who had owned it for twenty-five years. ``In those twenty-five years, they had passed all kinds of laws,'' he recalled. ``You
had to have separate rest rooms for men and women, you had to
have ratproof walls and everything on God's earth. We were hit
with all those regulations, and they cost us \$30,000. We had to
close the shop.'' He went on, ``It's obvious now that nobody, but
nobody, is buying into a decaying black ghetto except blacks
themselves. So the effect of some regulations is almost 100 percent to exclude blacks.'' Occupational licensing laws also work
like the medieval guilds to keep people out of good jobs. In
cities like Miami, Chicago, and New York, it costs tens of thousands of dollars to buy a taxicab license, so an otherwise easy
form of entrepreneurship is closed to people who don't already
have capital.
\switchcolumn
与此同时,我们还应当以批评的眼光来看待那些对长期苦
于政府之手的人所产生的非均衡的消极影响的政策。那些阻碍
创业和产生工作机会的税收和管制,实际上会对那些经济地位
不髙的人造成更大的伤害。即 将 担 任 “有色人种进步国家协
会” (NAACP)主席的胡克斯(Benjamin Hooks) 曾经在孟菲
斯那里买下一家甜甜圈店,在把店卖给他之前,店主经营了这
家店25年。“在25年当中,他们通过了各种各样的法律他
回忆道,“你必须把店里的休息室分成男女两部分,你必须有
防鼠墙,你必须有上帝的地球上的所有东西。我们不得不满足
所有这些规定,这些东西花掉了我3 万美元。我不得不关了这
家店。” 他继续说道, “很显然,现在除了黑人外没人愿意搬进衰败的黑人街区。因此,这些管制的结果就是彻头彻尾的排
斥黑人。” 职业资格法也是一种中世纪的行业管制,它让人们
无法得到好工作。在一些城市,如迈阿密、芝加哥和纽约,购
买一份出租车运营执照将花费数万美元,因此,对于那些缺乏
资金的人来说,那种轻松创业的机会就对他们关上了大门。
\switchcolumn*
One government policy whose discrimination against blacks
goes largely unnoticed is the politically untouchable Social Security system. I'll say more about the overall system in chapter
10, but let me note here that like any massive, government-run, one-size-fits-all monopoly, Social Security was designed for
a ``typical'' 1930s family. It doesn't work so well for people who
don't fit the pattern. Unmarried and childless people are required to pay for survivor's insurance that they wouldn't choose
to buy from a private insurer. Married working women can't
get both spouse's benefits and their own, even though they
must pay for them. And black people---because they have
lower life expectancies than whites---pay the same taxes but receive far fewer benefits than whites. A study by the National
Center for Policy Analysis found that a white male entering the
labor market in 1986 could expect to receive 74 percent more
in Social Security retirement benefits and 47 percent more in
Medicare than a black male. A white working couple could expect about 35 percent more benefits than a black working couple. The disparity is strong at every level of income. A private,
competitive retirement savings system would offer different
plans to meet the needs of different people instead of one plan
for everybody. As we eliminate racial preferences in the law, we
should also seek to repeal laws that disproportionately harm
poor people and minorities.
\switchcolumn
有一个政策当中所包含的对黑人的歧视很大程度上没有被
.注意到,这个政策就是政治上不可讨论的社会保障系统。我将
在第十章当中对整个社保系统进行更详细的讨论,在这里提起
注意的是:像其他庞大的国营的奉行“一刀切”政策的垄断
企业一样,社保系统是为1930年 代 的 “典型” 家庭而设计
的。对于那些装不进它的模式的人来说,这个系统就效果很
差。未婚的和没有孩子的人被要求购买遗属保险,而如果让他
们从私营保险公司购买的话,他们是不会选择这项保险的。已
婚的工作妇女不能同时得到她丈夫和她自己的受益金,尽管两
份保险他们都必须支付。黑人和白人支付同样的税,但是他们
只能得到比白人少得多的退休金,因为他们的预期寿命远低于
白人。美国国家政策研究中心的一项研究发现,一 个 1986年
开始工作的白人男人从社保系统当中有望拿到的退休金比一名
黑人男人多74\% ,而从医疗保险当中有望得到的受益金则多
47\% 。一对有工作的白人夫妇有望得到的受益金则比一对有工
作的黑人夫妇要多大约35\% 。这种社保收入上的差别在每个
收入级别的人当中都存在。一个私营的、有竞争力的退休储蓄
体系将会提供不同的退休计划,满足不同人的需要,而不是用
一个计划对付所有人。当我们在法律上消灭种族偏袒的时候,
我们同时也应当寻求消除法律对穷人和少数民族的不对称的
伤害。
\switchcolumn*
Now, in this as in so many areas, the libertarian solution is
not a panacea. Social conflict over education, childbearing, and race will not end even with a constitutional amendment separating all of them from government interference. After all, the
First Amendment has not ended legal and political battles over
the government's relationship with religion. But it surely has
limited and confined those battles, and the legal battles over
where to draw the line in the other areas would be fought on
narrower grounds than today's conflicts, where an expansive
government reaches into every corner of American life. Depoliticization of our cultural disagreements would go a long
way toward deescalating the cultural war.
\switchcolumn
现在,在以上讨论的以及其他许多领域,古典自由主义的解决办法不是一种万灵药。即便通过一个宪法修正案,将这些
领域与政府干预分开,在教育、小孩抚育和种族方面的社会冲
突也不会终止。毕竟,第一修正案也没有终结政府与宗教之间
的法律和政治斗争。但是它的确对这些斗争进行了限制,而今
后 ,在这个庞大的政府将触手扩展到人们生活每个角落的国
度 ,众多领域中关于如何划分政府界限的法律斗争将被限制在
比今天更狭窄的范围内。使文化争论非政治化,减少文化战争
的烈度,我们还有很长的路要走。

\end{paracol}