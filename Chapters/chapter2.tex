\chapter{THE ROOTS OF LIBERTARIANISM\\古典自由主义的渊源}
\begin{paracol}{2}
\hbadness5000

In a sense there have always been but two political
philosophies: liberty and power. Either people should
be free to live their lives as they see fit, as long as they respect
the equal rights of others, or some people should be able to use
force to make other people act in ways they wouldn't choose.
It's no surprise, of course, that the philosophy of power has always been more appealing to those in power. It has gone by many names---Caesarism, Oriental despotism, theocracy, socialism, fascism, communism, monarchism, ujamaa, welfarestatism---and the arguments for each of these systems have
been different enough to conceal the essential similarity. The
philosophy of liberty has also gone by different names, but its
defenders have always had a common thread of respect for the
individual, confidence in the ability of ordinary people to make
wise decisions about their own lives, and hostility to those who
would use violence to get what they want.
\switchcolumn
从某种意义上说,有且只有两种政治哲学:自由哲学和权
力哲学。或者是人们自由地按照他们认为适合的方式生活,同
时尊重别人的同等权利;或者是有人强制其他人做他们不愿意
做的事情。当然,很自然,权力哲学对那些拥有权力的人具有
相当的吸引力。权力哲学曾以各种面目出现:君主专制(Caearism)、东方专制主义、神权政治、社群主义、法西斯主义、
君主政治、乌贾马\footnote{乌贾马(ujamaa),在斯瓦西里话中意为“家庭、宗教”,是坦桑尼亚在1961年独立后由尼雷尔总统推动的社会经济运动的基础概念。}主义、福利国家主义,它们各不相同,以至于掩盖了他们之间本质上的相似之处。自由哲学也曾经有过不同的名字,但是它们的捍卫者的思路是一样的:对个人权利的尊重,对普通民众做出明智决策的信心,对那些试图使用暴
力获得他们所追求的东西的人的敌视。
\switchcolumn*
The first known libertarian may have been the Chinese
philosopher Lao-tzu, who lived around the sixth century B.C.
and is best known as the author of the \textit{Tao Te Ching}. Lao-tzu advised, ``Without law or compulsion, men would dwell in harmony.'' The Tao is a classic statement of the spiritual serenity
associated with Eastern philosophy. The Tao consists of yin and
yang; that is, it is the unity of opposites. It anticipates the theory of spontaneous order by teaching that harmony can be achieved through competition. And it advises the ruler not to
interfere in the lives of the people.
\switchcolumn
我们所知道的最早的古典自由主义者也许是中国的哲学家
老子。他大约生活在公元前6 世纪,因 为 其 著 作 《道德经》
而闻名于世。老 子 认 为 “没有法和强制,人们将和谐发展”
( “我无为而民自化”)。“道”是东方哲学中一种对超越的永
恒存在状态的描述。道由阴和阳所组成,它是对立双方的统
一。它超前地提出了自发秩序的理论,指出和谐可以通过竞争
来达到,告诫统治者不要干预人民的生活。
\switchcolumn*
Despite the example of Lao-tzu, libertarianism really arose in
the West. Does that make it a narrowly Western idea? I don't
think so. The principles of liberty and individual rights are universal, just as the principles of science are universal, even
though most of the discovery of those scientific principles took
place in the West.
\switchcolumn
除了老子这个例外,古典自由主义其实是从西方兴起的。
然而,这说明古典自由主义仅仅是一种西方思想吗?我不这么
认为。自由和个人权利的原理是普遍适用的,就像科学原理是
普遍适用的一样,尽管绝大多数科学原理都在西方被发现。

\switchcolumn*[\section{The Prehistory of Libertarianism\\ 古典自由主义的史前史}]
Both the two main lines of Western thought, the Greek and the
Judeo-Christian, contributed to the development of freedom.
According to the Old Testament, the people of Israel lived
without a king or any other coercive authority, governing themselves not by force but by their mutual adherence to their
covenant with God. Then, as recorded in the First Book of
Samuel, the Jews went to Samuel and said, ``Make us a king to
judge us like all the other nations.'' But when Samuel prayed to
God about their request, God said,
\switchcolumn
西方思想的两条主线:古希腊思想和犹太---基督教思想,
对自由的扩展做出了贡献。按 照 圣 经 《旧约全书》 的描述,
以色列人原本没有国王和任何其他强制性政权,而是通过共同
遵守与上帝的契约而进行自治。根 据 《撒母耳记上》 记载,
犹太人找到撒母耳,对他说:“为我们立一个王治理我们,像
列国一样。” 但是当撒母耳就这件事情对上帝祷告的时候,上
帝说:
\switchcolumn*
\begin{quotation}
	This will be the manner of the king that shall reign over you: He
	will take your sons, for his chariots. And he will take your daughters, to be cooks. And he will take your fields, and your olive-yards, and give them to his servants. And he will take the tenth
	of your seed, and of your vineyards, and of your sheep. And ye
	shall be his servants.
	
	And ye shall cry out in that day because of your king which ye
	shall have chosen, and the Lord will not hear you in that day.
\end{quotation}
\switchcolumn
\begin{quotation}
	管辖你们的王必这样行,他必派你们的儿子为他赶
	车、跟马,奔走在车前。又派他们作千夫长、五十夫长,
	为他耕种田地,收割庄稼,打造军器和车上的器械。必取
	你们的女儿为他制造香膏,做饭烤饼;也必取你们最好的
	田地、葡萄园、橄榄园賜给他的臣仆。你们的粮食和葡萄
	园所出的,他必取十分之一给他的太监和臣仆;你们的羊
	群,他必取十分之一,你们也必作他的仆人。
	
	那时你们必因所选的王哀求耶和华,耶和华却不应允
	你们。
\end{quotation}
\switchcolumn*
Although the people of Israel defied this awful warning and
created a monarchy, the story served as a constant reminder
that the origins of the state were by no means divinely inspired.
God's warning resonated not just in ancient Israel but on down
to modern times. Thomas Paine cited it in \textit{Common Sense} to remind Americans that ``the few good kings'' in the 3,000 years since Samuel could not ``blot out the sinfulness of the origin'' of
monarchy. The great historian of liberty, Lord Acton, assuming that all nineteenth-century British readers were familiar with it, referred casually to Samuel's ``momentous protestation.''
\switchcolumn
尽管以色列人最终没有听从这个可怕的警告,建立了一个君主制国家,但这个故事仍然成为一个永恒的提示:国家的起源根本不是神圣的。上帝的警告不仅回响在古代以色列而且回响在现代社会。潘恩在《常识》一书中引用这段话来提醒美国人:自撒母耳时代以来的三千年当中的“少数圣君”并不能 “洗清君主制原罪”。伟大的自由史学家阿克顿勋爵曾简单地用撒母耳的“历史性宣言” 来指代这段话,他 认为19世纪的英国读者都熟知这段话。

\switchcolumn*
Although they installed a king, the Jews may have been the
first people to develop the idea that the king was subordinate to
a higher law. In other civilizations, the king was the law, generally because he was considered divine. But the Jews said to the
Egyptian Pharaoh, and to their own kings, that a king is still
just a man, and every man is judged by God's law.
\switchcolumn
尽管他们拥立了一个国王,但是犹太人也许是第一个提出
国王应在更高的法之下的理念的民族。在其他文明中,国王就
是法律,因为人们认为国王就是神。但是犹太人对埃及法老说
(就像对他们自己的国王说的一样):国王只不过是一个人,
只要是人就应当被上帝的法律所审判。

\switchcolumn*[\subsection{Natural Law\\自然法}]
That concept of a higher law also developed in ancient Greece.
The playwright Sophocles, in the fifth century B.C., told the
story of Antigone, whose brother Polyneices had attacked the
city of Thebes and been killed in battle. For his treason the
tyrant Creon ordered that his body be left to rot outside the
gates, unburied and unmourned. Antigone defied Creon and
buried her brother. Brought before Creon, she declared that a
law made by a mere man, even a king, could not override ``the
gods' unwritten and unfailing laws,'' which had existed longer
than anyone could say.
\switchcolumn
“更高的法” 的概念也在古希腊发展了出来。公元前5 世
纪,剧作家索福克勒斯写了一个安提戈涅的故事。安提戈涅的
兄弟波吕涅克斯在一次进攻底比斯的战斗中死了。独裁者克利
翁认为他是背叛,下令把他的尸体扔到城门外任其腐烂,不许
埋葬,不许任何人去悼念。但安提戈涅无视克利翁的命令,埋
葬了她兄弟。她随后被带到了克利翁面前,她对克利翁说:由
人写的法律,哪怕是国王写的法律,不 能 凌 驾 于 “神的永恒
的法律之上,尽管神的法律不立文字” ,但神的法律永存。
\switchcolumn*
The notion of a law by which even rulers could be judged endured and grew throughout European civilization. It was developed in the Roman world by the Stoic philosophers, who
argued that even if the ruler is the people, they still may do only
what is just according to natural law. The enduring power of
this Stoic idea in the West was partly due to a happy accident:
The Stoic lawyer Cicero was regarded in later years as the greatest writer of Latin prose, so his essays were read by educated Europeans for many centuries.
\switchcolumn
这种统治者也要接受法律裁判的理论在整个欧洲文明史中
得到存续发展。古罗马斯多噶学派的哲学家们对这种理论进行
了发展,他们认为即便人民本身就是统治者,他们也只能依据
自然法去做应该做的事。斯多噶学派的理念之所以在西方世界
有长久的影响力,应该归因于一件很幸运的事情:斯多噶学派
的律师西塞罗被认为是最伟大的拉丁语散文作家,在很多个世
纪当中,无数受过教育的欧洲人都读过他的散文。
\switchcolumn*
Shortly after Cicero's time, in a famous encounter, Jesus was
asked whether his followers should pay taxes. ``Render unto
Caesar the things that are Caesar's and unto God the things
that are God's,'' he replied. In so doing he divided the world
into two realms, making it clear that not all of life is under the
control of the state. This radical notion took hold in Western
Christianity, though not in the Eastern Church, which was totally dominated by the state, leaving no space in society where
alternative sources of power might develop.
\switchcolumn
西塞罗时代之后不久,在一次著名的辩论当中,耶稣被问
到他的追随者是不是应当纳税。他回答道: “让凯撒的归凯
撒,让上帝的归上帝。” 这样,耶稣就把世界分为两个领域,
清楚地表明了并不是人们所有的生活都应该在国家的控制之
下。这种观念后来成为西方基督教会的基本理念,但东方的东
正教会却并不赞成这种理论。因此,东正教会一直由国家所控
制,没有给社会留下能够替代国家权力的资源生长的空间。

\switchcolumn*[\subsection{Pluralism\\多元主义}]
The independence of the Western Church, which came to be
known as Roman Catholic, meant that throughout Europe
there were two powerful institutions contending for power.
Neither state nor church particularly liked the situation, but
their divided power gave breathing space for individuals and
civil society to develop. Popes and emperors frequently denounced each other's character, contributing to a delegitimation of both. Again, this conflict between church and state was
virtually unique in the world, which helps to explain why the
principles of freedom were discovered first in the West.
\switchcolumn
西方教会的独立(后来被称为罗马天主教),意味着整个
欧洲开始有两种强大的势力在竞争权力。国家和教会都不喜欢
这种状况,但是它们之间的分权使得个人和市民社会的发展有
了些许空间。教皇和皇帝们经常对对方进行公开谴责,双方的
权威和声望因此都丧失殆尽。而实际上,欧洲的教会和国家之
间的这种冲突在世界上是独一无二的,这也部分解释了为什么
自由的理念首先在西方出现。
\switchcolumn*
In the fourth century the emperor Theodosius ordered the
bishop of Milan, St. Ambrose, to hand over his cathedral to the
empire. Ambrose rebuked the emperor, saying,
\switchcolumn
4 世纪的时候,罗马的狄奥多西皇帝命令米兰大主教圣安
布罗斯把他的教堂交给帝国。圣安布罗斯对皇帝的要求进行了
谴责,他说:
\switchcolumn*
\begin{quote}
	It is not lawful for us to deliver it up nor for your majesty to receive it. By no law can you violate the house of a private man. Do
	you think that the house of God may be taken away? It is asserted that all things are lawful to the emperor, that all things are
	his. But do not burden your conscience with the thought that
	you have any right as emperor over sacred things. Exalt not yourself, but if you would reign the longer be subject to God. It is
	written, God's to God and Caesar's to Caesar.
\end{quote}
\switchcolumn
\begin{quote}
	我们把教堂交给你是非法的,而陛下接受教堂同样也
	是非法的。没有任何法律支持您侵犯个人的住房。您认为
	上帝的房子能被夺走吗?有云,普天之下莫非王土,在法
	律上天下所有的东西都属于皇帝。但是请勿失却您的良
	知,认为您作为皇帝就能够拥有上帝的财产。荣耀您的不
	是您自己,如果您希望您的统治长久的话,请服从上帝。
	有云:上帝的归上帝,凯撒的归凯撒。
\end{quote}
\switchcolumn*
The emperor was forced to come to Ambrose's church and beg
forgiveness for his wrongdoing.
\switchcolumn
随后,皇帝被迫到圣安布罗斯的教堂,请求原谅他的
错行。
\switchcolumn*
Centuries later a similar conflict took place in England. The
archbishop of Canterbury, Thomas a Becket, defended the
church's rights against Henry II's usurpations. Henry wished
aloud that he could be rid of ``this meddlesome priest,'' where-upon four knights rode off to murder Becket. Within four years
Becket had been made a saint, and Henry had been forced to
walk barefoot through the snow to Becket's church as penance
for his crime and to back down from his demands on the
church.
\switchcolumn
几个世纪之后,一场类似的冲突发生在英国。坎特布雷大
主教贝克特(Thomas Becket) 反对国王亨利二世对教会权利
的剥夺。亨利发誓要除掉“这个多管闲事的牧师”,于是派了
四名骑士杀害了贝克特。而四年后,贝克特被教廷尊封为圣
徒,而亨利则被迫赤脚走过雪地到贝克特的教堂忏悔自己的罪
行 ,并且收回了对教会权利的要求。
\switchcolumn*
Because the struggle between church and state prevented
any absolute power from arising, there was room for autonomous institutions to develop, and because the church
lacked absolute power, dissident religious views were able to ferment. Markets and associations, oath-bound relationships,
guilds, universities, and chartered cities all contributed to the
development of pluralism and civil society.
\switchcolumn
教会和国家权力之间的斗争防止了绝对权力的产生,给自
治组织的发展留下了空间;同时,由于教会没有掌握绝对的权
力,异端宗教和异端理论才能够孕育发展。随后,市场和民间
组织、契约关系、行会、大学和自治城市发展起来,促成了多
元主义和市民社会的发展。

\switchcolumn*[\subsection{Religious Toleration\\宗教自由}]
Libertarianism is often seen as primarily a philosophy of economic freedom, but its real historical roots lie more in the struggle for religious toleration. Early Christians began to develop
theories of toleration to counter their persecution by the Roman
state. One of the earliest was Tertullian, a Carthaginian known
as ``the Father of Latin theology,'' who wrote around A.D. 200,
\switchcolumn
古典自由主义常常被认为主要是一种经济自由主义的理
论 ,但是它的历史渊源更多的来自争取宗教信仰自由的斗争。
早期基督教开始产生信仰自由的理论,以对抗罗马帝国的迫
害。最早主张宗教自由的哲学家之一迦太基人德尔图良\footnote{德尔图良(Tertullian, 150$\sim$230),基督教著名神学家和哲学家。生于迦	太基,也卒于此地$\cdot$被誉为罗马教会的教父和神学鼻祖之一。在神学上,他首先提出 “三位一体” 的理论。},被认为是 “拉丁神学之父” ,公元200年左右他写道:
\switchcolumn*
\begin{quote}
	It is a fundamental human right, a privilege of nature, that every
	man should worship according to his own convictions. One
	man's religion neither harms nor helps another man. It is assuredly no part of religion to compel religion, to which free will
	and not force should lead us.
\end{quote}
\switchcolumn
\begin{quote}
	每个人根据他自己的确信而信仰,这是一项基本的人
	权,一种自然的权利。一个人的信仰既不会伤害别人,也
	不会帮助别人。毫无疑问,没有任何一种宗教能够强制别人信仰。只有自由意志才能引导我们,而不是强制我们。
\end{quote}
\switchcolumn*
Already the case for freedom is being made in terms of fundamental, or natural, rights.
\switchcolumn
此时,自由已经成为一种最基本的、最自然的权利。
\switchcolumn*
The growth of trade, of varying religious interpretations, and
of civil society meant that there were more sources of influence
within each community, and that pluralism led to demands for
formal limitations on government. In one remarkable decade
there were major steps toward limited, representative government in three widely dispersed parts of Europe. The most fa-
mous, at least in the United States, took place in England in
1215, when the barons confronted King John at Runnymede
and forced him to sign Magna Carta, or the Great Charter,
which guaranteed every free man security from illegal interference in his person or property and justice to everyone. The
king's ability to raise revenue was limited, the church was guaranteed a degree of freedom, and liberties of the boroughs were
confirmed.
\switchcolumn
贸易的增长、对宗教不同解释的出现、市民社会的扩展,
意味着在每个社会中出现了越来越多的能够产生影响力的力
量 ,多元社会的出现就产生了限制政府权力的正式要求。于
是 ,在一个引人注目的十年当中,在欧洲三个不同地方出现了
朝向有限的代议制政府的重大进步。最 广 为 人 知 的 (至少在
美国是最广为人知的)一次出现在1215年的英国,贵族们与
英王约翰在兰尼米德(Runnymede)会面,迫使国王签署了
《大宪章》 (Magna Carta或 Great Charter)。大宪章保护每个自
由人的人身和财+ 安全不受非法侵害,保证司法公正。国王增
加税收的权力受到了限制,教会获得了一定程度的自由,自治
城市的自由也得到了确认。
\switchcolumn*
Meanwhile, around 1220 the German town of Magdeburg
developed a set of laws that emphasized freedom and self-government. Magdeburg law was so widely respected that it was
adopted by hundreds of the newly forming towns all over central Europe, and legal cases in some central-eastern European
towns were referred to Magdeburg judges. Finally, in 1222 the lesser nobles and gentry of Hungary---then very much a part of the European mainstream---forced King Andrew II to sign the Golden Bull, which exempted the gentry and clergy from
taxation, granted them freedom to dispose of their domains as
they saw fit, protected them from arbitrary imprisonment and
confiscation, assured them an annual assembly to present
grievances, and even gave them \textit{the Jus Resistendi}, the right to
resist the king if he attacked the liberties and privileges of the
Golden Bull.
\switchcolumn
与此同时,1220年左右 ,在德国一个城镇马格德堡(Magdeburg)产生了一系列主张自由和自治的法律。《马格德堡法》贏得了广泛的赞誉,并且被中部欧洲数以百计的新建立的城镇所采用,而一些中欧和东欧城镇的司法案件的判决也在引用马格德堡的判例。而 在 1222年,匈牙利的低等贵族和
士 绅 (在当时的欧洲,这些人很大程度上是主流阶层的一部分)迫使安德鲁二世国王签署了《金玺诏书》(Golden Bull),诏书免除了士绅和教士纳税的义务,确认了他们按照自己认为合适的方式处置自己领地的权利,保护他们不受任意监禁和没收财产的权利,确定每年召开会议处理他们的不满,甚至给了他们反抗权(Jus Resisten di),也就是当国王侵犯了他们的自由或者《金玺诏书》所授予的贵族特权的时候,他们反抗国王的权利。
\switchcolumn*
The principles found in these documents were far from full-fledged libertarianism; they still excluded many people from
their guarantees of liberties, and both Magna Carta and the
Golden Bull explicitly discriminated against Jews. Still, they
are milestones in a continuing advance toward liberty, limited
government, and the expansion of the concept of personhood to
include all individuals. They demonstrated that people all over
Europe were thinking about concepts of freedom, and they created classes of people jealous to defend their liberties.
\switchcolumn
当然,这些文件当中所体现的原则与真正的古典自由主义仍有相当的距离;它们仍然把多数人排除在保障自由的范围之
外,无 论 是 《大宪章》还 是 《金玺诏书》都公开歧视犹太人。
不过,它们仍然是朝向自由、有限政府、人权概念扩展到所有
个人的目标不断进步的历史进程中的里程碑。这些法律的出现
说明了欧洲各地的人们都在思考自由的概念,也激发了欧洲各
个阶层的人们保卫自己自由的情绪。
\switchcolumn*
Later in the thirteenth century, St. Thomas Aquinas, who
was perhaps the greatest of all Catholic theologians, and other
philosophers developed the theological argument for limits on
royal power. Aquinas wrote, ``A king who is unfaithful to his
duty forfeits his claim to obedience. It is not rebellion to depose
him, for he is himself a rebel whom the nation has a right to put
down. But it is better to abridge his power, that he may be unable to abuse it.'' Thus was theological authority put behind the
idea that tyrants could be deposed. Both John of Salisbury, an
English bishop who witnessed Becket's murder in the twelfth
century, and Roger Bacon, a thirteenth-century scholar-whom Lord Acton describes as the most distinguished English
writers of their respective epochs---defended even the right to
kill tyrants, an argument unimaginable virtually anywhere else
in the world.
\switchcolumn
13世纪后期,最伟大的基督教神学家托马斯$\cdot$阿奎那和
其他的哲学家从神学理论的角度提出了限制王权的思想。阿奎
那在著作中写道:“一个国王如果不能忠实于其责任,就无权要求别人服
从他。推翻这个国王就不能算是反叛,因为国王自己就是
一个反叛者,国家有权推翻他。但是更好的办法是剥夺其
权力,让他不能滥用。”看来神学思想对暴君应当被推翻的思想的产生起到了推动
作用。而见证了贝克特主教被杀害事件的英国的索尔兹伯里主
教约翰,以及被阿克顿勋爵称为那个伟大时代的最伟大的英国
作家的培根,这两个人甚至为杀死暴君的权利进行辩护。这样
的言论在世界其他任何地方都是不可想像的。
\switchcolumn*
The sixteenth-century Spanish Scholastic thinkers, sometimes known as the school of Salamanca, built on the work of
Aquinas to explore theology, natural law, and economics. They
anticipated many of the themes later found in the works of
Adam Smith and the Austrian School. From his post at the University of Salamanca, Francisco de Vitoria condemned the Spanish enslavement of the Indians in the New World in terms of individualism and natural rights: ``Every Indian is a man and thus
capable of achieving salvation or damnation$\ldots$ Inasmuch as
he is a person, every Indian has free will and, consequently, is
the master of his actions$\ldots$ Every man has the right to his
own life and to physical and mental integrity.'' Vitoria and his
colleagues also developed natural-law doctrine in such areas as
private property, profits, interest, and taxation; their works influenced Hugo Grotius, Samuel Pufendorf, and through them
Adam Smith and his Scottish associates.
\switchcolumn
16世纪西班牙的经院哲学家们,又被称为萨拉门卡学派
(Salamanca School), 在阿奎那的基础上对神学、自然法和经
济学进行了研究。他们在亚当$\cdot$ 斯密和奥地利学派之前,对亚
当 $\cdot$ 斯密和奥地利学派的许多领域都进行了探索。萨拉门卡大
学的弗朗西斯科$\cdot$ 德 $\cdot$ 维 多 利 亚 (Francisco de Vitoria) 从个
人主义和自然权利的角度对西班牙奴役新大陆印第安人的行为
进行了谴责:“每个印第安人都是人,都能获得神的拯救或者
惩罚……正因为他们是人,必然地,每个印第安人都有自由意志,都是他们自己行为的主人……每个人都有拥有自己的生
命的权利,都拥有完整的肉体和精神的权利。”维多利亚和
他的同事们还将自然法的原则应用到了私人产权、利润、利
息和税收等领域;他们的努力也影响到了格老秀斯和普芬多夫\footnote{萨缪尔$\cdot$冯$\cdot$普芬多夫(Samuel von Pufendorf, 1632$\sim$1694),德国法学家,古典自然法学派主要代表之一。著有《法学要论》《自然法和万民法》等。他力图使政治理论建筑在自我保存的基础上,强调依据自然法,人有保护自己生	命 、财产的权利。行为和社会关系的正当与否是由自然法的道德规范决定的,但国家的基础是人们共同的意愿。自然法对个人、国家和国家间的关系都是普遍适用的。国家的使命是保护人类的秩序和安全。他关于保证国家权力不受侵犯的论述为近代国际法提供了理论基础。},以及通过他们影响到了亚当$\cdot$ 斯密及其苏格兰学派。
\switchcolumn*
The prehistory of libertarianism culminates in the period of
the Renaissance and the Protestant Reformation. The rediscovery of classical learning and the humanism that marked the Renaissance are usually regarded as the emergence of the modern
world after the Middle Ages. With a novelist's passion, Ayn
Rand summed up one view of the Renaissance, that of the rationalist, individualist, secular strain of liberalism:
\switchcolumn
古典自由主义的史前史在文艺复兴和新教改革时期达到了
顶 峰 。 对古典时期经典的再发现以及人文主义的出现是文艺复
兴的标志,文艺复兴被认为是现代世界的开端和中世纪的结
束。安 $\cdot$兰德以小说家的热情把文艺复兴概括为一次理性主
义、个人主义以及世俗传统的自由主义运动:
\switchcolumn*
\begin{quote}
The Middle Ages were an era of mysticism, ruled by blind faith
and blind obedience to the dogma that faith is superior to reason.
The Renaissance was specifically the rebirth of reason, the liberation of man's mind, the triumph of rationality over mysticism---a faltering, incomplete, but impassioned triumph that led to the
birth of science, of individualism, of freedom.
\end{quote}
\switchcolumn
\begin{quote}
中世纪是一个神秘主义的时代,被迷信和对教条的盲
从所统治的时代,相信信仰高于理性。文艺复兴则是理性
的再生,人的头脑的解放,理性主义对神秘主义的胜利
---- 次虽然不稳固的、尚未完成,但是鼓舞人心的胜
利,导致了科学、个人主义和自由的诞生。
\end{quote}
\switchcolumn*
However, the historian Ralph Raico argues that the Renaissance
can be overrated as a progenitor of liberalism; the medieval
charters of rights and independent legal institutions provided a
more secure footing for freedom than the Promethean individualism of the Renaissance.
\switchcolumn
然而,历史学家拉尔夫$\cdot$ 瑞 科 (Ralph Raico) 认为文艺
箄兴不能被高估为自由主义的先声。相对于文艺复兴时期普罗
米修斯式的个人主义来说,中世纪国王和国家政权对贵族、市
民和其他人的特许权利,以及独立的司法制度才是自由的更加
坚实的基础。

\switchcolumn*
The Reformation contributed more to the development of
liberal ideas. The Protestant reformers, such as Martin Luther
and John Calvin, were by no means liberals. But by breaking
the monopoly of the Catholic Church they inadvertently encouraged a proliferation of Protestant sects, some of which---such as the Quakers and Baptists---did nurture liberal thought.
After the Wars of Religion people began to question the notion
that a community had to have only one religion. It had been
thought that without a single religious and moral authority, a
community would witness an endless proliferation of moral commitments and literally fall to pieces. That profoundly conservative idea has a long history. It goes back at least to Plato's
insistence on regulating even the music in an ideal society. It
has been enunciated in our own time by the socialist writer
Robert Heilbroner, who says that socialism requires ``a deliberately embraced collective moral goal'' to which ``every dissenting voice raises a threat.'' And it can also be heard in the fears of
the residents of rural Catlett, Virginia, who told the Washington
Post about their worries when a Buddhist temple was built in
their small town: ``We believe in one true God, and I guess we
were afraid with a false religion like that, maybe it would have
an influence on our children.'' Fortunately, most people noticed
after the Reformation that society did not fall apart in the presence of differing religious and moral views. Instead it became
stronger by accommodating diversity and competition.
\switchcolumn
而宗教改革运动对自由理念发展的贡献则更大。新教改革
家们如路德和加尔文绝不是自由主义者。但是他们通过打破天
主教会的垄断,无意中鼓励了各种新教派别的出现,其中的一
些教派例如贵格会、浸礼会的确孕育了自由主义思想。在宗教
战争之后,人们开始对一个社会只能有一种宗教观念产生质
疑。人们曾经认为,如果没有一个统一的宗教和道德权威,社
会将会陷入无休止的道德犯罪当中,并且最终四分五裂。这种
深厚的保守主义理念有着相当悠久的历史,至少可以追溯到柏
拉图,柏拉图坚持,在一个理想社会里甚至音乐都应该被管
制。而在我们这个时代,社会主义作家海尔布龙纳(Robert Heilbroner)则宣称:社会主义要求 “有意识地设立一个共同
的道德目标” ,而 “每一种不同的声音对社会都是一种威胁”。
这种观点在弗吉尼亚州卡特里特小镇的居民当中也能听到,他
们 告 诉 《华盛顿邮报》,他们对在他们小镇上建佛教寺庙表示
担忧:“我们只信仰一个真神 --- 上帝,我想我们对那种错误
的信仰很担忧,也许它会影响到我们的孩子。” 幸运的是,绝
大多数人在宗教革命后意识到社会并没有因为有不同的信仰和
道德准则而分裂。相反,它因为容纳了多样化和竞争而更加
强大。

\switchcolumn*[\subsection{The Response to Absolutism\\对专制主义的触动}]
By the end of the sixteenth century the church, weakened by its
own corruption and by the Reformation, needed the support of
the state more than the state needed the church. The church's
weakness created an opening for the rise of royal absolutism,
seen especially in the reigns of Louis XIV in France and the Stuart kings in England. Monarchs began to set up their own bureaucracies, impose new taxes, establish standing armies, and
make increasing claims for their own power. Drawing on the
work of Copernicus, who proved that the planets revolve
around the sun, Louis XIV called himself the Sun King because
he was the center of life in France, and he famously declared,
``\textit{L'etat, c'est moi}'' (``I am the state''). He banned Protestantism
and tried to make himself head of the Catholic Church in
France. During his reign of almost seventy years, he never
called a session of the representative assembly, the estates-general. His finance minister implemented a policy of mercantilism, under which the state would supervise, guide, plan, design,
and monitor the economy---subsidizing, prohibiting, granting
monopolies, nationalizing, setting wages and prices, and ensuring quality.
\switchcolumn
到 16世纪末,教会因其自身的腐败和宗教革命而衰落,
此时教会需要国家政权,甚于国家政权需要教会。教会的衰落
为王权专制的崛起创造了条件,尤其是法国的路易十四和英国
的斯图亚特王朝时期,君王们开始建立他们自己的官僚体系,
征收新的税收,建立常备军,不断扩张自己的权力。路易十四
模仿哥白尼创立的日心说,自 称 “太阳王”,意思是他自己是法国的中心。他还说出了那句臭名昭著的“朕即国家”。他禁
止了新教,并且试图让自己成为法国天主教会的领袖。他在位
的大约七十年当中,从未召集过一次各等级代表会议。他的财
政大臣颁布了一个重商主义的法令,根据这个法令,国家可以
对经济进行监督、引导、计划、设计和调节,国家可以设立垄
断企业、授予垄断权、给垄断企业补助,也可以禁止垄断,可
以实施国有化,可以管制价格和工资,可以管制商品质量。
\switchcolumn*
In England the Stuart kings also tried to institute absolute
rule. They sought to ignore the common law and to raise taxes without the approval of England's representative assembly, Parliament. But civil society and the authority of Parliament proved
more durable in England than on the Continent, and the Stuarts' absolutist campaign was stymied within forty years of
James I's accession to the throne. The resistance to absolutism
culminated in the beheading of James's son, Charles I, in 1649.
\switchcolumn
在英国,斯图亚特王朝的国王们也试图制定专制法律。他
们企图无视普通法,企图不通过英格兰的代表会议 --- 国会的
批准就提髙税收。但是,事实证明英国的市民社会和国会的权
威比欧洲大陆更加坚实。斯图亚特王朝的专制主义努力在詹姆
斯一世在位的四十年当中遭到了失败。而对专制主义的反抗在
1649年达到了顶点,这一年詹姆斯的儿子查理一世被送上了
断头台。
\switchcolumn*
Meanwhile, as absolutism took root in France and Spain, the
Netherlands became a beacon of religious toleration, commercial freedom, and limited central government. After the Dutch
gained their independence from Spain in the early seventeenth
century, they created a loose confederation of cities and
provinces, becoming the century's leading commercial power
and a haven for refugees from oppression. Books and pamphlets
by dissident Englishmen and Frenchmen were often published
in Dutch cities. One of those refugees, the philosopher Baruch
Spinoza, whose Jewish parents had fled Catholic persecution in
Portugal, described in his \textit{The ologico-Politkal Treatise} the happy
interplay of religious toleration and prosperity in seventeenth-century Amsterdam:
\switchcolumn
与此同时,由于专制主义在法国和西班牙生根发芽,荷兰
成为宗教自由、商业自由和有限中央政府理念的新灯塔。在
17世纪早期,荷兰从西班牙统治下获得独立之后,他们建立
了一种城市和各省份的松散联盟,成为17世纪领先的商业强权 ,一个容纳逃离压迫的难民的天堂。持不同政见的英国人和
法国人写的书和小册子常常是在荷兰的各个城市出版发行。其中的一个难民,哲学家斯宾诺莎(他的父母从葡萄牙的夭主
教迫害下逃到荷兰)在他的《神学政治论》 中描述了17世纪阿姆斯特丹的宗教自由、各种宗教的相互影响和繁荣的景象:
\switchcolumn*
\begin{quote}
The city of Amsterdam reaps the fruit of freedom in its own great
prosperity and in the admiration of all other people. For in this
most flourishing state, and most splendid city, men of every nation and religion live together in the greatest harmony, and ask
no questions before trusting their goods to a fellow-citizen. A citizen's religion and sect is considered of no importance: for it has
no effect before the judges in gaining or losing a cause, and there
is no sect so despised that its followers, provided that they harm
no one, pay every man his due, and live uprightly, are deprived of
the protection of the magisterial authority.
\end{quote}
\switchcolumn
\begin{quote}
阿姆斯特丹这个城市从自由当中获益甚丰,而这种自
由源于其自身伟大的繁荣和对所有人的尊重。在这个欣欣
向荣的国家和最辉煌的城市当中,不同民族、不同宗教的
人和谐地生活在一起,信任他们的商品就像信任自己的国人的商品一样,从不对他们的商品品质产生疑问。公民的
宗教和教派的不同被认为是无关紧要的:因为在法庭上这
并不会影响到他是会赢得官司还是输掉官司,如果他没有
伤害任何人,按时还债,生活正直,那么没有一种教派的
教徒会受到歧视以至于被剥夺受政府公正保护的权利。
\end{quote}

\switchcolumn*
Holland's example of social harmony and economic progress in-
spired protoliberals in England and other countries.
\switchcolumn
荷兰的社会和谐和经济发展的例子激励了英国和其他国家
自由主义雏形的产生。


\switchcolumn*[\section{The English Revolution\\英国革命}]
English opposition to royal absolutism created a great deal of
intellectual ferment, and the first stirrings of clearly protoliberal ideas can be seen in seventeenth-century England. Again, liberal ideas developed out of the defense of religious toleration.
In 1644 John Milton published Areopagitica, a powerful argument for freedom of religion and against official licensing of the
press. Of the relationship between freedom and virtue, an issue
that vexes American politics to this day, Milton wrote, ``Liberty
is the best school of virtue.'' Virtue, he said, is only virtuous if
chosen freely. On freedom of speech, he wrote, ``Who ever knew
Truth put to the worse in a free and open encounter?''
\switchcolumn
英国反对王权专制的斗争播下了大量知识的火种。最初的
清晰的自由主义理论雏形在17世纪的英国产生。 自由主义的
理念仍然产生于对宗教自由的辩护和桿卫过程当中。1644年,
弥尔顿\footnote{弥尔顿(John  Milton, 1608$\sim$1674),英国诗人、政论家,著有著名长诗《失乐园》《复乐园》《力士参孙》。政论作品有《论出版自由》 《论国王与官吏的职权》。}的对宗教自由进行强有力辩护、反对政府出版许可制度的著作《论出版自由》 出版。对于自由和美德之间的关系这个美国政客至今仍在争论不休的问题,弥尔顿写道:“自由是美德的最好学校。” 他说,美德只有当能够对其自由选择的时候才成其为美德。关于言论自由,他写道:“谁如果知道真理,请在自由公开的辩论中打败对手。”
\switchcolumn*
During the interregnum, the time after the beheading of
Charles I when England was between kings and under the rule
of Oliver Cromwell, there was tremendous intellectual debate.
A group known as the Levellers began enunciating the full set
of ideas that would come to be known as liberalism. They
placed the defense of religious liberty and the ancient rights of
Englishmen in a context of self-ownership and natural rights. In
a famous essay, ``An Arrow against All Tyrants,'' the Leveller
leader Richard Overton argued that every individual has a ``self-propriety''; that is, everyone owns himself and thus has rights
to life, liberty, and property. ``No man hath power over my
rights and liberties, and I over no man's.''
\switchcolumn
在查理一世被砍头之后没有国王的空位期,在克伦威尔统
治之下,英国出现了大量知识分子的辩论。一个被称为平权派
的群体开始发表后来被称为自由主义的一整套学说。他们把对
宗教自由的捍卫和英国人的古老权利置于自我所有权和自然权利的话语体系之下。在著名的文章“一箭射穿所有暴君” (An
Arrow against All Tyrants)中,平权派领导人欧威尔顿(Rich­
ard Overton) 说,每个个人都有“自我所有权” ,也就是说,每个人拥有自己的生命、自由和财产。 “没有人的权力能凌驾
于我的权利和自由之上,我也不会将权力凌驾于别人之上。”
\switchcolumn*
Despite the efforts of the Levellers and other radicals, the
Stuart dynasty returned to the throne in 1660, in the person of
Charles II. Charles promised to respect liberty of conscience and
the rights of landowners, but he and his brother, James II, again
tried to extend royal power. In the Glorious Revolution of
1688, Parliament offered the crown to William and Mary of
Holland (both grandchildren of Charles I). William and Mary
agreed to respect the ``true, ancient, and indubitable rights'' of
Englishmen, as put down in the Bill of Rights in 1689.
\switchcolumn
尽管有平权派和其他激进派的斗争,斯图亚特王朝仍然在
1660年重新上台,新的国王是查理二世。查理承诺尊重信仰
自由,尊重土地所有者的权利,但是他和他的弟弟詹姆斯二世
再一次试图扩张王权。在 1688年的光荣革命当中,国会给荷
兰的威廉和玛丽亚加冕(他们都是查理一世的孙辈)。威廉和
玛丽亚同意尊重英国人的“真实的、古老的、无可置疑的权
利” ,并且在1689年将其写人了《权利法案》。
\switchcolumn*
We can date the birth of liberalism to the time of the Glorious Revolution. John Locke is rightly seen as the first real liberal and as the father of modern political philosophy. If you
don't know the ideas of Locke, you really can't understand the
world we live in. Locke's great work \textit{The Second Treatise of Government} was published in 1690, but it had been written a few
years earlier, to refute the absolutist philosopher Sir Robert
Filmer, making its defense of individual rights and representative government that much more radical. Locke asked, what is
the point of government? Why do we have it? He answered, people have rights prior to the existence of government---thus
we call them natural rights, because they exist in nature. People
form a government to protect their rights. They could do that
without government, but a government is an efficient system
for protecting rights. And if government exceeds that role, people are justified in revolting. Representative government is the
best way to ensure that government sticks to its proper purpose. Echoing a philosophical tradition that had been entrenched in the West for centuries, he wrote, ``A Government is
not free to do as it pleases$\ldots$ The law of nature stands as an
eternal rule to all men, legislators as well as others.''
\switchcolumn
我们可以把光荣革命视作自由主义的生日。约 翰 $\cdot$洛克被
视为第一个真正的自由主义者,他也是现代政治学理论之父。
如果不知道洛克的理论,你就无法了解我们所生活的现代社
会。洛克的伟大著作《政府论》 (下篇)于 1690年出版,但
是这本书他在公开出版的几年前就写好了。在 《政府论》 当
中,洛克驳斥了专制主义哲学家罗伯特$\cdot$ 菲 尔 麦 爵 士 (Sir
Robert Filmer)的观点,这使得他对个人主义和代议制政府的
桿卫显得更加激进。洛克问道:政府是什么?我们为什么要有
政府?他自己回答道:人民拥有先于政府存在的个人权利 ---
我们称之为自然权利,因为这些权利是自然存在的。人们建立
政府是为了保护他们自己的权利。他们可以不通过政府来保护
自己的权利,但政府是一个保护权利的有效体制。如果政府的
角色超出这个限度,人民起义就是正义的。代议制政府是确保
政府忠实于其适当目标的最佳手段。他指出$\cdot$. “政府不能随其
喜好做事…… 自然法对所有人包括立法者和其他人都是永恒的律令”。这句话对西方存在了几个世纪的哲学传统也是一个
呼应。
\switchcolumn*
Locke also articulated clearly the idea of property rights:
\switchcolumn
洛克清晰地表述了财产权的概念:
\switchcolumn*
\begin{quote}
Every Man has a \textit{Property in his own Person}. This no Body has any
Right to but himself. The \textit{Labour} of his Body, and the \textit{Work} of his
Hands, we may say, are properly his. Whatsoever then he removes out of the State that Nature hath provided, and left it in,
he hath mixed his \textit{Labour} with, and \textit{joyned} to it something that is
his own, and thereby makes it his \textit{Property}.
\end{quote}
\switchcolumn
\begin{quote}
每个人都有建立\textbf{在自己本人的权利之上的所有权}。这
是除了他自己之外别人并不拥有的排他权利。他的身体进
行的\textbf{劳动}以及他的手做的\textbf{工作},可以说完全属于他自己。
那么,无论是他从这个国家挖出的任何自然出产的东西,
还是他种下的东西,他都已经在其中加入了自己的\textbf{劳动},
把自己所有的东西与自然的出产结合在了一起,使之成为了自己的\textbf{财产}。
\end{quote}
\switchcolumn*
People have an inalienable right to life and liberty, and they acquire a right to previously unowned property that they ``mix their labor with,'' such as by farming. It is the role of government to protect the ``Lives, Liberties, and Estates'' of the people.
\switchcolumn
人们对生命和自由拥有不可剥夺的权利,同时人们拥有对
事先没有所有权的无主财产“加入自己的劳动”,例如耕作,
从而获得财产的权利。政府的责任是保护人们的“生命、自
由和财产”。
\switchcolumn*
These ideas were enthusiastically received. Europe was still in
the grip of royal absolutism, but thanks to their experience with
the Stuarts, the English were suspicious of all forms of government. They warmly embraced this powerful philosophical defense of natural rights, the rule of law, and the right of
revolution. They also, of course, began carrying the ideas of
Locke and the Levellers on ships bound for the New World.
\switchcolumn
这些理念被人们热情地接受了。欧洲仍然在王权专制统治
的控制之下,但是由于他们在斯图亚特王朝下的经历,英国人
对所有形式的政府都充满怀疑。他们热情地拥抱了这种强有力
地捍卫自然权利、捍卫法治、拥护革命权利的理论。当然,他
们也开始把洛克和平权派的理念装上了船,驶向新大陆。

\switchcolumn*[\section{The Liberal Eighteenth Century\\自由的18世纪}]
England prospered under limited government. As Holland had
inspired liberals a century earlier, the English model began to
be cited by liberal thinkers on the Continent and eventually
around the world. We might date the Enlightenment from
roughly 1720, when the French writer Voltaire fled from French tyranny and arrived in England. He saw religious toleration,
representative government, and a prosperous middle class. He
noticed that trade was more respected than it was in France,
where aristocrats looked down their noses at those involved in
commerce. He also noticed that when you allow people to trade
freely, their prejudices may take second place to self-interest, as
in his famous description of the stock exchange in \textit{his Letters on England}:
\switchcolumn
英国在有限政府之下获得了繁荣。正如荷兰在一个世纪之
前激发了自由主义的思想一样,英国的例子开始被欧洲大陆的
自由主义思想家以及后来全世界的思想家所引用。我们也许可
以把启蒙运动的时间大致确定为1720年 ,这一年,法国作家伏尔泰逃离法国的专制统治,抵达英国。在那里,他看到了宗
教自由、代议制政府和数量庞大的中产阶级。他注意到贸易在
这里比在法国受到更多的尊重,在法国,贵族们对那些参与商
业活动的人嗤之以鼻。他还注意到,当你允许人们进行自由贸
易的时候,他们的偏见就会让位于对自己利益的追求。他在著
名 的《哲学通信》一书中这样描述股票交易的场面:
\switchcolumn*
\begin{quote}
Go into the London Stock Exchange---a more respectable place
than many a court---and you will see representatives of all nations
gathered there for the service of mankind. There the Jew, the Mohammedan, and the Christian deal with each other as if they were
of the same religion, and give the name of infidel only to those
who go bankrupt. There the Presbyterian trusts the Anabaptist,
and the Anglican accepts the Quaker's promise. On leaving these
peaceful and free assemblies, some go to the synagogue, others go
to drink . . . others go to their church to wait the inspiration of
God, their hats on their heads, and all are content.
\end{quote}
\switchcolumn
\begin{quote}
进入到伦敦股票交易所 --- 一个比许多法院更值得尊
敬的地方 --- 你可以看到来自所有国家的人们的代表聚集
在一起,目的是为人们服务。在这里,犹太教徒、穆斯林
和基督教徒相互之间进行交易,就好像他们是属于同一个
宗教一样,在他们看来,异教徒的称号只会给那些破产的
人。长老会派信徒信任再浸礼派信徒,圣公会信徒接受贵
格会信徒的承诺。离开这些和平和自由的集会之后,有的
人去犹太教堂,有的人去喝酒……另一些人则去基督教堂
等待上帝的启示,各取所需,各得其所。
\end{quote}
\switchcolumn*
The eighteenth century was the great century of liberal
thought. Locke's ideas were developed by many writers, notably John Trenchard and Thomas Gordon, who wrote a series of newspaper essays signed ``Cato,'' after Cato the Younger, the
defender of the Roman Republic against Julius Caesar's quest
for power. These essays, which denounced the government for
continuing to infringe upon the rights of Englishmen, came to
be known as \textit{Cato's Letters}. (Names reminiscent of the Roman
Republic were popular with eighteenth-century writers; compare the \textit{Federalist Papers}, which were signed ``Publius.'') 
\switchcolumn
18世纪是自由主义思想的伟大时代。洛克的思想被许多
思想家所发展,其中著名的是特伦查德(John Trenchard) 和
戈 登 (Thomas  Gordon), 后者用Cato的笔名在报纸上发表了
一系列文章。Cato的名字来源于古罗马时代反对恺撒攫取权
力的共和国捍卫者Cato。这些文章对政府不断干预英国人的
权利的行为进行了谴责,后来这些文章被称为《卡托信笺》
(取一个模仿罗马共和国时代的复古名字,是 18世纪作家的时髦;就像《联邦党人文集》的作者取笔名为“普布利尤斯”
一样)。\footnote{密尔顿在写作《联邦党人文集》时所用的笔名普布利尤斯来源于他所尊敬的古罗马执政官Publius Valerius Publicola。}
\switchcolumn*
In France the Physiocrats developed the modern science of economics. Their name came from the Greek \textit{physis} (nature) and \textit{kratos} (rule); they argued for the rule of nature, by which they meant that natural laws similar to those of physics governed society and the creation of wealth. The best way to increase the
supply of real goods was to allow free commerce, unhindered by
monopolies, guild restrictions, and high taxes. The absence of
coercive constraints would produce harmony and abundance. It
is from this period that the famous libertarian rallying cry ``laissez faire'' comes. According to legend, Louis XV asked a group of merchants, ``How can I help you?'' They responded, ``\textit{Laisseznous faire, laissez-nous passer, Le monde va de lui-meme}.'' (``Let us do,
leave us alone. The world runs by itself.'')
\switchcolumn
在法国,重 农 主 义 者 (physiocrats)开始发展出现代的经
济学。这个词来源于 希腊 语“ 自然”  (physis)和 “规则”
(kratos);他们讨论自然法则,认为自然法就像物理法则一样
统治着人类社会和创造财富的过程。增加实物供给的最好办法
是允许自由贸易,而不是通过垄断、行会限制以及高税收来实
现。消除强制将会创造出和谐和富裕。从这个时期开始,著名
的古典自由主义的口号“ 自由放任” 开始出现。相传路易十
五曾经问一群商人:“我能帮你们什么?” 商人回答:“让我们
做 ,甭管我们。世界将会自己运转”。
\switchcolumn*
The leading Physiocrats included Frangois Quesnay and
Pierre Du Pont de Nemours, who fled the French Revolution
and came to America, where his son founded a small business in
Delaware. An associate of the Physiocrats, A. R. J. Turgot, was
a great economist who was named finance minister by Louis
XVI, an ``enlightened despot'' who wanted to ease the burden
of government on the French people---and perhaps create more
wealth to be taxed, since, as the Physiocrats had pointed out,
``poor peasants, poor kingdom; poor kingdom, poor king.'' Turgot issued the Six Edicts to abolish the guilds (which had become calcified monopolies), abolish internal taxes and forced
labor (the \textit{corvee}), and establish toleration for Protestants. He
ran into stiff resistance from the vested interests, and he was
dismissed in 1776. With him, says Raico, ``went the last hope
for the French monarchy,'' which indeed fell to revolution thirteen years later.
\switchcolumn
著名的重农主义者包括魁奈和奈穆尔(Pierre Du Pont de
Nemours), 后者为了逃离法国革命而到了美国,他儿子在特
拉华州创建了一家小公司。重农学派的朋友杜尔哥(A.R.J. Turgot)是著名的经济学家,被路易十六任命为财政大臣。路易十六作为一个“开明的专制君王”希望减轻政府
对人民的负担,而这也许将会创造出更多的财富,也会创造出
更多的税收,因为正如重农学派所指出的那样,“农民穷则国
家穷,国家穷则国王穷”。杜尔哥颁布了六个法令废除行会
(当时的行会已经成了僵化的垄断组织),废除若干国内税和
强制徭役,允许人们信仰新教。随即他遭到了既得利益集团的
强烈反对,并 于 1776年被解职。就像拉尔夫$\cdot$瑞科所说的那
样 ,他 的 离 开 “带走了法国君主制的最后希望”,13年后法国
即陷入了大革命。
\switchcolumn*
The French Enlightenment is better known to history, but
there was an important Scottish Enlightenment as well. Scots
had long resented English domination, they had suffered
greatly under British mercantilism, and they had within the
past century achieved a higher literacy rate and better schools
than had the English. They were well suited to develop liberal
ideas (and to dominate English intellectual life for a century).
Among the scholars of the Scottish Enlightenment were Adam
Ferguson, author of \textit{Essay on the History of Civil Society}, who
coined the phrase ``the result of human action but not of human
design,'' which would inspire future scholars of spontaneous
order; Francis Hutcheson, who anticipated the utilitarians with
his notion of ``the greatest good for the greatest number''; and
Dugald Stewart, whose \textit{Philosophy of the Human Mind} was widely
read in early American universities. But the most prominent
were David Hume and his friend Adam Smith.
\switchcolumn
法国启蒙运动在历史上闻名遐迩,但是还有一场苏格兰启
蒙运动同样重要。苏格兰长期受到英格兰的统治,受到英国重
商主义伤害甚重,而他们在过去一个世纪当中达到了比英格兰
人更高的识字率,拥有更好的大学。因此在那里更适合发展出
自由主义的理念,而这种理念在一个世纪的时间内成为英国知识分子中的主流思想。苏格兰启蒙运动当中的著名学者包括
《市民社会的历史》(\textit{Essy on the History of Civel Society})的作
者 弗 格 森 (Adam  Ferguson),他 的 名 句 “人类行为的结果而
不是人类设计的结果”给后来研究自发秩序的学者们提供了
灵感;还 有 哈 奇 森 (Francis  Hutcheson) , 他 的 “为最大多数
人的最大利益” 的理论成为功利主义的先驱;以及斯图尔特
(Dugald  Stewart), 他 的 作 品 《人类思维论》(\textit{Philosophy of the Human Mind})在早期的美国大学中被广泛阅读。但是,在这些思想家中最著名的无疑是休漠和他的朋友亚当$\cdot$ 斯密。
\switchcolumn*
Hume was a philosopher, an economist, and a historian, in
the days before the university aristocracy decreed that knowledge must be divided into discrete categories. He is best known to contemporary students for his philosophical skepticism, but he also helped to develop our modern understanding of the productiveness and benevolence of the free market. He defended
property and contract, free-market banking, and the spontaneous order of a free society. Arguing against the balance-of-trade doctrine of the mercantilists, he pointed out that everyone
benefits from the prosperity of others, even the prosperity of
people in other countries.
\switchcolumn
休谟是一名哲学家、经济学家和历史学家、在以前的时
代,大学的贵族化决定了知识必须被分成互不相干的科目。当
代学生所熟悉的是他哲学上的怀疑论,但是他的理论贡献也大
大增进了我们对现代自由市场的生产和利他行为的理解。他为
自由社会的产权与契约、自由市场银行制度以及自发秩序而辩
护。对重商主义的贸易平衡的教条进行了批判,指出每个人都
会从别人的富裕当中受益,甚至也会从其他国家人民的富裕当
中受益。
\switchcolumn*
Along with John Locke, Adam Smith was the other father of
liberalism, or what we now call libertarianism. And since we
live in a liberal world, Locke and Smith may be seen as the architects of the modern world. In \textit{The Theory of Moral Sentiments}, Smith distinguished between two kinds of behavior, self-interest and beneficence. Many critics say that Adam Smith, or economists generally, or libertarians, believe that all behavior is
motivated by self-interest. In his first great book, Smith made
clear that that wasn't the case. Of course people sometimes act
out of benevolence, and society should encourage such sentiments. But, he said, if necessary, society could exist without
beneficence extending beyond the family. People would still get
fed, the economy would still function, knowledge would
progress; but society cannot exist without justice, which means
the protection of the rights of life, liberty, and property. Justice,
therefore, must be the first concern of the state.
\switchcolumn
与洛克一样,亚 当 $\cdot$ 斯 密 是 自 由 主 义 (也就是我们今天
称的古典自由主义)的另一个鼻祖。由于我们生活在一个自
由的世界,洛克和斯密也许应当被视为现代社会的设计师。在《道德情操论》 当中,斯密区别了两种行为:利己和利他。很
多人批评说,亚 当 $\cdot$ 斯密以及绝大多数经济学家、古典自由主义者都相信人类行为的所有动机都来自于利己。在他的第一本伟大著作中,斯密很明确地指出这是不正确的。人们当然会在有时表现出利他行为,人类社会也应当鼓励这样的情操。但是他说,如有必要的话,人类社会可以不需要这种由家庭而延伸出来的利他行为而存在,人们依然能够得到食物,经济仍然会运转,知识仍然会进步;但是人类社会如果没有公正就会消失,因为公正意味着对生命、自由和财产权利的保护。因此,公正应当是国家的第一要务。
\switchcolumn*
In his better-known book, \textit{The Wealth of Nations}, Smith laid
the groundwork for the modern science of economics. He said
that he was describing ``the simple system of natural liberty.'' In
the modern vernacular, we might say that capitalism is what
happens when you leave people alone. Smith showed how, when
people produce and trade in their own self-interest, they are led
``by an invisible hand'' to benefit others. To get a job, or to sell
something for money, each person must figure out what others
would like to have. Benevolence is important, but ``it is not
from the benevolence of the butcher, the brewer, or the baker,
that we expect our dinner, but from their regard to their own
interest.'' Thus the free market allows more people to satisfy
more of their desires, and ultimately to enjoy a higher standard
of living, than any other social system.
\switchcolumn
在他更广为人知的著作《国富论》 当中,斯密建立了现代经济学的基础。他说他在这本书中是在对“自然的自由社
会的朴素系统” 进行描述。用现代的话来说就是,资本主义是人们不受干涉而自然形成的状态。斯密论述了当人们为了利
己的目的而生产和交换的时候,他们是如何被“看不见的手”所引导而为他人造福的。为了得到一份工作,或者为了得到钱
而卖东西,每个人必须判断出别人希望得到什么。利他很重要 ,但 是 “我们并不是因为屠夫、酿酒师和面包师的利他行
为而获得晚餐,相反,我们是因为他们关心自己的利益而获得晚餐”。因此,自由市场比其他任何一种社会制度都能够让更
多的人满足自己的更多欲望,享受更高品质的生活。
\switchcolumn*
Smith's most important contribution to libertarian theory
was to develop the idea of spontaneous order. We frequently 
hear that there is a conflict between freedom and order, and
such a perspective seems logical. But, more completely than the
Physiocrats and other earlier thinkers, Smith stressed that order
in human affairs arises spontaneously. Let people interact freely
with each other, protect their rights to liberty and property, and
order will emerge without central direction. The market economy is one form of spontaneous order; hundreds or thousands---or today, billions---of people enter the marketplace or
the business world every day wondering how they can produce
more goods or get a better job or make more money for themselves and their families. They are not guided by any central authority, nor by the biological instinct that drives bees to make
honey, yet they produce wealth for themselves and others by
producing and trading.
\switchcolumn
斯密对古典自由主义的最重大贡献就是提出了自发秩序的
概念。我们常常听到人们说在自由和秩序之间存在着冲突,这
种观点看上去很合逻辑。但是斯密强调人类生活的秩序是自发
形成的。这一点斯密表达得比重农学派和其他早期思想家要更
加清楚和全面。斯密认为,只要让人们自由地交往,保护他们
的自由和财产权,秩序就会自动出现,而不需要中央指令。市
场经济是一种自发秩序;数以十万计的人们,或者在今天数以
十4乙计的人们每天进入市场或者商业的世界,操心如何为了自
己和家庭而生产更多的产品、得到一份更好的工作或者挣更多
的钱。并没有一个中央政府来引导他们,他们也不是像蜜蜂一
样由生物本能驱使来制造蜂蜜,他们是通过生产和交换来为自
己和他人创造财富。
\switchcolumn*
The market is not the only form of spontaneous order. Consider language. No one sat down to write the English language
and then teach it to early Englishmen. It arose and changed
naturally, spontaneously, in response to human needs. Consider
also law. Today we think of laws as something passed by Congress, but the common law grew up long before any king or
legislature sought to write it down. When two people had a
dispute, they asked another to serve as a judge. Sometimes juries were assembled to hear a case. Judges and juries were not
supposed to ``make'' the law; rather, they sought to ``find'' the
law, to ask what the customary practice was or what had been
decided in similar cases. Thus, in case after case the legal order
developed. Money is another product of spontaneous order; it
arose naturally when people needed something to facilitate
trade. Hayek wrote that ``if [law] had been deliberately designed, it would deserve to rank among the greatest of human
inventions. But it has, of course, been as little invented by any
one mind as language or money or most of the practices and
conventions on which social life rests.'' Law, language, money,
markets---the most important institutions in human society---
arose spontaneously.
\switchcolumn
市场并不是唯一的自发秩序。想想人类的语言吧。没有人坐下来写下英语并且把它教给古代英国人。语言的产生和改变
是自然的、自发的,是为满足人们的需要而出现的 6 再想想法
律。今天,我们会认为法律是国会通过的,但是普通法在任何
国王或立法机构企图制定法律之前很长时间就一直存在。当两
个人有了争端的时候,他们会找到另一个人作为法官,有时候
会组织一群人作为陪审团来听取这个案子。法官和陪审团并不
会 “制定” 法律,而 是 会 去 “发现” 法律,他们会问按照惯
例会怎么办,或者在以前的类似案子当中是怎么判决的。于
是,一个案例接着一个案例,法律秩序就产生了。货币是自发
秩序的另一个产物;当人们需要一种东西来方便交换的时候,
货币就自然出现了。哈耶克写道:“如 果 (法律)是被有意识
发明的,它就应当位列人类最伟大的发明之中,但是它并不是
被任何单独的个人的大脑所发明,就像语言、货币和其他绝大
多数人类社会生活所依赖的习俗和行为一样。”人类社会最重
要的制度 --- 法律、语言、货币、市场,都是自发形成的。
\switchcolumn*
With Smith's systematic elaboration of the principle of spontaneous order, the basic principles of liberalism were essentially
complete. We might define those basic principles as the idea of
a higher law or natural law, the dignity of the individual, natural rights to liberty and property, and the social theory of
spontaneous order. Many more specific ideas flow from these
fundamentals: individual freedom, limited and representative
government, free markets. It had taken a long time to define
them; it was still necessary to fight for them.
\switchcolumn
随着斯密对自发秩序原理的系统阐述,自由主义的基本原
理至此已经基本完善。我们也许可以把这些基本原理概括为更
高的法或者自然法的理念,个人的自由和财产的自然权利,以
及自发秩序的社会理论。很多更专门的理论都是由这些基本理
论阐发而来:个人自由、有限政府、代议制政府和自由市场。
界定这些原理花了很长的时间,而现在仍然有必要为了它们而
战斗。

\switchcolumn*[\section{Making a Liberal World\\创造一个自由世界}]
Like the English Revolution, the period leading up to the American Revolution was one of great ideological debate. Even more
than the seventeenth-century English world, eighteenth-century America was dominated by liberal ideas. Indeed, we might say that there were virtually no nonliberal ideas circulating in
America; there were only conservative liberals, who urged that
Americans continue to peacefully petition for their rights as
Englishmen, and radical liberals, who eventually rejected even a
constitutional monarchy and called for independence. The most
galvanizing of the radical liberals was Thomas Paine. Paine was
what we might call an outside agitator, a traveling missionary of
liberty. Born in England, he went to America to help make a
revolution, and when his task was done, he crossed the Atlantic
again to help the French with their revolution.
\switchcolumn
像英国革命一样,美国革命的那个时代也充斥着意识形态
的争论。18世纪的美国,自由主义思想占据了思想界的统治地位,甚至比17世纪的英国更甚。事实上,我们可以说当时
在美国根本就没有流行过非自由主义的理念;实际上只有保守
自由派和激进自由派两种,前者主张美国人继续和平地争取和
英国人同等的权利,而激进自由派彻底拒绝哪怕是立宪的君主
制,而是要求独立。激进自由派当中最活跃的是托马斯$\cdot$潘
恩。我们也许应当称潘恩为外来的鼓吹者,一个在路上的自由
使者。他出生在英国,到美国是为了帮助革命,当他的任务结
束之后,他又穿过大西洋去帮助法国革命去了。
\switchcolumn*[\subsection{Society versus Government\\社会V.S.政府}]
Paine's great contribution to the revolutionary cause was his
pamphlet \textit{Common Sense}, which is said to have sold some
100,000 copies within a few months, in a country of three million people. Everyone read it; those who could not read heard it read in taverns and participated in debating its ideas. \textit{Common Sense} was not just a call for independence. It offered a radically libertarian theory to justify natural rights and independence.
Paine began by making a distinction between society and government: ``Society is produced by our wants, and government by our wickedness$\ldots$ Society in every state is a blessing, but
government even in its best state is but a necessary evil; in its
worst state an intolerable one.'' He went on to denounce the
origins of monarchy: ``Could we take off the dark covering of
antiquity $\ldots$ we should find the first [king] nothing better than the principal ruffian of some restless gang, whose savage
manners or pre-eminence in subtlety obtained him the title of
chief among plunderers.''
\switchcolumn
潘恩对革命的最大贡献是他的小册子《常识》。据说,这
本小册子在几个月当中卖出了大约10万本,而这个成绩是在
一个300万人口的国家当中取得的。当时几乎每个人都读过这
本书;那些没有阅读能力的人则在小酒馆里听别人读,并且加
入对这本书的理念的争论当中。《常识》并不仅仅是一本要求
独立的宣言。它提出了一种为自然权利和独立进行辩护的激进
的古典自由主义理论。潘恩在书的开头就分辨了社会和政府的
区别:“社会是一种上帝的赐福,而政府即便是最好的政府也
是一种必要的恶;如果是最坏的政府,则是一种无法容忍的状
态。”他继续对君主制的起源进行批判:“当我们揭开古代的
黑暗面纱……我们就会发现第一任国王并不比那些流寇当中最
罪大恶极的匪徒更好,而这些匪徒往往凭借残暴的性格和比别
人更加狡诈阴险而在众多恶棍当中获得头领的地位。”
\switchcolumn*
In \textit{Common Sense} and in his later writings, Paine developed the
idea that civil society exists prior to government and that people can peacefully interact to create spontaneous order. His belief in spontaneous order was strengthened when he saw society
continue to function after the colonial governments were
kicked out of American cities and colonies. In his writings he
neatly fused the normative theory of individual rights with the
positive analysis of spontaneous order.
\switchcolumn
在 《常识》 和他后来的著作中,潘恩阐述了市民社会的
存在先于政府、人们能够通过和平的相互关系而创造自发秩序
的理论。当看到殖民政府被赶出美国城市和殖民地之后社会仍
然继续运转的事实之后,他对自发秩序的信仰更加增强了。在他的作品当中,他将个人权利的标准理论和对自发秩序的积极
分析巧妙地融合在了一起。
\switchcolumn*
Neither \textit{Common Sense} nor \textit{The Wealth of Nations} was the only
milestone in the struggle for liberty in 1776. Neither may even
have been the most important event in that banner year. For in
1776 the American colonies issued their Declaration of Independence, probably the finest piece of libertarian writing in history. Thomas Jefferson's eloquent words proclaimed to all the
world the liberal vision:
\switchcolumn
无 论 是 《常识》还 是 《国富论》,在 1776年争取自由的
战争当中,都不是唯一的里程碑。这两者都不是那个杰出年份
的最重大事件。1776年 ,北美殖民地发表了《独立宣言》,这
本宣言也许是历史上最好的古典自由主义文献。托 马 斯 $\cdot$杰斐
逊以雄辩的语言向全世界发布了自由的宣言:
\switchcolumn*
\begin{quote}
We hold these truths to be self-evident, that all men are created
equal, that they are endowed by their Creator with certain unalienable Rights, that among these are life, liberty, and the pursuit of happiness. That to secure these rights, governments are
instituted among men, deriving their just powers from the consent of the governed. That whenever any form of government becomes destructive of these ends, it is the right of the people to
alter or abolish it.
\end{quote}
\switchcolumn
\begin{quote}
我们认为下面这些真理不证自明,即人人生而平等,
造物主赋予他们若干不可剥夺的权利,包括生命、 自由以
及追求幸福的权利。为了保障这些权利,人类才在他们之
间建立政府,而政府之正当权力,是经被治理者的同意而
产生的。当任何形式的政府对这些目标具破坏作用时,人
民便有权力改变或废除它。
\end{quote}
\switchcolumn*
The influence of the Levellers and John Locke is obvious. Jefferson succinctly made three points: that people have natural
rights; that the purpose of government is to protect those
rights; and that if government exceeds its proper purpose, people have the right ``to alter or abolish it.'' For his eloquence in
stating the liberal case, and for his lifelong role in the liberal
revolution that changed the world, the columnist George F.
Will named Jefferson ``the man of the millennium.'' Far be it
from me to argue with that choice. But it should be noted that
in writing the Declaration of Independence, Jefferson did not
break much new ground. John Adams, perhaps resentful of the
attention Jefferson got, said years later that ``there is not an idea in the Declaration but what had been hackneyed in Congress for two years before.'' Jefferson himself said that while he ``turned to neither book nor pamphlet in writing it,'' his goal
was ``not to find out new principles, or new arguments,'' but
merely to produce ``an expression of the American mind.'' The
ideas in the Declaration were, he said, the ``sentiments of the
day, whether expressed in conversation, in letters, printed essays, or the elementary books of public right.'' The triumph of liberal ideas in the United States was overwhelming.
\switchcolumn
从中我们看到了平权派和洛克的明显影响。杰斐逊简洁地
提出了三个观点:人们拥有自然权利;政府的目的是保护这些
权利;如果政府超出了其适当的目标,人 们 有 权 “改变或者
废除它”。由于其对自由主义实践的雄辩阐述,由于其终其一
生在改变世界的自由主义革命过程当中扮演的角色,专栏作家
乔 治 $\cdot$ 威 尔 (George F. Will)提名杰斐逊为“千年人物”。我
并不想对这个提名提出异议。但是必须指出的是,杰斐逊的
《独立宣言》并没有提出新的思想。约 翰 $\cdot$亚当斯也许对杰斐
逊受到的关注表示不满,几年之后他说: “在 《独立宣言》
中,除了两年前的国会当中就已提出的理念之外,并没有什么
新思想。” 而杰斐逊本人说,当时他并不是在写一本书或者小
册子,他 的 目 的 “不是发现新的原理和新的观点”,而只是对
“美国人的思想” 进行表述而已。他说: “这些思想早已在当时美国人的日常对话、通信、发表的文章以及关于公众权利的
人门读物当中表达了出来,《独立宣言》 只不过是对其进行表
述而已。” 自由主义理念在美国的胜利是空前的。

\switchcolumn*[\subsection{Limiting Government\\有限政府}]
After their military victory, the independent Americans set
about putting into practice the ideas that English liberals had
been developing throughout the eighteenth century. The distinguished Harvard University historian Bernard Bailyn writes in his 1973 essay ``The Central Themes of the American Revolution'' that
\switchcolumn
在取得军事胜利之后,独立的美国人开始将英国在整个
18世纪发展起来的自由主义理念付诸实践。哈佛大学著名历
史学家伯纳德$\cdot$ 贝林(Bernard  Bailyn) 1973年 在 “美国革命
的中心理论” 一文中说:
\switchcolumn*
\begin{quote}
the major themes of eighteenth-century radical libertarianism were brought to realization here. The first is the belief that
power is evil, a necessity perhaps but an evil necessity; that it is
infinitely corrupting; and that it must be controlled, limited, restricted in every way compatible with a minimum of civil order.
Written constitutions; the separation of powers; bills of rights;
limitations on executives, on legislatures, and courts; restrictions
on the right to coerce and wage war---all express the profound
distrust of power that lies at the ideological heart of the American Revolution and that has remained with us as a permanent legacy ever after.
\end{quote}
\switchcolumn
\begin{quote}
18世纪激进的古典自由主义的重要理念都获得了关
注。第一个理念是认为权力就是罪恶。权力也许是必要的
恶,但是必要的恶必然也是恶。权力毫无例外地会腐败;
因此必须受到各种形式的控制、限制和制约,确保其和最
低程度的文明秩序不矛盾。成文宪法、权力的分立;权利
法案;对行政、立法和司法权力的限制;对实施强制和发
动战争的权力的限制 --- 这些都显示了美国革命的意识形
态内核当中对权力的极度不信任,而这种不信任从那以后
直到今天已经成为我们的永恒遗产。
\end{quote}
\switchcolumn*
The Constitution of the United States built on the ideas of
the Declaration to establish a government suitable for free people. It was based on the principle that individuals have natural
rights that precede the establishment of government and that
all the power a government has is delegated to it by individuals
for the protection of their rights. Based on that understanding,
the Framers did not set up a monarchy, nor did they create an
unlimited democracy, a government of plenary powers limited
only by popular vote. Instead, they carefully enumerated (in
Article I, Section 8) the powers that the federal government
would have. The Constitution, whose greatest theorist and architect was Jefferson's friend and neighbor James Madison, was truly revolutionary in its establishment of a government of \textit{delegated, enumerated}, and thus \textit{limited powers}.
\switchcolumn
美国宪法是依照《独立宣言》 中建立一个适合自由人民
的政府的理念而写成的。它所遵循的原则是个人拥有的自然权
利先于政府的建立,政府的一切权力均来源于个人为了保护自
己的权利而授予政府的权力。因此,建国者们既没有建立一个
君主制政体,也没有建立一个无限民主制政体,即政府拥有无
限的权力,其权力仅仅受到公民投票的限制。相反,他们审慎
地列举了联知政府应当拥有的权力(第 1 条,第 8 款)。美国宪法最革命性的贡献是规定了\textbf{政府权力的有限性},也就是说,政府权力必须是\textbf{经过授权的},权力的内容是\textbf{有限列举的}。而这部宪法的最伟大的理论家和设计师是杰斐逊的朋友和邻居詹姆斯$\cdot$麦迪逊。
\switchcolumn*
When a Bill of Rights was first proposed, many of the
Framers responded that one was not needed because the enumerated powers were so limited that government would be unable to infringe on individual rights. Eventually, it was decided
to add a Bill of Rights, in Madison's words, ``for greater caution.'' After enumerating specific rights in the first eight
amendments, the first Congress added two more that summarize the whole structure of the federal government as it was created: The Ninth Amendment provides that ``the enumeration
in the Constitution of certain rights shall not be construed to
deny or disparage others retained by the people.'' The Tenth
Amendment says, ``The powers not delegated to the United
States by the Constitution, nor prohibited by it to the States,
are reserved to the States respectively or to the people.'' Again,
the fundamental tenets of liberalism: People have rights before
they create government, and they retain all the rights they
haven't expressly delegated to government; and the national
government has no powers not specifically granted in the Constitution.
\switchcolumn
然而,当 《权利法案》第一次提出的时候,许多建国者
的反应是,这是没有必要的,因为有限列举的权力对政府的限
制太大了,以至于政府将无法干预个人的权利。最终,他们决
定在宪法中加入《权利法案》,用麦迪逊的话来说就是“为慎
重起见”。在前八个修正案中列举了特定的权利之后,第一届
国会决定再加人两条权利,这两条权利总结了联邦政府成立时
的整体结构:第九修正案规定“本宪法对某些权利的列举不
得被解释为否定或轻视由人民保有的其他权利。”第十修正案
则 说 “宪法未授予合众国,也未禁止各州行使的权力,分别
由各州或由人民保留。” $\cdot$于是,自由主义的基本原则又一次得
到了重申:人民拥有权利先于其创造政府,他们自然保有那些
没有明确声明授予政府的权利;而全国性政府只能拥有宪法当
中明确授予的权力。
\switchcolumn*
In both the United States and Europe, the century after the
American Revolution was marked by the spread of liberalism.
Written constitutions and bills of rights protected liberty and
guaranteed the rule of law. Guilds and monopolies were largely
eliminated, with all trades thrown open to competition on the
basis of merit. Freedom of the press and of religion was greatly
expanded, property rights were made more secure, international trade was freed.
\switchcolumn
无论是在美国还是欧洲,美国革命之后的一个世纪都是以
自由主义的扩张而名垂史册的。成文宪法和权利法案保护了自
由,捍卫了法治。行会和垄断被消除,贸易全面开放,商业竞
争建立在商品的价值之上。出版自由和宗教自由大大扩展,产
权得到了更大的保护,国际贸易更加自由。

\switchcolumn*[\subsection{Civil Rights\\公民权利}]
Individualism, natural rights, and free markets led logically to
agitation for the extension of civil and political rights to those
who had been excluded from liberty, as they were from
power---notably slaves, serfs, and women. The world's first antislavery society was founded in Philadelphia in 1775, and slavery and serfdom were abolished throughout the Western world within a century. During the debate in the British Parliament
over the idea of compensating slaveholders for the loss of their
``property,'' the libertarian Benjamin Pearson replied that he
had ``thought it was the slaves who should have been compensated.'' Tom Paine's Pennsylvania Journal published a stirring
early defense of women's rights in 1775. Mary Wollstonecraft, a
friend of Paine and other liberals, published \textit{A Vindication of the
Rights of Woman} in England in 1792. The first feminist convention in the United States took place in 1848, as women began
to demand the natural rights that white men had claimed in
1776 and that were being demanded for black men. In the
phrase of the English historian Henry Sumner Maine, the world
was moving from a society of status to a society of contract.
\switchcolumn
个人主义、自然权利以及自由市场发展的逻辑结果,必然
是公民权利乃至政治权利扩展到那些以前被排斥在自由和权力
之外的阶层,特别是奴隶、农奴和女人。世界上第一个反蓄奴
地 区 1775年在费城出现,随后,在一个世纪的时间里,奴隶制和农奴制在整个西方世界被废除。当时,英国国会激烈讨论
如何赔偿奴隶主的“财产” 损失。古典自由主义者本杰明$\cdot$
皮尔逊 (Benjamin  Pearson) 回答道:“我以为应该是对奴隶的损失进行赔偿才对吧。” 汤 姆 $\cdot$ 潘 恩 所 在 的 《宾夕法尼亚杂志》在 1775年发表了一份对妇女权利进行辩护的早期文章。
潘恩的朋友玛丽$\cdot$ 沃尔斯通克莱夫特(Mary  Wollstonecraft)
和其他一些自由主义者也于1792年在英国出版了《为妇女权利辩护》一书。而第一次女权运动大会也于1848年在美国召
开,妇女幵始对那些白种男人在1776年所宣称的自然权利提出要求。同时黑人也开始主张自己的权利。在英国历史学家亨
利 $\cdot$ 梅因\footnote{亨利$\cdot$梅因(Henry Maine, 1822$\sim$1888),英国著名法律史学家,著有《古代法》 《国际法》,被认为是英国历史法学派的奠基人和主要代表人物。}的时代,这个世界正在从一个等级制的时代过渡到
契约的时代。  
\switchcolumn*
Liberals also took on the ever-present specter of war. In England, Richard Cobden and John Bright tirelessly argued that
free trade would bind people of different nations together
peacefully, reducing the likelihood of war. The new limits on
governments, and greater public skepticism toward rulers,
made it more difficult for political leaders to meddle abroad and
to go to war. After the turmoil of the French Revolution and the
final defeat of Napoleon in 1815, and with the exception of the
Crimean War and the wars of national unification, most of the
people of Europe enjoyed a century of relative peace and
progress.
\switchcolumn
自由主义者也对当时仍然存在的战争的幽灵进行了讨论。
在英国,理查德$\cdot$ 科布登\footnote{理查德$\cdot$科布登(Richard Cobden,1804$\sim$1865),19世纪英国历史上的重要人物,领导了反谷物法运动,促成了谷物法的废除。谷物法的废除结束了重商主义的经济政策,从此英国建立了自由贸易制度,开启了自由贸易的时代。}和 约 翰 $\cdot$ 布赖特\footnote{约翰$\cdot$布赖特(John Bright, 1811$\sim$ 1889), 19世纪英国著名政治家、演说家,反谷物法运动的另一位重要领导人。}对这个问题不知疲倦地进行了辩护。他们认为:自由贸易能够将不同国家的人们和平地联系在一起,降低发生战争的可能性。对政府的新限制以及公众对统治者怀疑的加深,使得政治领导人更难以插手海外和发动战争。经过法国大革命的混乱以及拿破仑在1815年的最终失败之后,除了克:里米亚战争和一些国家的统一战争之外,绝大多数欧洲人享受了长达一个世纪之久的相对和平和进步,尽管其间发生过克里木战争以及国家统一战争之类的局部战争。

\switchcolumn*[\subsection{The Results of Liberalism\\自由主义的结果}]
This liberation of human creativity created astounding scientific and material progress. The Nation magazine, which was
then a truly liberal journal, looking back in 1900, wrote, ``Freed
from the vexatious meddling of governments, men devoted
themselves to their natural task, the bettering of their condition, with the wonderful results which surround us.'' The technological advances of the liberal nineteenth century are
innumerable: the steam engine, the railroad, the telegraph, the
telephone, electricity, the internal combustion engine. Thanks
to the accumulation of capital and ``the miracle of compound
interest,'' in Europe and America the great masses of people
began to be liberated from the backbreaking toil that had been
the natural condition of mankind since time immemorial. Infant mortality fell and life expectancy began to rise to unprecedented levels. A person looking back from 1800 would see a
world that for most people had changed little in thousands of
years; by 1900, the world was unrecognizable.
\switchcolumn
人类创造力的解放创造出了惊人的科学进步和物质的空前
繁荣。一本真正的自由主义杂志《国家》(\textit{Nation}) 在回顾
1900年的时候写道:“随着人们从令人困扰的政府的干预中解
放出来,他们开始全身心地投入到自然事物当中,改善他们的
生活条件,由此产生出来的大量神奇成果至今仍然存在于我们
的周围。” 自由的19世纪所创造出来的技术进步数不胜数:蒸
汽机、铁路、电报、电话、电、内燃机等。得益于资本的积累
和 “复利的奇迹” ,在欧洲和美国,数量庞大的人们开始从太
古以来就一直在进行的、依赖于人本身身体自然条件的辛勤劳
作中解放出来。婴儿死亡率迅速下降,人类的预期寿命也上升
到了空前的水平。如果 从1800年回头看以前的历史,会发现
大多数人的生活在几千年里几乎没有发生变化;而到了 1900
年,这个世界已经变得面目全非了。
\switchcolumn*
Liberal thought continued to develop throughout the nineteenth century. Jeremy Bentham propounded the theory of
utilitarianism, the idea that government should promote ``the
greatest happiness for the greatest number.'' Although his
philosophical premises were different from those of natural
rights, he came to most of the same conclusions about limited
government and free markets. Alexis de Tocqueville came to
America to see how a free society worked and published his brilliant observations as \textit{Democracy in America} between 1834 and
1840. John Stuart Mill published \textit{On Liberty}, a powerful case for
individual freedom, in 1859. In 1851 Herbert Spencer, a towering scholar whose work is unjustly neglected and often misrepresented today, published Social Statics, in which he set forth his
``law of equal freedom,'' an early and explicit statement of the
modern libertarian credo. Spencer's principle was ``that every
man may claim the fullest liberty to exercise his faculties compatible with the possession of like liberty by every other man.''
Spencer pointed out that ``the law of equal freedom manifestly
applies to the whole race---female as well as male.'' He also extended the classical liberal critique of war to distinguish between two kinds of societies: industrial society, where people
produce and trade peacefully and in voluntary association, and
militant society, in which war prevails and the government controls the lives of its subjects as means to its own ends.
\switchcolumn
自由主义思想在19世纪仍然在继续发展。边沁提出了功
利主义的理论,主 张 政 府 应 当 促 进 “最大多数人的最大幸
福”。尽管理论前提不同于其他自由主义理论的自然权利论,
但他得出的结论是几乎一样的:有限政府和自由市场。1834 ~
1840年 ,托克维尔到美国考察自由社会的运转机制,并且出
版了他的伟大著作《论美国的民主》。1859年,密尔出版了
《论自由》,为个人自由进行了强有力的辩护。1851年,斯宾
塞出版了《社会静力学》。在这本书中,他提出了 “同等自由
的法则” 的概念,这是对现代的古典自由主义理念的早期表
述。斯宾塞是一位里程碑式的学者,但他的工作长期以来被不公正地忽视,今天还在被误读。斯宾塞提出的原则是:“每个
人都可以主张最大限度自由,施展其最大的能力,但必须尊重
别人也拥有的同等的自由。” 斯宾塞指出: “同等自由的法则
毫无疑问适用于所有人 --- 男人和女人。”他还扩展了古典自
由主义对战争的批判,他分辨出两种社会:工业社会,人们通
过自发联合和平地生产和贸易;军事社会,在军事社会里战争
压倒一切,政府控制人民的生命,把人民当作附属品和实现其
目标的手段。
\switchcolumn*
In its golden age, Germany produced great writers such as
Goethe and Schiller, who were liberals, and it contributed to
liberal philosophy in the ideas of philosophers and scholars such
as Immanuel Kant and Wilhelm von Humboldt. Kant emphasized individual autonomy and attempted to ground individual
rights and liberties in the requirements of reason itself. He
called for a ``legal constitution which guarantees everyone his
freedom within the law, so that each remains free to seek his
happiness in whatever way he thinks best, so long as he does
not violate the lawful freedom and rights of his fellow subjects.''
Humboldt's classic work \textit{The Sphere and Duties of Government}, which heavily influenced Mill's \textit{On Liberty}, argued that the full nourishing of the individual requires not only freedom but ``a
manifoldness of situations,'' by which he meant that people
should have available to them a wide variety of circumstances
and living arrangements---the modern term might be ``alternative lifestyles''---which they can continually test and choose. 
\switchcolumn
德国的黄金时代产生了许多伟大的作家,歌德和席勒都是自由主义者。这个时期的哲学家和学者 --- 如康德和洪堡,对
自由主义理论的发展做出了重大贡献。康德强调个人自治,试图通过逻辑论证来建立个人权利和自由的基础。他主张建立
“合法的宪法,在法律范围之内确保每个人的自由,让每个人都能以他自己认为最好的方式来追求幸福,同时,他不能侵犯
其他人的合法自由和权利。” 洪堡在经典著作 《论国家的作用》(\textit{The Sphere and Duties of Government}) --- 这本书对密尔的 《论自由》产生了深刻的影响 --- 中认为,个人的充分发展不仅仅需要自由,而且需要一个“多样化的环境”,意思
是人们应当拥有多种多样的环境和生活方式,以供不断的试错和选择 --- 用现代的话来说就是“多元化的生活方式”。
\switchcolumn*
In France, Benjamin Constant was the best-known liberal on the
Continent in the early part of the century. ``He loved liberty as
other men love power,'' a contemporary said. Like Humboldt,
he saw liberty as a system in which people could best discover
and develop their own personalities and interests. In an important essay, he contrasted the meaning of liberty in the ancient
republics---equal participation in public life---with modern liberty---the individual freedoms to speak, write, own property,
trade, and pursue one's private interests. An associate of Constant was Madame de Stael, a novelist, perhaps best known for
the saying, ``Liberty is old; it is despotism that is new,'' referring
to the attempt of the royal absolutists to take away the hard-won chartered liberties of the Middle Ages.
\switchcolumn
在法国,贡斯当是19世纪前期欧洲大陆最著名的自由主
义者,他的同代人这样描述他:“他热爱自由就像其他人热爱
权力一样。” 和洪堡一样,贡斯当把自由看作一个系统,在这
个系统当中人们能够最大限度地发现和发展自己的个性,追求
个人的利益。在一篇重要的文章中,他比较了两种不同的自
由:在古代的共和主义者当中,自由的意思是平等地参与公共
生活;现代的自由则是指个人自由,包括言论的自由、写作的自由、拥有财产的自由、交易的自由,以及追求个人利益的自
由。小说家斯塔尔夫人\footnote{塔尔夫人(Madame de Stafel, 1766$\sim$ 1817),法国著名小说家。主要作品有文学批评作品《论文学》《论德国》,小说 《黛尔芬》《柯丽娜》。}是贡斯当的同路人。她说过的最著名的一句话是: “自由是古老的传统,而专制主义才是新的事物。”她指的是当时王权专制主义者试图把中世纪人们艰苦获得的特许自由(chartered liberties) 夺走。
\switchcolumn*
Another French liberal, Frederic Bastiat, served in Parliament as an avid free-trader and wrote a multitude of witty and
hard-hitting essays attacking the state and all its actions. His
last essay, ``What Is Seen and What Is Not Seen,'' offered the
important insight that whatever a government does---build a
bridge, subsidize the arts, pay out pensions---has simple and
obvious effects. Money is circulated, jobs are created, and people think that the government has generated economic growth.
The task of the economist is to see what is not so easily seen---
the houses not built, the clothes not bought, the jobs not created---because money was taxed away from those who would
have spent it on their own behalf. In ``The Law,'' he attacked the
concept of ``legal plunder,'' by which people use government to
appropriate what others have produced. And in ``The Petition of
the Candlemakers against the Competition of the Sun,'' he
mocked French industrialists who wanted to be protected from
competition by pretending to speak on behalf of the candle-makers who wanted Parliament to block out the sun, which was
causing people not to need candles during daytime---an early
refutation of ``antidumping'' laws.
\switchcolumn
另一个法国自由主义者弗里德里克$\cdot$ 巴斯夏\footnote{弗雷德里克$\cdot$巴斯夏(Frederic Bastiat, 1801$\sim$1850), 19世纪法国古典由主义理论家、政治经济学家、立法议会的议员。代表作有《财产、法律与政府》。巴斯夏不属于法国唯理主义的思想传统,在思想气质上更接近于英国人,尤其接近当时鼓吹自由贸易的曼彻斯特学派。他主张自由贸易,反对政府工程和政府垄断。巴斯夏对主张破窗理论的经济学家进行了严厉批判。},国会议员,活跃的自由贸易鼓吹者,写了大量机智幽默地尖锐批评国家及其行为的文章。他在最后一篇文章《看得见的和看不见的》 中提出了一个洞见:无论政府做什么,无论是建一座桥、支持艺术还是支付补偿,都会产生直接和明显的后果:货币在流通,工作机会被创造了出来,人们认为政府引发了经济增长。而经济学家的任务是帮助人们看见那些不那么容易看见的事实 --- 房子并没有建起来,衣服并没有卖出去,工作机会并没有被创造出来,因为钱是被政府从本来会自己来花这笔钱的
那些人手里以税收的形式拿走了。他还对政府通过法律进行“合法的抢劫” 进行了批评,人们通过政府进行的“合法的抢
劫” 占有了其他人生产的东西。在他的著名寓言《蜡烛制造商抗议太阳竞争的陈情书》 当中,他以一个蜡烛制造商的口气向国会提出请求,要求将太阳遮起来,因为有了太阳,人们在白天就不需要蜡烛了。巴斯夏通过这个故事挖苦了法国工业家们企图通过政府保护来逃避竞争。这也许是对“反倾销法”的最早驳斥。
\switchcolumn*
In the United States, the abolitionist movement was naturally led by libertarians. Leading abolitionists called slavery
``man stealing,'' in that it sought to deny self-ownership and
steal a man's very self. Their arguments paralleled those of the
Levellers and John Locke. William Lloyd Garrison wrote that
his goal was not just the abolition of slavery but ``the emancipation of our whole race from the dominion of man, from the
thraldom of self, from the government of brute force.'' Another
abolitionist, Lysander Spooner, proceeded from the natural-rights argument against slavery to the conclusion that no one
could be held to have given up any of his natural rights under
any contract, including the Constitution, that he had not personally signed. Frederick Douglass likewise made his arguments for abolition in the terms of classical liberalism:
self-ownership and natural rights.
\switchcolumn
在美国,废奴运动很自然由古典自由主义者所领导。废奴
运动的领导者们把奴隶制度称为“抢人”,也就是说它通过否
定人们的自我拥有权,把人本身抢走了。他们的观点与平权派
和洛克的观点是一致的。加里森 (Wiliam  Lloyd Garrison)写
道 ,他的目标不仅仅是废除奴隶制度,而 且 是 要 “把我们的
整个种族从男人的统治,从人的奴役,从政府的暴力强制下解
放出来。”另一个废奴主义者斯普纳(Lysangder Spooner)用
自然权利反对奴隶制度,认为人不能通过任何他自己没有签署
的契约包括宪法放弃他的自然权利。道 格 拉 斯 (Frederick
Douglass)则用传统的自由主义理念来为废奴运动进行辩护:
自我拥有权和自然权利。

\switchcolumn*[\section{The Decline of Liberalism\\自由主义的衰落}]
Toward the end of the nineteenth century, classical liberalism
began to give way to new forms of collectivism and state power.
If liberalism had been so successful---liberating the great mass
of humanity from the crushing burden of statism and unleashing an unprecedented improvement in living standards---what
happened? That question has vexed liberals and libertarians
throughout the twentieth century.
\switchcolumn
到 19世纪末期,传统的自由主义开始让位于各种新形式
的集体主义和国家权力。如果说自由主义曾经如此成功 --- 将
无数人从国家主义令人窒息的负担中解放出来,让人类的生活
水平得到前所未有的提高 --- 那么是什么使得自由主义渐趋衰
落呢?这个问题困扰了整个20世纪的自由主义者和古典自由
主义者。
\switchcolumn*
One problem was that the liberals got lazy; they forgot Jefferson's admonition that ``eternal vigilance is the price of liberty'' and figured that the obvious social harmony and
abundance brought about by liberalism would mean that no
one would want to revive the old order. Some liberal intellectuals gave the impression that liberalism was a closed system,
with no more interesting work to be done. Socialism, especially
the Marxist variety, came along, with a whole new theory to develop, and attracted younger intellectuals.
\switchcolumn
一个原因是自由主义者变得懒惰了;他们忘记了杰斐逊的
忠告: “永远不放松警惕是自由必须付出的代价。”他们想当
然地以为,由于自由主义带来了显而易见的社会和谐和财富,
就不会有人想回到旧秩序之下。一些自由主义知识分子给人的
印象是:自由主义是一个已终结的系统,在这个系统下已经没有什么事情好做了。而社会主义,特别是马克思主义出现了,
有着全新的理论可以发展。它们吸引了年轻的知识分子。
\switchcolumn*
It may also be that people forgot how hard it had been to create a society of abundance. Americans and Britons born in the
latter part of the nineteenth century entered a world of rapidly
improving wealth, technology, and living standards; it was not
so obvious to them that the world had not always been like
that. And even those who understood that the world was different may have assumed that the age-old problem of poverty had
been solved. It was no longer important to maintain the social
institutions that had solved it.
\switchcolumn
另一个原因也许是人们已经忘记了创造一个丰裕的社会是
多么艰难。生 于 19世纪后期的美国人和英国人生活在一个财
富迅速增长、技术飞速进步、生活水平大幅提高的时代。对他
们来说,这个世界并非历来如此这一事实并不是显而易见的。
甚至那些理解这个时代和旧时代的不同的人,也可能会认为古
老的贫困问题早已经被解决。既然问题已经解决,那么维持这
种社会制度就不是那么重要了。
\switchcolumn*
A related problem was the separation of the issue of production from that of distribution. In a world of abundance, people
began to take production for granted and discuss ``the problem
of distribution.'' The great philosopher Friedrich Hayek once
told me in an interview,
\switchcolumn
还有一个原因就是生产和分配的分离。在一个丰裕的社
会 ,人们会认为生产出丰富的产品是理所当然的,不用考虑,
而开始讨论“分配问题”。伟大的思想家哈耶克曾经在一次访
谈中对我说:
\switchcolumn*
\begin{quote}
I am personally convinced that the reason which led the intellectuals, particularly of the English-speaking world, to socialism
was a man who is regarded as a great hero of classical liberalism,
John Stuart Mill. In his famous textbook \textit{Principles of Political Economy}, which came out in 1848 and for some decades was a
widely read text on the subject, he makes the following statement as he passes from the theory of production to the theory of distribution: ``The things once there, mankind, individually or
collectively, can do with them as they like.'' Now, if that were
true I would admit that it is a clear moral obligation to see that it
is justly distributed. But it isn't true, because if we did do with
that product whatever we pleased, people would never produce
those things again.
\end{quote}
\switchcolumn
\begin{quote}
我个人确信,将知识分子尤其是英语世界的知识分子引向社会主义的原因在于一个人,一个被认为是古典自由主义的伟大英雄的人,他就是约翰斯图亚特$\cdot$ 密尔。他著名的作品《政治经济学原理》 1848年出版,在很长时间内这本书被广泛阅读。在这本书中,当从生产理论转向分配理论的时候,密尔是这样论述的:“东西一旦已经生产出来,人们,无论是个人还是集体都可以按照他们的意愿来处理它。” 而我说,如果(东西已经生产出来)是事实的话,我承认公平分配显然是符合道德的。但这不是事实,因为如果我们按照我们的意愿来处理那些东西,那么人们就再也不会生产那些产品了。
\end{quote}
\switchcolumn*
Besides, for the first time in history people began to question the tolerability of poverty. Before the Industrial Revolution, everyone was poor; there was no problem to study. Only
when most people became rich---by the standards of history---did people begin to wonder why some were still poor.
Thus Charles Dickens bemoaned the already waning practice
of child labor that kept alive many children who in earlier eras
would have died, as most children had from time immemorial;
and Karl Marx offered a vision of a world of perfect freedom
and plenty. Meanwhile, the success of science and business
gave rise to the notion that engineers and corporate executives
could design and run a whole society as well as a large corporation.
\switchcolumn
另外,人们有史以来第一次开始质疑忍耐贫困的必要性。
工业革命之前,所有人都很穷;因此没有研究这个问题的必要。只有当大多数人(以历史的标准来看)变得富裕了之后,
人们才会疑惑为什么有人仍然很穷。于是狄更斯对使用童工进
行谴责,而他却不提正是使用童工才帮助了很多儿童生存下
来 ,这些儿童从前只会早早夭折,就像自古以来一直在发生的
那样,并且使用童工在当时已经是呈减少的趋势。而马克思则
提供了一个完全自由和物质丰富的理想社会的前景。与此同
时,科学的进步和商业的成功使得一些理论认为:工程师和公
司管理人员能够像管理一家大公司那样设计整个社会,并且让
它良好运转。
\switchcolumn*
Bentham and Mill's utilitarian emphasis on ``the greatest good for the greatest number'' caused some scholars to begin
questioning the need for limited government and protection of
individual rights. If the point of it all was to generate prosperity
and happiness, why take the roundabout way of protecting
rights? Why not just aim directly at economic growth and
widespread prosperity? Again, people forgot the concept of
spontaneous order, assumed away the problem of production,
and developed schemes to guide the economy in a politically
chosen direction.
\switchcolumn
边沁和密尔的功利主义不断强调的“最大多数人的最大
幸福”也使得一些学者开始质疑有限政府和对个人权利的保
护。如果我们的目标是繁荣和幸福,为什么我们还要绕个弯儿
间接地通过保护个人权利的方式来实现它呢?为什么不直接来
促进经济增长和扩大繁荣呢?于是,人们又一次忘记了自发秩
序的概念,他们拋开生产问题,而通过设计蓝图来引导经济朝
着政治上选定的方向前进。
\switchcolumn*
Of course, we must not neglect the age-old human desire
for power over others. Some forgot the roots of economic
progress, some mourned the disruption of family and community that freedom and affluence brought, and some genuinely
believed that Marxism could make everyone prosperous and
free without the necessity of work in dark satanic mills. But
many others used those ideas as a means to power. If the divine right of kings would no longer persuade people to hand
over their liberty and property, then the power seekers would
use nationalism, or egalitarianism, or racial prejudice, or class
warfare, or the vague promise that the state would alleviate
whatever ailed you.
\switchcolumn
当然,我们不应该忽略人们那古老的希望凌驾于他人之上
的权力欲望。有人忘记了经济发展的根源,有人抱怨自由和富
裕带来的家庭和社会的分裂,也有的人真诚地相信马克思主义
能够让人们不需要在黑暗丑陋的煤矿中工作就会都得到自由和
幸福。但是也有很多人把这些理想作为获取权力的手段。如果
国王的神圣权利再也不能说服人民让渡自由和财产,那么追求
权力的人就会用民族主义、平均主义、种族歧视或者阶级斗争
来达到目的,或者笼统地许诺国家会减轻困扰你的那些烦恼,
解决你面临的问题。
\switchcolumn*
By the turn of the century the remaining liberals despaired of
the future. The \textit{Nation} editorialized that ``material comfort has
blinded the eyes of the present generation to the cause which
made it possible'' and worried that ``before [statism] is again repudiated there must be international struggles on a terrific
scale.'' Herbert Spencer published \textit{The Coming Slavery} and
mourned at his death in 1903 that the world was returning to
war and barbarism.
\switchcolumn
在那个世纪之交,仅剩的自由主义者对未来充满了悲观。
《国家》杂志发表社论说: “物质上的舒适蒙上了这一代人的双眼,让他们看不到带来物质繁荣的原因。” 忧虑“在国家主义再次被推翻之前,必然会发生规模惊人的国家间的战争”。斯宾塞出版了《行将到来的奴隶制》 (The Coming Slavery),
在 1903年临死之前,他哀叹这个世界正在回到战争和野蛮的
时代。
\switchcolumn*
Indeed, as the liberals had feared, the century of European
peace that began in 1815 came crashing down in 1914, with
the First World War. The replacement of liberalism by statism
and nationalism was in large part to blame, and the war itself
may have delivered the death blow to liberalism. In the United
States and Europe, governments enlarged their scope and
power in response to the war. Exorbitant taxation, conscription,
censorship, nationalization, and central planning---not to mention the 10 million deaths at Flanders fields and Verdun and elsewhere---signaled that the era of liberalism, which had so recently supplanted the old order, was now itself supplanted by the era of the megastate.
\switchcolumn
事实正如自由主义者所担忧的那样。欧洲开端于1815年、
持续了长达一个世纪的和平终于随着1914年第一次世界大战
的爆发而终结。这在很大程度上应该归因于自由主义被国家主
义和民族主义所替代,而战争则给了自由主义致命的一击。在
美国和欧洲,为了应对战争的威胁,政府扩大了权力的领域。
过度征税、征兵、审查、国有化和中央计划,还 有 超 过 1000
万人在佛兰德尔和凡尔登以及其他战场战死。标志着曾经代替
旧秩序的自由主义时代已经被大国家时代所代替。

\switchcolumn*[\section{The Rise of the Modern Libertarian Movement\\现代古典自由主义运动的崛起}]
Through the Progressive Era, World War I, the New Deal, and
World War II, there was tremendous enthusiasm for bigger
government among American intellectuals. Herbert Croly, the
first editor of the \textit{New Republic}, wrote in\textit{ The Promise of American Life} that that promise would be fulfilled ``not by ... economic
freedom, but by a certain measure of discipline; not by the
abundant satisfaction of individual desires, but by a large measure of individual subordination and self-denial.'' Even the
awful collectivism beginning to emerge in Europe was not repugnant to many ``progressive'' journalists and intellectuals in
America. Anne O'Hare McCormick reported in the \textit{New York
Times} in the first months of Franklin Roosevelt's New Deal,
\switchcolumn
经过进步主义时代、第一次世界大战、罗斯福新政和第二
次世界大战,美国知识分子中许多人对大政府充满热情。《新
共和》(\textit{New Republi}c)的第一任编辑克罗利(Herbert Croly)
在《美国生活的希望》中写道:希望将会得到满足,“不是通过经济自由……而是通过一定程度的纪律;不是通过无休止地
满足个人欲望,而是通过大童个人服从和自我否定”。很多“进步”的新闻记者和知识分子甚至对那些已经开始在欧洲登
场的可怕的集体主义也没有反感。在罗斯福新政的最初几个月,安妮 $\cdot$ 欧黑尔 $\cdot$ 麦克米克(Anne O’Hare McCormick) 在
《纽约时报》上撰文说:
\switchcolumn*
\begin{quotation}
The atmosphere (in Washington) strangely reminiscent of
Rome in the first weeks after the march of the Blackshirts, of
Moscow at the beginning of the Five-Year Plan$\ldots$


Something far more positive than acquiescence vests the President with the authority of a dictator. This authority is a free gift,
a sort of unanimous power of attorney$\ldots$ America today literally asks for orders$\ldots$ Not only does the present occupant of
the White House possess more authority than any of his predecessors, but he presides over a government that has more control
over more private activities than any other that has ever existed
in the United States$\ldots$ The Roosevelt administration envisages a federation of industry, labor and government after the
fashion of the corporative State as it exists in Italy.
\end{quotation}
\switchcolumn
\begin{quotation}
华盛顿的气氛让人很奇怪地联想起黑衫军大进军之后的罗马,想起第一个五年计划开始时的莫斯科……
	
最让人悲观的是总统被认可授予类似于独裁的权力。
这个授权是一个免费的礼物,一种没有异议的代表人民的
权力 ...... 美国今天确实在要求秩序 … 不仅仅是现在占据白宫的那个人(总统)拥有了他的任何一位前任都无法企及的权力,而且统辖了一个比美国历史上任何一个时期$\cdot$ 对个人行为的控制都要多得多的政府…… (罗斯福政府)正在实施工业、劳工和政府的联合,追随着正在流行的国家社团主义,而这种模式已经存在于意大利。
\end{quotation}
\switchcolumn*
Although a few liberals---notably the journalist H. L. Mencken---remained outspoken, there was indeed a general intellectual and popular acquiescence in the trend toward big government. The government's apparent success in ending the Great Depression and winning World War II gave impetus to the notion that government could solve all sorts of problems.Not until twenty-five years or so after the end of the war did
popular sentiment start to turn against the megastate.
\switchcolumn
尽管包括新闻记者门肯(H.L. Mencken) 在内的自由主
义者仍然在大声疾呼,但是知识分子和大众的潮流已经倾向于
大政府。由于政府结束大萧条的表面上的胜利以及第二次世界
大战的胜利,那些认为政府能够解决所有问题的观点占了上
风。从第二次世界大战结束一直到战后大约25年,公众的感
情一直倾向于大政府。

\switchcolumn*[\subsection{The Austrian Economists\\奥地利学派经济学家}]
Meanwhile, even in the darkest hour of libertarianism, great
thinkers continued to emerge and to refine liberal ideas. One of
the greatest was Ludwig von Mises, an Austrian economist who
fled the Nazis, first to Switzerland in 1934 and then to the
United States in 1940. Mises's devastating book \textit{Socialism}
showed that socialism could not possibly work because without
private property and a price system there is no way to determine what should be produced and how. His student Friedrich
Hayek related the influence that \textit{Socialism} had on some of the
most promising young intellectuals of the time:
\switchcolumn
甚至是在古典自由主义最黑暗的时刻,伟大的思想家们仍
然在吸收和完善自由主义的理念。其中最伟大的思想家之一就
是路德维希$\cdot$ 冯 $\cdot$ 米瑟斯,一位奥地利经济学家。1934年,
他逃离纳粹的统治,最开始到了瑞士,随后于1940年到了美
国。米瑟斯在其伟大著作《社会主义》(\textit{Socialim})中论证社
会主义不可能成功,因为如果没有私人产权和价格系统就没有
办法决定应当生产什么以及如何生产。他的学生哈耶克描述了
《社会主义》这本书在一些最前途无量的年轻知识分子当中产
生的影响:
\switchcolumn*
\begin{quote}
When \textit{Socialism} first appeared in 1922, its impact was profound.
It gradually but fundamentally altered the outlook of many of
the young idealists returning to their university studies after
World War I. I know, for I was one of them$\ldots$ Socialism
promised to fulfill our hopes for a more rational, more just world.
And then came this book. Our hopes were dashed.
\end{quote}
\switchcolumn
\begin{quote}
当 《社会主义》1922年刚刚出版的时候,它带来的
冲击是深远的。它逐渐地但是根本地改变了许多在第一次
世界大战后返回大学学习的年轻理想主义者的思想。我知
道,我就是其中的一个……社会主义承诺,满足我们要求
一个更理性、更公正世界的希望。然而这本书出现了。我
们的希望被打碎了。
\end{quote}
\switchcolumn*
Another young intellectual whose faith in socialism was
dashed by Mises was Wilhelm Roepke, who went on to be the
chief adviser to Ludwig Erhard, the German economics minister after World War II and chief architect of the ``German economic miracle'' of the 1950s and 1960s. Others took longer to
learn. The American economist and bestselling author Robert
Heilbroner wrote that in the 1930s, when he was studying economics, Mises's argument about the impossibility of planning
``did not seem a particularly cogent reason to reject socialism.''
Fifty years later, Heilbroner wrote in the \textit{New Yorker}, ``It turns out, of course, that Mises was right.'' Better late than never.
\switchcolumn
另一位被米瑟斯打碎了社会主义信仰的年轻知识分子是勒普克(Wilhelm Roepke)。他后来成为第二次世界大战后德国经济部长艾哈德(Lidwig Erhard)的首席顾问,而艾哈德正是20世纪50$\sim$60年代 “德国经济奇迹”的首席设计师。而其他人则经过了很长时间才领悟到这一点。美国经济学家和畅销书作家海尔布罗纳\footnote{尔布罗纳(Robert Heilbiwier, 1919$\sim$2005),美国经济学家、畅销书作家、古典自由主义学者。其著作有《美国资本主义的极限》《尘世哲学家》《几位著名经济思想家的生平、时代和思想》等。}在20世 纪30年代的文章中谈到,当他学习经济学的时候,看到米瑟斯在一系列文章中论述计划经济是不可能的,当时他觉得这些“反对社会主义的理由看上去并不特别有说服力”。五十年后之后,海尔布龙纳在《纽约客》(\textit{New Yorker})杂志发表的文章中说:“当然,后来证明米瑟斯是正确的。” 认识得晚总比从来没有认识到要好得多。
\switchcolumn*
Mises's magnum opus was \textit{Human Action}, a comprehensive
treatise on economics. In it he developed a complete science of
economics, which he considered to be the study of all purposeful human action. He was an uncompromising free-marketer,
who forcefully pointed out how every government intervention
in the marketplace tends to reduce wealth and the overall standard of living.
\switchcolumn
米瑟斯的杰作《人类行为》(\textit{Human  Action}) 是一部全面
的经济学巨著。在这本书中,他建立了一个全面的经济学体
系,他认为经济学就是对所有有目的的人类行为的研究。他是
一位不妥协的自由市场鼓吹者。他雄辩地指出了每一次政府对
市场的干预行为都会减少财富和降低整体的生活水平。
\switchcolumn*
Mises's student Hayek became not only a brilliant economist---he won the Nobel Prize in 1974---but perhaps the
greatest social thinker of the century. His books \textit{The Sensory Order}, \textit{The Counter-Revolution of Science}, \textit{The Constitution of Liberty},
and \textit{Law, Legislation, and Liberty} explored topics ranging from
psychology and the misapplication of the methods of the physical sciences in the social sciences to law and political theory. In
his most famous work, The \textit{Road to Serfdom}, published in 1944,
he warned the very countries that were then engaged in a war
against totalitarianism that economic planning would lead not
to equality but to a new system of class and status, not to prosperity but to poverty, not to liberty but to serfdom. The book
was bitterly attacked by socialist and left-leaning intellectuals
in England and the United States, but it sold very well (perhaps
one of the reasons the writers of academic books resented it)
and inspired a new generation of young people to explore libertarian ideas. Hayek's last book, \textit{The Fatal Conceit}, published in
1988 when he was approaching ninety, returned to the problem
that had occupied most of his scholarly interest: the spontaneous order, which is ``of human action but not of human design.'' The fatal conceit of intellectuals, he said, is to think that
smart people can design an economy or a society better than the
apparently chaotic interactions of millions of people. Such intellectuals fail to realize how much they don't know or how a market makes use of all the localized knowledge each of us
possesses.
\switchcolumn
米瑟斯的学生哈耶克不仅仅是一位杰出的经济学家,羸得了 1974年的诺贝尔经济学奖,而且也许是这个世纪的最伟大
的社会思想家。他著有《感觉的秩序》《科学的反革命》《自由宪章》 《法律、立法与自由》。这些著作探索的领域非常广
泛,从心理学、社会科学错误应用物理学的方法,到法律、政
治学理论等。他最著名的著作《通往奴役之路》 于 1944年出
版,在这本书中,他对那些正在和极权主义国家进行战争的国
家提出警告:计划经济将不会带来平等,只会带来一种新的等
级制度;不会带来富裕,而只会带来贫穷;不会带来自由,只
会带来奴役。这本书在英国和美国遭到了社会主义者和左翼知
识分子的激烈攻击,但是这本书相当畅销(也许正是因为学
院派学者的憎恨吧),激发了新一代年轻人研究古典自由主义
理论的热情。 《致命的自负》于 1988年出版,这时哈耶克已
经将近90岁。在这本书中,哈耶克回到了那个占据了他学术
生涯大部分时间的问题:自发秩序。他指出:“自发秩序是人
类行为而不是人类设计的秩序”,而知识分子的致命自负恰恰
在于,认为聪明人能够设计出一种经济和社会体制,比数以百
万计的人们那看上去混乱不堪的互动关系要好。这些知识分子
没有认识到哪些是他们不知道的,以及市场是怎样使我们每个
人拥有的个体化知识产生作用的。

\switchcolumn*[\subsection{The Last Classical Liberals\\最后的古典自由主义者}]
A group of writers and political thinkers was also keeping libertarian ideas alive. H. L. Mencken was best known as a journalist and literary critic, but he thought deeply about politics; he
said his ideal was ``a government that barely escapes being no
government at all.'' Albert Jay Nock (the author of \textit{Our Enemy,the State}), Garet Garrett, John T. Flynn, Felix Morley, and Frank Chodorov worried about the future of limited, constitutional
government in the face of the New Deal and what seemed to be
a permanent war footing that the United States had assumed
during the twentieth century. Henry Hazlitt, a journalist who
wrote about economics, served as a link between these schools.
He worked for the Nation and the \textit{New York Times}, wrote a column for \textit{Newsweek}, gave Mises's \textit{Human Action} a rave review, and popularized free-market economics in a little book called \textit{Economics in One Lesson}, which drew out the implications of Bastiat's
``what is seen and what is not seen.'' Mencken said of him, ``He
was one of the few economists in human history who could really write.''
\switchcolumn
有一群作家和政治思想家仍然在努力保持古典自由主义的活力。门肯以新闻记者和文学批评家而闻名,但是关于政治学他也进行了深入的思考;他说他的理想是“仅仅比无政府多一点点的政府”。其他的自由主义学者还有诺克 --- 《我们的敌人 --- 国家》(\textit{Our Enemy, the State}) 一书的作者、加略特(Garet Garrett)、弗 莱 恩 (John T. Flynn)、莫 里 (Felix  Morley )和肖多罗夫(Frank Chodorov)等。这些学者都在担心罗斯福新政之下有限的宪政政府的未来,而关于这个问题的斗争看来在美国将持续整个20世纪。
\switchcolumn*
In the dark year of 1943, in the depths of World War II and
the Holocaust, when the most powerful government in the history of the United States was allied with one totalitarian power
to defeat another, three remarkable women published books
that could be said to have given birth to the modern libertarian
movement. Rose Wilder Lane, the daughter of Laura Ingalls
Wilder, who had written Little House on the Prairie and other stories of American rugged individualism, published a passionate
historical essay called \textit{The Discovery of Freedom}. Isabel Paterson, a
novelist and literary critic, produced \textit{The God of the Machine},
which defended individualism as the source of progress in the
world. And Ayn Rand published \textit{The Fountainhead}.
\switchcolumn
在 1943年的黑暗岁月里,在第二次世界大战和纳粹大屠
杀的阴影当中,当美国历史上最强大的政府与一个极权主义政
权结盟以击败另一个极权主义政权的时候,三位非凡的女性出
版了她们的著作,这些著作可以说代表了现代古典自由主义的
诞生。萝 丝 $\cdot$ 蓝恩(Rose Wilder Lane) 发表了一篇充满激情的历史性著作《发现自由》(\textit{The Discovery of Freedom}) ---
她母亲劳拉(Laura Ingalls Wilder)著有《草原小屋》(\textit{Little House on the Prairie})和其他一些讲述美国质朴的个人主义的
故事。而小说家和文学批评家伊萨贝尔$\cdot$帕特森则出版了
《机器之神》,这本书捍卫了个人主义,认为个人主义是世界
进步之源。安 $\cdot$ 兰德则出版了《源泉》(\textit{The Fountainhead})。

\switchcolumn*[\subsection{Ayn Rand\\安$\cdot$兰德}]
\textit{The Fountainhead} was a sprawling novel about architecture and
integrity. The book's individualist theme did not fit the spirit of
the age, and reviewers savaged it. But it found its intended
readers. Its sales started slowly, then built and built. It was still
on the \textit{New York Times} bestseller list two full years later. Hundreds of thousands of people read it in the 1940s, millions eventually, and thousands of them were inspired enough to seek
more information about Ayn Rand's ideas. Rand went on to
write an even more successful novel, \textit{Atlas Shrugged}, in 1957,
and to found an association of people who shared her philosophy, which she called Objectivism. Although her political philosophy was libertarian, not all libertarians shared her views on
metaphysics, ethics, and religion. Others were put off by the
starkness of her presentation and by her cult following.
\switchcolumn
《源泉》是一本关于建筑学和诚实正直精神的小说。这本
书中的个人主义思想和时代显得格格不入,评论家们对此进行
了猛烈的批评。但是它找到了它的目标读者。开始,这本书销
售很慢,然后销量不断地不断地增加。后 来 在 《纽约时报》的畅销书排行榜上竟然盘踞了两年。数以十万计的读者在20
世纪40年代读了这本书,而最终的读者数量达到了数百万人,
其中数以千计的读者受到启发开始寻找更多的信息来研究安$\cdot$
兰德的思想。安 $\cdot$ 兰德接下来在1957年出版了一本更加成功
的作 品《阿特拉斯耸耸肩》(\textit{Atlas Shrugged}),并且建立了一
个学会来分享她的哲学,她称之为客观主义。尽管她的政治理
论是古典自由主义,但是并不是所有的古典自由主义者都赞同
她在形而上学、美学和宗教上的观点。其他人则被她刻板的表
达和狂热的崇拜者所吓退。
\switchcolumn*
Like Mises and Hayek, Rand demonstrates the importance of
immigration not just to America but to American libertarianism. Mises had fled the Nazis, Rand fled the Communists who
came to power in her native Russia. When a heckler asked her
after a speech, ``Why should we care what a foreigner thinks?''
she replied with her usual fire, ``I chose to be an American. What
did you ever do, except for having been born?''
\switchcolumn
像米瑟斯和哈耶克一样,安$\cdot$兰德的例子显示了移民不仅
仅对美国至关重要,而且对美国古典自由主义的发展也至关重
要。米瑟斯逃离纳粹的统治,安 $\cdot$ 兰德则从苏俄逃出来。曾经
有一个人在她演讲完毕之后质问她:“为什么我们要关心一个
外国人怎么想呢?” 她用她那一贯的充满火药味的口气回答:
“我选择成为美国人。你又做了些什么呢,除了出生在这里
之外?”

\switchcolumn*[\subsection{The Postwar Revival\\战后的复兴}]
Not long after the publication of Atlas Shrugged, the University of
Chicago economist Milton Friedman published \textit{Capitalism and Freedom}, in which he argued that political freedom could not exist
without private property and economic freedom. Friedman's
stature as an economist, which won him a Nobel Prize in 1976,
was based on his work in monetary economics. But through \textit{Capitalism and Freedom}, his long-running \textit{Newsweek} column, and the
1980 book and television series \textit{Free to Choose}, he became the most
prominent American libertarian of the past generation.
\switchcolumn
在 《阿特拉斯耸耸肩》 出版之后不久,芝加哥大学的经
济学家米尔顿$\cdot$ 弗里德曼 (Milton  Friedman) 出版了《资本主义与自由》(\textit{Capitalism  and Freedom}) 一书。在书中他论证
道,如果没有私人产权和经济自由,政治自由也不会存在。作
为一名经济学家,弗里德曼的地位是建立在他对货币经济学的
研究之上的。1976年,他获得了诺贝尔经济学奖。但是通过《资本主义与自由》一书,通过他在 《新闻周刊》 上长期专栏,以及1980年的著作和电视系列片《自由选择》(\textit{Free to choose}), 他成为上一优最杰出的美国古典自由主义者。

\switchcolumn*
Another economist, Murray Rothbard, achieved less fame
but played an important role in building both a theoretical
structure for modern libertarian thought and a political movement devoted to those ideas. Rothbard wrote a major economic
treatise, \textit{Man, Economy, and State}; a four-volume history of the
American Revolution, \textit{Conceived in Liberty}; a concise guide to the
theory of natural rights and its implications, \textit{The Ethics of Liberty}; a popular libertarian manifesto, \textit{For a New Liberty}; and countless pamphlets and articles in magazines and newsletters.
Libertarians compared him to both Marx, the builder of an integrated political-economic theory, and Lenin, the indefatigable organizer of a radical movement.
\switchcolumn
另一位经济学家罗斯巴德(Murry Rothbard) 的名声稍逊 ,但是他无论对于现代古典自由主义思想理论框架的建立,还是在基于这些理念的政治运动中,都扮演了重要的角色。罗斯巴德写了一本重要的经济学著作《人、经济与国家》(\textit{Man, Economy and State}), 四卷本的美国革命历史书《在自由中孕育》(\textit{Conceived in Liberty});关于自由权利理念及其推论的简明导读《自由的伦理》(\textit{The Ethics of Liberty}),大众化的古典自由主义宣言《为了新自由》(\textit{For a New Liberty}),以及难以计数的小册子和在杂志、福纸上发表的文章。古典自由主义者把他比作古典自由主义马克思和列宁。马克思是统一的政治经济学理论的创立者,而列宁则是激进运动不屈不挠的组织者。
\switchcolumn*
Libertarianism got a major boost in scholarly respect in 1974 with the publication of \textit{Anarchy, State, and Utopia} by the Harvard University philosopher Robert Nozick. With wit and fine-toothed logic, Nozick laid out a case for rights, which concluded that
\switchcolumn
1974年 ,古典自由主义在学术上取得了重大发展。这一年,哈佛大学的哲学家罗伯特$\cdot$ 诺齐克(Robert  Nozick) 的
《无政府、国家和乌托邦》(\textit{Anarchy, State,and Utopia}) 出版。诺齐克以机智和严密的逻辑提出了关于权利的问题,他总结道:
\switchcolumn*
\begin{quote}
a minimal state, limited to the narrow functions of protection
against force, theft, [and] fraud, enforcement of contracts, and so
on, is justified; that any more extensive state will violate persons' rights not to be forced to do certain things, and is unjustified;
and that the minimal state is inspiring as well as right.
\end{quote}
\switchcolumn
\begin{quote}
一个将其功能限制在以下领域:保护公民不受暴力、
盗窃、欺诈的侵害,保护契约等领域的最小国家,是正当
的;任何超出这些功能的国家政权都将侵犯人们不被强制
做某事的权利, 因而是不正当的;而最小国家既是激动人
心的,也是正当的。
\end{quote}
\switchcolumn*
In a catchier vein, he called for the legalization of ``capitalist
acts between consenting adults.'' Nozick's book---along
with Rothbard's \textit{For a New Liberty} and Rand's essays on political
philosophy---defined the ``hard-core'' version of modern libertarianism, which essentially restated Spencer's law of equal freedom: Individuals have the right to do whatever they want to, so
long as they respect the equal rights of others. The role of government is to protect individual rights from foreign aggressors and from neighbors who murder, rape, rob, assault, or defraud
us. And if government seeks to do more than that, it will itself
be depriving us of our rights and liberties.
\switchcolumn
在一个总结性的章节中,他主张 “ 双方自愿的成年人之间的资本主义行为的合法化”。诺齐克的书和罗斯巴德的《为了新自由》 和 安 $\cdot$兰德的政治哲学文章一起构成了现代古典自由主义的“硬核” ,重申了斯宾塞的同等自由法则:个人有权做他们想做的事情,同时必须尊重别人的同等权利。政府的角色是保护个人自由,防止我们受到来自外国侵略者和周围人的谋杀、强奸、抢劫、威胁或者欺诈。如果政府试图超过这个范围,它就会剥夺我们的权利和自由。

\switchcolumn*[\section{Libertarianism Today\\当代古典自由主义}]
Libertarianism is sometimes accused of being rigid and dogmatic, but it is in fact merely a basic framework for societies in
which free individuals can live together in peace and harmony,
each undertaking what Jefferson called ``their own pursuits of
industry and improvement.'' The society created by a libertarian framework is the most dynamic and innovative ever seen on
earth, as witness the unprecedented advances in science, technology, and standard of living since the liberal revolution of the
late eighteenth century. A libertarian society is marked by
widespread charity undertaken as a result of personal benevolence, not left to state coercion.
\switchcolumn
古典自由主义有时被批评为过于僵化和教条,但是它其实
仅仅是社会运转的基本制度框架,在这个框架下,自由的个人
能够和平、和谐地生活在一起,每个人都可以追求杰斐逊所说
的 “他们自己的事业和生活的改善”。建立在古典自由主义框
架下的社会是地球上所能见到的最有活力和创新精神的社会。
自从18世纪的自由主义革命以来,人类社会经历了前所未有
的科技进步和生活水平提髙。一个古典自由主义社会的慈善事
业也会获得发展,这是个人善行的结果,而与政府的强制
无关。
\switchcolumn*
Libertarianism is also a creative and dynamic framework for
intellectual activity. Today it is statist ideas that seem old and
tired, while there is an explosion of libertarian scholarship in
such fields as economics, law, history, philosophy, psychology,
feminism, economic development, civil rights, education, the
environment, social theory, bioethics, civil liberties, foreign policy, technology, the Information Age, and more. Libertarianism
has developed a framework for scholarship and problem solving, but our understanding of the dynamics of free and unfree
societies will continue to develop.
\switchcolumn
古典自由主义也为知识活动提供了最有创造力和活力的制
度框架。今天,国家主义的理念显得陈旧过时、令人厌倦,而
自由主义的学术研究迅速扩展到经济学、法学、历史学、哲
学 、心理学、女权主义、经济发展、公民权利、教育、环境、
社会学、信息时代等研究领域。古典自由主义建立了一种学术
研究和问题解决的框架,但是我们对自由和不自由社会的驱动
力的研究还将继续进行。
\switchcolumn*
Today, the intellectual development of libertarian ideas continues, but the broader impact of those ideas derives from the
growing network of libertarian magazines and think tanks, the
revival of traditional American hostility to centralized government, and most important, the continuing failure of big government to deliver on its promises.
\switchcolumn
今天,古典自由主义理念的知识发展仍然在继续,但是这
些理念更广泛的影响来自不断增长的古典自由主义的杂志和思
想库网络的扩展,来自于美国传统中警惕中央政府这一观念的
复兴,而最重要的是大政府在兑现其承诺过程中的不断失败。

\end{paracol}