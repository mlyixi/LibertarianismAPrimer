\chapter*{FOR FURTHER READING}

\section*{Chapter 1}
A basic introduction to libertarian ideas can be found in Murray N. Rothbard, \textit{For a New Liberty: The Libertarian Manifesto} (New York: Collier, 1978).

A more current and less radical presentation is Charles Murray, \textit{What It Means to Be a Libertarian} (New York: Broadway, 1997). 

For contemporary libertarian policy ideas, see David Boaz and Edward H. Crane, eds., \textit{Market
Liberalism: A Paradigm forthe 21st Century} (Washington, D.C.: Cato Institute,
1993). 

Other introductory libertarian works include F. A. Hayek, \textit{The Road
to Serfdom} (Chicago: University of Chicago Press, 1944); Milton Friedman,
\textit{Capitalism and Freedom} (Chicago: University of Chicago Press, 1962); and
David Friedman, \textit{The Machinery of Freedom} (La Salle, 111.: Open Court, 1989).

British readers might want to consult Geoffrey Sampson, \textit{An End to Allegiance} (London: Temple Smith, 1984). And for a light-hearted libertarian look at big government, see E J. O'Rourke, \textit{Parliament of Whores} (New York:
Atlantic Monthly Press, 1991).

\section*{Chapter 2}
For an introduction to the history of liberty, see Lord Acton, \textit{Essays in the History of Liberty} (Indianapolis: Liberty Classics, 1985); Alexander Rustow,
\textit{Freedom and Domination} (Princeton, N.J.: Princeton University Press, 1980);
and Ralph Raico, ``The Epic Struggle for Liberty'' (New York: Laissez Faire
Books, 1994), audiotape series. 

Many of the works discussed in this chapter
are excerpted in David Boaz, ed., \textit{The Libertarian Reader: Classic and Contemporary Writings from Lao-tzu to Milton Friedman} (New York: Free Press, 1997).

A different selection of documents and excerpts from classical liberal writings is E. K. Bramsted and K. J. Melhuish, eds., \textit{Western Liberalism: A History
in Documents from Locke to Croce} (New York: Longman, 1978). On the rise of liberty and commerce in Europe, see E. L. Jones, \textit{The European Miracle} (Cambridge: Cambridge University Press, 1981); Douglas Irwin, \textit{Against the Tide:
An Intellectual History of Free Trade} (Princeton, N.J.: Princeton University
Press, 1996); and Nathan Rosenberg and L. E. Birdzell, Jr., \textit{How the West Grew Rich} (New York: Basic Books, 1986). On the libertarian origins of the
United States, see Bernard Bailyn, \textit{The Ideological Origins of the American Revolution} (Cambridge: Harvard University Press, 1967) and Arthur Ekirch, \textit{The Decline of American Liberalism} (New York: Atheneum, 1967).

The key books of classical liberalism are available in many editions, including John Milton, \textit{Areopagitica}; John Locke, \textit{The Second Treatise of Government}; David Hume, \textit{A Treatise of Human Nature}; Adam Smith, \textit{The Theory of
Moral Sentiments and The Wealth of Nations}; Thomas Paine, \textit{Common Sense and
The Rights of Man}; Alexander Hamilton, James Madison, and John Jay, \textit{The Federalist Papers}; Alexis de Tocqueville, \textit{Democracy in America}; John Stuart
Mill, On Liberty; Herbert Spencer, \textit{Social Statics and The Man versus the State};
Wilhelm von Humboldt, \textit{The Sphere and Duties of Government}; and various writings of Thomas Jefferson, Benjamin Constant, Frederic Bastiat, William
Lloyd Garrison, and Mary Wollstonecraft. Leveller writings can be found in
G. E. Aylmer, ed., \textit{The Levellers in the English Revolution} (Ithaca, N.Y: Cornell
University Press, 1975), and Cato's Letters are now available in Ronald
Hamowy, ed., \textit{Cato's Letters} (Indianapolis: Liberty Press, 1995).

Important twentieth-century libertarian books include Ludwig von
Mises, \textit{Socialism} (1922; Indianapolis: Liberty Classics, 1981), \textit{Human Action}
(New Haven, Conn.: Yale University Press, 1963), and other works; F. A.
Hayek, \textit{The Road to Serfdom} (1944), \textit{The Constitution of Liberty} (1960), \textit{The
Fatal Conceit} (1988) and \textit{Law, Legislation, andLiberty} (1973, 1976, 1979) (all
from the University of Chicago Press), and many other works; Isabel Paterson, \textit{The God of the Machine} (1943; New Brunswick, N.J.: Transaction,
1993); Rose Wilder Lane, \textit{The Discovery of Freedom} (1943; New York: Laissez
Faire Books, 1984); Ayn Rand, \textit{The Fountainhead} (New York: Bobbs-Merrill, 1943), \textit{Atlas Shrugged} (New York: Random House, 1957), \textit{Capitalism:
The Unknown Ideal} (New York: New American Library, 1967), and other
works; Milton Friedman, \textit{Capitalism and Freedom} (Chicago: University of
Chicago Press, 1962); Milton and Rose Friedman, \textit{Free to Choose} (New York:
Harcourt Brace Jovanovich, 1980); Murray Rothbard, Man, \textit{Economy, and
State} (Los Angeles: Nash, 1972), For a New Liberty (New York: Collier,
1978), and \textit{The Ethics of Liberty} (Atlantic Highlands, N.J.: Humanities
Press, 1982); and Robert Nozick, \textit{Anarchy, State, and Utopia} (New York: Basic Books, 1974).

Contemporary libertarian scholarship is too voluminous to list. A basic
list might include work in economics (Thomas Sowell, \textit{Knowledge and Decisions}; Israel Kirzner, \textit{Competition and Entrepreneurship}), law (Richard Epstein,
\textit{Simple Rules for a Complex World}; Ellen Frankel Paul, \textit{Property Rights and Eminent Domain}), history (Robert Higgs, \textit{Crisis and Leviathan: Critical Episodes in
the Growth of American Government}), philosophy (Loren Lomasky, \textit{Persons, Rights, and the Moral Community}; Tara Smith, \textit{Moral Rights and Political Freedom}; Tibor Machan, \textit{Individuals and Their Rights}; Jan Narveson, \textit{The Libertarian Idea}), psychology (Thomas Szasz, \textit{Law, Liberty, and Psychiatry}), feminism (Joan Kennedy Taylor, \textit{Reclaiming the Mainstream}; Wendy McElroy, \textit{Sexual
Correctness}), economic development (E T. Bauer, \textit{Dissent on Development}; Hernando de Soto, \textit{The Other Path}; Deepak Lai, \textit{The Poverty of Development Economics}), civil rights (Walter Williams, \textit{The State against Blacks}; Clint Bolick,
\textit{Changing Course}), the First Amendment (Jonathan Emord, \textit{Freedom, Technology, and the First Amendment}), education (Myron Lieberman, \textit{Beyond Public Education}; Sheldon Richman, \textit{Separating School and State}), the environment
(Julian Simon, \textit{The Ultimate Resource}; Terry Anderson and Don Leal, \textit{Free Market Environmentalism}), social theory (Charles Murray, \textit{In Pursuit: Of Happiness and Good Government}), bioethics (Tristram Engelhardt, \textit{The Foundations
of Bioethics}), civil liberties (Stephen Macedo, \textit{The New Right v. the Constitution}; James Bovard, \textit{Lost Rights}), foreign policy (Earl Ravenal, \textit{Defining Defense};
Ted Galen Carpenter, \textit{A Search for Enemies}), new technologies and the Information Age (Michael Rothschild, \textit{Bionomics: The Economy as Ecosystem};
Lawrence Gasman, Telecompetition: \textit{The Free Market Road to the Information Highway}), and more.

\section*{Chapter 3}
For more on the libertarian view of rights, see Robert Nozick, \textit{Anarchy,
State, and Utopia} (New York: Basic Books, 1974); Murray N. Rothbard, \textit{The Ethics of Liberty} (Atlantic Highlands, N.J.: Humanities Press, 1982); and Ayn Rand, ``Man's Rights,'' in \textit{Capitalism: The Unknown Ideal} (New York: New American Library, 1967). More recent treatments include Douglas B. Rasmussen and Douglas J. Den Uyl, \textit{Liberty and Nature} (La Salle, III: Open
Court, 1991); Jan Narveson, \textit{The Libertarian Idea} (Philadelphia: Temple University Press, 1988); David Conway, \textit{Classical Liberalism}: The Unvanquished
Ideal (New York: St. Martin's, 1996); and Richard Epstein, \textit{Simple Rules for a Complex World} (Cambridge: Harvard University Press, 1995). Of course, the
works of Locke, Hume, Paine, Spencer, and others cited in chapter 2 are also essential to an understanding of libertarian rights theory.

\section*{Chapter 4}
On individualism, see Felix Morley, ed., \textit{Essays on Individuality} (Indianapolis: Liberty Press, 1977). See also Wendy McElroy, ed., \textit{Freedom, Feminism, and the State} (Washington, D.C.: Cato Institute, 1982); Joan Kennedy
Taylor, \textit{Reclaiming the Mainstream: Individualist Feminism Rediscovered} (Buffalo: Prometheus, 1992); and Clint Bolick, \textit{Changing Course: Civil Rights at the
Crossroads} (New Brunswick, N.J.: Transaction, 1988).

\section*{Chapter 5}
On the appropriate rules for a free society, see F. A. Hayek, \textit{The Constitution of Liberty} (Chicago: University of Chicago Press, 1960). On the meaning
of toleration and pluralism in specific areas, see George H. Smith, ``Philosophies of Toleration,'' in \textit{Atheism, Ayn Rand, and Other Heresies} (Buffalo:
Prometheus, 1991); Sheldon Richman, \textit{Separating School and State} (Fairfax, Va.: Future of Freedom Foundation, 1994); and H. Tristram Engelhardt, Jr.,
\textit{The Foundations of Bioethics} (Oxford: Oxford University Press, 1986).
\section*{Chapter 6}
On law and liberty, see F. A. Hayek, \textit{The Constitution of Liberty} (Chicago:
University of Chicago Press, 1960) and \textit{Law, Legislation, and Liberty}, vols. 1
and 2 (Chicago: University of Chicago Press, 1973 and 1976); and Bruno Leoni, \textit{Freedom and the Law} (Indianapolis: Liberty Press, 1991). On modern
constitutional law, see Richard Epstein, \textit{Simple Rules for a Complex World}
(Cambridge: Harvard University Press, 1995) and \textit{Takings: Private Property
and the Right of Eminent Domain} (Cambridge: Harvard University Press,
1985); Henry Mark Holzer, \textit{Sweet Land of Liberty?} (Costa Mesa, Calif: Common Sense, 1983); Stephen Macedo, \textit{The New Right v. the Constitution} (Washington, D.C.: Cato Institute, 1987); Roger Pilon, ``Freedom, Responsibility,
and the Constitution: On Recovering Our Founding Principles,'' in David Boaz and Edward H. Crane, eds., \textit{Market Liberalism: A Paradigm for the 21st
Century} (Washington, D.C.: Cato Institute, 1993) and ``A Government of
Limited Powers,'' in \textit{The Cato Handbook for Congress} (Washington, D.C.: Cato
Institute, 1995). See also, of course, \textit{The Federalist Papers} and Herbert Storing, ed., \textit{The Anti-Federalist} (Chicago: University of Chicago Press, 1985), a
collection of Anti-Federalist writings.
\section*{Chapter 7}
On civil society, see (once again) F. A. Hayek, \textit{The Constitution of Liberty}
(Chicago: University of Chicago Press, 1960); Ernest Gellner, \textit{Conditions of
Liberty: Civil Society and Its Rivals} (New York: Viking Penguin, 1994); and
Charles Murray, \textit{In Pursuit: Of Happiness and Good Government} (New York:
Simon \& Schuster, 1988). For earlier treatments, see Adam Ferguson, \textit{An
Essay on the History of Civil Society} (1773); Alexis de Tocqueville, \textit{Democracy in
America} (1835); and Benjamin Constant, ``The Liberty of the Ancients Compared with That of the Moderns'' (1833) in \textit{Benjamin Constant: Political Writings}, Biancamaria Fontana, ed. (New York: Cambridge University Press,
1988). On mutual aid, see David Green, \textit{Reinventing Civil Society: The Rediscovery of Welfare without Politics} (London: Institute of Economic Affairs, 1993);
David Green and Lawrence Cromwell, \textit{Mutual Aid or Welfare State: Australia's
Friendly Societies} (Sydney: Allen \& Unwin, 1984); and David Beito, ``Mutual
Aid for Social Welfare: The Case of American Fraternal Societies,'' Critical
Review 4, no. 4.
\section*{Chapter 8}
There are three short books that provide an easy introduction to economics: Henry Hazlitt, \textit{Economics in One Lesson} (New York: Crown, 1979);
Faustino Ballve, \textit{Essentials of Economics} (Irvington, N.Y: Foundation for Economic Education, 1963); and James D. Gwartney and Richard L. Stroup,
\textit{What Everyone Should Know about Economics and Prosperity} (Tallahassee, Fla.:
James Madison Institute, 1993). The serious student should consult two
outstanding treatises: Ludwig von Mises, \textit{Human Action} (New Haven,
Conn.: Yale University Press, 1963) and Murray Rothbard, \textit{Man, Economy,
and State} (Los Angeles: Nash, 1972), along with its sequel, \textit{Power and Market}
(Menlo Park, Calif.: Institute for Humane Studies, 1970). Two good textbooks are Paul Heyne, \textit{The Economic Way of Thinking} (Chicago: Science Research Associates, 1983) and James D. Gwartney and Richard L. Stroup,
\textit{Economics: Private and Public Choice} (Orlando, Fla.: Dryden Press, 1992). Of
course, the classic source for economics is Adam Smith, \textit{An Inquiry into the
Nature and Causes of the Wealth of Nations} (1776).
\section*{Chapter 9}
On the libertarian view of coercive government, see Thomas Paine, Common Sense (1776); Albert Jay Nock, \textit{Our Enemy, the State} (1935); Herbert
Spencer, \textit{The Man versus the State} (1884); and Murray N. Rothbard, \textit{For a New
Liberty: The Libertarian Manifesto} (New York: Collier, 1978). On Public
Choice economics see James M. Buchanan and Gordon TuUock, \textit{The Calculus
of Consent} (Ann Arbor: University of Michigan Press, 1962) and James L.
Payne, \textit{The Culture of Spending} (San Francisco: Institute for Contemporary
Studies, 1991). On war and the growth of the state, see Robert Higgs, \textit{Crisis
and Leviathan: Critical Episodes in the Growth of American Government} (New
York: Oxford University Press, 1987) and Bruce D. Porter, \textit{War and the Rise
of the State} (New York: Free Press, 1994). On how the U.S. government presently deprives Americans of their rights, see James Bovard, \textit{Lost Rights} (New York: St. Martin's, 1994).
\section*{Chapter 10}
On libertarian approaches to public policy issues, I can heartily recommend David Boaz and Edward H. Crane, eds., \textit{Market Liberalism: A New Paradigm for the 21st Century} (Washington, D.C.: Cato Institute, 1993) and \textit{The
Cato Handbook for Congress} (Washington, D.C.: Cato Institute, 1995).
\section*{Chapter 11}
On the problem of market failure and public goods, see Tyler Cowen, ed.,
\textit{The Theory of Market Failure} (Fairfax, Va.: George Mason University Press,
1988), which includes, among other essays, both Coase on lighthouses and
Cheung on beekeepers. Allen Wallis's analysis can be found in \textit{Welfare Programs: An Economic Appraisal} (Washington, D.C.: American Enterprise Institute, 1968). On the Postal Service, see Edward L. Hudgins, ed., \textit{The Last
Monopoly: Privatizing the Postal Service for the Information Age} (Washington,
D.C.: Cato Institute, 1996). On education, see Sheldon Richman, \textit{Separating
School and State} (Fairfax, Va.: Future of Freedom Foundation, 1994); Lewis
Perelman, \textit{School's Out: Hyperlearning, the New Technology, and the End of Education} (New York: Morrow, 1992); and Myron Lieberman, \textit{Public Education: An
Autopsy} (Cambridge: Harvard University Press, 1993). On private communities, see Fred Foldvary, \textit{Public Goods and Private Communities: The Market
Provision of Social Services} (Brookfield, Vt.: Edward Elgar, 1994).
\section*{Chapter 12}
For libertarian perspectives on the Information Age, see Lawrence Gasman, \textit{Telecompetition: The Free Market Road to the Information Highway} (Washington, D.C.: Cato Institute, 1994); Peter Huber, \textit{Orwell's Revenge} (New
York: Free Press, 1994); and Norman Macrae, \textit{The 2025 Report: A Concise
History of the Future, 1975-2025} (New York: Macmillan, 1985).
\\
\\
A good resource for libertarian books and general information on freemarket economics and libertarian political theory is Laissez-Faire Books, 938
Howard Street, San Francisco, CA 94103, (800) 326-0996.