\chapter{LAW AND THE CONSTITUTION\\法律与宪法}
\begin{paracol}{2}
\hbadness5000

Closely tied to questions of the state's scope is the venerable libertarian principle of the rule of law. In its simplest
form, this principle means that we should be governed by generally applicable laws, not by the arbitrary decisions of rulers---``a government of laws, not of men,'' as the Massachusetts Bill
of Rights of 1780 put it.
\switchcolumn
与国家的界限问题密切相关的是古典自由主义的一项古老
原则:法治。用最简单的话来概括这个原则的意思就是:我们
应该由普遍的法律来统治,而不是由统治者的专断决策来统
治。“政府是法律的政府,而不是人的政府” ,1780年马萨诸
塞 州 《权利法案》如是说。
\switchcolumn*
In \textit{The Constitution of Liberty}, Friedrich Hayek discusses the
rule of law in detail. He lays out three aspects of the principle:
Laws should be general and abstract, not intended to
command specific actions by citizens; they should be known
and certain, so that citizens can know in advance that their actions comply with the law; and they should apply equally to
all persons.
\switchcolumn
在《自由宪章》中,哈耶克详细讨论了法治。他举出三
个方面的原则:法律应当是普世的和抽象的,而不是针对公民
的特定行为;法律应当被普遍告知和确定,公民才能事先知道
他们的行为是否合法;法律应平等地适用于每个人。
\switchcolumn*
These principles have important implications.
\switchcolumn
这些原则有一些重要的推论:
\switchcolumn*
\begin{itemize}
	\item The laws must apply to everyone, including the rulers.
	\item No one is above the law.
	\item To guard against the accumulation of arbitrary power, power should be divided.
	\item The laws should be made by one body and administered by another.
	\item An independent judiciary is necessary to ensure that the laws are administered fairly.
\end{itemize}
\switchcolumn
\begin{itemize}
	\item 法律应适用于每一个人,包括统治者。
	\item 没有人能高于法律。
	\item 为对抗不断增长的专断权力,权力应分立。
	\item 法律应当由一个主体所立,由另一个主体执行。
	\item 司法独立是必须的,以确保法律公正。
\end{itemize}
\switchcolumn*
Those who administer the law should have little discretion,
because discretionary power is the very evil that the rule of
law is intended to prevent.
\switchcolumn
执行法律的人应当有很少的自由裁量权,因为自由裁量权是法治所应防范的恶。

\switchcolumn*[\section{Judge-Made Law\\法官造法}]

There is a confusion in our modern language over the meaning
of the word ``law.'' We tend to think of law as something written
by Congress or the state legislature. But in fact law is much
older than any legislative body. As Hayek notes, ``only the observance of common rules makes the peaceful existence of individuals in society possible.'' Those rules are the law, which
originally developed through the process of deciding disputes.
Laws were not laid down in advance by a lawgiver or legislative
body; they were built up one by one, as each dispute was decided. Each new decision helped to delineate what rights people
had, especially with regard to how they could use property and
how contracts would be interpreted and enforced.
\switchcolumn
“法律” 这个词的含义在现代语言中有些混乱。我们倾向于认为法律是由国会或者州议会所写的某些东西。但是事实上
法律比任何立法机构都要古老。正如哈耶克所注意到的,“只
有对共同规则的遵守才使得个人在社会中的和平存在成为可
能。” 那些规则就是法律,最初是通过纠纷的解决过程而发展
起来的。法律并不是由立法者或者立法机构事先制定的,而是
通过每一个纠纷的判决而一个接一个逐步建立起来的。每一个
新的纠纷的判决都有助于界定人们有哪些权利,尤其是那些关
于如何使用财产、契约如何解释和执行的判例。
\switchcolumn*
The evolution of law in this manner began before recorded
history, but it is best known in the form of Roman law, especially the Justinian Code (or \textit{Corpus Juris Civilis}), which still underlies Continental European law, and of the English common
law, which continued to develop in the United States and other
former English colonies. Codification of law, such as the Uniform Commercial Code, usually reflects an attempt to collect
and set down in one place the decisions that judges and juries
have made in myriad cases and the terms of contracts in evolving areas of the economy. The American Law Institute, a private
organization, regularly recommends commercial-code revisions
to legislatures. According to Hayek, even the great lawgivers of
history, such as Hammurabi, Solon, and Lycurgus, ``did not intend to create new law but merely to state what law was and
had always been.''
\switchcolumn
法律如此的演变开始于有记录的历史之前,但最广为人知
的是罗马法和英国普通法的产生。罗马法特别是《查士丁尼法典》\footnote{查士丁尼法典,指拜占庭皇帝查士丁尼一世主持下于529$\sim$565年完成的法律和法律解释的汇编。},现在仍是欧洲大陆法的基础;英国普通法则在美国
和其他前英国殖民地得到了继续发展。法典的编纂,如 《美
国统一商法典》\footnote{美国统一商法典(Uniform Commercial Code, U. C. C),一部被看作是英美法与大陆法日趋合一的最具代表性的法典。美国统一商法典不同于大陆法国家的商法典,它不是美国国会通过的法律,而是由一些法律团体起苹,供各州自由采用的一部样板法。尽管如此,现在在美国50个州中,除保持大陆法传统的路易斯安娜州外,其他各州均已通过本州立法采用了这部统一商法典。}常常反映了一种试图把法官和陪审团曾经判决过的大量判例和不断变化的经济领域中的合同条款集中到一起并制定法律的努力。美国法律研究所(一个私人组织)也定期向立法机构提供修改商业法典的建议。根据哈耶克的观
点,甚至历史上伟大的立法者如汉谟拉比、所罗门和吕库古\footnote{吕库古(Lycurgus),传说是公元前7 世纪的斯巴达立法者。他为斯巴达建立了一套独特的法律和政治制度,这套制度不仅保证了相当长时期里斯巴达的军事优势和实力,而且这套制度所塑造的严正和艰苦的生活方式也吸引了当时和后世的许多人。}“都不会试图创造新的法律,而仅仅是表明法律当时的样子和它一直以来的状态。”
\switchcolumn*
As English jurists such as Coke and Blackstone pointed out,
the common law is part of the constitutional check on the
concentration of power. A judge doesn't issue edicts; he can
only rule when a dispute is brought to him. That limitation keeps the judge's power in check, and the fact that the law is
made by many people involved in many disputes limits the
potential for arbitrary power wielded by a lawgiver, whether a
monarch or a legislature. Generally, people go to court only
when their lawyers identify a problem---an unsettled area---in
the law. (A lawyer's job is frequently to tell a client, ``The law
is clear. You have no case. You'll be wasting everybody's time
and money if you go to court.'') In that way, many people participate in the evolution of the law to deal with new circumstances and problems.
\switchcolumn
正如英国法学家如柯克(Coke)和 布 莱 克 斯 通 (Blackstone) 指出,普通法是对权力集中的宪法性检查的一部分。
法官并不发布命令,只有当纠纷提交给他的时候他才会进行裁
决。这种限制让法官的权力仅限于检查,而法律是由在无数纠
纷中的无数人所创制的事实,限制了立法者所掌握的独断权力
的范围,无论这个立法者是君主还是立法机构。通常只有当他
们的律师在法律中找到一个未解决的漏洞时,人们才会上法
院。律师的工作之一常常是告诉客户:“法律很清楚。你没有
机会。如果上法庭的话,你会浪费所有人的时间和i 钱。”通
过这样的方式,人们参与了法律的演进。而法律通过这样的演
进来适应新环境和新问题。
\switchcolumn*
Legislation---which is unfortunately called law by most people---is a different process. Much legislation involves rules for
running the government, in which case it is similar to the internal rules of any organization. Some other legislation, as noted
above, amounts to codifying the common law. But increasingly,
legislation involves commands directing how people shall act,
with the purpose of effecting specific outcomes. In that way
legislation moves a society away from general rules that protect
rights and leave people free to pursue their own ends, toward
detailed rules specifying how people should use their property
and interact with others.
\switchcolumn
立法---很不幸这被大多数人称为“法律” --- 是一个
完全不同的程序。很多立法涉及政府如何运作的规则,这就很
像任何一个组织的内部规章。其他的立法则就像上面提到的那
样把普通法编成法典。但是立法越来越多地涉及一些命令,试
图指导人们应怎样行动,以达到特定的结果。通过这样的方
式 ,立法把社会从一个遵守普遍规则、保护权利、让人们自由
追求自己目标的社会,变成了一个用繁琐的规则来规定人们应
如何使用财产和与他人互动的社会。


\switchcolumn*[\section{The Decline of Contract Law\\合同法的退步}]

As legislation has superseded common law in regulating our
relations with one another, legislators have taken more and
more of our income in taxes and circumscribed property rights
through regulations aimed at securing everything from low-cost housing to panoramic views. Judges, unfortunately, have
not only upheld those legislative decisions, ignoring provi-
sions of the U.S. Constitution that protect property rights;
they have also voided contracts that they thought reflected
``unequal bargaining power'' or that otherwise were not in
``the public interest.'' In any given case, if the legislator or
judge thought his values would be served by transferring
property from its rightful owner to a more sympathetic
claimant or releasing someone from the contractual obligations he had assumed, the great benefits of a system of property and contract were dismissed.
\switchcolumn
由于立法代替了普通法规定我们与别人的关系,为了确保
包括从低成本住房到环境景观在内的所有东西,议员们从我们
那里拿走越来越多的收入作为税收,通过法令限定我们的财产
权。很不幸的是,法官只会确认那些立法决定,无视美国宪法
明文规定的对财产权的保护;他们还宣告许多合同无效,因为
他们认为这些合同反映了双方“谈判能力不对等”,或者不符
合 “ 公共利益”。在上面这些例子中,如果议员或者法官认为
他的价值观应该得到应用,就可以把财产从合法所有者那里转
移给一个更值得同情的申请者那里,或者解除某个人应承担的
合同义务,那么产权和契约制度给我们带来的巨大利益就会烟
消云散。
\switchcolumn*
In his book \textit{Sweet Land of Liberty}? the legal scholar Henry
Mark Holzer identifies several milestones in the government's
erosion of the sanctity of contract. Before the Civil War, he
points out, money in the United States consisted of gold and silver coin. To finance the Civil War, Congress authorized the issuing of inflationary paper money, which it declared to be ``legal
tender,'' meaning that it had to be accepted in payment of
debts, even if the lender had expected to be repaid in gold or silver. In 1871 the Supreme Court upheld the Legal Tender Act,
effectively rewriting every loan agreement---and putting people with money on notice that the government could unilaterally change the terms of future loans. Then in 1938, despite the
explicit provision in the Constitution forbidding the states to
enact any ``law impairing the obligation of contracts,'' the
Supreme Court upheld a Minnesota law giving borrowers more
time to pay their mortgages than the contract specified, leaving
lenders no choice but to wait for the money they were owed.
\switchcolumn
在《自由乐土?》(\textit{Sweet Land of Liberty?}) 一 书中,法学家霍尔兹(Henry Mark Holzer) 界定了政府对契约神圣的不断侵蚀过程中的几个“里程碑”。他指出,在内战之前美国的货
币包括金币和银币。为了给内战提供经费,国会授权发行膨胀
的纸币并将它称为“法定货币”,意思是人们必须接受这种纸
币来支付债务,即便是债权人应收回的是金币或银币。1871
年最高法院确认了《法定货币法》,实际上改写了每一份借贷
合同 --- 让人们看到政府原来可以单方面改变巳生效的借贷合
同的条款。随后,在 1938年,尽管宪法中的条款明文禁止政
府颁布任何“减少合同义务的法律”,最高法院仍然确认了明
尼苏达州的一项法律,该法律给了债务人超过合同限定的时间
来赎回抵押物 , 让债权人别无选择只能等待他们应得的钱还
回来。
\switchcolumn*
Around the same time, the Court delivered yet another blow
to freedom of contract. One major concern of any lender is to
make sure that the money repaid will be worth as much as the
money lent, which may not be the case if inflation has reduced
the value of money in the meantime. After the Legal Tender decision, many contracts included a ``gold clause'' specifying the
amount of repayment in terms of gold, which holds its value
better than government-issued dollars. In June 1933 the Roosevelt administration persuaded Congress to expunge the gold
clause from all contracts, effectively transferring billions of dollars from creditors, who had lent money in good faith, to borrowers, who would be able to repay the money in inflationary
dollars. In each of these cases legislators and judges said that in
their opinion the apparent need of one group of contracting
parties should outweigh the obligations those parties had voluntarily assumed. Such decisions have progressively undermined economic progress, which depends on security in one's
property and confidence that contractual obligations will be
carried out.
\switchcolumn
大约在同一时期,法院对契约自由进行了另一次打击。债
权人最关心的一个问题是确信还回来的钱的价值和借出去时一
样,而这可能不能实现,因为这期间通货膨胀可能会降低货币
的价值。在 “法定货币”政策颁布之后 , 很多合同都加入了
“黄金条款” ,特别强调必须用黄金还款,因为黄金的保值能
力超过政府发行的美元。1933年 6 月罗斯福政府说服国会
从所有的合同中删除 “黄金条款”,实际上等于把数以十亿计的美元从债权人那里转到了债务人头上,因为债权人借出的是
以黄金为信用的钱,而债务人现在只需要用不断贬值的美元还
款了。在每一个案例当中,议员和法官们都说合同一方的显见
需要应该超过双方因签订合同而自动承担的义务。这样的政策
极大地破坏了经济增长,因为经济增长依赖于人们的财产安全
和对契约义务必然会兑现的信心。
\switchcolumn*[\section{Special-Interest Law\\特殊利益法}]
In broad measure, the United States is a nation governed by the
rule of law. But one can point to laws---Hayek would call them
legislation, not true laws---that seem to conflict with the rule of
law. There are outright, up-front subsidies and bailouts for specific companies, such as Congress's 1979 guarantee of \$1.5 billion in loans to the Chrysler Corporation. Somewhat less
obviously, there are clauses in many bills along the lines of ``but
this requirement shall not apply to any corporation incorporated in the state of Illinois on August 14, 1967''---that is, one
firm is being exempted from a requirement imposed on its
competitors. There are large incentives in the tax code for particular products such as ethanol, a corn-based gasoline substitute, 65 percent of which is produced by one company, the
generous political contributor Archer-Daniels-Midland. There
are valuable parts of the broadcast frequency spectrum set aside
for minority-owned businesses, and there are government contracts reserved for small businesses.
\switchcolumn
总体上说,美国是一个法治国家。但是人们能够看到一些
法 律 (哈耶克把这叫做立法,不是真正的法律)与法治存在
互相冲突。政府曾经给特定的公司提供全额的预先补贴和紧急
援助,例 如 1979年国会给克莱斯勒公司提供了 15亿美元的贷
款担保。有的则不那么显眼,在很多法案上都有这样的条款,
“但该请求不适用于任何于1967年 8 月 14日在伊利诺伊州注
册的公司。” 也就是说,某家公司避开了施加于其他竞争对手
之上的法律要求。税收法典对某些特定商品有很大的税收优
惠,例如乙醇汽油,一种用玉米炼制的汽油替代品,而 65\%
的乙醇汽油是由一家公司所生产的,这家公司就是慷慨大方的
政治捐款者美国ADM公 司(Archer Daniels Midland)。在广
播频率波段中,商业价值极高的部分频率预留给了少数民族拥
有的公司,还有部分政府合同预留给小公司。
\switchcolumn*
The Fifth Amendment states that if private property is taken
for public use, the owner must be compensated. Yet regulations
take the value of property all the time, and governments have
resisted paying owners for their loss. Property-rights advocates
say, ``If the government wants to preserve the coastline by forbidding me to build a house on my property, or wants to create
a bicycle path through my land, fine---pay me for the value of
my property that is taken.'' But courts have generally allowed
the government to get away with such takings, and they are
often imposed arbitrarily, after an owner has bought a piece of
property with a particular plan in mind. Even if property is
being taken for a public purpose, the owner ought to be compensated; but often the purpose is clearly private, not public---
as when the city of Detroit condemned the homes and
businesses in a Polish-American neighborhood called Poletown
so that General Motors could build a plant there. To add insult
to injury, after people were forced to move from the neighborhood where they had lived all their lives, GM decided not to
build the plant after all.
\switchcolumn
宪法第五修正案规定,私人财产如因公共需要而征用,应
对所有人进行补偿。然而尽管政府管制总是让私人财产贬值,
各级政府却拒绝对所有人的损失进行赔偿。财产权的拥护者们
说 :“如果政府因为保护海岸线而禁止我在自己的地产上建造
房屋,或者想修建一条自行车道穿过我的土地,那好--- 把我产业价值的损失赔偿给我。”但是法院通常允许政府可以不管
这样的损失,他们还经常在土地所有者购买了一片土地有了开
发建造的计划之后专横地实施他们的管制。即便财产是因公共
目的而被征用,所有者也应当被补偿;但是实际上很多这样的
征用其目的常常是私人的,而不是公共的 --- 底特律就曾经征
用波兰裔美国人聚居地“波兰城”的一片建有住房和公司的
土地,为了让通用汽车公司在上面建工厂。后来的事情更是在
伤口上撒盐,当人们被强迫搬出他们世世代代居住的地方之
后 ,通用汽车公司却决定不在上面盖工厂了。
\switchcolumn*
Occupational licensing laws often conflict with the spirit of
the rule of law. Requiring individuals to comply with specific
state regulations in order to offer their services to the public as
lawyers, cabdrivers, cosmetologists, or some 800 other occupations may not run afoul of the rule of law, though it is surely a
violation of economic liberty. But requiring a hairdresser who is
licensed in Tennessee to live in Kentucky for a year before she
can practice her trade there clearly seems to treat citizens differently under the law and is obviously intended to create the
equivalent of a protective tariff for the benefit of hairdressers already residing in Kentucky.
\switchcolumn
各种执业资格法也常常与法治精神相冲突。要求个人在给
公众提供服务的时候,例如律师、出租车司机、美容师以及其
他大约800个职业,必须遵守特殊的政府规定,也许并不违背
法治,尽管这的确侵犯了经济自由。但是要求一名持田纳西州
执照的美发师在肯塔基州居住一年以上才能在那里开业,显然
是对公民在法律上的不平等对待,同时很明显,目的是在于给
已经居住在肯塔基州的美发师制造一种类似于贸易壁垒的
保护。
\switchcolumn*
Perhaps the most serious way that current American law violates the rule of law is in the delegation of legislative and judicial power to unelected and invisible administrators. In 1948
Winston Churchill complained, ``I am told that 300 officials
have the power to make new regulations, apart altogether from
Parliament, carrying with them the penalty of imprisonment
for crimes hitherto unknown to the law.'' We should be so lucky
today as to have only 300 officials with the power to make laws.
Until Franklin Roosevelt's New Deal, it was understood that
the U.S. Constitution gave the exclusive power of lawmaking to
Congress. In conformity with the rule of law, it gave the president the power to execute the laws and the judiciary the power
to interpret and enforce them. In the 1930s, however, Congress
started passing broad laws and leaving the details up to administrative agencies. Such agencies---the Agriculture Department, the Federal Trade Commission, the Food and Drug
Administration, the Environmental Protection Agency, and
countless more---now churn out rules and regulations that
clearly have the force of law but were never passed by the constitutional lawmaking authority. Sometimes Congress didn't
know how to make its broad promises real, sometimes it didn't
want to vote on the actual trade-offs involved in giving some
people what they wanted at the expense of other people, sometimes it just couldn't be bothered with the details. The result is
tens of thousands of bureaucrats churning out laws---60,000
pages of them in a typical year---for which Congress takes no
responsibility.
\switchcolumn
把立法和司法的权力委托给非经选举产生的官员和秘密情
报机构,这也许是现行美国法律对法治最严重的侵犯。1948
年 ,温斯顿$\cdot$ 丘吉尔抱怨说:“他们告诉我,有 300名官员有
权不通过议会就制定新法令,按照这些法律,他们可以根据一
些迄今为止法律上没有的罪名指控人并将其监禁。”假如今天
美国只有300名官员有权制造法律,我们真应该感到庆幸。

罗斯福新政以前,美国宪法将唯一的、排他的立法权力賦
予了国会,并根据法治原则,将执行法律的权力赋予总统,将
解释和实施法律的权力陚予司法机构。然而,到 1930年代,
国会开始通过原则性的法律,而把法律细节留给行政机构来裁量。这些行政机构包括农业部、联邦贸易委员会、食品与药品
管理局、环境保护局,以及其他一些无法一一列举的机构。迄
今为止,这些机构炮制了一大批具有法律效力的规定和规章,
而这些规定和规章从未经过宪法规定的立法机关审核通过。有
时是因为国会不知道怎样实现他们的原则性承诺,有时是因为
它不想就如何花一部分人的钱满足另一部分人要求的现实权衡
进行表决,有时候则只是因为不愿意纠缠在细节当中。结果,
实际上是数以万计的官僚在制造法律。每年都会从他们那里产
生出6 万页的此类法律,而国会则对此不负责任。
\switchcolumn*
Compounding the insult to the rule of law is that these agencies then interpret and enforce their own rules, deciding how
they will apply in each individual case. They are legislator, prosecutor, judge, jury, and executioner, all in one---as clear a violation of the rule of law as one could imagine. A particular
problem is the federalization and criminalization of environmental law over the past three decades. In its zeal to protect the
environment, the federal government has created a web of regulations so dense that compliance with the law is, essentially,
unachievable. Prosecutors and courts have stripped environmental criminal suspects of such traditional legal defenses as
good faith, fair warning, and double jeopardy, while effectively
requiring potential suspects to incriminate themselves. It is
when pursuing a goal as public-spirited as environmental protection that we must remind ourselves to be most careful in following rules and abiding by constitutional protections, lest the
worth of the goal lead us to erode the principles that allow us to
achieve all our goals.
\switchcolumn
对法治更严重的羞辱是,这些政府机构自己解释和实施这
些规定,自己决定这些规定如何适用于每个个人的案件。他们
集立法者、执法者、法官、陪审团的角色于一身,对法治赤裸.
裸的侵犯超出了你的想像。其中一个问题是,过去三十年当中
环保法的联邦化和刑罚化。在保护环境的热情驱使下,联邦政
府用规章法令编织了一张庞大的法网,这张网是如此严密,以
至于人们基本上无法达到这些法律的要求。执法人员和法院并
不给环保违法嫌疑人以传统的法律保护,如善意原则、合理警
告 、双重危境条款\footnote{双重危境条款(double  jeopardy),是植根于西方古老法律传统的一个原则,意思是一个人不可以因为同一件犯罪被起诉两次。美国宪法第五修正案指出,任何人不得因同一罪行而两次遭受生命或身体的危害,也就是说不能因同一犯罪而受到两次审判。}等,而是实际上要求嫌疑人必须自己证明自己无罪。当我们追求一个公共目标如环境保护的时候,我们才恰恰应当提醒自己,必须保持最髙的警惕,遵循法治,坚持宪法保护,以避免在某个单一目标的引导下,去侵蚀那些可以
让我们达到所有目标的原则。

\switchcolumn*[\section{Constitutional Limits on Government\\特殊利益法}]

Perhaps the most remarkable American contribution to protecting individual rights and the rule of law was our written Constitution. The purpose of government was made clear in the Declaration of Independence: ``to secure these rights, governments are instituted among men.'' Having concluded that government was necessary, the Americans sought to devise a constitution that would limit the government to just that
purpose.
\switchcolumn
\footnote{这里有缺失,下面四段是机器翻译}也许美国为保护个人权利和法治最显着的贡献是我们的成文宪法。政府的目的被明确定义在独立宣言中:“为了保障这些权利,人们创立了政府。”美国人得出结论认为政府是必要的,并试图制定一部宪法试图限制政府从而达到这一唯一目的。
\switchcolumn*
The power to protect rights is naturally held by each individual, and it is \textit{delegated} to government in the Constitution. To make it clear that the Constitution was not a general grant of power to government, the specific powers granted to the federal government are \textit{enumerated} in Article I, Section 8. Because they are delegated and enumerated, the powers of the federal government are \textit{limited}. A government of delegated, enumerated, and limited powers---that is the great American contribution to the development of liberty under law.
\switchcolumn
每个人自然持有保障权利的权力,并\textbf{委托}给政府的宪法。要清楚,宪法不是将权力赋予政府,而是联邦政府被授予特定权力((\textbf{列举}在第一章,第8节),因为权力是被委托和枚举的,所以联邦政府的权力是\textbf{有限}的---那就是美国对法治下自由发展的伟大贡献。
\switchcolumn*
The legal scholar Roger Pilon lays out the meaning of the Constitution in his 1995 essay ``Restoring Constitutional Government'':
\switchcolumn
法律学者罗杰·皮隆在他1995年的文章《恢复宪政》中勾画出宪法的含义:
\switchcolumn*
\begin{quotation}
Congress may act in any given area or on any given subject,
therefore, only if it has authority under the Constitution to do so.
If not, that area or subject must be addressed by state, local, or
private action.

The doctrine of enumerated powers, as just stated, was meant by the Framers to be the centerpiece of the Constitution. As such, it serves two basic functions. First, it explains and justifies federal power: flowing from the people to the government, power is legitimate insofar as it has been thus delegated. But second, the very doctrine that justifies federal power serves also to limit it, for the government has only those powers that the people have given it. Indeed, it was the enumeration of powers, not the enumeration of rights in the Bill of Rights, that was meant by the Framers to be the principal limitation on government power. For the Framers could hardly have enumerated all of our rights, whereas they could enumerate federal powers. By implication, where there is no power, there is a right belonging to the states or the people.
\end{quotation}

\switchcolumn
\begin{quotation}
只要宪法授权,国会可以在任何区域或任何主题制定法律。如果没有,该区域或主题必须由国家,地方或私人的诉讼来解决。

正如刚才所说,列举的权力是制宪者制定宪法的核心。因此,它有两个基本功能。首先,它为联邦权力解释和证明:权力从人民流动到政府是合法的,只要它被这样委托。其次,列举的权力也应限制联邦政府,因为这些权力只是人们委托给政府的。事实上,列举权力而不是人权的列举,才是建国先贤所说的对政府权力最重要的限制。因为建国先贤们几乎不可能列举我们的所有权利,但是他们可以列举联邦的权力。这意味着:哪里没有政府权力,哪里的权利就属于州或人民。
\end{quotation}
\switchcolumn*
Today, when a new federal law is proposed, many libertarian-minded people on both the right and the left look to the Bill of
Rights to see whether the law will violate any constitutional
rights. But we should look first to the enumerated powers to see
if the federal government has been granted the power to undertake the proposed action. Only if it has such a power should we
move on to ask whether its proposed action would violate any
protected right.
\switchcolumn
今天,每当一项新联邦法律提出的时候,许多具有古典自
由主义理念的人 --- 无论属于左派还是右派,都 会 翻 出 《权
利法案》看看这项法律会不会侵犯宪法权利。但是更应该首
先 翻 出 “列举的权力”,看看联邦政府是不是被授予了权力去
提出那个法案。只有当它的确有这项权力的时候,才应该接着
追问它提出的法案会不会侵犯任何受到保护的权利。
\switchcolumn*
Much---perhaps most---of what the federal government
does today is not authorized in Article I, Section 8. That is to
say, the federal government has assumed many powers that
were not delegated by the people and not enumerated in the
Constitution. It would be hard to find in the Constitution any
authorization for economic planning, aid to education, a government-run retirement program, farm subsidies, art subsidies,
corporate welfare, energy production, public housing, or most
of the rest of the panoply of federal undertakings.
\switchcolumn
联邦政府今天所做的许多(也许是绝大多数)事情在宪
法 第 1条第8 款都没有授权。也就是说,联邦政府已经攫取了
很多未得到人民授权、没有在宪法中列举的权力。在美国宪法
中很难找到任何对经济计划、教育补贴、政府经营的退休金计
划、农业补贴、艺术补贴、强制的公司福利、能源生产、公共
住房或者其他绝大多数花里胡哨的联邦国营事业的授权。
\switchcolumn*
For much of our history the limits on federal powers were taken for granted. As early as 1794, James Madison, the princi-
pal author of the Constitution, rose in the House of Representatives to oppose a bill because he could not ``undertake to lay his
finger on that article of the Federal Constitution which granted
a right to Congress of expending, on objects of benevolence, the
money of their constituents.'' As late as 1887, President Grover
Cleveland vetoed a bill to provide seeds for drought-stricken
farmers because he could ``find no warrant for such an appropriation in the Constitution.'' Things had changed by 1935, when
Franklin Roosevelt wrote to the chairman of the House Ways
and Means Committee, ``I hope your committee will not permit
doubts as to constitutionality, however reasonable, to block the
suggested legislation.'' Thirty-three years later, Rexford Tugwell, one of Roosevelt's principal advisers, admitted, ``To the extent that these [New Deal policies] developed, they were
tortured interpretations of a document intended to prevent
them.''
\switchcolumn
美国历史中,大部分时候对联邦权力进行限制被认为是理
所当然的。早 在 1794年,美国宪法的主要作者之一麦迪逊就
在众议院反对一项拨款援助法国难民的法案,因为他不能
“在宪法中找到哪个条款授予国会以慈善的名义花费国民财富
的权利。” 1887年 ,克利夫兰总统否决了一项给受到干旱打击
的农民提供种子的法案,因 为 他 “无法在宪法当中找到对此
类拨款的授权”。情况到1935年发生了改变,当年罗斯福给美
国众议院筹款委员会主席写信说: “我希望你的委员会不要因
为怀疑已提交的议案是否符合宪法而阻碍议案的通过,无论理
由看上去多么充分。” 33年之后,罗斯福的首席智囊塔格威尔(Rexford Tugwell)承认: “就这些新政政策的结果来看,它们是对宪法的曲解,而宪法本意就是阻止这一切的。”
\switchcolumn*
Today, it seems, we do not even ask where Congress finds the
constitutional authority to pass the laws it does. It's hard to remember when a member of Congress rose to ask, ``Where in the
Constitution do we find this power?'' Should an outside critic do
so, he will likely be referred to the Constitution's preamble:
\switchcolumn
而今天,我们甚至连问都不会问国会是在哪里找到的宪法
授权通过那些法律的。这样的情形已经是一个很久远的记忆
了:一名国会议员站起来问:“从宪法的什么地方能找到这个
权力的呢?”而外界的批评者问同样的问题他很可能会被告知
宪法序言:
\switchcolumn*
\begin{quote}
We, the people of the United States, in order to form a more perfect Union, establish justice, insure domestic tranquility, provide
for the common defence, promote the general welfare, and secure
the blessings of liberty to ourselves and our posterity, do ordain
and establish this Constitution of the United States.
\end{quote}
\switchcolumn
\begin{quote}
我们美利坚合众国的人民,为了组织一个更完善的联
邦,树立 正 义 , 保障国内的安宁,建立共同的国防,增进
全民福利和确保我们自己及我们的后代能安享自由带来的
幸福,乃为美利坚合众国制定和确立这一部宪法。
\end{quote}
\switchcolumn*
The mention of ``general welfare,'' it will be said, authorizes virtually anything Congress wants to do. But that is a misreading
of the general welfare clause. Of course, as Locke and Hume argued, we create government to enhance our welfare in the
broadest sense. But what will enhance our welfare is the opportunity to live in a civil society, where our life, liberty, and property are protected and we are left free to pursue happiness in
our own way. Our welfare is decidedly not enhanced by a limitless government, arrogating to itself the power to decide that
anything from a Chrysler bailout to a V-chip to a job-training
program would be good for us. A narrower criticism of this expansive reading of the general welfare clause is that by ``general
welfare'' the Framers were making clear that the government
must act in the interest of all, not on behalf of any particular
person or group---but virtually everything Congress does today
involves taking money from some people to give it to others.
\switchcolumn
其 中 提 到 “全民福利”,有人说这意味着它实际上已向国
会授权做任何想做的事情。但 是 这 是 对 “全民福利”这个词
的误解。当然,如洛克和休谟所论证的那样,在最宽泛的意义
上来说,我们创建政府就是为了增进我们的福利。但是能够增
进我们福利的是生活在一个市民社会的机会,在这个市民社会
里,我们的生命、自由和财产得到保护,我们能够自由地用自
己的方式来追求幸福。我们的福利决不是由一个无限政府所增
进的,这个无限政府冒称有权颁布任何政策,包括给克莱斯勒
公司紧急援助、电视节目过滤芯片(V-chip)\footnote{V-chip,violence chip,暴力、色情节目过滤芯片,一种电子锁定装置,可帮助父母防止青少年从电视中看到暴力、色情节目。}、工作培训计
划,还说是为了我们好。对 “全民福利”概念的这种任意发
挥 ,较温和的批评是:建国先贤们提出“全民福利”是为了
明确政府的行为应当是为了所有人的利益,而不是代表任何一
个特定的人或集团,但是实际上今天国会所做的每一件事都是
把钱从一些人那里拿走然后交给另一些人。
\switchcolumn*
The value of a written constitution is that is lays out precisely
what the government's powers are and, at least by omission, indicates what they are not. It sets up orderly procedures for the
operation of government and, more important, systems for
checking any attempt to exceed constitutional authority. But
the real check on any government's power is the eternal vigilance of the people. The U.S. Constitution was a brilliant design
not just because its Framers were geniuses but because the
American people in the founding era were well aware of the
dangers of tyranny and steeped in the rights theory of Locke
and the experience with British constitutionalism. A friend of
mine told me around 1990 that he had been engaged by people
in the newly liberated Bulgaria to help them write a constitution that would protect liberty. ``I'm sure you'll write a great
constitution,'' I told him, ``even better than the U.S. Constitution, but it's not just a matter of writing a good document and
handing it to the popular assembly. It took 500 years to write
the U.S. Constitution---from Magna Carta in 1215 to the Constitutional Convention in 1787.'' The question is whether the
people of Bulgaria appreciate the importance to liberty and
prosperity of guaranteeing individual rights through a government of delegated, enumerated, and limited powers. Here in
the United States, the question is whether Americans \textit{still} appreciate the Constitution and the thinking that underlies it.
\switchcolumn
成文宪法的价值是它准确地提出了政府权力是什么,并且
至少通过不列举而指出政府权力不是什么。它为政府的运作建
立了有规则的程序,更重要的是它建立了一种体制,对任何超
越宪法权威的企图进行制约。但是真正制约政府权力的是人民
永不懈怠的警惕。美国宪法的设计非常伟大,不仅仅是因为建
国先贤们的天才,而是因为美国人民在建国时期对独裁危险的
警惕,再加之对洛克的权利理论和英国宪政制度经验的接受。
我的一位朋友大约在1990年时告诉我,新获得自由的保加利
亚人找到他,让他帮助写一部宪法以保护自由。“我相信你可
以写一部很好的宪法”,我告诉他,“甚至比美国宪法还要好,
但是问题不在于仅仅写一份好的文件并交给国民议会。美国花
了五百年的时间才写成这部宪法 --- 从 1215年 的 《英国大宪
章》到 1787年的立宪会议。”问题在于保加利亚人民是否认
识到自由和通过一个权力经过授权、列举和限制的政府来保护
个人权利而带来繁荣的重要性。而在美国,问题则在于美国人
民是否\textbf{仍然}欣赏宪法及其背后的思想?
\switchcolumn*
How could the U.S. Constitution be improved? Hayek warns
us to be cautious in our attempts to improve long-standing institutions, and any of us would be well advised to approach with
humility the task of improving upon the work of Washington,
Adams, Madison, Hamilton, Mason, Randolph, Franklin, and
their colleagues. But with 200 years of experience, we can perhaps suggest some minor improvements. The general framework of delegated, enumerated, and thus limited powers is
obviously in keeping with libertarian values. A libertarian would enthusiastically endorse the separation of powers and
would have no obvious criticism of the structure of a legislative
body with two houses apportioned differently, a president with
a veto, a reasonably difficult amendment process, and so on.
\switchcolumn
那么美国宪法怎样才能得到改进?哈耶克警告我们,对改
变长期有效运行的制度的企图要保持警惕,而我们当中的任何
人都应该听从这个建议,对由华盛顿、亚当斯、麦迪逊、汉密
尔顿、兰道夫\footnote{兰道夫(Edmund  Jennings Randolph) , 在起草和批准美国宪法的过程中扮演了重要角色。}、梅森\footnote{乔治$\cdot$梅森(Geodge Mason,1725$\sim$1792), 美国宪法的缔造者之一,参与了宪法的起草,但因为宪法中没有保障人民权利的条款而拒绝在宪法上签字。后来由于他的坚持和积极努力,被称为权利法案的美国宪法修正案头十条才得到通过,并在以后的年代里成为保障美国人民个人权利至高无上的法律。}、富兰克林以及其他先贤所完成的工作一定要保持谦虚,企图改变这些先贤成果的时候一定要足够谨慎。但是在二百多年的经验之上,我们也许可以建议进行一些微小的改进。权力须经授权、列举和限制的总体框架显然已成为古典自由主义的基本价值。古典自由主义者应当会对权力
分立的制度表示热情的支持:基本赞同将立法机构分为参众两院,并且有不同的份额安排,拥有否决权的总统,修改程序保持合理难度,等等。
\switchcolumn*
Someone has suggested that on top of the safeguards against
excessive government already in the Constitution---the structure of enumerated and limited powers, the Bill of Rights, the
Ninth Amendment clarifying that all other rights are retained
by the people, the Tenth Amendment reserving unenumerated
powers to the states or the people---one more layer be added:
an amendment reading, ``And we mean it.'' In that spirit, were
one revising the U.S. Constitution either for Americans or for
some other country, one might add a clause clarifying that the
powers granted in Article I, Section 8, are indeed all the powers of the federal government. And in case that, too, was insufficient, one might expand the Bill of Rights to guarantee not
just separation of church and state but separation of family and
state, school and state, race and state, art and state, even economy and state. One might also want to amend the Constitution to
\switchcolumn
对政府过度权力的防范存在于宪法当中:列举的、有限的
权力;权利法案;第九修正案确认了除列举的权利之外其他所
有权利均为人民所保有;第十修正案将未列举的权力保留给州
或人民。除此之外,曾经有人建议在上述条款之上再加上一层
保护:这项修改的内容是“这些条款不再修改”。在这个精神
下,如果有人要修订美国宪法,无论是为了美国人还是为了其
他国家,都应加入一个条款,确认在宪法第1条第8 款所授予
的权力是联邦政府的全部权力。而且为防万一,这还不够,他
还 可 以 将 《权利法案》扩展,不仅仅确保政教分离,而且要
确保政治与家庭分离、政治与教育分离、政治与种族分离、政
治与艺术分离,甚至政治与经济分离。他也许还应对宪法作下
面的修改:
\switchcolumn*
\begin{itemize}
	\item require a balanced budget, as Thomas Jefferson recommended and as almost all state constitutions do;
	\item forbid Congress to delegate its lawmaking authority to administrative agencies;
	\item revive the colonial principle of rotation in office by limiting the terms of members of Congress as well as the president;
	and
	\item give the president a line-item veto so he could veto individual parts of a bill, or clarify that when Article I refers to a ``bill,'' it means a single piece of legislation dealing with a single subject, not a massive amalgamation of subjects and appropriations.
\end{itemize}
\switchcolumn
\begin{itemize}
	\item 联邦政府的财政必须收支平衡。杰斐逊曾经提到过这一条,并且现在几乎所有的州宪法都是如此规定;
	\item 禁止国会将立法权委托给行政部门;
	\item 通过限制国会议员和总统的任期来恢复殖民地时期的职务轮换原则;
	\item 授予总统单项条款否决权以使他能够只能否决法案的部分条款,或者确认恢复宪法第一章“法案” 的定义,也就是一条立法针对一个问题,而不是一大堆的目标和拨款混在一起。
\end{itemize}
\switchcolumn*
The Framers of the Constitution and the Bill of Rights wrote
their limits on government and their guarantees of specific
rights based on their experience with the depredations of liberty
by the British government. With 200 more years' experience
with the ways governments seek to break the bounds we place on them, we see new rights to enumerate and new limits to
place on power.
\switchcolumn
建国先贤们在《美国宪法》和 《权利法案》 中写下限制
政府和保护基本权利的条款,是因为他们经历过英国政府对殖
民地人民自由的剥夺。而在此后二百多年当中,在经历了政府
多次突破这些边界的企图之后,我们看到了列举新的权利和对
政府权力设置新限制的必要。
\switchcolumn*
For now, however, enforcing the Constitution as it stands
would be a big step in the libertarian direction, that is, in the
direction of protecting every American's liberty and keeping
the coercive power of the state out of civil society.
\switchcolumn
无论如何,就现在来说,只要严格按照宪法的精神来实施
宪法,就会是朝向古典自由主义方向前进的一大步,也就是说
是朝向捍卫每一个美国人的自由和使得国家强制力远离我们市
民社会的一大步。

\end{paracol}

