\chapter{THE MARKET PROCESS\\市场经济}
\begin{paracol}{2}
\hbadness5000

When I go to the supermarket, I encounter a veritable
cornucopia of food---from milk and bread to Wolfgang Puck's Spago Pizza and fresh kiwis from New Zealand.
The average supermarket today has 30,000 items, double the
number just ten years ago. Like most shoppers, I take this abundance for granted. I stand in the middle of this culinary festival
and say something like, ``I can't believe this crummy store doesn't have Diet Caffeine-free Cherry Coke in 12-ounce cans!''
\switchcolumn
走进这家超市,我就像进入了神话中的丰饶角\footnote{丰饶角(cornucopia),希腊神话传说中哺乳了宙斯的山羊的角,脱落并装满了水果、花朵和谷物后来装满了其主人希望得到的任何东西。}。这里堆
满了各种食物---从牛奶、面包到世界知名厨师沃夫甘$\cdot$帕克做的Spago比萨饼,还有来自新西兰的新鲜几维鸟肉,应有尽
有。今天,这家普通的超市供应三万种商品,比十年前增加了
一倍。像绝大多数购物者一样,我认为这是理所当然的。站在
这个食品的盛典之中,我对自己说:“我不相信这个可爱的商
店没有12盎司罐装的无咖啡因櫻桃口味的健怡可乐!”
\switchcolumn*
But how does this marvelous feat happen? How is it that I,
who couldn't find a farm with a map, can go to a store at any
time of day or night and expect to find all the food I want, in
convenient packages and ready for purchase, with extra quantities of turkey in November and lemonade in June? Who plans
this complex undertaking?
\switchcolumn
但是这些神迹是如何出现的呢?我这样一个在地图上找不
到一片农场的家伙,怎么能在任何时间,无论白天还是晚上走
进一家商店就可以找到所有想要的食物,装在方便的包装里随
时买走,而且在11月得到特别供应的火鸡,6 月得到特别供
应的柠檬呢?是谁计划了这么复杂的工程呢?
\switchcolumn*
The secret, of course, is precisely that no one plans it---no
one \textit{could} plan it. The modern supermarket is a commonplace
but ultimately astounding example of the infinitely complex
spontaneous order known as the free market.
\switchcolumn
其中的秘密当然是一恰恰没人计划,也没人\textbf{能够}计划。
现代化的超市是一个普通的地方,但绝对是无限复杂的自发秩
序让人惊叹的例证,这种自发秩序被称为自由市场经济。
\switchcolumn*
The market arises from the fact that humans can accomplish
more in cooperation with others than we can individually, and
the fact that we can recognize this. If we were a species for
whom cooperation was not more productive than isolated
work, or if we were unable to discern the benefits of cooperation, then we would not only remain isolated and atomistic, but, as Ludwig von Mises explains, ``Each man would have been
forced to view all other men as his enemies; his craving for the
satisfaction of his own appetites would have brought him into
an implacable conflict with all his neighbors.'' Without the possibility of mutual benefit from cooperation and the division of
labor, neither feelings of sympathy and friendship nor the market order itself could arise. Those who say that humans ``are
made for cooperation, not competition'' fail to recognize that
the market is cooperation. (Indeed, it is people competing to
cooperate better!)
\switchcolumn
市场的产生来自一个事实:人们与他人进行合作比独自工
作能够完成更多的任务。并且,我们能够意识到这种合作的优
势。如果人类是一个这样的物种:互相合作后,生产出来的东西还没有独自工作多,或者人们不能分辨出合作的优势,那么
人类就不仅会保持互相孤立和原子化的状态,而且还会像米瑟
斯所说的那样:“每个人将被迫把别人看作敌人;他满足自己
欲望的追求将把他带入和所有邻居间难解的冲突状态。”如果
没有合作和分工带来的相互利益,同情和友谊不会产生,市场
秩序本身也不会出现。那 些 说 “人天生适合合作而不是竞争”
的人没有认识到市 场就是 合作。事实上,正是竞争让人们合作
得更好!
\switchcolumn*
The economist Paul Heyne compares planning with spontaneous order this way: There are three major airports in the San
Francisco Bay area. Every day thousands of airplanes take off
from those airports, each one bound for a different destination.
Getting them all in the air and back on the ground on time and
without colliding with each other is an incredibly complex task,
and the air traffic control system is a marvel of sophisticated organization. But also every day in the Bay area people make
thousands of times as many trips in automobiles, with far more
individuated points of origin, destinations, and ``flight plans.''
\textit{That} system, the coordination of millions of automobile trips, is
far too complex for any traffic control system to manage, so we
have to let it operate spontaneously within a few specific rules:
drive on the right, stop at lights, yield when making a left turn.
There are accidents, to be sure, and traffic congestion---much
of which could be alleviated if the roads themselves were built
and operated according to market principles---but the point is
that it would be simply impossible to plan and consciously coordinate all those automobile trips. Contrary to our initial impression, then, it is precisely the less complex systems that can be
planned and the \textit{more} complex systems that must develop spontaneously.
\switchcolumn
经济学家保罗$\cdot$海恩\footnote{保罗$\cdot$海恩(Paul T. Heyne, 1931$\sim$2000),美国经济学家。著有《经济学的思维方式》一书。}这样对比计划和自发秩序:旧金山
湾区有三个大机场。每天数千架飞机从这些机场起飞,每架飞机都飞向不同的目的地。让它们全部准时起飞并且返回地面,
不发生互相碰撞是一个无比复杂的任务。空中交通管理系统是
一个极其复杂的组织系统。但同时,每天旧金山湾区人们驾车
出行的次数超过飞机起降次数几千倍,还有远超过飞行任务的
个人的出发地、目的地和“飞行计划”。这个包含数以百万计
汽车出行的系统的复杂程度远远超过了任何交通管理系统的管
理能力,因此,我们不得不让它自发运转,只有少数几条特定
的规则:汽车靠右行驶、遇红灯停、左转让直行等。当然会有
交通事故和拥堵 --- 而且这些事故和拥堵大部分可以通过按照
市场原则建设和使用道路来减轻 --- 但是关键是计划和有意识
地协调所有的机动车行驶是根本不可能的。因此,与我们的最
初印象相反,恰恰是复杂程度较低的系统是可以计划的,而\textbf{越
复杂}的系统越必须自发运转。
\switchcolumn*
Many people accept that markets are necessary but still feel
that there is something vaguely immoral about them. They fear
that markets lead to inequality, or they dislike the self-interest
reflected in markets. Markets are often called ``brutal'' or ``dog-
eat-dog.'' But as this chapter will demonstrate, markets are not
just essential to economic progress, they are more consensual
and lead to more virtue and equality than government coercion. Markets are based on consent. No business sends an invoice for
a product you haven't ordered, like an income tax form. No
business can force you to trade. Businesses try to find out what
you want and offer it to you. People who are trying to make
money by selling groceries, or cars, or computers, or machines
that make cars and computers need to know what consumers
want and how much they would be willing to pay. Where do
businesses get the information? It's not in a massive book. In a
market economy, it isn't embodied in orders from a planning
agency (though of course, theoretically, in socialist economies
producers do act on orders from above).
\switchcolumn
很多人接受市场是必须的这个观点,但是仍然模糊地感觉到这里面有些不道德的东西。他们担心市场导致不公平,或者
不喜欢人们在市场中的自私自利。市场常常被称作“野蛮的”
或 者 “损人利己” (dog-eat-dog)。但是就像我要在本章当
中阐述的那样,市场不仅对经济增长至关重要,而且它能够比
政府强制带来更多的共识、美德、和平等。

\switchcolumn*[\section{Information and Coordination\\信息与协作}]
Markets are based on consent. No business sends an invoice for
a product you haven't ordered, like an income tax form. No
business can force you to trade. Businesses try to find out what
you want and offer it to you. People who are trying to make
money by selling groceries, or cars, or computers, or machines
that make cars and computers need to know what consumers
want and how much they would be willing to pay. Where do
businesses get the information? It's not in a massive book. In a
market economy, it isn't embodied in orders from a planning
agency (though of course, theoretically, in socialist economies
producers do act on orders from above).
\switchcolumn
市场是建立在共识基础之上的。如果你没有预订一个商
品,没有企业会像政府给你寄税单一样单方面给你寄账单。没
有哪个企业能强迫你交易。企业会试着发现你的需求,并把你
需要的东西提供给你。如果你想通过销售日用百货、汽车、电
脑或机器来赚钱,你就必须在制造汽车或电脑之前知道消费者
到底需要什么,以及他们愿意为此付多少钱。那么企业从哪里
得到这些信息呢?这些信息不在一本巨大的书里面。在市场经
济中,这些信息并不包含在中央计划机构的命令书里 --- 当
然,至少理论上,社会主义经济的生产者确实是按照上级指令
来生产的。

\switchcolumn*[\subsection{Prices\\价格}]
This vitally important information about other people's wants
is embodied in prices. Prices don't just tell us how much something costs at the store. The price system pulls together all the
information available in the economy about what each person
wants, how much he values it, and how it can best be produced.
Prices make that information usable to both producer and consumer. Each price contains within it information about consumer demands and costs of production, ranging from the
amount of labor needed to produce the item to the cost of labor
to the bad weather on the other side of the world that is raising
the price of the raw materials needed to produce the good. Instead of having to know all the details, one is presented with a
simple number: the price.
\switchcolumn
关于人们需求的最重要信息包含在价格里面。价格不仅告
诉我们某个东西在商店里面卖多少钱。价格体系将经济体中所
能提供的所有关于每个人的需求、每个人对商品如何佔价、商
品应该如何生产这些信息集中在一起。价格使得这些信息能够
被生产者和消费者所使用。每一个价格都包含了关于消费者需
求和生产成本的信息。从制造这件东西所需要的劳动力数量,
到劳动力成本,到地球另一面坏天气造成的原材料价格上涨。而一个人并不需要知道所有的细节,他只需要知道一个简单的
数字:价格。
\switchcolumn*
Market prices tell producers when something can't be produced at a cost less than what consumers will pay for it. The
real cost of anything is not the price in dollars; it is whatever
could have been done instead with the resources used. Your cost
of reading this book is whatever you would have done with your
time otherwise: gone to a movie, slept late, read a different
book, cleaned the house. The cost of a \$15 CD is whatever you
would have done with that \$15 otherwise. Every use of time or
other resources to produce one good incurs a cost, which economists call the\textit{ opportunity cost}. That resource can't be used to produce anything else.
\switchcolumn
市场价格会告诉生产者,什么时候某个东西的生产成本不
能少于消费者愿意支付的价格。任何东西的真实成本并不是用
货币所标注的价格,而是任何本来可以用这些资源来做的事
情。你读这本书的成本是本来可以用这段时间来做的其他任何
事情:看电影,睡懒觉,读另一本书,打扫房间等。一 张15美元的CD的成本是你本来可以用这15美元做的任何事情。每一次使用时间或其他资源来生产产品都是成本,经济学家称
之为\textbf{机会成本}。这部分资源就不能被用来生产任何其他东西。
\switchcolumn*
The information that prices deliver allows people to work together to produce more. The point of an economy is not just to
produce more things; it's to produce more things that people
want. Prices tell all of us what other people want. When prices
for certain goods rise, we tend to reduce our consumption of
those goods. Some of us calculate whether we could make
money by starting to produce those goods. When prices (that
is, wages or salaries) for some kinds of labor rise, we consider
whether we ought to move into that field. Young people think
about training for jobs that are starting to pay more, and they
move away from training that prepares them for jobs for which
wages are declining.
\switchcolumn
价格所传递的信息让人们一起工作生产更多的东西。一个
经济体的重点不仅仅是制造更多的东西,而且是制造更多人们
需要的 东西。当某种商品的价格上升,我们就会降低对这些商
品的消费。另一些人则会计算是否能够开始通过生产这些商品
来赚钱。当某些种类的劳动力价格(也就是工资或收入)上
升的时候,我们就会考虑是否应该进入这些领域。年轻人会考
虑参加与那些收入开始上升的职业有关的培训,而会从那些与
薪水在下降的职业有关的培训中退出来。
\switchcolumn*
In any economy more complex than a village---maybe even
more complex than a nuclear family---it's difficult to know just
what everyone wants, what everyone can do, and what everyone is willing to do at what price. In the family, we love one another, and we have an intimate knowledge of each person's
abilities, needs, and preferences, so we don't need prices to determine what each person will contribute and receive. Beyond
the family, it is good that we act benevolently toward other people. But no matter how much preachers and teachers exhort us
to love one another, we will never love everyone in society as
much, or know their needs as well, as the people in our family.
The price system reflects the choices of millions of producers,
consumers, and resource owners who may never meet and coordinates their efforts. Although we can never feel affection for---
or even meet---everyone in the economy, market prices help us
to work together to produce more of what everyone wants.
\switchcolumn
在任何一个比村庄更复杂 --- 甚至也许仅仅比一个核心家
庭复杂一些----的经济体里,都很难知道:每个人需要什么、
每个人能够做什么以及每个人愿意做什么、以什么价格来做。
在家庭里面,我们互相爱对方,对每个人的能力、需要和爱好
有直接的了解,因此我们不需要价格来确定每个人贡献多少和
得到多少东西。在家庭之外,我们对别人乐善好施是很好的。
但是不论牧师和老师们多么苦口婆心地教导我们去爱别人,我
们都不会像爱自己家人一样爱社会上的每一个人,也不会像了
解自己的家人一样了解他们的需要。价格系统反映了数以百万计的生产者、消费者和资源所有者的选择,而这些人也许从来
没有见过面,也从来没有一起合作过。尽管我们永远不会感受
到经济体中每一个人的感觉,甚至从来也不会见面,但是市场
价格能够让我们一起为制造人们所需要的东西而工作。
\switchcolumn*
Unlike government, which at best takes the will of the majority (and more often acts according to pressure from a small
group) and imposes it on everyone, markets use prices to let
buyers and sellers freely decide what they want to do with their
money. Nobody can afford everything, and some people can afford much more than others, but each person is free to spend his
money as he chooses. And if 51 percent of the people like black
cars, or Barry Manilow, dissenters are free to buy something
else; they don't have to organize a political movement to get
the whole country to switch to blue cars or Willie Nelson.
\switchcolumn
市场与政府的不同之处在于,在最好的情况下,政府也不
过是把大多数人的意愿强加给所有人,而更经常的情况则是,
在少数人的利益集团的压力下,政府把少数人的意愿强加给所
有人。而市场则是用价格来让买卖双方自由决定如何使用他们
的钱。没有人能够提供所有的产品,而有的人能够比其他人提
供多得多的产品,但是每个人都是自由地选择如何花自己的
钱。如果51\%的人喜欢黑色汽车,或者喜欢巴瑞$\cdot$ 曼尼洛\footnote{巴瑞$\cdot$曼尼诺(Bairy Manilow),美国著名流行歌星。},
持不同意见者可以自由选择购买其他东西。他们不需要组织一
次政治运动来让整个国家都转而接受蓝色汽车或者威利$\cdot$尼
尔森\footnote{威利$\cdot$尼尔森(Willie Nelson),美国著名流行歌星。}。

\switchcolumn*[\subsection{Competition\\竞争}]
All this talk about the marvel of coordination shouldn't leave
the impression that the market process isn't competitive. Our
individual plans are always in conflict with those of other people; we plan to sell our services or our goods to customers, but
other people are also hoping to sell to the same customers. It is
precisely through competition that we find out how things can
be produced at the least cost, by discovering who will sell us
raw materials or labor services for the lowest price.
\switchcolumn
上面对合作好处的讨论不应该给大家留下一个印象:市场
过程是没有竞争的。个人的计划经常和其他人的计划相冲突;
我们想要把服务和产品卖给客户 , 但是其他人也想把他们的服
务和产品卖给同样的客户。而我们正是通过竞争来发现如何以
最低的成本来生产产品,也就是说,通过发现从谁那里能以最
低价格买到原材料或者劳务来发现最低成本。
\switchcolumn*
The basic economic question is how to combine all the resources in society, including human effort, to produce the greatest possible output---not the most pounds of steel, or the most
computers, or the most exciting movies, but the combination of
output that will satisfy people most. We want to produce as
much as we can of each good \textit{that people want}, but not so much
that it would be better to produce something else instead. The
prices we're willing to pay for a good or service, and the prices
we're willing to accept for our labor or for what we've produced, guide entrepreneurs toward the right solution.
\switchcolumn
基本的经济问题是如何将社会中包括人力资源在内的所有
资源结合起来,以最大可能地生产产品 --- 不是最多数量的钢材、计算机或者最激动人心的电影,而是最能\textbf{满足人们需要}的
产品。我们希望尽可能生产出人们需要的东西,但实际上不必
生产那么多,因为生产其他的东西会更好。我们愿意为一件商
品或服务所支付的价格,以及愿意接受的劳动或产品价格,会
弓I导企业家提供正确的解决方案。
\switchcolumn*
When we make decisions in the market, each decision is
made incrementally, or ``on the margin'': do I want \textit{this} steak,
one more magazine, a three-bedroom house? Our willingness to
pay, and the point at which we're not willing to buy another
unit, tells producers how much they can afford to spend on producing the product. If they can't produce another one for less
than the ``market-clearing'' price, they know not to devote
more resources to production of that product. When consumers
show rising interest in computers and declining interest in televisions, firms will pay more for raw materials and labor to produce computers. When the cost of hiring more labor and
materials reaches the limit of what consumers are willing to pay
for the finished product, firms stop drawing more resources in.
As these decisions are repeated thousands, millions, billions, of
times, a complex system of coordination develops that brings to
consumers everything from kiwis to Pentium chips.
\switchcolumn
当我们在市场中作出决策的时候,每一个决策都是增量的,也就是说是“在边际上”的决策:我想要\textbf{这块}牛排、这本杂志、这套三居室房子吗?我们愿意为商品付钱的意愿,以
及不愿意购买更多数量同样商品的意愿,都告诉生产者他们应
该花多少成本来生产这个产品。如果他们不能以低于“市场
出清” 价格的成本多生产一件产品,他们就会知道,不能投
人更多资源来生产那件产品。当消费者显示出对计算机兴趣上
升 、对电.视机兴趣下降的时候,工厂就会将更多成本用于生产
计算机的原材料和劳动力上。当雇用更多劳动力和购买更多原
料的成本达到消费者愿意为最终产品支付的价格限度的时候,
工厂就会停止投入更多的资源。这些决策被数千次、数百万
次、数十亿次地重复,一个复杂的协作系统就出现了,给消费
者带来了从新西兰几维鸟肉到Intel芯片在内的所有东西。
\switchcolumn*
It is the competition of all firms to attract new customers
that produces this coordination. If one firm senses that consumer demand for computers is increasing, and it is the first to produce more computers, it will be rewarded. Conversely, its
television-producing competitor may find its sales declining. In
practice, tens of thousands of firms do well, and thousands go
out of business, every year. This is the ``creative destruction'' of
the market. Harsh as the consumers' judgment may feel to
someone who loses a job or an investment, the market works on
a principle of equality. In a free market no firm gets special privileges from government, and each must constantly satisfy consumers to stay in business.
\switchcolumn
正是企业为吸引新的客户而进行的竞争产生了这种协作。
如果一家企业感觉到消费者对电脑的需要在上升,首先它会生
产出更多的电脑,同时它也将获得回报。与此相反,它的竞争
者电视生产商会发现它的销售在下降。实际上,每一年都有数
以万计的公司做得很好,而同时有数以千计的公司退出竞争。
这就是所谓市场的“创造性破坏” (creative  destruction)。消
费者选择的结果对那些失去工作或者投资失败的人来说是严酷
的,但市场是公平的。在自由市场经济当中没有一家企业会得
到政府授予的特权,每家企业都必须满足消费者的需求以保证企业生存。
\switchcolumn*
Far from \textit{inducing} self-interest, as critics charge, in the marketplace \textit{the fact} of self-interest induces people to serve others.
Markets reward honesty because people are more willing to do
business with those who have a reputation for honesty. Markets
reward civility because people prefer to deal with courteous
partners and suppliers.
\switchcolumn
市场远非如批评者所指责的那样\textbf{引发自私},\textbf{事实上}正是自
私引导人们为他人服务。市场给诚实的人带来回报,因为人们
更愿意与那些有诚实名声的人做生意。市场给懂礼貌的人带来
回报,因为人们更喜欢与有礼貌的伙伴和供应商打交道。

\switchcolumn*[\subsection{Socialism\\社会主义}]
It is the absence of market prices that makes socialism unworkable, as Ludwig von Mises pointed out in the 1920s. Socialists
have often considered the question of production an engineering question: Just do some calculations to figure out what
would be most efficient. It's true that an engineer can answer a
specific question about the production process, such as, What's
the most efficient way to use tin to make a 10-ounce soup can,
that is, what shape of can would contain 10 ounces with the
smallest surface area? But the economic question---the efficient
use of all relevant resources---can't be answered by the engineer. Should the can be made of aluminum, or of platinum?
Everyone knows that a platinum soup can would be ridiculous,
but we know it \textit{because} the price system tells us so. An engineer
would tell you that silver or platinum wire would conduct electricity better than copper. Why do we use copper? Because it
delivers the best results for the cost. That's an economic problem, not an engineering problem.
\switchcolumn
米瑟斯在1920年代指出,正是市场价格的缺位使得社会
主义无法运转。社会主义者常常认为生产问题是一个工程学问
题:只要通过计算找出什么是最有效率的就可以了。的确,工
程师可以回答关于生产过程的特定的问题,例如,如何最有效
地使用白铁皮制造一个10盎司容量的铁汤罐,即制造什么形
状的铁罐才能让使用铁皮的面积最小,而同时有10盎司的容
量?但是经济学问题,即对相关资源如何有效利用的问题,工
程师是无法回答的。例如,汤罐是该用铝皮还是铂金制造?人
人都知道用铂金制造汤罐是荒唐的,但是我们之所以知道这个
是因为价格体系告诉了我们这个信息。工程师会告诉你银线或
铂金线比铜线导电性能更好。那么为什么我们要用铜线制造电
线呢?\textbf{因为}它从成本上考虑是最佳的。这是一个经济学问题,
而不是工程学问题。
\switchcolumn*
Without prices, how would the socialist planner know what
to produce? He could take a poll and find that people want
bread, meat, shoes, refrigerators, televisions. But how much
bread and how many shoes? And what resources should be used to make which goods? ``Enough,'' one might answer. But, beyond absolute subsistence, how much bread is enough? At what
point would people prefer a new pair of shoes to more food? If
there's a limited amount of steel available, how much of it
should be used for cars and how much for ovens? And most important, what \textit{combination} of resources is the least expensive
way to produce each good? The problem is impossible to solve
in a theoretical model; without the information conveyed by
prices, planners are ``planning'' blind.
\switchcolumn
没有价:格的话,社会主义计划官员是如何知道该生产什么
呢?他可以做一项调查,然后发现人们需要面包、肉、鞋、电冰箱、电视机。但是该生产多少个面包、多少双鞋呢?该使用什么样的资源来生产什么东西呢? “够用就行”,有人也许会
这么回答。但是在超过最低生存需要之上,多少个面包才算是足够呢?在什么时候人们宁可要一双新鞋而不是更多的食物呢?如果可用钢材的数量是有限的,其中多少钢材应该用来生产汽车,多少应该用来生产炉子呢?而最重要的是,什么样的\textbf{资源组合}才是生产每一件产品的最节省办法呢?这个问题用一
个理论模型是无法解决的;没有价格来传递信息的话,计划官
员就是胡乱“计划”。
\switchcolumn*
In practice, Soviet factory managers had to establish markets
illegally among themselves. They were not allowed to use money
prices, so marvelously complex systems of indirect exchange---or
barter---emerged. Soviet economists identified at least eighty different media of exchange, from vodka to ball bearings to motor
oil to tractor tires. The closest analogy to such a clumsy market
that Americans have ever encountered was probably the bargaining skill of Radar O'Reilly on the television show M*A*S*H.
Radar was also operating in a centrally planned economy---the
U.S. Army---and his unit had no money with which to purchase
supplies, so he would get on the phone, call other M*A*S*H
units, and arrange elaborate trades of surgical gloves for C rations
for penicillin for bourbon, each unit trading something it had
been overallocated for what it had been underallocated. Imagine
running an entire economy like that.
\switchcolumn
在实践当中,苏联的工厂管理人员不得不在他们之间建立
非法的市场。由于不允许使用货币价格,于是就产生了极端复
杂的非直接交换系统 --- 以物易物。苏联经济学家确认,至少
存在着80种不同的交换媒介,从伏特加、滚珠轴承、汽油到
拖拉机轮胎等。美国人所知道的与这种笨拙的市场最接近的也
许是电视节目《M * A * S * H》当中奥赖利 $\cdot$雷达先生的 讨价还价技巧。雷达先生也在一个中央计划经济 --- 美国陆军当中
工作。当他所在的部队没有钱购买补给的时候,他就打电话给
其他的M * A * S * H 单位,安排进行交易,交易的东西包括
用医用手套装着的C级军粮、抗生素、波旁威士忌酒等。每
个单位都把分配给他们的多余东西拿出来交换他们分配不足的
东西。想像一下以这种方式管理整个经济体吧!

\switchcolumn*[\section{Property and Exchange\\财产与交换}]
One major reason that economic calculation is impossible under
socialism is that there is no private property, so there are no
owners to indicate through prices what they would be willing
to accept in exchange for some of their property. In chapter 3
we examined the right to hold private property. Here we look at
the economic importance of the institution of private property.
Property is at the root of the prosperity produced by a free market. When people have secure title to property---whether it is
land, buildings, equipment, or anything else---they can use
that property to achieve their ends.
\switchcolumn
社会主义制度下的经济计算是不可能的,一个主要原因是
没有私人产权,因而就没有所有者来表示他愿意以什么价格来
交换他的部分财产。在第三章当中我们讨论了保有私人财产的
权利。在这里我们看一下私人产权制度在经济上的重要性。产
权是自由市场创造繁荣的根源。当人们的财产权利受到保护的
时候,无论是土地、房屋、设备还是其他东西,他们都能够使
用这些财产来达到自己的目标。
\switchcolumn*
All property must be owned by someone. There are several reasons to prefer diverse private ownership to government ownership. Private owners tend to take better care of their property
because they will reap the benefits of any increase in its value, or
suffer if its value declines. If you let the condition of your house
deteriorate, you will not be able to sell it for as much as if you
had kept it in good condition---which serves as a strong incentive to maintain it well. Owners generally take better care of
property than renters do; that is, they maintain the capital
value rather than, in effect, using up its value. That's why many
rental agreements require the renter to put down a deposit, to
ensure that he, too, will have an incentive to maintain the property value. Privately owned rental apartments are much better
maintained than public housing. The reason is that no one really owns ``public'' property; no individual will lose his investment if the value of public property declines.
\switchcolumn
所有的财产都必然被某个人所拥有。多样化的私人所有比国家所有要好,有这样几个原因:私人财产所有者倾向于更好
地照顾自己的财产,因为他们将获得其财产价值上升的收益、
承担价值下降的后果。如果你让自己房子的状况恶化,你就不
能把房子卖到保持良好状态的同样价钱 --- 这会大大激励你努
力维护房子。财产所有人对财产的爱护总的来说好于租赁者;
也就是说,他们会维护资产的价值而不是实际上损耗其价值。
这就是为什么很多租赁合同都要求承租人缴纳一笔押金,以确
保他也有维护财产价值的激励。私人拥有的出租公寓比公共租
屋维护得好得多。原因是没有人真正拥有“公共” 财产;如
果公共财产的价值下降,没有谁的个人投资会受到损失。
\switchcolumn*
Private ownership allows people to profit from improving
their property, by building on it or otherwise making it more
valuable. People can also profit by improving themselves, of
course, through education and the development of good habits,
as long as they are allowed to reap the profits that come from
such improvement. There's not much point in improving your
skills, for instance, if regulations will keep you from entering
your chosen occupation or high taxes will take most of your
higher income.
\switchcolumn
私人产权使人们从自己财产的升值中获利,例如通过在上
面建造房屋或者其他方式。人们也可以通过自我提升,如通过
接受教育和发展良好的习惯来获利,只要允许他们收获这些自
我提升所带来的收益。实际上,如果管制让你无法进入所选择
的职业,或者高税收拿走你大部分收入,你就不会有太多动力
来提升自己的技能。
\switchcolumn*
The economic value of an asset reflects the income it will produce in the future. Thus private owners, who have the right to
that income, have an incentive to maintain the asset. When
land is scarce and privately owned, owners will seek to extract
value from it now and also to ensure that they will be able to
continue receiving value from it in the future. That's why timber companies don't cut all the trees on their land and instead
continually plant more trees to replace the ones cut down. They
may be moved by a concern for the environment, but the future
income from the property is probably a more powerful incentive. In the socialist countries of Eastern Europe, where the government controlled all property, there was no real owner to
worry about the future value of property; and pollution and environmental destruction were far worse than in the West. Vaclav Klaus, the prime minister of the Czech Republic, said in 1995, ``The worst environmental damage occurs in countries
without private property, markets, or prices.''
\switchcolumn
一份资产的经济价值反映了它将来会产生的收人。于是私
人产权所有者,即对收入拥有权利的人,就有维护这份资产的
激励。当土地是稀少的并且是个人拥有的时候,所有者就不仅
仅希望在当前获益,也希望他们将来也能继续获益。这就是为
什么木材公司不会砍掉他们土地上的所有树木,而是继续种植
更多的树来代替砍掉的树。他们也许是被对环境的忧虑所打
动,但是来自这份财产的未来收入可能才是更强的激励。在东
欧社会主义国家,政府控制着所有的财产,没有一个真正的所
有者来操心财产的未来价值;因而东欧的污染和环境破坏就远
比西方严重。捷 克 共 和 国 总 理 瓦 茨 拉 夫 $\cdot$ 克 劳 斯(Vaclav Klaus) 1995年时说: “对环境最严重的破坏发生在没有私人产权、市场和价格的国家。”
\switchcolumn*
Another benefit of private property, not so clearly economic,
is that it diffuses power. When one entity, such as the government, owns all property, individuals have little protection from
the will of the government. The institution of private property
gives many individuals a place to call their own, a place where
they are safe from depredation by others and by the state. This
aspect of private property is captured in the axiom, ``A man's
home is his castle.'' Private property is essential for privacy and
for freedom of the press. Try to imagine ``freedom of the press''
in a country where the government owned all the presses and
all the paper.
\switchcolumn
私人财产的另一个好处是它消解了权力。这并不能直接算
是经济上的好处。如果一个主体例如政府拥有所有的财产,对
来自政府的意志,个人几乎就没有可能违抗。私人产权制度给
了很多人一块自己的土地,在这块土地上他们不受来自其他人
和国家的掠夺和侵犯。私人产权的这个原则被收人一句格言:
“一个人的家就是他的城堡。”私人财产对保护隐私权和新闻
出版自由来说是必需的。想像一下,在一个政府拥有所有新闻
媒体和报纸的情况下,“新闻出版自由” 是什么?

\switchcolumn*[\subsection{Division of Labor\\劳动分工}]
Because people have different abilities and preferences, and natural resources are distributed unevenly around the world, we
can produce more if we work at different tasks. Through \textit{the division of labor}, we all seek to produce what we're best at, so we'll
have more to trade with others. In \textit{The Wealth of Nations}, Adam
Smith described a pin factory where the production of pins was
broken into ``about eighteen different operations,'' each performed by specific workers. With such specialization, the workers could produce 4,800 pins per worker per day; without the
division of labor, Smith doubted that one pinmaker could make
20 pins in a day. Note that there are gains to be had from specialization even if one person is better at everything. Economists call this the principle of \textit{comparative advantage}. If Friday
can catch twice as many fish as Crusoe but can find three times
as many ripe fruits in a day, then both of them will be better off
if Crusoe specializes in fishing and Friday specializes in foraging.
As they do specialize, of course, each is likely to improve by repetition and experimentation.
\switchcolumn
人们有不同的能力和偏好,同时自然资源在世界上的分配是不均等的,如果我们按照不同的任务来工作.(分工),就能生产更多的东西。通过\textbf{劳动分工} ,我们生产最擅长的东西,于是就有了更多东西与别人交易。在 《国富论》 中,亚当$\cdot$斯密描绘了一个别针厂,在那里别针的生产被分成“大约十八个不同的操作步骤” ,每个步骤都由一组特定的工人来做。通过这样的专业化,平均每名工人每天就能够生产出4800根别针;如果没有劳动分工,斯密估计一名工人一天就只能生产20个别针。请注意,即便一个人擅长做所有的事情,他也会从专业化分工当中获益。经济学家称这个原则为\textbf{比较优势}。如果 “星期五”在一天当中能够抓到鲁滨孙两倍的鱼但是能够找到三倍的成熟果子,那么如果鲁滨孙专门抓鱼, “星期五”专门收集粮食,对他们两个都更好。当然,如果进行专业化分工 ,他们的技能还会通过不断重复和实验而改进。
\switchcolumn*
People engage in exchange because they expect to become
better off. As Adam Smith put it in a famous passage quoted
earlier but relevant here as well,
\switchcolumn
人们进行交换,是因为他们希望状况变得更好。亚 当 $\cdot$斯
密在一段著名的话当中进行过阐明,这段话我们先前引用过,但这里要再次引用:
\switchcolumn*
\begin{quote}
It is not from the benevolence of the butcher, the brewer, or the
baker that we expect our dinner, but from their regard to their own interest. We address ourselves, not to their humanity but to
their self-love, and never talk to them of our own necessities but
of their advantages.
\end{quote}
\switchcolumn
\begin{quote}
	我们的晚餐不是来自于屠夫、酿酒商人,或面包师傅
	的仁慈之心,而是来自于他们对自身利益的关切。我们不
	依赖于他们的仁慈,而是诉诸他们的自利 之 心 ; 我们从来
	不向他们谈论自己的需求,而是谈论他们自己的利益。
\end{quote}
\switchcolumn*
That doesn't mean that people are always selfish and unconcerned about their fellow man. As noted earlier, the fact that
the butcher must persuade you to buy his meat encourages him
to pay attention to your wants and needs. Store clerks in the
West are famously more pleasant than were their Soviet-era
counterparts.
\switchcolumn
这并不是说人们总是自私的,不关心他们的顾客。正如前
面提到的,实际上,屠夫必须说服你买他的肉,这就会鼓励他
注意你的愿望和需求。众所周知,西方国家的商店店员远比苏
联的店员更加热情。
\switchcolumn*
Still, it makes sense that social institutions operate effectively
when people \textit{do} act in their self-interest. In fact, when people
act in their own interest in a free market, they improve the
well-being of the whole society. Because people trade things
they value less for things they value more, every trade increases
the value of both goods. I will only trade my book for your CD
if I value the CD more than the book, and if you value the book
more than the CD. We're both better off. Similarly, if I trade my
labor for a paycheck from Microsoft, it's because I value the
money more than the time, and the shareholders of Microsoft
value my labor more than the money they give up. Through
millions of such transactions, goods and services move to people
who value them most, and the whole society is made better off.
\switchcolumn
而且,只有当人们\textbf{确实}在追求他们的私利的时候,社会组织才能有效运作。事实上,当人们在自由市场中追求他们自己利益的时候,他们同时也改变了整个社会的状况。人们总是在
用对自己价值较低的东西来交换对自己价值较高的东西,因此每一次交易都提高了双方所交易的东西的价值。只有在我认为
你的CD 比我的书对我更有价值、同时你认为我的书比你的CD对你更有价值的时候,我才能用我的书换到你的CD。而我们双方都会从中受益。类似的情况,如果我用我的劳动交换
微软公司给我的工资,那是因为我认为钱比我付出的时间对我更有价值,而微软的股东们则认为我的劳动比他们放弃的那部
分钱对他们更有价值。通过数以百万计的这种交易,商品和服
务就转移到了那些对它们估值最高的人那里,于是整个社会都
从中得到了好处。
\switchcolumn*
Capitalism encourages people to serve others in order to
achieve their own ends. Under any system, talented and ambitious people are likely to acquire more wealth than others. In a
statist system, whether the old precapitalist regimes or a ``modern'' socialist country, the way to get ahead is to get your hands
on the levers of power and force other people to do your bidding. In a free market, you have to \textit{persuade} others to do what
you want. How do you do that? By offering them something
\textit{they} want. So the most talented and ambitious people have an
incentive to find out what others want and try to supply it.
\switchcolumn
资本主义鼓励人们为别人提供服务来达到他们自己的目
标。在任何制度下,有才千和有雄心的人通常都会比其他人得到更多的财富。在国家主义体制下,无论是旧的前资本主义国家还是 “现代的” 社会主义国家,脱颖而出的办法都是想办法把手放在权杖上,强迫其他人遵照你的命令。在自由市场经济中,你不得不通过\textbf{说服}别人按照你的愿望来做事。怎么才能做到这一点呢?通过提供\textbf{他们}想要的东西。因此,最有才干和最有雄心的人会有动机去发现其他人想要什么,并且努力给他们提供这些东西。
\switchcolumn*
Private ownership under the rule of law \textit{prevents} the kind of
selfishness that involves taking what you want from those who
own it. It also encourages people who want to get rich to produce goods and services that other people want. And so they
do---Henry Ford with his cheap, efficient automobiles; Bill
Cosby with his popular television show; Sam Walton with his discount stores; Bill Gates with his computer operating system;
and many obscure people in a complex economy like ours, such
as Philip Zaffere, who cleared \$200 million when he sold the
company he had founded, which made Stove Top stuffing and
Mrs. Paul's fish sticks. Leona Helmsley may not \textit{be} a nice person, but to get rich in the hotel business, she has to provide a
comfortable room and clerks who \textit{seem} nice.
\switchcolumn
法治之下的私人所有制能够\textbf{防止}想从所有者那里夺走东西的那种自私,同时鼓励那些想致富的人向其他人提供他们需要的产品和服务。而那些富人的确是这么做的。亨 利 $\cdot$福特通过
提供价廉物美的汽车而致富,比尔$\cdot$科斯则通过带来受欢迎的电视节目而变得有钱,山姆$\cdot$华生则通过开折扣商店成为富
翁,比尔$\cdot$盖茨则因为给人们提供了计算机操作系统而获得财
富;在美国这样的复杂经济体当中还有很多并不那么有名的富
豪,如菲利浦$\cdot$扎法雷,他 的 公 司 生 产 “炉顶食材” 和 “保
罗太太” 鱼肉棒。在卖掉公司之后,他得到了 2 亿美元。利昂
娜 $\cdot$ 赫尔姆斯莉\footnote{利昂娜$\cdot$赫尔姆斯莉(Leona  Helmslcy) ,纽约的亿万富婆,波兰犹太人的后裔,生前曾是纽约城市饭店的掌管者和帝国大度的建造人,被人们称为“吝啬皇后” (Queen  of Mean)。她著名的一句话是:“只有平民百姓才会交税”(only  the little people pay taxes), 曾因为逃税而人狱19个月。她把遗产留给了她的小狗。}\textbf{也许并不}招人喜欢,但是为了从宾馆生意中获得财富,她不得不给人们提供舒适的房间和\textbf{看上去}招人喜欢的服务员。

\switchcolumn*[\section{Profits, Losses, and Entrepreneurs\\利润、亏损和企业家}]
Everyone can see what roles consumers and producers---whether farmers, laborers, craftsmen, or factory owners---play
in a market system, but sometimes the role of the entrepreneur
or middleman is not so well understood. Historically, there has
been a lot of hostility directed at middlemen. (Often this has
taken the form of racial or ethnic prejudice, against Jewish entrepreneurs in Europe and the United States, Indians and
Lebanese in Africa, Chinese in much of Asia, and Koreans in
today's inner cities, as Thomas Sowell points out in \textit{Race and Culture}. Economic ignorance is obviously not the only source of
such attitudes, but a better understanding of economics would
help to alleviate it.) The feeling seems to be, the farmer grows
the wheat, the miller grinds it, the baker makes bread, but
what value is added by the traders and distributors who move
the wheat along the path to the consumer? Of what value is the
Wall Street trader who spends his time exploiting price discrepancies between markets?
\switchcolumn
人们都能明白消费者和生产者 --- 无论是农民、工人、手
工业者还是工厂主 --- 在市场体制当中的作用,但有时候对企
业家或者中间商的作用并不是那么清楚。在历史上常常存在对中间商的敌视。而 正 如 索 维尔(Thomas Sowell)在《种族与文化》(\textit{Race and Culture}) 一 书中所指出的那样,这种敌视常
常表现为种族或民族偏见,例如在欧洲、美国对犹太企业家,在非洲对印度人和黎巴嫩人,在大部分亚洲地区对中国人,以
及今天美国内陆城市中对韩国人的偏见等。对经济学的无知显
然不是这种态度的唯一原因,但是更好地了解经济学将有助于
减少这种偏见。他们的感觉也许是,农民种出了小麦,磨坊主
把小麦磨成面粉,面包师傅把面粉做成面包,但交易商和批发
商只是把小麦通过这个链条转移到了消费者那里,他们增加了
什么价值呢?华尔街的期货股票交易者们只是在利用不同市场
的价差转手赚钱,他们增加了什么价值呢?
\switchcolumn*
In a complex economy the role of the entrepreneur is vitally
important. He might even be viewed as the person who actually
performs the coordination that is the market process, the person who directs resources to where they're most needed. In a
sense, we are all entrepreneurs. Every person tries to forecast
the future and allocate his own resources wisely. Even Robinson
Crusoe had to predict whether future weather conditions meant
that he had better spend more time building a shelter at the expense of eating better today. Each of us forecasts where our
skills will be most in demand, what potential customers will be
willing to pay for our products, whether products we want will cost more or less next week, where we should invest our retirement savings. Of course, no one is really the much-derided
``economic man,'' making calculations only on the basis of monetary return. We may take a less remunerative job because it involves interesting work or is near our home; we may start a
photography business because we like photography, even
though we could make more money selling business equipment; we may be willing to pay more for products sold by
friends or by environmentally sensitive companies. We make
most of our economic decisions on the basis of a combination of
factors, including price, convenience, enjoyment, personal relationships, and so on. The only thing economic analysis assumes
is that we all make choices that are in our own interest, \textit{however we define our interest}.
\switchcolumn
在一个复杂的经济体里面,企业家的作用是至关重要的。
他甚至应当被看作实际上操盘协作过程(即市场过程)的人,
将资源引导到最需要的地方的人。从某种意义上来说,我们都
是企业家。每个人都希望能够预测未来并明智地安排自己的资
源。就连鲁滨孙也不得不预测,未来的天气条件是不是意味着
他最好花时间建一个窝棚,而把今天的美餐牺牲掉。每个人都
会预测哪里最需要自己的技能,什么样的潜在客户会愿意为我
们的产品掏钱,想要买的商品下周会涨价还是跌价,应该把退
休准备金投资到哪里。当然,没有一个人真的 是“经济人”(这个概念广受嘲笑),成天只是计算着挣多少钱。我们也许
会做一份收入比较少的工作,因为这个工作比较有意思或者离
家比较近;我们可以开一家照相馆,因为我们喜欢摄影,尽管
我们卖掉这些摄影设备可能挣更多的钱;我们也许愿意为朋友
出售的商品付更高的价钱,或者为更注重环保的公司的产品多
付钱。我们做出大多数经济决策的时候考虑的是综合因素,包
括价格、方便性、喜好、人际关系等。经济学分析的唯一假设是 :我们是按照自己的利益来做出决策,\textbf{无论我们对“自己的利益” 如何界定}。
\switchcolumn*
But economists use the term ``entrepreneur'' to denote a specific participant in the market process, one who is neither producer nor consumer but someone who sees and acts on an
opportunity to move resources from where they are less valuable to where they are more valuable. He may see that kiwis sell
for 30c on the West Coast and 50c on the East Coast, and that
he can transport them for 10c each, so he can make 10c a kiwi
by buying them in the West and shipping them east. He may
discover that one company wants to buy an office building for
up to \$10 million, and that another company has an appropriate office building that it would sell for \$8 million. By buying
and reselling it (or simply by bringing buyer and seller together
for a fee), he can make a tidy profit. He may see that radios
could be produced very cheaply in Malaysia and sold in the
United States for less than they currently cost, so he contracts
with a manufacturer to produce radios and ship them here. He
or another entrepreneur may then see that American firms
could supply insurance in Malaysia cheaper than any Malaysian
firm, so another profitable exchange can be made.
\switchcolumn
但是经济学家使用“企业家”这个词时指的是市场过程
中某些特定的参与者,他既不是生产者也不是消费者,而是那
种通过观察和把握机会,把资源从价值较低的地方转移到价值
较髙的地方的人。他也许看到几维鸟肉在西海岸售价30美分
而在东海岸售价50美分,然后他会在西海岸买进几维鸟肉,
以每只10美分的运输费用运到东海岸卖掉,每只几维鸟肉赚
10美分的利润。他也许会发现一家公司愿意以最高1000万美
元的价格买一栋写字楼,而另一家公司有一栋合适的写字楼愿
意以800万美元的价格卖出。通过买进和卖出,或者只是把买
家和卖家撮合在一起,他就会挣到一笔净利润。他也许看到收
音机如果在马来西亚便宜地生产并拿到美国销售,成本会比现
在更低,那么他就会和一家制造商签约生产收音机,然后把它
们运到这里。他或者另一位企业家如果看到美国公司能够在马
来西亚提供比任何一家当地公司费用更低的保险,又一个有利
可图的交易就会产生。
\switchcolumn*
In each case the entrepreneur's role is to see a situation in
which resources could be used in a way more valuable than they
are now being used. His reward for seeing that is a portion of
the value that he adds to both sides. If, to satisfy some people's
skepticism about middlemen, we outlawed entrepreneurial activity, what would happen? Easterners would be deprived of kiwis they would gladly pay for, Americans would pay more for
radios, one company wouldn't get an office building it could
use, and another wouldn't get cash that it would value more
than a building. But these are just the surface manifestations.
What would really happen is that our complex modern economy would grind to a halt. Middlemen exist for a reason, because their services are worth something to the people they
trade with. Farmers could bring their own goods to market, but
most of them find it more efficient to concentrate on farming
and sell their produce to middlemen. Consumers could go to
farm states and buy produce from farmers there, but it's clearly
more efficient to go to the grocery.
\switchcolumn
在每个案例中,企业家的作用就是能够发现某种局面下,
资源能够比现有方式更有价值地被利用。对他发现的回报是他
为双方增加的价值的一部分。如果为了消除某些人对中间商的
疑心,把这种企业家行为视为非法,将会发生什么情况?东海
岸的人将吃不到几维鸟肉,尽管他们愿意付钱;美国人将会为
购买收音机付更高的价钱;那家公司将得不到写字楼,另一家
公司则得不到现金,尽管他们认为现金比写字楼对他们来说更
有价值。但这只是外在表现而已,真正致命的是复杂现代经济
的运转将会被粉碎、中断。中间商的存在自然有其理由,因为
他们的服务对与他们进行交易的人来说是有价值的。农民可以自己把产品送到市场上,但是他们发现,自己专注于种地而把
出产的货物卖给中间商是最有效率的。消费者可以亲自到农业
州的农场去,从农民手里直接购买农产品,但很显然到食品杂
货店去购买更有效率。
\switchcolumn*
The role of entrepreneurs in allocating capital goods---the
resources that are used to produce consumer goods---is even
more necessary. As an economy gets wealthier and more complex, its structure of production lengthens. That is, there are
more steps between raw materials and consumer goods. The
first capital goods were probably nets for catching fish. By
Adam Smith's time, there were several more steps involved in
producing the machines that would help workers make pins in
a factory. Today, just imagine the steps involved in getting a
computer to the consumer: the store, which someone has to invest in; the transportation system; the firm that produces the
computer; the software engineers, who had to be educated; the
chips, which had to be designed and produced; the metal, glass,
and plastic, which had to be produced, refined, and molded,
and so on and so on. As the structure of production lengthens,
requiring investments in production processes long before consumers will decide whether to purchase a product, it becomes
ever more essential to have people constantly looking for opportunities to use resources more efficiently.
\switchcolumn
企业家配置资产 --- 即用来生产消费品的资源 --- 的作用
甚至更为必要。随着一个经济体变得越来越富裕和越来越复
杂,它的生产链条就会拉长。也就是说,在原材料和消费品之
间会产生更多的环节。最初的资产也许是用来抓鱼的渔网。到
了亚当$\cdot$ 斯密时代,机器生产的过程当中包含了更多的环节,
这让工人们能够在一个工厂中生产更多的别针。今天,想像一
下给消费者提供一台电脑的所有环节:商店,必须有人投资来
建商店;交通运输系统;生产电脑的工厂;软件工程师,他必
须经过教育培训;芯片,必须经过设计并生产出来;金属、玻
璃、塑料,这些都必须生产、提纯和压模,等等。由于生产链
条的拉长,各个环节的生产商在消费者购买某件产品之前很长
时间就得投资进行生产。于是,有人不断地寻找机会让资源的
利用更有效率就更为必要。
\switchcolumn*
People engage in economic activity to get something they
want---more goods and services, ultimately, but in the immediate situation, a paycheck or a purchase. Workers get paid for
their labor, farmers sell their products. The reward an entrepreneur gets is profit. The word ``profit'' can mean different things.
To an accountant it just means the money left over after a period of economic activity. Often that money is really the salary
paid to the business owner for his labor, or interest earned on money lent to borrowers. Pure entrepreneurial profit comes out
of the gap between the lower-valued and higher-valued use of a
resource that the entrepreneur has spotted and acted on. It reflects his correct forecast about what consumers would prefer.
The flip side is that entrepreneurs sometimes estimate wrongly,
in which case they sustain entrepreneurial losses.
\switchcolumn
人们参与经济活动是为了得到他们想要的东西 --- 更多的
商品和服务,最 终 (但是最直接地)是为了一份工资或者卖
出货物。工人获得报酬是因为付出了劳动,农民是因为卖出了
他们的产品,而企业家的回报是利润。从不同的角度看,“利
润” 这个词会有不同的含义。对会计来说,它的意思仅仅是
在一段时间的经济活动之后留下的钱。这笔钱常常真的是对企
业所有者的劳动所支付的工资,或者是通过把钱借给别人而获
取的利息。而纯粹的企业家利润是来自对资源的低效利用和高
效利用之间的差距,这种差距是由企业家发现并加以利用的。
它反映了企业家对消费者喜好的正确预测。不幸的是,企业家有时候会估计错误,这时候他们就会遭到亏损。
\switchcolumn*
Sometimes people get upset about high profits. They want to
limit profits or tax them away, especially those notorious ``windfall profits.'' (You rarely hear people saying that society should
chip in to help those businessmen who make ``windfall losses.'')
In fact, we should be grateful to those who make profits. As the
economist Murray Rothbard put it, ``profits are an index that
maladjustments (that is, less efficient uses of resources) are
being met and combatted by the profit-making entrepreneurs.''
Or, as Israel Kirzner of New York University explains, ``The entrepreneurial search for profits implies \& \textit{search for situations where
resources are mis-allocated}.'' The higher the profit an entrepreneur
makes, the bigger the gap he discovered between how resources
were being used and how they \textit{could} be used, and thus the more
he has benefited society. When critics complain that drug companies' profits are too high, they imply that high profits on such essential products are immoral. In fact, high profits signal the need for more investment in making drugs and curing disease. The drug companies that were making the highest profits were
filling the \textit{greatest} gap between what consumers needed and
what the market was hitherto producing. A limit on drug company profits would discourage investment just where it was most needed.
\switchcolumn
人们有时候会对髙额利润感到不安。他们希望对利润,尤
其是名声不好的“意外利润”进行限制或者通过抽税把它们
拿走。但你很少听到人们说社会应当帮助那些遭到“意外亏
损” 的商人。实际上,我们应该对那些创造利润的人表示感
激。正如经济学家罗斯巴德所说,“利润是一个指标,证明经
济 失 调 (即对资源的低效利用)正在被创造利润的企业家所
发现和制止。” 或者如纽约大学的柯兹纳\footnote{伊斯雷尔$\cdot$柯兹纳(Israel  M Kirzner, 1930 $\sim$),纽约大学经济学教授,当代奥地利学派经济学的代表人物之一。}所解释的那样,“企业家寻求利润实际上是在\textbf{搜寻资源被错误配置}的状况 。” 企业家获得的利润越高,说明他所发现的资源被利用的现状与资源\textbf{能够}被利用的状况之间的差距越大,他对社会所作的贡献也就
越大。当批评者们抱怨制药公司利润过高的时候,他们隐含的意思就是,在这样至关重要的产品上赚取高利润是不道德的。实际上,高额利润是一种信号,说明需要更多投资来生产药物和治疗疾病。获得最高利润的制药公司填补了消费者需求和市场现有生产之间 的\textbf{最大}鸿沟。对制药公司利润的限制将会阻碍对最需要投资的地方获得投资。
\switchcolumn*
It's not the profit maker we should criticize, but the loss
maker. But we don't need any windfall-loss tax. The market
punishes entrepreneurs who make wrong predictions in the
form of losses, and enough losses will remove him from the role
of entrepreneur and encourage him to go to work for somebody
who is better at allocating resources.
\switchcolumn
我们应该指责的不是创造利润的人,而是制造亏损的人。
但是不需要征收任何的“意外亏损税”。市场会通过亏损惩罚
那些做出错误决策的企业家,亏损够大的话就会把他开除出企
业家的行列,让他转而去为某个更善于配置资源的人打工。
\switchcolumn*
Through this immensely complex process---which looks so
simple on the surface, with an endless stream of consumer products flowing into stores---free-market prices help us all to coordinate our efforts and raise our standard of living.
\switchcolumn
这个极其复杂的过程 --- 从表面上看是如此简单,像河流
一样地带着消费品永不停歇地流向万千商店 --- 自由市场的价
格帮助所有人互相协作,提高人们的生活水准。
\switchcolumn*
Enthusiasts for the market process sometimes refer to ``the magic of the marketplace.'' But there's no magic involved, just
the spontaneous order of peaceful, productive people freely interacting, each seeking his own gain but led to cooperate with
others in order to achieve it. It doesn't happen overnight, but
through years and centuries the market process has brought us
from a society characterized by backbreaking labor to achieve
bare subsistence and an average life expectancy of twenty-five
years to today's truly amazing level of abundance, health, and
technology.
\switchcolumn
市场经济的热情支持者有时候会说“市场的奇迹”,但是
这里面其实不包含任何奇迹,只不过是一种自发形成的秩序。
在这种秩序当中,和平的、有生产能力的人们自由地建立关
系,虽然每个人追求自己的利益,但被引导去与其他人进行合
作来达到其自利的目的。这不是在一夜之间发生的,而是持续
了很多年、很多个世纪。市场经济带领我们从一个人人辛苦劳
作、在生存线上挣扎、平均预期寿命25 岁的社会,走到今天
的这个财富、健康和技术水平都达到了神奇高度的社会。

\switchcolumn*[\section{Economic Growth\\经济增长}]
How does economic growth happen? How did we get from a
world in which people had only their own labor, land, and readily apparent natural resources to today's complex economic
structure supporting an unprecedented standard of living?
\switchcolumn
经济增长是如何发生的?我们如何从一个人们只能利用自
己的劳动、土地和一些简单自然资源的世界变成今天这样一个
支撑了史无前例的生活水准的复杂经济结构的世界?
\switchcolumn*
In their highly readable little book, What \textit{Everyone Should Know about Economics and Prosperity}, the economists James D.
Gwartney and Richard L. Stroup offer a concise guide to the
sources of prosperity. A first point to note is that to consume
more, we must produce more. Scarcity is a basic part of the
human condition; that is, our wants always exceed the resources
available to satisfy them. To satisfy more of our wants, we must
learn to use resources more efficiently.
\switchcolumn
在一本被广泛阅读的小书《人人都该知道的经济学常识
与繁荣》(\textit{Everyone Should Know about Economics and Prosperity}) 中,经济学家格瓦特尼(James D. Gwartney) 和
斯特鲁普(Richard L.Strmip) 对繁荣的源泉提供了一个简明
指南。第一点需要提出是:为了消费得更多,我们必须生产得
更多。稀缺是人类面临的一个最基本条件,也就是说,资源所
能够提供的东西永远无法满足需要。为了更多地满足我们的需
要,我们必须学会更有效地利用资源。
\switchcolumn*
We should also note that our goal is not to increase ``growth
of the economy,'' much less gross national product, national income, or any other statistical aggregate. There are many problems with such statistics (even though I will occasionally use
them in this book), and they can lead people to make gross errors in economic observation, such as the notorious estimates
that the Soviet economy was much larger than it was or the
ridiculous statistics showing that the East German economy
was half the size of West Germany's per capita. The goal of economic activity is to increase the supply of consumer goods for
people, which will entail also increasing the supply of capital
goods with which to produce consumer goods. It may be hard
to measure that accurately, but we should remember that our
concern is real goods, not statistics.
\switchcolumn
还需要注意的是,我 们 的 目 标 不 是 提 高 “经济增长速
度” ,更不是国民生产总值(GDP)、国民收入或者任何其他
综合统计数字。这些统计方法都有很多问题(尽管在这本书
里我会偶尔使用这些数字),会导致人们进行经济观察时犯重
大的错误。这方面臭名昭著的例子是苏联对自己经济规模的估
计远远超过其实际水平,还有一套荒唐的统计数字显示,东德
的人均GDP相当于西德的一半。经济活动的目的是为人们提
供更多的消费品,同时这也会带来资本品供应的增长,而资本
品是用来生产消费品的。要准确区分这两者是很难的,但是应
该记住我们关注的是真实产品,而不是统计数字。
\switchcolumn*
One way to produce real economic growth is through saving
and investment. By consuming less today, we can produce and
consume more tomorrow. There are two basic benefits of saving. The first is to set something aside ``for a rainy day,'' a
metaphor that recalls a primitive, even Robinson Crusoe econ-
omy. Crusoe sets aside some of the fish and berries he gathers
today in case he is sick tomorrow or the weather prevents him
from gathering more food. The second benefit is even more important. We save and invest so we can produce more in the future. If Crusoe saves food for a few days, he can take a day to
produce a net, which would allow him to catch many more fish.
In a complex economy, savings allow us to open a business or to
invent or purchase equipment to make us more productive. The
higher our level of saving (either as individuals or as a society),
the more investments we can make in future production, and
the higher our future standard of living---and that of our children---can be.
\switchcolumn
产生真实经济增长的一个办法是储蓄和投资。通过减少今
天的消费,我们可以在明天生产和消费更多。储蓄有两个基本
的好处。第一是未雨绸缪,为可能发生的意外做储备。所谓未
雨绸缪即便对原始经济甚至鲁滨孙经济来说也是必要的。鲁滨
孙今天把采集来的鱼和浆果储备一些,是为了防止万一明天他
生病或天气不佳无法出去采集食物。第二个好处甚至更加重
要。储蓄和投资是为了在将来能够生产更多的东西。如果鲁滨
孙储藏几天的食物,他就能够用一天的时间制作一张渔网,帮
助他抓到多得多的鱼。在一个复杂经济体当中,储蓄让我们能
够开一家公司或者投资或者购买设备,提高我们的生产能力。
储蓄水平越高(不论是个人还是社会),就能将越多的投资用
于未来的生产,也就越有能力提高未来的生活水平,以及我们
孩子们的生活水平。
\switchcolumn*
A complex economy needs an efficient capital market to attract savings and channel them into investments that will pro-
duce new wealth. The capital market includes markets for
stocks, real estate, and businesses, and financial institutions
such as banks, insurance companies, mutual funds, and investment firms. As Gwartney and Stroup write, ``The capital market coordinates the actions of savers who supply funds to the
market and investors seeking funds to finance various business
activities. Private investors have a strong incentive to evaluate
potential projects carefully and search for profitable projects.''
Investors are rewarded for making the right decisions---for
channeling capital to projects that serve consumers' needs---
and penalized with losses for channeling scarce capital to the
wrong projects. We often hear disparaging references to ``paper
entrepreneurs,'' with a sort of macho disdain for people who
don't ``make things'' like steel and automobiles. But in an increasingly complex economy, no task is more important than allocating capital to the right projects, and it is entirely
appropriate that the market rewards people handsomely for
making the right investment decisions.
\switchcolumn
一个复杂经济体需要一个高效的资本市场来吸引储蓄,以
及把资金引导到能够产生新财富的投资当中去。资本市场包括
股票市场、房地产市场以及企业债券市场,还有金融机构如银
行 、保险公司、互助基金和投资公司等。正如格瓦特尼和斯特
鲁普所说:“资本市场是给市场提供资金的储蓄者和寻找资金
给不同商业活动提供财务支持的投资者共同作用的结果。私人
投资者有更强的动机来对可选项目进行慎重评估,以寻找可盈
利的项目。” 投资者因为做出正确的决定 --- 把资金投入到满
足消费者需求的项目当中 --- 而得到回报,或者因为把稀缺资本投入到错误项目当中而遭受亏损的惩罚。我们常常听到有人
带着鄙视的情绪称那些不从事“产品生产”,不直接生产如钢
铁和汽车等产品的投资者为“纸上资本家”。但是在一个越来
越复杂的经济体中,再没有比把资本配置到正确项目中更重要
的任务了,市场对那些漂亮地做出正确投资决定的人给予回报
是完全正当的。
\switchcolumn*
Another source of economic growth is improvements in
human capital, that is, the skills of workers. People who improve their skills---by learning to read and write, or learning
carpentry or computer programming, or going to medical
school---will usually be rewarded with higher earnings.
\switchcolumn
经济增长的另一个源泉是人力资本即劳动者技能的提升。
那些提高自己技能的人 --- 通过学习阅读和写作、学习木工、
计算机编程或者进入医学院学习 --- 将会得到更高收入作为
回报。
\switchcolumn*
Improvements in technology also contribute to economic
growth. Beginning with the Industrial Revolution about 250
years ago, technological changes have transformed our world.
The steam engine, internal combustion, electricity, and nuclear
power have replaced human and animal power as our principal
sources of energy. Transportation has been revolutionized by the
railroad, the automobile, and the airplane. Labor-saving devices
such as washing machines, stoves, microwave ovens, computers, and a whole panoply of industrial machines have allowed us
to produce more in less time. Entertainment has been changed
beyond recognition by records, tapes, compact disks, movies,
and television. In the eighteenth century only the Austro-Hungarian emperor and his court could hear Mozart; today anyone
can hear Mozart, Mancini, or Madonna for a few dollars. Hollywood may produce plenty of trash (though we should remember that it's trash that people choose to watch), but more people
have seen Shakespeare's \textit{Richard III} performed in movies featuring Laurence Olivier and Ian McKellen than saw all the stage
performances in history.
\switchcolumn
技术进步同样也会对经济增长作出贡献。开始于大约250
年前的工业革命的技术进步已经彻底改变了我们的世界。蒸汽
机、内燃机车、电和原子能已经代替了人力和畜力,成为最基
本的动力来源。交通已经通过铁路、汽车和飞机而发生了革命
性的变化。节约人力的机器如洗衣机、加热器、微波炉、电脑
和其他不可胜数的工业机器让我们能用更少的时间生产更多的
东西。娱乐也因为录音机、录像带、CD、电影和电视的出现
而发生了根本改变。18世纪只有奥匈帝国的皇帝和皇室成员
才能听到莫扎特,今天任何人只要花几个美元就能听到莫扎
特、曼奇尼或者麦当娜。好莱坞也许制造出了大量的垃圾
(但应该记住,这些垃圾是人们自愿选择看的),但是更多的
人通过电影看到了由劳伦斯$\cdot$ 奥 立 弗 (Laurence  Olivier) 和伊
安 $\cdot$ 迈 克 兰(Ian McKellen)主演的莎士比亚的《查理三世》,
人数比历史上所有通过舞台看到这个话剧的人还多。
\switchcolumn*
An often-overlooked source of growth is improvements in
economic organization. A system of property rights, the rule of
law, and minimal government allows maximum scope for people to experiment with new forms of cooperation. The development of the corporation allowed larger economic tasks to be
undertaken than individuals or partnerships could achieve. Organizations such as condominium associations, mutual funds,
insurance companies, banks, worker-owned cooperatives, and
others are attempts to solve particular economic problems by
new forms of association. Some of these forms turn out to be inefficient; many of the corporate conglomerates of the 1960s, for
instance, proved to be unmanageable, and shareholders lost
money. The rapid feedback of the market process ensures that
successful forms of organization will be copied and unsuccessful
forms will be discouraged.
\switchcolumn
经济增长的一个常常被忽视的原因是经济组织的改进。一
个由财产权、法治和最小政府所构成的体制允许人们在最大范
围内试验新型的合作方式。公司组织的发展可以让人们完成比个人和合伙企业所能达到的规模大得多的经济任务。经济组织
如公寓业主协会、互助基金、保险公司、银行、工人合作社,
以及其他形式的组织,试图通过建立新型组织来解决特定的经
济问题。其中有些形式的组织被证明是无效的,1960年代很
多这样的多元化企业被证明是无法管理的。股东们的投资都遭
到了损失。市场经济的快速反馈机制确保了成功的组织形式能
够被复制,不成功的组织形式的发展能够被阻止。
\switchcolumn*
All these sources of growth---saving, investment, improvements in human capital, technology, and economic organization---reflect the choices of individuals spurred by their own in-
terest in a free market. The market in the United States and
Western Europe is hardly as free as it could be, but its relative
freedom has produced huge increases in output. As Gwartney
and Stroup point out, ``Workers in North America, Europe, and
Japan produce about five times more output per capita than
their ancestors did 50 years ago.'' So, not surprisingly, ``their inflation-adjusted per capita income---what economists call real
income---is approximately five times higher.''
\switchcolumn
所有这些经济增长的来源 --- 储蓄、投资、人力资源的提
升、技术进步、经济组织的改进 --- 反映了在自由市场当中个
人因自利而进行的选择。美国和西欧的市场并没有达到其应有
的自由程度,但是他们的相对自由已经产生了巨大增长。正如
格瓦特尼和斯特鲁普指出的那样:“北美、欧洲和日本的工人
比他们五十年前的先辈们多创造出了五倍以上的人均产值。”
因此,毫不奇怪,“他们去掉通胀因素后的人均收入(即经济
学家认为的真实收入)大约比五十年前高五倍。”

\switchcolumn*[\section{Government's Discoordination\\政府破坏协调}]
What is the role of government in the economy? To begin with,
it plays a very important role: protecting property rights and
freedom of exchange, so that market prices can bring about coordination of individual plans. When it goes beyond this role,
trying to supply particular goods or services or encourage particular outcomes, it not only doesn't help the process of coordination, it actually does the opposite---it \textit{discoordinates}. Prices convey information. If prices are controlled or interfered with by the government, then they won't convey \textit{accurate information}.
The more interference, the more inaccurate the information,
the less economic coordination, and the less satisfaction of
wants. Interference in the information conveyed by prices is just
as destructive to economic progress as interference in language
would be to having a conversation.
\switchcolumn
政府在经济当中的作用是什么?它扮演了一个非常重要的
角色:保护产权和自由交易,以便市场价格能带来无数个人计
划之间的协调。如果它超出这个角色,试图提供商品或服务,
或者鼓励某些结果的出现,那它就不仅仅无助于市场的协调,而且实际上做了相反的事情---\textbf{破坏协调} ((discoordinates)。
价格的功能是传递信息。如果价格被政府控制或干预,它们就
不能传递\textbf{正确的信息}。干预越多,信息越失真,经济就越不协调 ,就越不能满足人们的需求。干预价格对经济发展的破坏,
就像干预语言对对话的破坏一样。

\switchcolumn*[\subsection{Preserving Jobs\\就业保护}]
Whenever a better way is found to satisfy any human need (or
when demand for any product falls), some of the resources previously employed in satisfying it will no longer be needed.
Those no-longer-needed resources may be machines or factories
or labor services. Individuals may lose their investments or their
jobs when a competitor comes along with a cheaper way of
meeting consumers' needs. We should be sympathetic to those
who find themselves unemployed or faced with a loss of their
investment in such a situation, but we should not lose sight of
the benefits of competition and creative destruction. People in such a situation often want the government to step in, to main-
tain demand for their product, or bar a competitor from the
market, or somehow preserve their jobs.
\switchcolumn
只要一种满足人们需求的更好方式被发现(或者当对某
种产品的需求下降),以前用来满足旧的需求的那些资源就不
再需要了。这些不再需要的资源也许是工厂的机器,也许是劳
动力。当一个竞争者带来一种满足消费者需求的成本更低的方
式的时候,有A 也许会失去他们的投资或者丢掉工作。在这种
情况下,我们应该对那些失去工作或者损失了投资的人表示同
情,但是我们不应该无视竞争和创造性破坏的好处。处在这种
情形下的人们常常希望政府出面,维护对他们产品的需求,或
者禁止市场上的竞争者,或者设法保住他们的工作。
\switchcolumn*
In the long run, however, it makes no sense to try to preserve
unnecessary jobs or investments. Imagine if we had tried to preserve the jobs in the buggy industry when the automobile came
along. We would have been keeping resources---land, labor,
and capital---in an industry that could no longer satisfy con-
sumers as well as other uses of those resources. To take a more
recent example---one that should be familiar to those who en-
tered school in the 1960s, though perhaps entirely unfamiliar
to younger people---the slide rule was completely replaced by
the calculator in a matter of just a few years in the 1970s.
Should we have preserved the jobs of those making slide rules?
For what purpose? Who would have bought slide rules once
calculators became available and inexpensive? If we did that
every time a firm or an industry became uneconomical, we
would soon have a standard of living comparable to that of the
Soviet Union.
\switchcolumn
然而,从长期来说,没有任何理由去保护不必要的工作或
投资。想一想如果在汽车出现的时候去保护二轮马车产业会是
什么结果。本来可以把土地、劳动力和资本等资源保留在一个
不那么能够满足消费者需要的产业当中。举一个较近的例子
--- 这个例子I960年代上学的人应该很熟悉,也许年轻人完
全 不 熟 悉 计 算 尺 在 1970年代很短的几年内完全被计算器
所代替。我们应该保护制造计算尺的工作岗位吗?为什么呢?
一旦计算器随处都有,价格便宜,谁还会买计算尺呢?如果每
次都这样做,那么一个公司或者行业就会变得不经济,很快,
我们的生活水平就可以和苏联相媲美了。
\switchcolumn*
It's often said that the point of an economy, or at least of economic policy, is to create jobs. That's backward. The point of an
economy is to produce things that people want. If we really
wanted to create lots and lots of jobs, the economist Richard
McKenzie points out, we could do it with a three-word federal
policy: Outlaw farm machinery. That would create about 60
million jobs, but it would mean withdrawing workers from
where they are most productive and using them to produce
food that could be produced much more efficiently by fewer
workers and more machinery. We would all be much worse off.
\switchcolumn
常常听到的一个说法是经济的目的,或者至少是经济政策
的目标应该是创造工作机会。这是一种曲解。经济的目的应该
是生产人们需要的东西。经济学家理查德$\cdot$ 麦肯泽(Richard McKenzie)指出,如果我们真的是想创造大量的工作机会,
我们只需采用一个简单的联邦政策一禁止使用农业机械。那
将会创造出大约6000万个工作机会,但所有人的状况都会因此恶化。因为这意味着把劳动者从生产力高的地方撤出来,让
他们生产粮食,而这些粮食本来使用极少的人力和更多的机器
就可以更有效率地生产出来。
\switchcolumn*
Norman Macrae, long-time deputy editor of the Economist,
has pointed out that in England, since the Industrial Revolution, about two-thirds of all the jobs that existed at the beginning of each century have been eliminated by the end of the
century, yet there have been three times as many people employed at the end of the century. He notes that ``in the late
1880s, about 60 percent of the work force in both the United
States and Britain were in agriculture, domestic service, and
jobs related to horse transport. Today, only 3 percent of the
work force are in those occupations.'' During the twentieth century most workers moved from those jobs to manufacturing
and then service jobs. During the twenty-first century it's likely
that many, perhaps most, workers will move from hands-on
production work to information work. Along the way many
people will lose their jobs and their investments, but the result
will be a higher standard of living for everyone. If we're lucky,
fifty years from now we will be producing five times as much
output per person as we do today---unless government distorts
price signals, impedes coordination, and holds resources in unproductive uses.
\switchcolumn
长 期 担 任 《经济学家》 杂 志 副 主 编 的 马 克 莱(Norman Macrae)指出,英格兰自工业革命以来,大约三分之二在世纪
初还存在的工作岗位到世纪末就已经消失了,而在世纪末有相
当于世纪初三倍的人被雇用。他说: “19世 纪 80年代后期,
无论是美国还是英国,都有大约60\%的劳动力在农业、家政
服务以及与马车交通有关的行业工作。而今天,只有3\% 的劳
动力仍然在上述行业工作。” 20世纪中,绝大多数劳动者都从
制造业和服务业中转移出来。而在21世纪,看 起 来 很 多 (也
许是绝大多数)劳动者会从手工生产工作转向使用信息的工
作。在这条道路上,很多人会失去他们的工作和投资,但是结
果将会是每个人享受到更高的生活水平。如果幸运的话,50
年之后我们每个人将会生产出相当于现在5 倍的产品一 除非
政府扰乱价格信号,破坏协调以及把持资源用于非生产的
方面。
\switchcolumn*
In other words, the best way to ``preserve'' jobs is to unleash
the economy. Jobs will \textit{change}, but there will always be more
new jobs created than old ones lost. This is true even in cases of
technological progress; people get replaced by machines in one
field, but the higher level of capital investment in the economy
means a rising level of wages for other jobs.
\switchcolumn
换句话说,最好的 “保护”就业的办法是解开对经济的
束缚。工作会\textbf{改变},但是总会有比消失的旧的就业机会更多的
新就业机会被创造出来。在技术进步的情况下更是如此。人们
在一个领域被机器所代替,但是更高层次的资本投入经济当中
则意味着其他工作的工资标准的上升。

\switchcolumn*[\subsection{Price Controls\\价格管制}]
Controls on prices---including wages, the price of labor---are
perhaps the most direct way in which government distorts price
signals. Sometimes governments try to set minimum prices,
but more often they want to limit maximum prices. Price controls are usually implemented in response to rising prices. Prices
rise for several reasons. In a free market, a rising price usually
indicates either rising demand for the product or a reduction in
supply. In either case, there will be a tendency for resources to
move into that market to take advantage of the rising price,
which will tend to reduce or even reverse the price increase. (We
might note that, over the long run, in real terms, the only price
that consistently seems to rise is the price of human labor.
Looking back a hundred years or so, we see that prices of
goods---from wheat to oil to computers---have fallen, while the
real wage rate has quintupled in fifty years. The only thing getting more scarce in economic terms, that is, relative to all other
factors, is people.)
\switchcolumn
价格管制,包括管制劳动力价格即工资,也许是政府扭曲
价格信号的最直接方式。有时候政府会试图规定最低价格,但
是更多的时候他们是想限制最高价格。价格管制是为了应对价
格上涨而引入。价格上涨有几个原因。在自由市场中,价格上
涨常常既反映了对产品的需求上升,又反映了产品供应的减
少。无论是哪方面的原因,都会有人想把资源投入到那个市场
中去,以从价格上涨中获益,而这将会减缓甚至扭转价格上涨
的趋势。我们也许注意到,就长期趋势来说,在真实情况下,
唯一保持持续上涨的是劳动力价格。回头看看,大约一百年的
时间里,商品的价格都下降了,从小麦到石油到计算机价格都
下降了,唯有真实的工资水平在过去50年当中增长了 5 倍。
在经济条件下,唯一相对于其他要素总是更稀少的是人。
\switchcolumn*
\textit{Rent control} is a particularly pervasive example of price control. Every economist understands that rent controls produce
shortages of rental housing. If the controls are set so as to hold
rents below their market value, then people will demand more rental housing than they would otherwise. That is, the price set
by the state is not the market-clearing price: more people will
come to the city, or look for bigger apartments than they would
be willing to pay a market price for, or stay in large apartments
after the children move out, or seek to rent even though they
could afford to buy a house. But since rents are being held
below market value, investors prefer to invest in something on
which they can get a full market return, so the supply of housing won't increase to meet the demand.
\switchcolumn
\textbf{租金管制}是 一个很有说服力的价格管制例子。任何一个经
济学家都知道,租金管制会造成出租房屋数量的短缺。如果设
置管制是为了让租金低于市场价,那么人们会比不管制的情况
下要求更多的出租房屋。也就是说,政府规定的价格不是市场
出清价格:更多的人会涌入这个城市,或者寻找比他们在市场
价格下愿意承担的面积更大的公寓,或者在孩子们搬出去之后
仍然住在大面积的公寓里,或者在他们有能力买房子的时候仍
然租房。同时,由于租金被控制在市场价以下,投资者宁愿投
资到别的领域以得到完全的市场回报,于是住房的供应就不会
增长来满足需求。
\switchcolumn*
In fact, if rent control remains in effect, the supply may
shrink, as owners decide to live in their property rather than
rent it out, or deteriorating housing is not maintained or replaced. If landlords can't rent apartments to the highest bidder,
they will find other ways to choose among potential tenants;
they may take under-the-table bribes, known in New York City
as ``key money,'' or they may discriminate on the basis of race,
sexual favors, or some other non-price factor. In extreme circumstances, which may be seen in some neighborhoods of the South
Bronx, owners of apartment buildings that don't bring in
enough rent to cover the property taxes and thus can't even be
sold simply abandon them and try to disappear.
\switchcolumn
事实上,如果租金管制继续实施的话,住房供应也许会减
少,因为房主会决定住在他自己的房子里面而不是租出去,或
者是损坏的房屋得不到维护或者重建。如果房主不能把公寓租
给出价最高的人,他们就会用其他方式在申请人当中进行选
择;他们也许会接受桌子下的贿赂,这种贿赂在纽约市被称为
“钥匙费” ,或者他们会按照种族、性取向或者其他非价格因
素来进行歧视。在极端的情况下,例如在南布朗克斯的一些居
住区,公寓的业主得到的租金甚至还不足以用来缴纳物业税,因此无法把公寓出售,只能放弃自己的房产然后人间蒸发。
\switchcolumn*
As with so many kinds of government intervention, the
problems created by rent control lead to more intervention.
Landlords try to convert their unprofitable apartment buildings
to condominiums, so city councils pass laws restricting condo
conversions. In the market, tenants and landlords have good
reason to try to keep one another happy, but rent control means
that tenants are just a burden on the landlord, so landlords and
tenants end up fighting, and governments create landlord-ten-
ant commissions to regulate every aspect of their interaction.
Bribery and inside information become the best way to find an
apartment. The city council in Washington, D.C., once passed
an ordinance that would repeal rent control as soon as the vacancy rate rose above a certain level---indicating a sufficient
supply of available housing---but of course the supply of housing won't increase as long as rent controls are in place. It's no
wonder that the Swedish economist Assar Lindbeck wrote,
``Next to bombing, rent control seems in many cases to be the
most efficient technique so far known for destroying cities.''
\switchcolumn
就像政府很多其他的干预一样,由租金管制所产生的问题
导致了更多的干预。房主试图把他们的不赚钱公寓楼转成共管
公寓,于是市参议会通过了法律,限制普通公寓转换成共管公
寓。在市场条件下,租户和房主有很多理由保持愉快的关系,
但是租金管制意味着租户对房主来说是一个负担,于是租户和
房主就会不断争吵冲突,于是政府就建立房主----租户委员会来
规范他们之间关系的每一个细节。.腐败和内部信息成了找到一
套公寓的最佳办法。华盛顿特区的市参议会曾经通过一个法
令 ,规定一旦房屋空置率超过某个标准 --- 表示住房供应过剩
--- 政府就会进行租金管制,但是住房供应不会因为租金管制
而上升。毫无疑问,正如瑞典经济学家林德贝克(Assar Lindbeck)所说:“租金管制看起来是迄今为止已知的摧毁城市的
最有效技术之一,仅次于轰炸。”
\switchcolumn*
Controls aren't always designed to keep prices down. Some-
times government tries to set minimum prices, such as the minimum wage law. Perhaps no issue better illustrates the sometimes
counterintuitive nature of spontaneous order, the market
process, and the coordination function of prices. Eighty percent
of Americans consistently support an increase in the minimum
wage, and why not? The idea sounds good: it's hard to make a
living on, say, \$4 an hour, so why not set a minimum wage of
\$5? But just as maximum prices create shortages, minimum
prices produce surpluses. Workers whose productivity to an
employer is less than the legal minimum won't be hired at all.
Again, the price signal is distorted, and coordination can't
occur. We noted earlier that the market process produces a job
for everyone who wants one---except when the process isn't allowed to work. As economists William Baumol and Alan
Blinder (later a member of the Clinton administration) wrote in
their textbook \textit{Economics: Principles and Policies}, ``The primary
consequence of the \textit{minimum wage law} is not an increase in the
incomes of the least skilled workers but a restriction of their
employment opportunities.'' Employers will hire a skilled
worker instead of two unskilled workers, or invest in machinery,
or just let some jobs go undone. Older people say there used to
be ushers in movie theaters; maybe there still would be, if theaters could offer less than the minimum wage to people looking
for a first job. Instead, the teenage unemployment rate is several times as high as it was in the 1950s. The way to raise wages
is not to outlaw work for less than a certain wage but to increase
the accumulation of capital so that each employee can produce
more, and to increase the skills of each employee so he can produce more with the same tools.
\switchcolumn
价格管制并不总是用来降低价格。有时候政府会规定最低
价格,例如最低工资法。也许没有什么能比最低工资问题更能
显示自发秩序即市场经济有时与直觉相反的特性,以及价格的
协调功能了。80\%的美国人坚决支持提高最低工资,为什么不
呢?最低工资的主意听上去很不错:假如一小时挣4 美元很难
维持生活,那么我们为什么不把最低工资规定为5 美元呢?但
是如果说最高价格限制制造短缺,那么最低价格限制就会制造
过剩。就雇主来说,生产能力低于法定最低工资的工人,他根
本就不会雇用。价格信号在这里又一次被扭曲,经济协调也就
不会出现。我们先前注意到,市场过程会为每一个想要工作的
人创造出工作机会 --- 除非市场过程被阻止。正如经济学家鲍
摩 尔(William Baumol)和 布 赖 恩 德(Alan Blinder,后来成为克林顿政府的一员)在他们写 的教材《经济学:原理与政策》(\textit{Economics: Principles and Policies})中所说:“最低工资
法最主要的后果不是增加最低技能工人的收人,而是限制了他们被雇用的机会。”雇主会雇用一个熟练工人而不是两个非熟
练工人,或者投资购买机器,或者仅仅是削减一些职位。老人们说从前在电影院里有领位员。如果电影院可以给那些找第一
份工作的人提供低于最低工资的薪水,也许现在电影院仍然还会有领位员。相反,今天十几岁年龄段的失业率是1950年代
的好几倍。提高工资的真正办法不是禁止人们为少于某个标准的薪水而工作,而是增加资本的积累,让每个雇员能够生产更
多产品,让每个雇员的技能得到提升,以使他能够用同样的工
具生产出更多的东西。
\switchcolumn*
\textit{Farm price supports} are another example of minimum prices. In
any growing, noninflationary economy, we would expect prices
to fall gently; more production means that the real price of
everything, in terms of labor, is falling. Agricultural produce,
being the ``first'' products in any economy, would be the clearest
example of this. Indeed, over the past 200 years supplies of
grain and other basic farm produce have been rising, and prices
have been falling (when measured by hours of labor needed to
buy units of produce). With food more plentiful, we need fewer people working on farms. Falling prices send that signal to
farmers. That's why 53 percent of Americans were farmers in
1870, and only about 2.5 percent are today. That's good news;
it means all those people can produce something else, making
themselves and all the rest of us richer.
\switchcolumn
\textbf{农产品价格管制}是另一个规定最低价格的例子。在任何一
个不断增长的、没有通货膨胀的经济体中,商品价格都会逐渐
下降;更多的产品意味着所有东西的价格相对于劳动力价格来
说都是下降的。农产品在所有经济体当中的“第一” 产品,
是最清晰的例子。事实上,在过去二百年当中,谷物和其他基
本农产品的供应都在持续上升,价 格 在 持 续 下 降 (用购买一
定单位的农产品需要多少劳动时间来衡量)。随着食物越来越
丰富,需要在农业上工作的人越来越少。不断下降的价格向农
民们传递出了这个信号。这就是为什么1870年有53\% 的美国
人是农民,而今天农民只占2.5\%。这是件好事;因为这意味
着所有那些人可以制造其他的东西,让他们自己和其他所有人
更加富有。
\switchcolumn*
But starting in the 1920s the federal government decided to
keep farm prices high, to keep farmers happy. That is, it decided
to block the price signals that were telling farmers to move to
more profitable endeavors. It set minimum prices for farm produce and promised to buy enough of each product to keep the
price at that level. In return, farmers took some of their land
out of production. That's where we get the popular jibe that the
farm program ``pays farmers not to farm.'' Of course, farmers
aren't dumb. They put their worst land in the ``soil bank'' and
farmed their best land. Then, since the government would pay
an above-market price on anything they could produce on the
land in production, farmers improved their technology, fertilizer, and seeds to increase production. The government ended
up buying more crops than it had intended, piling up billions of
dollars in surplus produce. (Perhaps the only consolation to
American consumers and taxpayers is that the European Community has pursued similar but even more uneconomic programs, producing what European critics call ``wine lakes'' and
``butter mountains.'') Some of the surplus food was sent to poor
countries such as India---which sounds nice, except that it \textit{lowered} prices there and discouraged local farmers from producing,
thus helping to keep the countries poor and in need of the surpluses that American farmers kept producing and selling to the
U.S. government.
\switchcolumn
但是从20世纪20年代开始,为了让农民们高兴,联邦政
府决定让农产品价格保持高位。也就是说,政府决定阻止价格
信号的传递,而正是价格信号告诉农民应该转移到更有利可图
的工作上去。政府规定了农产品的最低价格,并且承诺购买各种农产品,数量足以保持农产品的价格水平。作为回报,农民
让他们的部分土地休耕。这就是那句流行的讽刺农业补贴计划
的 “给农民钱,让他别种地” 的来历。农民当然不是傻子,他们把最差的土地送进“土地银行” ,把最好的地拿来耕种。于是,由于政府愿意以高于市场的价格收购他们土地出产的任
何东西,农民们就不断改进他们的技术、化肥和种子,以提高
产量。政府不断买进比它预料的更多的粮食,将数以十亿计的
美元花在过剩农产品上。其中部分过剩食物被送到了穷国,如印度 --- 这听上去不错,只不过这些援助\textbf{压低}了那里的粮食价格 ,降低了当地农民种植粮食的意愿,因此这些穷国继续贫穷;而美国农民由于有人购买这些过剩农产品,于是继续不断地生产,继续把粮食卖给美国政府。也许对美国消费者和纳税人来说,唯一的安慰是欧盟也在执行类似的而且更加不经济的
计划,生产出的过剩农产品被欧洲批评家们称为“酒池肉林”。
\switchcolumn*
Farm programs have changed over the years, but the goal has
typically been to keep prices high and thus distort the price signals that would otherwise encourage farmers to go into more
productive lines of work.
\switchcolumn
很多年后,农业补贴计划进行了改革,但是其目标基本上
仍然是保持农产品的高价,因此扭曲了价格信号,而这个价格
信号本来应该鼓励农民进入生产价值更高的领域。
\switchcolumn*
Wage and price controls are the clumsiest possible intervention into the market's coordination process. They're the economic equivalent of Michael Jordan standing between you and
a friend, waving his arms, as you try to toss a basketball back
and forth.
\switchcolumn
工资和价格管制是对市场协调过程的最笨拙的干预手段。
他们就像经济上的迈克尔$\cdot$ 乔丹,站在你和你朋友中间,挥舞
着他的胳膊,而你则试图带球前后突破把球传给你的朋友。

\switchcolumn*[\subsection{Taxation\\税收}]
The clumsy interventions described above should sound
patently unfair and inegalitarian. Now let's consider an ever-popular form of coercion by which governments extract money
directly from those who earn it: taxation. Taxes reduce the return each individual gets from economic activity. Since one of
the important functions of income---including profits and
losses---is to direct resources toward their most highly valued
uses, an artificial reduction in the return has a distorting effect
on economic calculation. Defenders of taxation may argue that
a tax levied equally on all economic activity would be neutral in
its effects. The diverse and uncountable array of taxes levied by
contemporary governments---sales taxes, property taxes, inheritance taxes, luxury taxes, sin taxes, business-incorporation
taxes, corporate income taxes, Social Security taxes, and income
taxes levied at different rates on different people---would suggest that governments are not trying very hard to achieve a
neutral system of taxation. But even if they did try, they would
fail. Taxes always have different effects on different economic
actors. They drive the marginal supplier or the marginal purchaser out of the market. Since taxation is always coupled with
government expenditure, the combination can only have the effect of diverting resources from where consumers wanted them
used to some other use chosen by political officials.
\switchcolumn
上面描述的笨拙的干预手段看上去显然是不公平的,也是
不平等的。现在让我们来看一看一种曾经很受欢迎的强制形
式 ,通过这种强制,政府从那些挣钱的人那里直接把钱抽走,
这种形式就是税收。税收减少了每个个人从经济活动中所得到的回报。由于收入(包括利润和亏损)的一个最重要功能是
引导资源朝着最高效的利用方式流动,人为减少经济活动的回
报就会在经济计算中产生扭曲。为税收辩护的人也许会说,税
收的征集对所有经济活动都是平等的,因此它的影响是中性
的。可是,当代政府征收的税收种类繁多、不可胜数,销售
税 、财产税、遗产税、奢侈税、罪恶税\footnote{罪恶税 (sin tax),对 “有罪” 的商品如烟、酒 、赌博等开征的一种惩罚性税收。}、公司注册税、公司营业税、社会保障税以及所得税等,以不同的税率对不同的人强行征收。这说明政府并没有努力建立一个中性的、没有偏向的税收系统。但是即便他们尝试了,也肯定会失败。因为税收对不同经济要素的影响总是不同的。它们将边际上的卖方或者
边际上的买方赶出市场。由于税收常常伴随着政府开支,这些税收的唯一效果就是把资源从消费者希望使用的地方转移到政府官员选择的另外地方。
\switchcolumn*
Taxes inhibit the vital function of entrepreneurship by reducing the return the entrepreneur can earn by noticing and remedying a mis-allocation of resources. If you tax something, you
get less of it; taxing the rewards of entrepreneurship means that
we will get less entrepreneurship, less alertness to ways that resources could be shifted to serve consumers' needs better.
\switchcolumn
企业家通过发现和纠正资源的错误配置而获得回报,而税
收通过拿走这些回报而抑制了企业家精神的基本活力。你对什
么东西征税,什么东西就会减少。对企业家精神的回报征税,
企业家精神就会减少,也就意味着对资源的错误配置发出的警
报会减少,而这些资源本来可以转到更好地满足消费者需求的
使用方式上去。
\switchcolumn*
Taxes create a wedge between buyers and sellers, including
employers and employees, that can prevent productive ex-
changes from being made. If I'm willing to pay up to \$200 for
a suit, and you're willing to sell it for any price above \$190, we
have an obvious opportunity for an exchange that will benefit
both of us. But add on a 10 percent sales tax, and there will be
no price that we can agree on. If I'm willing to work for as little as \$30,000, and you value my services at \$35,000, then we
should be able work out a deal somewhere between those two
figures. But add on a Social Security tax of 15.3 percent, and a
federal income tax of 28 percent, and a state income tax, and
maybe a city income tax, and we won't be able to agree on a
price. If taxes were lower, there would be more money in the
private sector being directed to the satisfaction of consumer
demand, and more demand for workers and thus less unemployment.
\switchcolumn
税收在买卖双方包括雇主和雇员之间打人了一个楔子,阻
止了能够产生更多生产力的交换的出现。如果我愿意为一套西
服支付最高200美元,你也愿意以任何高于190美元的价格卖
掉它,我们显然很可能达成对双方有利的交换。但是如果加上
10\%的营业税,那么我们之间就不会再达成一致的价格了。如
果我愿意以3 万美元的薪水工作,而你认为我的服务值3. 5 万
美元,那么我们就能够在这两个数字之间达成协议。但是如果
再加上15. 3\%的社会保障税、28\% 的联邦所得税和一份州所
得税,也许再加上城市所得税,我们就不会就工资价格达成一
致了。如果税收低一些,就会有更多的钱留在私人部门用来满
足消费者的需求,还会产生对工人的更多需求,从而降低失
业率。
\switchcolumn*
High tax rates discourage work effort. Why work overtime if
the government will take half of what you earn? Why invest in
a risky business opportunity when the government promises to
take half of any profit but to let you bear the losses? In all these
ways, taxes reduce the productive effort directed toward serving
human needs.
\switchcolumn
高税率会打击工作努力。如果政府将拿走你一半收入的
话 ,你为什么要加班工作呢?如果政府拿走你的一半利润却让
你独自承担亏损的话,你为什么要把钱投到一个有风险的商业
机会上呢?税收就是这样减少了直接满足人们需要的生产
努力。
\switchcolumn*
High taxes may also encourage investors to put their money
into tax-sheltered investments rather than into projects whose
real return is greater in the absence of the tax differential. They
also induce people to spend money on wasteful but tax-deductible purchases like offices fancier than their business really
requires, vacations disguised as business travel, company auto-mobiles, and so on. Such expenditures may be worthwhile to
the people who make them; we know that when they spend
their own money on them. But the tax laws may encourage
over-investment in things for which people wouldn't spend their
own money. Finally, compliance with tax laws diverts resources
from producing other goods. Businesses and individuals spend
5.5 billion worker-hours each year on tax paperwork---the
equivalent of 2,750,000 workers who could be producing
goods and services that consumers want.
\switchcolumn
高税收也会鼓励投资者把钱放到不征税或者有税收优惠的
投资当中,而不是投到那些没有税收差异时真实回报本来应该
更大的项目当中去。高税收还会诱导人们把钱花在浪费但是能
够减少税收的购买上,如超过实际需要的办公室豪华装修,伪
装成出差的休假,公司配车等。这些花费也许对生产这些消费
品的人来说是好事;我们也知道他们是在花自己的钱。但是税
法也许会鼓励对某些东西的过度投资,而那些东西,如果是人
们花自己钱的话就不会买。最后,很多资源用在了执行税法
上 ,而这些资源本来可以用来生产其他东西。企业和个人每年
要花费55亿个工作小时用在填写税收表格之类的文书工作上,
相当于275万个工人的劳动力,这些本来可以用在生产消费者
需要的产品和提供消费者需要的服务上面。

\switchcolumn*[\subsection{Regulation\\管制}]
A book could easily be written on the effects of government
regulation on the market process. Here we can look only at a
few basic points. We should begin by noting that some rules,
commonly known as ``regulations,'' are an inherent part of the
market process in a system of property rights and the rule of
law. Prohibitions on polluting other people's air, water, and
land, for instance, are an acknowledgment of their property rights (chapter 10 will discuss in slightly more detail the kinds
of rules that are effective and appropriate). Rules requiring people to live up to the terms of contracts, such as prohibitions on
fraud, are also part of the common-law framework of the market process.
\switchcolumn
关于政府管制对市场经济的影响,随随便便就可以写成一本书。在这里我们只看几个基本的要点。首先,我们注意到,
一 些 规 则 (常常被认为是“法规”)是产权和法治制度下市场
经济的一个固有部分。禁止污染别人的空气、水和土地实质上
是对他们产权的承认(在第十章中我们将更详细地讨论那些
有效运行和恰当的规则)。那些要求人们遵守合同条款的规
则 ,如禁止欺诈,也是市场经济下的普通法体系的一部分。
\switchcolumn*
Unfortunately, most of the regulations promulgated by legislative bodies and administrative agencies these days don't fall
into those categories. The regulations that concern us here are
explicitly designed to bring about an economic outcome different from what the market process would have produced. Sometimes we can point to specific problems generated by such
regulations: rent controls reduce the supply of housing; airline
regulation raises the cost of air travel; a lengthy drug-approval
process keeps lifesaving and pain-relieving drugs out of the
hands of consumers. Often, however, it is more difficult to assess the effect of a regulation, which is to say, to figure out what
\textit{would} have happened if the market's coordination process had
been allowed to work. It is precisely the \textit{least} obvious absences
of coordination that regulation may invisibly prevent market
participants from discovering and remedying. If we are persuaded that the market process works to satisfy consumer demands---that is, to allocate resources in the way that will
produce the most value for a given level of resources---then we
will conclude that there are always costs to regulations that prevent voluntary exchanges from being made.
\switchcolumn
不幸的是,今天绝大多数由立法机构和行政部门颁布的管
制法规不在此列。这些管制明显是为了某些经济目标而设计
的,而这种经济目标是不同于市场过程的自然结果的。有时候
我们明确能够知道某些问题是由某项管制而产生的:租金管制
减少了住房供应;航空管制提高了航空旅行的价格;冗长的药
品审批程序让消费者得不到能够救命和减轻痛苦的药物。但更常见的情况是我们很难对一项管制的影响做出评估,也就是说 ,我们很难描述出假如市场协作过程能够正常运转的话\textbf{本来应该}发生什么。管制在无形中让市场参与者无法发现和纠正\textbf{并不明显}的市场协作失调。如果承认市场经济的任务是满足消费者的需求 --- 即让给定的资源条件产生最大价值的资源配置方
式,那么就应该得出结论:管制的代价是总会阻止自发交换的出现。
\switchcolumn*
Robert Samuelson wrote in \textit{Newsweek} in 1994:
\switchcolumn
罗伯特$\cdot$ 萨缪尔森1994年 在 《新闻周刊》上指出:
\switchcolumn*
\begin{quote}
The totality of federal regulations now comes to 202 volumes
numbering 131,803 pages. This is 14 times greater than in 1950
and nearly four times greater than in 1965. There are 16 volumes
of environmental regulations, 19 volumes of agricultural regulations and 2 volumes of employment regulations.
\end{quote}
\switchcolumn
\begin{quote}
联邦政府的法规总数现在达到了 202部 ,131803页。
比 1950年多出14倍,几乎比1965年多出4 倍。其中有
16部环境保护法规、19部农业法规和2 部雇佣法规。
\end{quote}
\switchcolumn*
If you run a business, you'd better know what's in all those
regulations---and the 60,000 or so pages of new regulations
(some of which replace or alter old regulations) published each
year in the \textit{Federal Register}. About 130,000 people work in federal regulatory agencies, and the economist Thomas D. Hopkins, writing in the Journal of Regulation and Social Costs,
estimates that regulation costs our economy some \$600 billion a year in lost output---resources that could have gone to satisfying consumer needs. Clifford Winston of the Brookings Institution estimates that ``society has gained at least \$36--\$46
billion (1990 dollars) annually from deregulation,'' which suggests that for all the recent deregulation of transportation,
communications, energy, and financial services, the regulatory
burden has barely been reduced.
\switchcolumn
如果你开一家公司,你最好知道所有相关的法规 --- 在每
年 出版 的《联邦公报》 当中,最 新 的 法 规 (旧法规每年都会进行修改或用新法替代)有大约6 万页。经济学家霍普金斯(Thomas D. Hopkins)在 《管制与社会成本》月刊中指出:大
约有13万人在联邦管制机构工作。估计这些管制让经济每年
损失6000亿美元的产值,而这些资源本来可以用来满足消费
者的需求。布鲁克林研究所的克利福德$\cdot$ 温 斯 顿(Clifford Winston)估计,“如果解除管制,整个社会将会每年增加360亿$\sim$460亿美元的收入(按 1990年的美元价格计算)”。对当前交通、通信、能源和金融服务来说,管制的负担根本没有
减少。
\switchcolumn*
Winston also writes that ``economists found it difficult to
predict, or even consider, changes in firms' operations and technology, and consumers' responses to these changes, that developed in response to regulatory reform.'' That is, the discoordinations produced by interference with the market
process are so great and so complex that it is very difficult to assess them and predict the improvements in coordination that
would occur under deregulation. To take just one example,
economists recognized that the regulation of trucking prices
and routes by the Interstate Commerce Commission was producing major inefficiencies. They predicted that deregulation
could save consumers and businesses \$5 billion to \$8 billion a
year by making trucking more efficient. They were right; in
fact, a 1990 study for the U.S. Department of Transportation
estimated annual savings from the 1980 deregulation at about
\$10 billion. What the economists did not predict was a far
more important outcome: cheaper, more reliable trucking allowed firms to reduce their inventory, knowing that they would
be able to \textit{get} their products to buyers when they were needed.
The inventory savings, which amounted to some \$56 billion to
\$90 billion a year by the mid-1980s, dwarfed the direct savings
in trucking costs.
\switchcolumn
温斯顿还提到:“经济学家发现,很难预测对管制进行改
革会有什么样的结果,放松管制对公司运作和技术发展会产生
的变化,以及消费者对这些变化如何反应,都很难作出预测和
应对。” 也就是说,干预市场进程所产生的紊乱是如此巨大、
如此复杂,以至于很难评估和预测在解除管制的情况下市场的
自动协调改进会出现什么结果。我只举一个例子,经济学家发
现州际贸易委员会对价格和路线进行的管制制造了卡车运输的
低效率。他们预计解除管制将提高卡车运输的效率,会为消费
者和企业每年节约50亿$\sim$80亿美元的费用。他们的估计是正
确的,1990年一份为美国交通部所作的研究估算,从 1980年
解除管制以来,每年卡车运输节省的费用大约是100亿美元。
而经济学家没有预测到一个远为重要的结果:更便宜、更可靠
的卡车运输让公司减少他们的库存,因为他们知道能够在需要
的时候及时把产品\textbf{运给}买方。而库存减少节省出来的费用远远超过了卡车运输直接节省的费用。1980年代中期每年节约的费用大约是560亿$\sim$900亿美元。
\switchcolumn*
The real motivation for regulation is often self-interest in the
worst sense, an attempt to get something through government
coercion that you couldn't get through the actions of consumers. There are all kinds of such so-called transfer-seeking,
many of which are discussed in chapter 9. You can get a higher
tax imposed on a competitive industry than on your own. If
you're a big company, you can support regulations that will cost
large and small companies similar amounts of money, hurting
the small companies proportionally more. You can get a tariff to
protect your product from foreign competition. You can get a regulation that makes it cheaper for customers to buy your
product than your competitor's. You can get a licensing law to
limit the number of people in your industry, and on and on. All
these regulations distort the market process and move resources
away from their highest-valued use.
\switchcolumn
管制的真实动机常常来自人们自利动机最坏的方面,也就
是人们试图通过政府强制来得到通过市场竞争无法得到的东
西,其中包括各种各样的所谓的“寻求转移” 行为。我们将在第九章当中进行讨论。你可以寻求让更高的税收加在与你竞
争的行业身上,而不是加在你自己所在的行业身上。如果你是
一家大公司,你就可以支持那种让公司开支巨大但小公司也将
支出大致相同费用的管制,因为按比例来说小公司将会受伤害
更重。你可以寻求一种管制,让你的产品价格比竞争对手更便
宜。你还可以寻求通过一个执业资格法,限制你所在行业的从
业者数量,等等。所有这些管制都扭曲了市场进程,让资源从
得到最有效利用的地方转移走。
\switchcolumn*
But these days many regulations are advanced by people
who generally believe them to be in the public interest, people
who may even believe firmly in the market process except when
regulation seems really necessary. Regulations are enacted to
guarantee safety in consumer products; to forbid discrimination
on the basis of race, sex, religion, national origin, marital status,
sexual orientation, personal appearance, or Appalachian origin;
to reduce inconveniences faced by disabled people; to ensure
the efficacy of pharmaceutical drugs; to guarantee access to
health insurance; to discourage corporate layoffs; and for myriad other noble causes. It's hard to argue with the goals of any
of these regulations. We all want a society of safe and effective
products, free from discrimination, where everyone has health
insurance and a secure job.
\switchcolumn
但是最近,很多管制却是被那些真心相信自己是为公众利
益考虑的人所推动的,这些人当中有的也许甚至还坚定地相信
市场,只不过认为只有在必要的时候才能实行管制。例如,为
了确保消费品的安全,为了防止消费行为中的种族歧视、性别
歧视、宗教歧视、民族歧视、婚姻状况歧视、性取向歧视、长
相歧视、对阿巴拉契亚山区人的歧视,为了减少残疾人的不
便,为了确保药物疗效,为了人人参加医疗保险,为了减少公
司雇员失业,以及无数其他的崇髙理由,你很难就这些管制的
目标进行争论。我们都希望商品安全好用、没有歧视、人人拥
有健康保险、人人都有一份稳定的工作。
\switchcolumn*
But the attempt to realize such goals by regulation is self-defeating. It substitutes the judgment of a small group of fallible
politicians for the results of a market process that coordinates
the needs and preferences of millions of people. It sets up static,
backward-looking rules that can never deal with changing circumstances as well as voluntary exchange and contract. No one
regulation will destroy the market process. But each one acts
like a termite, eating away at the structure of a system that is
rugged but not indestructible. And if regulation does indeed
cost our economy anywhere near \$600 billion, then it is costing
lives. A 1994 study from Harvard University's Center for Risk
Analysis found that our command-and-control regulatory system may be costing as many as 60,000 lives a year, by spending
resources on negligible risks, leaving less money for people to
spend on protecting themselves from bigger but less dramatic
risks. As Aaron Wildavsky of the University of California at
Berkeley wrote, wealthier is healthier and richer is safer. As people get richer, they purchase more health and safety---not just
medical care, but better nutrition, better sanitation, shorter
work hours, safer workplaces and kitchens. The cost of every regulation proposed to improve health or safety should be
weighed against the health costs that will be incurred by individuals' having less wealth. Also, Wildavsky argued, competitive institutions and processes produce better results over time
than centralized systems, so the competitive market process is
more likely to develop advances in health and safety than are
more heavily regulatory or bureaucratic systems.
\switchcolumn
但是通过管制来实现这些目标无疑会弄巧成拙。因为管制
是用一小群容易犯错的政客的判断来代替那结合了数以百万计
人们需求和偏好的市场过程。它建立的是一种静止的、向后看
的规则,不可能应对不断变化的环境,以及自愿的交换和合
同。没有一种管制能摧毁市场经济。但是每一项管制就像一只
白蚁,在整个系统的结构内部不停地蛀蚀,外面看上去虽然完
好,但系统并不是不可摧毁的。如果管制实际上让我们损失
6000亿美元的话,那么它也实际上是在让我们的生命遭到损
失。哈佛大学风险分析中心的一项研究发现,依靠命令与控制的管制系统每年造成6 万个生命损失。这方面的主要原因是政
府把资源用在防止一些微不足道的风险上,只给人们留下很少
的钱用来在更大的、但不那么戏剧性的风险面前保护自己。正
如加州大学伯克利分校 的阿龙$\cdot$ 维 达 夫 斯 基(Aaron  Wildavsky)所说,越富裕就越健康,越有钱就越安全。不仅仅对
医疗制度来说如此,而且更多的营养、更卫生的环境、更短的
工作时间、更安全的工作场所和厨房,都依赖于财富的增长。
每一项以改善健康或者安全为目的的管制,其成本都应该把人
们因为贫穷导致健康状况恶化而造成的医疗成本的增加计算在
内。维达夫斯基继续分析,随着时间的推移,竞争的制度和过
程会比中央计划体制产生更好的结果,因此相比管制或者官僚
体制,竞争的市场过程更能在健康和安全上取得进步。

\switchcolumn*[\subsection{International Trade\\国际贸易}]
One of the important applications of the principle of comparative advantage is international trade. To an economist there is
nothing really special about international trade; individuals
make trades when both of them expect to benefit, whether they
live across the street, in different states, or in different countries.
\switchcolumn
比较优势原则最重要的一个应用就是国际贸易。对一个经
济学家来说,国际贸易并没有什么特别的;只要双方都认为可
以从中获利,个人就会与其他人进行贸易,无论他们是生活在
街对面、在别的州还是在别的国家。
\switchcolumn*
Since 1776, when Adam Smith demonstrated the benefits of
free trade, there has been little \textit{intellectual} debate on the subject.
More than most economic topics, the debate over trade has
been spurred by special interests seeking advantages from government that they could not gain in the marketplace.
\switchcolumn
自从1776年 亚 当 $\cdot$ 斯密证明了自由贸易的好处之后,就
极少会有\textbf{知识分子}对这个问题进行争论。相比较绝大多数经济
学话题而言,关于贸易问题的争论多是因一些特殊利益而起,
目的多半是为了通过政府来寻求从市场上得不到的好处。
\switchcolumn*
Whenever two individuals make a trade, both expect to benefit; and both theory and observation tell us that more often
than not both parties do benefit, and the level of wealth in society is enhanced. The division of labor allows people to specialize
in what they're best at and to exchange with those who specialize in something else. As Smith wrote, ``It is the maxim of every
prudent $\ldots$ family, never to attempt to make at home what it
will cost $\ldots$ more to make than to buy$\ldots$ What is prudence
in the conduct of every private family can scarce be folly in that
of a great kingdom.''
\switchcolumn
无论何时两个人进行交易,双方希望从中获益;而无论是
理论还是观察都告诉我们常常都是双方受益,整个社会的财富
水平因此得到了提高。劳动分工让人们能够在他最擅长的领域
进行专业化的工作,并且与那些在别的领域进行专业化工作的
人进行交换。正如斯密所说:“这是每个谨慎的家庭的原则,如果一件东西自己生产的成本超过了购买的价格,就永远不要
自己生产……对每个私人家庭来说是谨慎的行为,对一个大的
王国来说也绝不是愚蠢的。”
\switchcolumn*
That is, it's usually best to sell where you can get the highest
price and buy where you can get the lowest price. But somehow, the drawing of national boundaries confuses people's
thinking on the benefits of trade. Maybe it's because ``balance of
trade'' statistics are calculated on a national basis. We could just
as well calculate the balance of trade between New York and
New Jersey, or between Massachusetts and California. For that
matter, you could calculate your own balance of trade between
yourself and everyone you deal with. If I did that, I would have
huge trade deficits with my grocer, my dentist, and my department store, because I buy a great deal from them and they
never buy anything from me. My only trade surpluses would be
with my employer and the publisher of this book, because I buy
almost nothing from them. What would be the sense of such
calculations? I expected to benefit from each transaction, and
the only balance I care about is that my income exceed my expenditures. The best way to make that happen is to concentrate
on doing what I do best and let others do what they do best.
\switchcolumn
也就是说,以最高的价格卖出、以最低的价格买进总是最
好的。但是不知为什么,国家边界的划分竟会扰乱人们对贸易
的好处的看法。也 许 这 是 因 为 “贸易平衡” 的数据是以国家
为基础进行统计。但谁能计算纽约和新泽西,或者马萨诸塞和
加利福尼亚之间的贸易平衡吗?如果你可以计算自己和与你有
过交易的人之间的贸易平衡,就会发现,你与杂货店、牙医和
百货商场之间有巨大的贸易逆差,因为你从他们那里买了大量
的商品和服务,而他们从来没从你那里买过任何东西。我的唯
一贸易顺差是和我的老板之间,以及和出版这本书的出版商之
间,因为我没有从他们那里购买过任何东西。可是这样的计算
有什么意义呢?我会从每一次交易当中得到获益,我所关心的
唯一平衡是我的收入和支出之间的平衡。保持这种平衡的最好
办法是集中精力把我自己的工作做到最好,同时让别人也把他
们的工作做到最好。
\switchcolumn*
The very notion of a ``balance of trade'' is misguided. Trade
has to balance. Just as an individual cannot long consume more
than he produces (except if he is a thief or the beneficiary of
gifts, charity, or government transfer payments), all the individuals in a country cannot consume more than they produce, or
import more than they export. As pleasant as it would be to
imagine, producers in other countries will not give us their
products for free or in return for dollars that are never exchanged for our goods and services. A national ``balance of
trade'' is just a composite of all the trades made by individuals
in the nation; if each of those trades makes economic sense, the
aggregate cannot be a problem.
\switchcolumn
“贸易平衡” 的概念本身纯粹是一个误导。贸易必然是平
衡的。就像一个人不可能长期地消费超过他的生产,除非他是
一个贼,或者得到了大礼包、慈善捐助,或者是政府转移支付
系统的受益者。一个国家的所有人不可能消费掉超过他们生产
的东西,或者进口超过出口。想象一下,别国的生产者不再免
费把他们的产品给我们,也不会把换回的美元拿回去,却从不
拿来交换我们的产品和服务,事情本来该如此愉快。国家间的
“贸易平衡” 就像在国家内部的个人之间的所有贸易一样,如
果每一次这样的交易在经济上是可行的,集合起来也不可能有
问题。
\switchcolumn*
Frederic Bastiat pointed out that a nation could improve its
balance of trade by loading a ship with exports, recording the
departure of the ship, and then sinking it outside the three-mile
boundary. Goods were exported, none were imported, and the
balance of trade is favorable. Clearly that would not be a sensible policy.
\switchcolumn
巴斯夏指出,一个国家要改善它的贸易平衡,只需要把一
艘船装满出口货物,出发的时候登记一下,然后让它沉到三英
里领海线之外就可以了。货物已经出口了,没有进口 一件东
西,贸易是顺差。显然那并不是一个明智的政策。
\switchcolumn*
The real problem may be a fundamental economic mistake:
regarding exports as good and imports as bad. We see this fallacy in every discussion of trade negotiations. Newspapers always report that the United States ``gave up'' some of its
restrictions on imports in return for similar ``concessions'' from
other countries. But we're not giving up anything when the
U.S. government lets American consumers buy from foreign
suppliers. The point of economic activity is consumption. We
produce in order that we may consume. We sell in order to buy.
And we export to pay for our imports. For each participant in
international trade, the goal is to acquire consumption goods as
cheaply as possible. The benefit of trade is the import; the cost
is the export.
\switchcolumn
真正的问题也许是一个基本的经济学错误:出口是好事,
进口是坏事。我们在每一次贸易谈判的讨论中都看到了这个错
误。报纸经常报道美国“放弃” 了对某些进口商品的限制,
换回了其他国家的类似“让步”。但是当美国政府允许美国消
费者从外国供应商那里购买商品的时候,我们并没有放弃任何
东西。经济活动的目的是消费。我们生产是为了消费。我们卖
东西是为了买东西。我们出口是为了有钱进口。对每一个国际
贸易的参与者来说,目的就是以尽可能便宜的价格得到消费
品。进口是贸易的收益,而出口则是贸易的成本。
\switchcolumn*
During his 1996 presidential campaign, Pat Buchanan stood
at the port of Baltimore and said, ``This harbor in Baltimore is
one of the biggest and busiest in the nation. There needs to be
more American goods going out.'' That's fundamentally mistaken. We don't want to send any more of our wealth overseas
than we have to in order to acquire goods from overseas. If
Saudi Arabia would give us oil for free, or if Japan would give us
televisions for free, Americans would be better off. The people
and capital that used to produce televisions---or used to produce things that were traded for televisions---could then shift
to producing other goods. Unfortunately for us, we don't get
those goods from other countries for free. But if we can get
them cheaper than it would cost us to produce them ourselves,
we're better off.
\switchcolumn
在 1996年总统竞选中,帕 特 $\cdot$布坎南在巴尔的摩的港口
上说:“巴尔的摩港是这个国家最大和最繁忙的港口之一。要
让更多的美国货物从这里出口。” 这基本是错误的。我们不想
送到海外去的财富超过我们为了从海外得到货物而必要的出
口。如果沙特阿拉伯愿意免费给我们石油,或者日本愿意免费
给我们电视机,美国人的生活状况只会更好。用来生产电视机
(或者用来生产交换电视机的产品)的劳动力和资本就能够转
移出来生产其他商品。不幸的是,我们不会免费从别的国家得
到这些东西。但是如果我们能够以比我们自己生产这些东西的
成本更低的价格得到它们,我们的生活状况也会得到改善。
\switchcolumn*
Sometimes international trade is seen in terms of competition between nations. We should view it, instead, like domestic
trade, as a form of cooperation. By trading, people in both
countries can prosper. And we should remember that goods are
produced by individuals and businesses, not by nation-states.
``Japan'' doesn't produce televisions; ``the United States'' doesn't
produce the world's most popular entertainment. \textit{Individuals},
organized into partnerships and corporations in each country,
produce and exchange. In any case, today's economy is so globally integrated that it's not clear even what a ``Japanese'' or
``Dutch'' company is. If Ford Motor Company owns a controlling interest in Mazda, which produces cars in Malaysia and
sells them in Europe, which ``country'' is racking up points on
the international Scoreboard? The immediate winners would
seem to be investors in the United States and Japan, workers in
Malaysia, and consumers in Europe; but of course the broader
benefits of international trade will accrue to investors, workers,
and consumers in all those areas.
\switchcolumn
国际贸易有时候被看作某种国家间的竞争。实际上与此相
反 ,我们应当把国际贸易看成像国内贸易一样,是合作的一种
形式。通过贸易,双方国家的人民都会变得更富裕。我们应该记住商品是由个人和企业,而不是由国家所生产的。 “ 日本”并不生产电视机,“美国” 也不生产最受世界欢迎的娱乐。是无数\textbf{个人}在各个国家当中组成合作组织或者企业,进行生产和
交换。而且,今天的经济已经在全球范围内进行整合,已经分
不清哪个是“ 日本” 公司、哪 个 是 “荷兰” 公司。如果福特
汽车公司在马自达公司拥有控制性的股份,而在马来西亚生
产 ,在欧洲销售,那么在国际记分牌上,其销售成绩应该算在
哪 个 “国家” 头上呢?当前的贏家看上去是美国和日本的投
资者、马来西亚工人和欧洲消费者,但是国际贸易当然将会给
所有这些领域的投资者、工人和消费者带来更广泛的利益。
\switchcolumn*
The benefit of international trade to consumers is clear: we
can buy goods produced in other countries if we find them better or cheaper. There are other benefits as well. First, it allows
the division of labor to work on a broader scale, enabling the
people in each country to produce the goods at which they have
a comparative advantage. As Mises put it, ``The inhabitants of
[Switzerland] prefer to manufacture watches instead of growing wheat. Watchmaking is for them the cheapest way to acquire
wheat. On the other hand the growing of wheat is the cheapest
way for the Canadian farmer to acquire watches.''
\switchcolumn
国际贸易对消费者的好处是显而易见的:我们可以购买其
他国家生产的更好或者更便宜的东西。除此之外,还有其他的
好处。首先,它使得劳动分工的范围更广,让每个国家的人们
能够生产拥有比较优势的产品。就像米瑟斯所说:“瑞士的居
民愿意制造手表而不是种小麦。制造手表对他们来说是得到小
麦的成本最低的办法。从另一方面来说,种植小麦对加拿大农
民来说是得到手表的成本最低的办法。”
\switchcolumn*
A great advantage of the price system is that it gives us one
standard by which to determine what goods any of us should
produce. Should we produce coffee, corn, radios, movies, or
flange-making machines? The answer is, whichever one will
give us the greatest profit. The economist Michael Boskin of
Stanford University got in hot water when he was chairman of
President George Bush's Council of Economic Advisers for reportedly saying something that was absolutely true: A dollar's
worth of potato chips is worth just as much as a dollar's worth
of computer chips, and it doesn't matter which one you produce. A country as technologically advanced as the United
States is going to produce a lot of high-tech products, though in
many cases we produce the designs here---which is where we
get the most profit---and then have the actual computer chips,
televisions, and so on produced where production wages are
cheaper. We also seem to have a huge comparative advantage in
producing popular culture: movies, television, music, computer
games, and so on. And despite our technological advancement,
we have vast amounts of rich farmland and highly productive
farmers, so we also produce many agricultural products more
cheaply than anyone else. Contrary to mercantilist notions, a
number of economies have prospered through the export
mainly of relatively unprocessed materials such as timber, meat,
grain, wool, and minerals. Just think of Canada, the United
States, Australia, and New Zealand. Others have prospered as
traders and manufacturers, despite a decided lack of natural resources. Think of Holland, Switzerland, Britain, Japan, and
Hong Kong. The key is free markets, not specific resources or
products.
\switchcolumn
价格体系的一大好处是提供了一个标准,让我们能够决定
自己应该生产什么东西。我们应该生产咖啡、棉花、收音机、
电影还是做珐琅边的机器?答案是:任何能够给我们带来最大
利润的东西。斯坦福大学的经济学家博斯金(Michael Boskin)
在担任老布什总统经济顾问委员会主席时说了一句话,让他备
受指责,而这句话是绝对正确的:一美元的薯片和一美元的芯
片价值是一样的,你生产哪个都无所谓。技术先进的国家如美
国会生产很多高科技产品,尽管在很多时候我们只是在国内进
行 设 计 (这是我们获得最高利润的方式),然后把实际的电脑
芯片、电视机等产品生产放在生产线工人工资较低的国家。看
起来美国在生产流行文化上也拥有巨大的比较优势:电影、电视 、音乐、电脑游戏等。而除了技术很先进之外,美国还有面
积巨大的肥沃农场和生产效率极高的农民,因此,美国的很多
种农产品也比其他任何国家都要便宜。与重商主义理论相反的
是,很多经济体主要通过出口未经加工的原料如木材、肉、谷
物、羊毛、矿物等获得了繁荣。想想加拿大、美国、澳大利亚
和新西兰。其他很多国家尽管极度缺乏自然资源,但却通过贸
易和制造业而获得繁荣。想想荷兰、瑞士、英国、日本和中国
香港。关键是自由市场而不是某些特定源或产品。
\switchcolumn*
Remember, it's not necessary that every country have an \textit{absolute} advantage at producing something; it will always have a
\textit{comparative} advantage in something. Even if Liz Claiborne is the
best typist in her company, she's still going to design clothes
and hire someone else to type. Even if Americans can produce
every conceivable product more cheaply than Mexicans, both
countries will still gain from trading, because Mexican firms will make the goods they are \textit{relatively}---even if not absolutely---more efficient at producing.
\switchcolumn
请记住,并不是每个国家都必须在生产某些产品上有\textbf{绝对优势},但它总会在某些产品的生产上有\textbf{比较优势}。尽管丽诗 $\cdot$ 卡邦\footnote{丽诗$\cdot$卡邦(Liz  Claiborne),美国著名服装和奢侈品牌。1976年创立。丽诗$\cdot$卡邦是公司创始人之一。}是她们公司最好的打字员( 但她仍然需要别人给她设计服装,也需要雇别人替她打字。尽管美国人在每一种所能想到的产品生产上都能够比墨西哥人成本更低,两个国家仍然会从贸易当中获益,因为墨西哥的公司能够以\textbf{相对}有效率(尽管不是绝对有效率)的方式生产某些产品。
\switchcolumn*
International trade also makes possible economies of scale
(that is, the efficiencies that companies can achieve by producing in large quantities), which couldn't be achieved in smaller
national economies. That's less important for American companies, which already have the world's largest market, than for
companies in Switzerland, Hong Kong, Taiwan, and other
small nations. But even American companies, especially if they
produce something for a narrow market, can reduce their unit
costs by selling internationally.
\switchcolumn
国际贸易还会扩大经济体的范围,也就是说公司的生产能
力通过制造更多产品所能达到的效率,而这在一个小国家的经
济体中是无法达到的。这一点对美国公司来说重要性相对不那
么高,因为美国已经拥有了世界上最大的市场,但对瑞士、中
国香港、中国台湾和其他小国家和地区的公司来说就尤为重要
了。但 是 美 国 公 司 ,尤其是那些为专业化的狭窄市场生产
产品的时候,也能够通过在国际范围内销售产品而降低他们的
单位成本。
\switchcolumn*
Free international trade is an important competitive spur to
domestic companies. American cars are better than they were
twenty years ago because of the competition from Japanese and
other foreign companies. According to Brink Lindsey, a trade
lawyer, integrated steel manufacturers have also improved their
efficiency in response to foreign competition, and ``American
semiconductor manufacturers, faced with brutal Japanese competition in high-volume memory chips, have improved their
manufacturing efficiency and concentrated resources on their
own strengths in design-intensive logic chips.''
\switchcolumn
自由的国际贸易对本地企业来说是一个重要的竞争刺激。现在美国生产的汽车比二十年前更好,是因为有来自日本和其
他外国公司的竞争。按照贸易律师布林克$\cdot$ 林赛(Brink Lind­sey) 的说法,美国联合钢铁生产厂商为了应对来自国外的竞
争提髙了生产效率,“美国的半导体生产厂商面对来自日本的
髙容量存储芯片的严酷竞争,也提高了他们的制造效率,并且
将资源集中在他们自己的强项 --- 精密设计逻辑芯片上。”
\switchcolumn*
When governments restrict international trade at the behest
of domestic interest groups, they impede the information and
coordination process of the market. They ``protect'' some industries and jobs, but only at the expense of the whole economy.
Protectionism prevents capital and labor from moving to uses
that would better satisfy consumer demand. Like laborsaving
machinery, imports reduce employment in one part of the economy, allowing those workers to move to more productive jobs.
\switchcolumn
当政府在国内利益集团的压力下对国际贸易进行限制的时
候 ,他们就阻碍了市场的信息传递和协作过程。他 们 “保护”
了一些产业和工作职位,却让整个经济付出代价。贸易保护主
义阻止了资本和劳动力向更能满足消费者需求的地方转移。就
像节省劳动力的机器一样,进口减少了某部分经济的就业,让
这些工人转移到生产率更高的岗位上。
\switchcolumn*
The nineteenth-century economist Henry George pointed
out in \textit{Protection or Free Trade} that nations try to embargo their
enemies to restrict their foreign trade in time of war, which is
much like protectionism: ``Blockading squadrons are a means
whereby nations seek to prevent their enemies from trading;
protective tariffs are a means whereby nations attempt to prevent their own people from trading. What protectionism
teaches us, is to do to ourselves in time of peace what enemies
seek to do to us in time of war.''
\switchcolumn
19世纪的经济学家亨利$\cdot$ 乔治(Henry George)在 《贸易保护还是贸易自由》(\textit{Protection or Free Trade} )  一 书中指
出,国家试图对敌人进行禁运,来限制他们在战争期间与国外
的贸易,这和贸易保护主义非常相像:“用军舰封锁对方的港
口是国家阻止敌人贸易的一种手段;保护性关税则是国家阻止
自己人民进行贸易的手段。贸易保护主义教给我们的是,用战
时对待敌人的方式,在和平时期对付我们自己。”
\switchcolumn*
Finally, a great benefit of international trade is to reduce the
chances of war. Nineteenth-century liberals said, ``When goods
cannot cross borders, armies will.'' Trade creates people on both
sides of national borders with an interest in peace and increases
international contacts and understanding. That doesn't mean
there will never be a war between countries practicing free
trade, but commercial relations do seem to improve the
prospects for peace.
\switchcolumn
最后,国际贸易的一大好处是降低战争的可能性。19世
纪的自由主义者说:“当商品无法跨越边界的时候,军队就会
跨过去。”贸易给国家边界两边的人们带来的是和平的利益和
不断增加的国际接触和相互理解。这并不是说在进行自由贸易
的两个国家之间永远都不会发生战争,但是商业关系的确能够
促进和平的前景。

\switchcolumn*[\subsection{Government and the Productive Process\\政府与生产过程}]
In all these ways and more, government interferes with the cooperation and coordination that are the market process. Introducing government intervention into the market is like
introducing a monkey wrench into a complex machine. It can
only reduce its efficiency. Fortunately, the market process is
more like a computer network than a machine; instead of coming to a complete stop, the market process routes information
around the destructive intervention. Its efficiency is reduced
but not halted. Each intervention into the market may cost a
great economy only a little. Adam Smith once encountered a
young man who bemoaned some new policy, saying, ``This will
be the ruin of Great Britain,'' to which Smith replied, ``Young
man, there is a deal of ruin in a nation.'' Similarly, the great
British historian Thomas Babington Macaulay wrote, ``It has
often been found that that profuse expenditure, heavy taxation,
absurd commercial restriction, corrupt tribunals [etc.] have not
been able to destroy capital so fast as the exertions of private
citizens have been able to create it.''
\switchcolumn
通过上面这些途径以及其他更多的途径,政府干预市场经
济的合作与协调。把政府干预引入市场当中就像把一只猴子带
到一台复杂机器面前让它乱扳乱按,只能降低效率。幸运的
是,市场经济更像一个计算机网络而不是一台机器。市场经济
不会因为破坏性的干预而停止,市场过程的信息会在政府干预
的周围绕过。它的效率会降低,但是不会停止。每一项进人市
场的干预也许只不过是让经济承担一些成本。亚 当 $\cdot$斯密曾经
遇到过一个年轻人,他对某项新颁布的政策忧心忡忡,说 :
“这将是英国的毁灭啊!” 对此斯密回答道:“年轻人,一个国
家总会有很多这样的毁灭的。” 与此相似,伟大的英国历史学
家麦考莱\footnote{托马斯$\cdot$巴宾顿$\cdot$麦考莱(Thomas Babington Macaulay, 1800$\sim$1895),英国历史学家、辉格党人、政治人物、演说家、政府官员、政论家。他所著的《英国史》使他成为辉格党历史学派的创始人。}写道: “人们常常发现奢侈的开支、沉重的税收、荒谬的商业禁令、腐败的法庭等都不能摧毁资本,因为无数公民个人努力的创造速度超过了他们的毁坏速度。”
\switchcolumn*
It is our great good fortune that the market process is so resilient, that it can continue to progress and produce despite the
burden of so much taxation and regulation. But there are real
costs. If we look only at the slowdown in U.S. productivity per
worker, and thus in economic growth, that began in the early
1970s---largely because of a dramatic growth in taxes and regulation in the 1960s and 1970s---the average American could
be 40 percent richer today, if productivity had continued to increase as fast as it did during the preceding twenty-five years.
Prosperous people may think that a 40 percent increase in wealth and income wouldn't be all that important (though I
would certainly like to see the new technologies and products
that would make up part of that increase), but lower-income
Americans would undoubtedly have their lives improved by
that kind of growth.
\switchcolumn
对我们来说万幸的是,市场经济已经变得如此的有弹性,
尽管承担了如此多的税收和管制负担,它仍然能够继续进步和
生产,但这的确是有代价的。仅仅看一下开始于1970年代早
期的美国工人人均劳动生产率增长的减速以及由此带来的经济
增长的减速 --- 很大程度上是因为税收和管制在I960年代和
1970年代的急剧上升 --- 就会知道,如果生产率按照前25年
的速度继续增长的话,到现在美国人应该比实际富裕40\%。
富裕阶层的人们也许认为40\% 的财富和收入增长并不是那么重 要 (尽管我当然愿意看到新科技和产品的出现,并为经济
增长作出贡献),但是低收入美国人的生活将无疑因为那样的
增长而改善很多。
\switchcolumn*
Each new tax, each new regulation, makes property a little
less secure, gives each individual a little less incentive to create
wealth, makes our society a little less adaptable to change, concentrates power a little more. There is a deal of ruin in a nation,
but civil society is not infinitely resilient.
\switchcolumn
每一项新税收,每一项新的管制,都会减少对财产的保
障,降低个人创造财富的热情,使我们的社会更难于适应变
化 ,权力更加集中。的确,一个国家总会有很多这样的毁灭,
但是市民社会的弹性终归是有限度的。

\switchcolumn*[\section{What Is Seen and What Is Not Seen\\看得见的和看不见的}]
Every proposal for government intervention in the economy involves a sleight of hand. Like a magician, the politician who proposes a tax, a subsidy, or a program wants the voters to look only
at his right hand and to be diverted from observing his left hand.
\switchcolumn
每一个政府干预经济的提案都是一种戏法。政客就像魔术
师一样,在提出征收一项税收、增加一项补贴或者提出一项计
划的时候,故意引导选民们只看他的右手,不看他的左手。
\switchcolumn*
In the early nineteenth century, Frederic Bastiat wrote a brilliant essay that inspired Henry Hazlitt's bestselling book Economics in One Lesson. As Hazlitt put it, ``The whole of economics
can be reduced to a single lesson $\ldots$: \textit{The art \textit{of economics consists in looking not merely at the immediate but at the longer effects of any act or policy; it consists in tracing the consequences of that policy not merely for one group but for all groups}}'' (emphasis in original).
\switchcolumn
19世纪早期,巴斯夏写了一篇精彩的文章,这篇文章给
了赫兹里特的畅销书《经济学一课》 以极大的灵感。赫兹里
特说: “整个经济学可以简化为一堂课……\textbf{经济学的艺术在
于:对任何法律或政策,不要只看当前效果,还要看长期效
果;不要只追踪政策对一个集团带来的后果,还要看政策对所
有集团带来的后果}。” (赫兹里特原文即为黑体强调)
\switchcolumn*
Bastiat and Hazlitt both began with the story of the broken
window. In a small town a teenager breaks a shop window. At
first everyone gathers out front and calls him a vandal. But then
someone says that, after all, someone will have to replace the
window. The money that the shopkeeper pays him will allow
the window installer to buy a new suit. The tailor then will be
able to buy a new desk. As the money circulates, everyone in
town may come to benefit from the boy's vandalism. What \textit{is seen} is the money circulating from replacing the window; what \textit{is not seen} is what would have been done with the money if no window had been broken. Either the shopkeeper would have saved
it, adding to investment capital and producing a higher standard of living later, or \textit{he} would have spent it. Perhaps he would have bought a new suit or a new desk. The town is not better off; people in the town have had to spend some money replacing something rather than producing new wealth.
\switchcolumn
巴斯夏与赫兹里特的书都以一个破窗的故事开始。在一个小镇上,一个十几岁的孩子打破了一家商店的窗户玻璃。最开
始每个人都跑出家门围过来,骂他是野蛮人,就知道搞破坏。后来,有人说,不管怎样都应该有人换掉这块窗户。商店老板
付给安装窗户的工人的钱可以让他买一套新衣服。然后裁缝就可以买一张新桌子了。随着这笔钱的流转,镇上的每一个人都
会从这个孩子的破坏行为当中得到好处。我们\textbf{看得见的}是换窗户导致了这笔钱的流通;\textbf{看不见的}是如果窗户没有被打破我们会把钱拿来做什么。或许商店老板会把这笔钱存起来,加到投
资资本当中以便今后创造更好的生活水平,或许他会花掉它。也许\textbf{他}会买一套新衣服或者一张新桌子。这个小镇并没有因此变得更好;小镇上的人不得不花钱来换掉某些东西,而不是创造新的财富。
\switchcolumn*
In such simple form, the fallacy may sound obviously absurd.
Who would claim that a broken window could benefit society?
But as Bastiat and Hazlitt pointed out, the same fallacy can be
found in the newspapers every day. The clearest example is the
story that always appears two days after a natural disaster. Yes,
Hurricane Andrew was awful, people reflect on the second day,
but think of all the construction jobs that will be created as we
rebuild our homes and factories. Indeed, a Florida newspaper
headline read, ``Hurricane Andrew Good News for S. Florida
Economy.'' The \textit{Washington Post} reported that Japan is considering building a new capital somewhere other than Tokyo. There
may be good arguments for the idea, but not this one: ``Supporters argue a new capital would boost Japan's sluggish econ-
omy. The massive construction project would create many jobs,
and the ripples would be felt throughout the nation's economy.'' They would indeed, but in both these cases we must look
at what is not seen. A hurricane destroys real wealth in society---houses, factories, churches, equipment. The capital and
labor that go into rebuilding them are not being used to produce \textit{additional} wealth. As for building a new capital, it would
create as many jobs as constructing the pyramids; but if there's
no good reason for a new capital, then capital and labor are
being diverted from more productive uses.
\switchcolumn
在这个简单的例子当中,这种谬见能够被明显地看出很荒
唐。谁会主张破窗会有利于社会呢?但是正如巴斯夏和赫兹里
特指出的那样,,同样的错误我们每天都可以在报纸t 找到。最
清楚的例子常常是一场自然灾害之后两天登出的报道。是啊,
安德鲁飓风的确很可怕,就像人们第二天反应的那样,但是想
想当我们重建住房和厂房的时候给建筑业创造的所有工作机会
吧。一家佛罗里达报纸的头条的确是这么写的: “安德鲁飓
风,南佛罗里达经济的福音”。 《华盛顿邮报》报道说,日本
正在考虑在某个地方建一个新首都,把首都从东京迁过去。这
个想法也许有很多好的理由,但绝不会是下面这个:“支持者
们说,一个新首都将拉动日本萧条经济的增长。大规模的建设
计划将创造大量的就业,而涟漪效应将会波及整个国家的经
济。” 的确会如此,但是在这些例子当中我们必须看到那些看
不见的东西。一场飓风摧毁了社会上的真实财富 --- 住房、工
厂、教堂、设备。投入到重建的资本和劳动力并没有被用来创
造\textbf{更多}的财富。对于建一个新首都来说,它将创造与建设金字
塔一样多的工作机会,但是如果并没有建一个新首都的好理由
的话,资本和劳动力就是被从更有效率的使用方式上挪用了。
\switchcolumn*
A related fallacy is the claim that West Germany and Japan
grew so fast after World War II not because they had lower
taxes and freer markets than some of the war's winners, but because their factories were destroyed and they built newer, more
modern factories. To my knowledge, the people making such
claims never actually urged bombing the factories of, say,
France and Great Britain in order to boost their economic
growth.
\switchcolumn
一个与破窗理论相关的谬见是有人声称西德和日本在第二
次世界大战后经济的高速增长不是因为他们与战胜国相比税收
更低、市场更自由,而是因为他们的工厂遭到了破坏而他们建
立了更新、更现代化的工厂。据我所知,发出这样种言论的人从来不会主张炸掉法国和英国的工厂以拉动他们的经济增长。
\switchcolumn*
The broken-window fallacy has far broader application:
\switchcolumn
破窗谬论还得到了更广泛的应用:
\switchcolumn*
\begin{itemize}
	\item Every time local politicians propose to tax people in order
	to build a stadium for a centimillionaire major-league owner, they hold out in their right hand the promise that the increased
	business activity will more than replace the money spent. But
	they don't want you to look at the left hand---the jobs and
	wealth created by the money that people would have spent if it
	hadn't been taxed away for the stadium.
\end{itemize}
\switchcolumn
\begin{itemize}
	\item 每次地方政客们提议向人民征税来为一个拥有职业棒
	球队的亿万富翁建造一座体育馆时,他们都会像耍戏法那样亮
	出右手,保证商业活动的增加将会弥补花掉的钱,但是他们并
	不想让你看到他们的左手 --- 这笔创造出工作机会和财富的钱
	如果不被征税拿走去建造体育场的话,本来是可以拿来花在别
	的地方的。
\end{itemize}
\switchcolumn*
\begin{itemize}
	\item After the federal government gave the Chrysler Corporation \$1.5 billion in loan guarantees, newspapers reported that
	the effort was a success because Chrysler stayed in business.
	What they didn't report---what they couldn't report---was \textit{what was not seen}: the homes that weren't built, the businesses
	that weren't expanded, with the money that other people
	couldn't borrow because the government directed scarce savings to Chrysler.
\end{itemize}
\switchcolumn
\begin{itemize}
	\item 在美国联邦政府给克莱斯勒公司提供15亿美元的贷款
	担保之后,报纸报道说这是一场胜利,因为克莱斯勒公司能够
	继续存在。他们没有报道的是 --- 他们也没有能力报道 --- 因为那是他们\textbf{看不见的}:没有建成的住房、没有扩展的企业,因
	为政府直接把稀缺的储蓄给了克莱斯勒,导致其他人不能从银
	行借出这笔钱。
\end{itemize}
\switchcolumn*
\begin{itemize}
	\item In every generation since the Industrial Revolution, people have worried that automation was going to eliminate jobs.
	In 1945 First Lady Eleanor Roosevelt wrote, ``We have
	reached a point today where labor-saving devices are good
	only when they do not throw the worker out of his job.'' It
	would seem they couldn't save much labor in that case. Gunnar Myrdal, who actually received a Nobel Prize in economics,
	wrote in 1970 in \textit{The Challenge of World Poverty}, that laborsaving machines should not be introduced in underdeveloped
	countries because they ``decrease the demand for labor.'' Of
	course automation reduces the demand for particular labor,
	but that means it frees up labor to do something else. If things
	can be produced with fewer resources, then more things can
	be produced---more clothes, more houses, more vaccinations
	to keep children from dying, more food for malnourished people, more water-treatment plants to combat cholera and
	dysentery.
\end{itemize}
\switchcolumn
\begin{itemize}
	\item 工业革命之后的每一代都有人担忧自动化将消灭就业
	机会。1945年,第一夫人伊莲娜$\cdot$ 罗斯福写道:“我们今天达
	成了一个共识,节省劳动力的机器只有在不把工人从他们的工
	作岗位上赶走的时候才是好的。” 如果有这样的机器的话,它
	们显然并不能节省多少劳动力。经济学家缪尔达尔 --- 他还获
	得过诺贝尔经济学奖 --- 1970年 在 《世界贫困的挑战》(\textit{The Challenge of World Poverty}) 一书中说,节省劳动力的机器不
	应该被应用到欠发达国家,因 为 它 们 “减少了对劳动力的需
	求”。自动化的确会减少对某些劳动力的需求,但也意味着将
	劳动力解放出来去做其他的事情。如果能够用更少的资源来生
	产产品,那么更多的产品将会被制造出来 --- 更多的衣服、更
	多的房子、更多的防止儿童死亡的疫苗、更多的给营养不良的
	人的食物、更多的防止霍乱和痢疾的水处理厂。
\end{itemize}
\switchcolumn*
Every scheme to create jobs through government spending
means that people will be taxed to pay for the project. The
money spent by government is then not spent by the people
who earned it, on projects \textit{they} would have chosen. Television
stations can send cameras to film the people who got jobs or
services from the program; they can't find the people who \textit{didn't} get a job because a little bit of money was diverted from
everyone in society to pay for the visible program.
\switchcolumn
每一个通过政府开支来创造工作机会的计划都意味着人们
将被征税来为这个计划付钱。于是,被政府用掉的钱就没有被
挣这笔钱的人用在\textbf{他们}会选择的其他计划上。电视台可以把镜
头给那些从这个项目中得到工作或服务机会的人;但是他们不可能去拍那些因为从社会上每个人那里拿走一小笔钱去支付看
得见的项目而\textbf{没有得到}工作的人。

\switchcolumn*[\section{Capitalism and Freedom\\资本主义与自由}]
In his pathbreaking essay, ``The Use of Knowledge in Society,''
Friedrich Hayek wrote,
\switchcolumn
在开创性的文章“知识在社会中的利用”(The  Use of Knowl­edge in Society)中,哈耶克写道:
\switchcolumn*
\begin{quote}
We often take the working of \textit{the price system} for granted. I
am convinced that if it were the result of deliberate human design, and if the people guided by the price changes understood
that their decisions have significance far beyond their immediate
aim, this mechanism would have been acclaimed as one of the
greatest triumphs of the human mind.
\end{quote}
\switchcolumn
\begin{quote}
我们常常把\textbf{价格系统}的运转看作理所当然。我确信,
如果这种机制是人类审慎设计的结果,如果人们在价格变
化的引导下懂得他们的决策之意义远远超出其直接目的的
范围,则这种机制早已会枝誉为人类智慧的一个最伟大的
功绩了。
\end{quote}
\switchcolumn*
But as I stress throughout this book, the great spontaneous
institutions of society---law, language, and markets---were not
designed by anyone. We all participate, unwittingly, in making
them work, and we do indeed take them for granted. That's
fine. They do evolve spontaneously, after all. We need simply to
remember to \textit{let} the market process work its apparent magic
and not let the government clumsily intervene in it so deeply
that it grinds to a halt.
\switchcolumn
但是正如我在本书中通篇强调的那样,人类社会伟大的自
发制度 --- 法律、语言和市场 --- 不是任何个人设计出来的。
所有人都在无意中参与了让这些制度有效运转的过程,而我们
的确是把它们视为理所当然。这很好。毕竟,它们是在自发地
演进。我们只需要记住,\textbf{让}市场自己完成它的奇迹,不要让笨
手笨脚的政府干预过深,否则它就会慢慢停下来。
\end{paracol}