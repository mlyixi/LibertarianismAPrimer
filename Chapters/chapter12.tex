\chapter{THE LIBERTARIAN FUTURE\\古典自由主义的未来}
\begin{paracol}{2}
\hbadness5000

Political society has failed to usher in the new age of
peace and plenty it promised. The failure of coercive
government has been proportional to the level of coercion and
the grandiosity of its promises. Fascist and communist governments that sought to eliminate civil society and to subsume individuals entirely in a larger cause are now recognized as abject
failures; they promised community and prosperity but delivered poverty, stagnation, resentment, and atomism.
\switchcolumn
政治社会(Political Society)没有如它所承诺的那样把我
们带入和平富裕的新时代。强制性政府的失败与其强制的程度
和它的承诺的浮夸程度成正比。极权主义国家消灭市民社会,
用一个宏大目标把全社会动员起来,在今天看来是彻底失败
了;他们承诺带来团结合作和富裕,可实际上带来的却是贫
穷、停滞、相互仇恨和社会原子化。
\switchcolumn*
The libertarian critique of socialism, long derided by left-leaning intellectuals, has been proven correct. Now the challenge to libertarianism is greater. With fascism and socialism
largely off the political scene, the conflict in the twenty-first
century will be between libertarianism and social democracy, a
watered-down version of socialism whose advocates accept the
necessity of civil society and the market process but find constant reasons to limit, control, shape, and obstruct the decisions
individuals make. (Social democracy is often called liberalism in
the United States, but I prefer not to tarnish the memory of a
word that once stood for individual freedom.) As for modern
American conservatism, we can expect to see its adherents divide into supporters of civil society and advocates of political
intervention to bring about a particular social order. Eventually, the statist conservatives will find themselves aligned with
the social democrats as defenders of political society against civil society, a trend that has already begun with the protectionist Buchanan movement and the growing tendency among conservatives not to limit government but to use it to impose
conservative values.
\switchcolumn
古典自由主义对社会主义的批评在很长时间里遭到了左派
知识分子的嘲笑,但这种批评已被证明是正确的。而现在古典
自由主义所面临的挑战更大。21世纪的主要冲突将发生在古
典自由主义和社会民主主义之间。社会民主主义只不过是一种
打了折扣的社会主义,它的拥护者同意市民社会和市场经济存
在的必要性,却不断寻找理由来限制、控制、影响和阻碍个人
作 出 决 策 (美国的社会民主主义常常被称作自由主义,但我
不愿意玷污这个曾经代表着个人自由的词汇,所以我宁愿称之
为社会民主主义)。现代美国的保守主义逐渐分裂为两部分,
一部分是市民社会的支持者,另一部分则主张通过政治干预带
来某种社会秩序。最终,国家主义的保守主义者会发现自己和
社会民主主义者站在一起支持政治社会而反对市民社会,这种
趋势开始于保护主义的布坎南运动。保守主义者中越来越强烈
的倾向不是限制政府权力,而是利用政府权力来加强保守主义
的价值观。
\switchcolumn*
Because social democracy in the United States and Western
Europe never entirely replaced civil society and markets, its failures have been less obvious. That is good news for the American
and European peoples, but it presents a bigger challenge for libertarians who want to point out the problems of intervention
and make the case for greater individual freedom and strictly
limited government. Still, the evidence of the failure of political
society has become overwhelming, and new examples appear
every day.
\switchcolumn
由于美国和西欧的社会民主主义从未替代市民社会和市
场,它的失败就不那么明显。对美国和欧洲的人民来说这是一
件好事,但是它给那些想指出政府干预的问题、为更大的个人
自由和对政府严格限权寻找案例的古典自由主义者带来了更大
挑战。当然,政治社会失败的证据是压倒性的,而新的例子也
层出不穷。
\switchcolumn*
Welfare-state transfer programs are becoming unsustainable
around the world, and the impending retirement of the baby
boomers will make Social Security's commitments impossible
to meet, even after massive tax increases. Information technology is being revolutionized, except for those forms monopolized
by the state---the schools and the post office---which get a little
less efficient and a lot more expensive every year. Examples
from Watergate to Whitewater, from Waco to the War on
Drugs, remind us that power corrupts. Taxes and regulation
have dramatically slowed down economic growth, just when
improved technology, better communications, and more efficient capital markets ought to give us \textit{increased} rates of growth.
Slower growth and the increasing perception that rewards are
handed out by government on the basis of identity politics and
political pull, rather than earned in the competitive marketplace, encourage group resentments and social conflict.
\switchcolumn
在世界范围内,福利国家的转移支付计划正在变得不可持
续。即将到来的婴儿潮一代人的整体退休将会使社保不可能满
足需求,即便是大规模加税也无济于事。信息技术正在让整个
经济发生根本改变,但那些由国家建立的垄断机构如学校和邮
局却僵化不变。它们耗费的金钱每年都大幅上升,但效率却越
来越低。从水门事件、白水门事件到韦科事件、反毒战争的例
子都在提醒我们:权力总是会腐败。税收和管制大幅度地\textbf{拉低了}经济增长,而在信息技术取得进步、通信更快捷、资本市场
更加高效的今天,本来经济增长率是应该大幅提升的。经济增
长的减速以及越来越多的人感觉到自己的收入依赖政府的施
舍,以政治身份和政治拉拢为基础,而不是在竞争市场中挣
得,这当然加剧了利益集团之间的互相憎恨和社会冲突。

\switchcolumn*[\section{The Washington That Roosevelt Built\\罗斯福建起来的华盛顿}]
The widespread disillusionment with big government, and the
growing attraction of the libertarian critique, have caused the
defenders of political society to launch a counterattack. What's
interesting about the most popular recent defenses of activist
government is their modesty. Gone are the sweeping calls for
social change of the 1930s and the starry-eyed crusades of the 1960s. Although such old-fashioned models can still be found
among tenured professors, politicians and authors who want to
appeal to a wide audience now make only modest claims for
what government can do.
\switchcolumn
由于对大政府失望的人越来越多,以及古典自由主义的批
评得到越来越多人的关注,政治社会的辩护者开始了新的反
击。目前最流行的对积极政府进行辩护的理论,其中最有趣的
是他们的克制。看 来 1930年代压倒一切的社会变革要求以及
I960年代不切实际的十字军圣战早已不复存在。尽管这些过
时的模式今天仍然能够在职业教授、政治家和作家那里找到。但现在他们为了吸引更多的听众,在提出政府能做哪些事情时
必须有所克制。
\switchcolumn*
Consider the 1992 book by David Osborne and Ted Gaebler,
\textit{Reinventing Government: How the Entrepreneurial Spirit Is Transforming the Public Sector}, which was widely hailed by such ``new
Democrats'' as Bill Clinton and Al Gore. Osborne and Gaebler
recognize that ``the kinds of governments that developed during the industrial era, with their sluggish centralized bureaucracies, their preoccupations with rules and regulations, and their
hierarchical chains of command, no longer work very well.''
They lay out ten things government should become: catalytic,
community owned, competitive, mission driven, results oriented, customer driven, enterprising, anticipatory, decentralized, and market oriented. The striking thing about that list is
that it's very close to a description not of government but of the
market process. The leading theorists of government activism
in our time promise that we can make government act like the
market.
\switchcolumn
我们可以看看奥斯本(David Osborne)和加百列(Ted Gaebler) 1992年 出 版 的 《再造政府:企业家精神如何改变公
共部门》(\textit{Reinventing Government: How the Entrepreneurial Spirit Is Transfroming the Public Sector}) 一书,这本书受到了“新民主党人” 如克林顿和戈尔的一致欢呼。奥斯本和加百列
认识到,“这种政府是在工业时代发展起来的、效率低下的中
央集权官僚机构。它们对法律和法规的关注、层级制的命令链
条都不再能有效运转。”他们提出了政府应该达到的十个要
求:反应迅速、共同所有、竞争性、任务驱动、结果导向、客
户驱动、创业精神、有预见性、分散决策、市场导向。这个单
子最让人印象深刻的是它非常像是在描述市场过程,而不是在
描述政府。当代研究政府活动的权威理论家们许诺,可以让政
府行为变得像市场。
\switchcolumn*
Or consider Jacob Weisberg's 1996 book \textit{In Defense of Government}, which sets forth five principles for ``resurrecting government'': (1) accept that life is risky and stop trying to legislate
risk out of existence; (2) stop promising more than government
can deliver; (3) be willing to abolish failed, outdated, or low-priority programs; (4) stop delegating Congress's lawmaking
authority to the bureaucracy; and (5) promise that government
won't get any bigger than it is now, in terms of government
share of GNE While Weisberg retains a touching belief in a
``wise, effective, and benevolent federal government,'' his policy
program is restrained compared with those of previous generations of enthusiasts for state activism.
\switchcolumn
再让我们看看维斯伯格(Jacob Weisberg) 1996年写的一本 书 《为政府辩护》(\textit{In Defense of Government}),书中提出了“复兴政府” 的五项原则:(1)承认生活有风险,不再试图通过立法来消灭风险;(2)不再做出无法实现的承诺;(3) 主动取消失败的、过时的或者不那么优先的项目;(4)不再将国会的立法权授予官僚机构;以及(5)承诺按政府支出占GNP的比例计算,政府不会比现在更大。而与此同时,维斯伯格保留了对“明智、高效和仁慈的联邦政府” 的动人信仰。与前几代国家积极行动主义的热情支持者相比,他的政策计划无疑是克制的。
\switchcolumn*
Despite these chastened interventionists, however, and despite President Clinton's proclamation that ``the era of big government is over,'' government in fact remains bigger than ever.
The federal government forcibly extracts \$1.6 trillion a year
from those who produce it, and state and local governments
take another trillion. Every year, Congress adds another 6,000 pages of statute law and regulators print 60,000 pages of new
regulations in the\textit{ Federal Register}. Lawyers agree that no business can possibly be in full compliance with federal regulation.
\switchcolumn
然而,尽管这些干预主义者的言论有所抑制,尽管克林顿
总 统宣布“大政府的时代结束了”,但事实上政府规模仍然比历史上任何时候都要大。联邦政府每年从那些创造财富的人手里强行抽走1. 6 万亿美元,而州和地方政府又拿走另一个1 万
亿。每年国会都要增加6000页的成文法律,而管制官员们则
每年都要在《联邦公报》 中增加6 万页新的管制规定。律师
们都承认,没有任何一家企业能完全遵守这些联邦管制规定。
\switchcolumn*
Most of our political leaders are still living in the Washington
that Roosevelt built, the Washington where, if you think of a
good idea, you create a government program. Consider a few
examples:
\switchcolumn
大多数政治领导人仍然居住在罗斯福建成的华盛顿。在这
里只要你想到一个好主意,就可以创造一个政府项目。看看下
面的几个例子:
\switchcolumn*
\begin{itemize}
	\item Senator Bob Dole reads the Tenth Amendment (``The powers
	not delegated to the United States by the Constitution, nor
	prohibited by it to the States, are reserved to the States respectively, or to the people'') on the campaign trail but introduces bills to federalize criminal law, welfare policy, and the
	definition of marriage.
\end{itemize}
\switchcolumn
\begin{itemize}
	\item 参议员鲍勃$\cdot$多尔(Bob Dole) 在竞选演说中引用宪
	法第十修正案(宪法未授予合众国,也未禁止各州行使的权
	力 ,分别由各州或由人民保留),却提出让刑法、福利政策和
	法定婚姻联邦化的议案。
\end{itemize}
\switchcolumn*
\begin{itemize}
	\item Vice President Gore announces a plan to tear down public
	housing projects, saying, ``These crime-infested monuments
	to a failed policy are killing the neighborhoods around
	them.'' He reminds his listeners, ``In years past, Washington
	told people around the country what to do, dictating wisdom
	from on high. And let's be honest: some of that wisdom really wasn't very wise.'' Then he announces a plan to $\ldots$ build
	new public housing projects.
\end{itemize}
\switchcolumn
\begin{itemize}
	\item 副总统戈尔在宣布拆除一片公共住宅时说:“这些犯罪
	横行的建筑是标志着政策失败的遗迹,它们毁灭着周围的社
	区。” 他提醒他的听众, “以前,华盛顿告诉全国人民应该做
	什么,高高在上地传达智慧的福音。但我们不得不承认:那些
	所谓的智慧很多其实并不很智慧。” 然后,他宣布了一个计划
	……建造新的公共住宅。
\end{itemize}
\switchcolumn*
\begin{itemize}
	\item Senator Dan Coats (R-Ind.) says that Republicans ``need to
	offer a vision of rebuilding broken communities---not
	through government, but through those private institutions
	and ideals that nurture lives'' and argues that ``even if government undermined civil society, it cannot directly reconstruct it.'' Then he proposes nineteen federal laws to establish
	a model school for at-risk youth, implement a waiting period
	for divorcing couples, fund religious maternity shelters, set
	up savings accounts for the poor, and more.
\end{itemize}
\switchcolumn
\begin{itemize}
	\item  参议员丹$\cdot$ 寇 兹 (Dan  Coats)说 共 和 党 “应该有重建
	那些分裂社区的理想 --- 不是通过政府,而是通过那些私人机
	构以及那些引领我们生活的理想”。并论证道, “即使政府已
	经破坏了市民社会,它也不能直接重建市民社会。”然后他提
	出 19项联邦法律草案,包括:为处境危险的年轻人建立模范
	学校,规定申请离婚的夫妇必须有一个冷静期,为教会产妇庇
	护所提供资助,为穷人建立储蓄账户等。
\end{itemize}
\switchcolumn*
\begin{itemize}
	\item Secretary of Housing and Urban Development Henry Cisneros promises to ``decentralize with a vengeance'' because
	churches, neighborhood groups, and small businesses ``know
	at least as much and are better positioned than the organizationally encumbered government in Washington'' to improve their own communities. But then he proposes to set up
	classrooms in public housing units and require all residents to attend class every day in prenatal training, educational
	day care, high school equivalency sessions, or seminars for
	the elderly.
\end{itemize}
\switchcolumn
\begin{itemize}
	\item 住房与城市发展部部长希斯纳罗斯(Henry Cisneros) 答
	应 把 “ (城市规划)彻底地分散化”,因为至少教会、社区组织和小公司“比华盛顿那臃肿的政府更清楚,什么样的规划才能
	更好地改善自己社区的品质”。但随后,他就提出在公屋单元里
	设立教室,要求所有居民每天上课,内容包括胎教、带有学习
	课程的幼儿日托、高中程度的课程、老人研修班等。
\end{itemize}
\switchcolumn*
\begin{itemize}
	\item Christian Coalition executive director Ralph Reed writes that
	America is united around ``a vision of a society based on two
	fundamental beliefs. The first belief is that all men, created
	equal in the eyes of God with certain inalienable rights, are
	free to pursue the longings of their heart. The second belief is
	that the sole purpose of government is to protect those
	rights.'' But his political program includes banning abortion,
	forbidding gay people to marry, and censoring the Internet.
\end{itemize}
\switchcolumn
\begin{itemize}
	\item 基督教联盟(Christian Coalition)执行主席拉尔夫$\cdot$里
	德 (Ralph Reed)在其文章中写道:“美国由建立于两个基本
	信仰之上的社会理想而联合起来。第一个信仰是在上帝面前人
	人生而平等,拥有某些不可剥夺的权利,拥有追求内心渴望的
	目标的自由。第二个信仰是政府的唯一目标是保护这些权
	利。” 但是,他的政治计划却包括禁止堕胎、禁止同性恋婚姻
	以及对互联网进行审查。
\end{itemize}
\switchcolumn*
And on and on it goes, in any day's newspaper: the president
has a plan to reduce the price of gasoline and to raise the price
of beef; the administration wants Japan and China to set specific targets for U.S. imports; a panel of experts wants to reduce
the number of doctors; county planners require developers to
build ``affordable'' housing, then a few years later develop a plan
to encourage ``upscale'' housing. The era of big government is
over, but the government doesn't seem to know it yet.
\switchcolumn
这些例子举不胜举,在每天的报纸上都能看到:总统计划
降低汽油价格、提高牛肉价格;政府要求日本和中国进口美国
商品的数量达到某个目标;一群专家在电视上辩论要求减少医
生的数量;县政府的计划官员要求开发商建设“人民负担得
起” 的住房, 过几年又出台一个计划鼓励建设“高档次” 的
住房。“大政府的时代结束了” ,不过似乎政府自己还不知道。
\switchcolumn*
Meanwhile, activists organize marches and rallies for all good
things under the sun: jobs, children, housing, health care, the
environment. It's hard to organize a rally for civil society and
the market process---the source of the ideas and the wealth that
allow us to provide better jobs, health care, child care, and
homes and use scarce resources more efficiently.
\switchcolumn
与此同时,活跃分子们还在组织游行示威,要求太阳底下
所有美好的东西:工作、孩子、住房、医疗、环境。但他们不
会为了市民社会和市场经济进行示威游行 --- 而这才是思想和
财富的真正源泉,让我们能够得到更好的工作、医疗、儿童照
顾和住房以及更有效利用稀缺资源的源泉。

\switchcolumn*[\section{Centralization, Devolution, and Order\\集权、分权与秩序}]
Two competing tendencies can be seen in world politics in the
1990s: centralization and devolution. Despite the talk in Washington about devolution and the Tenth Amendment, both Republicans and Democrats in Congress continue to offer federal
solutions to the problems that concern them, eliminating local
control, experimentation, and competing solutions. State
courts increasingly demand that all the schools in the state be
funded equally and be regulated by state guidelines. The bureaucrats of the European Union in Brussels try to centralize regulation at the continental level, partly to prevent any European government from making itself more attractive to investors by offering lower taxes or less regulation.
\switchcolumn
1990年代,世界政治中存在两种相互竞争的趋势:集权
与分权。尽管共和党和民主党的政治家都在华盛顿滔滔不绝地谈论分权,引用第十修正案,但是两党继续在国会中就一些问
题提出联邦一级的解决方案,排斥地方上可控的、试验性的和
竞争性的解决方案。州法院越来越多地要求本州所有学校都得
到平等的拨款并且接受州政府的指导。布鲁塞尔的欧盟官员试
图在全欧洲实施中央管制,在一定程度上阻止了任何欧洲政府
以低税收或放松管制的方式来吸引投资者。
\switchcolumn*
Paradoxically, nation-states today are too big \textit{and} too small.
They're too big to be responsive and manageable. India has
more than 1 million voters for each of its more than 500 legislators; can they possibly represent the interests of all their constituents or write laws that make sense for almost a billion
people? In any country larger than a city, local conditions vary
greatly and no national plan can make sense everywhere. At the
same time, even nation-states are often too small to be effective
economic units. Should Belgium, or even France, have a national railroad or a national television network, when rails and
broadcast signals can so easily cross national boundaries? The
great value of the European Union is not the reams of regulation produced by Eurocrats but rather the opportunity for businesses to produce and sell across a market larger than the
United States. A common market doesn't require centralized
regulation; it only requires that national governments not prevent their citizens from trading with citizens of other countries.
\switchcolumn
悖论的是,今天的民族国家\textbf{既可以}说太大了,\textbf{也可以}说太
小了。太大了以至于反应迟缓、无法管理。印度的议员超过500名,每一名都代表着100万以上的选民;他们真能代表自己选区所有选民的利益吗?或者说,他们真能制定出适合近10亿人的法律吗?在任何一个国家,只要规模超过一个城市,各地方的条件就会相差很大,没有任何一个全国性计划会适合每一个地方。与此同时,民族国家也常常太小了,以至于不能成为有效的经济单元。当铁路和广播信号能够轻易穿越国家边界的时候,比利时甚至法国还应该拥有一个全国铁路网或者全国电视网吗?欧盟的伟大价值不在于欧盟官僚制定出来的大量管制规定,而是给了企业在一个比美国还要大的市场里生产和销售的机会。一个共同市场并不需要中央集权的管制;它仅仅需要国家政府不阻止公民与其他国家的公民进行贸易。
\switchcolumn*
Meanwhile, as centralized governments from Washington to
Ottawa to Brussels to New Delhi try to centralize control and
squelch regional differences and small-scale experiments, another trend is also visible. Businesspeople try to ignore government and find their natural trading partners, be it across the
street or across national borders. Businesses in the triangle between Lyon, France; Geneva, Switzerland; and Turin, Italy, do
more business among themselves than with the political capitals of Paris and Rome. Dominique Nouvellet, one of Lyon's
leading venture capitalists, says, ``People are rebelling against
capitals that exercise too much control over their lives. Paris is
filled with civil servants, while Lyon is filled with merchants
who want the state to get off their back.'' Other cross-border
economic regions include Toulouse and Montpellier, France,
and Barcelona, Spain; Antwerp, Belgium, and Rotterdam, the
Netherlands; and Maastricht, the Netherlands, with Liege, Belgium, and Aachen, Germany. National governments and national borders impede the creation of wealth in those areas.
\switchcolumn
同时,中央集权的政府,从华盛顿到渥太华到布鲁塞尔到
新德里都在试图加强中央控制,压制地方差异和小规模的试
验 ,因此另一种趋势也显而易见。商人希望抛开政府的管制,
找到他们合适的贸易伙伴,这些贸易伙伴可能在街对面,也可
能在国家边界的对面。法国里昂一瑞士日内瓦一意大利都灵三
角地带的企业之间的商务往来也许比他们和本国的政治首都巴
黎和罗马之间还要密切。里昂杰出的风险资本家多米尼克$\cdot$诺
夫 勒 (Dominique  Nouvellet)说 : “人们正在对那些过多控制他们生活的首都进行反抗。巴黎到处都是公务员,而里昂到处
都是商人,他们希望国家从他们的生活中消失。”其他跨边界
的经济区包括法国的图卢兹和蒙比利埃、西班牙巴塞罗那、比
利时安特卫普一荷兰鹿特丹;以及荷兰马斯特里赫特一比利时
列日一德国亚琛。国家政府和国家边界阻碍了这些地区的创富
进程。
\switchcolumn*
Many regions are coming up with an old solution to the problems of out-of-touch, out-of-control government: secession. The French-speaking people of Quebec agitate for independence from Canada. So do a growing number of people in
British Columbia, who see that their trade ties to Seattle and
Tokyo are greater than those with Ottawa and Toronto. The
Lombard League has achieved rapid electoral success with its
call for the secession of productive northern Italy from what it
regards as Mafia-dominated, welfare-addicted southern Italy.
There's an increasing likelihood of devolution or even independence for Scotland. National breakup may well be a solution to
some of the problems of Africa, whose national boundaries were
carved by colonial powers with little regard to ethnic identity or
traditional trading patterns.
\switchcolumn
很多地区对这个问题采取了一种“脱离接触”  “脱离控制” 的解决办法,那就是:分离。魁北克的法语民众发起了脱离加拿大的独立运动。加拿大的不列颠哥伦比亚省认为他们与西雅图和东京的贸易联系超过和渥太华以及多伦多,因此不列颠哥伦比亚主张脱离加拿大的人越来越多。伦巴第联盟\footnote{伦巴第联盟(Lombard  League),指意大利北部地区,意大利北方工业化程度较高,比南方发达得多。}要求北部意大利与南部意大利相分离,因为意大利北部生产力水平高,而意大利南部被认为是在黑手党统治之下,而且沉溺于福利。他们在投票中很快取得了成功。苏格兰主张某种形式的分离甚至是独立的呼声也越来越高。对于非洲来说,解决他们问题的办法很可能是国家分裂,因为非洲国家的边界是由殖民政权在无视当地种族分布和传统贸易关系的情况下划分出
来的。
\switchcolumn*
Even in the United States, we see more agitation for secession than we've seen in a long time. Staten Island voted to secede from New York City in 1993, but the state legislature
blocked its path. Nine counties in western Kansas have petitioned Congress to be split off as a separate state. Activists in
both northern and southern California have proposed that the
giant state be split into two or three more manageable units.
The San Fernando Valley of \textit{American Graffiti} fame is brimming
with demands to secede from the city of Los Angeles.
\switchcolumn
甚至在美国,近年来分离主义的要求也比以前很长一段时
间都要多。1993年 ,史坦顿岛(Staten Island) 投票决定脱离纽约市,但是州参议会阻止了他们。堪萨斯州西部的九个县曾请求国会准许他们分离出来单独组成一个州。加州北部和南部
都有活跃分子要求把这个巨大的州分成两个或三个更容易管理
的部分。因电影《美国风情画》(\textit{American Graffiti}) 而闻名于世的圣费尔南多谷地区(San Fernando Valley) 也积极要求从
洛杉矶市分离出去。
\switchcolumn*
One of the most important lessons of America's economic
success is the value of broadening the geographic area in which
trade can flow freely while keeping government close to the
communities that will have to live with its decisions. Switzerland may be an even better example of the benefits of free trade
and decentralized power. Although it has only 7 million people,
Switzerland has three major language groups and people with
distinctly different cultures. It has solved the problem of cultural conflict with a very decentralized political system---
twenty cantons and six half-cantons, which are responsible for
most public affairs, and a weak central government, which handles foreign affairs, monetary policy, and enforcement of a bill
of rights.
\switchcolumn
美国经济成功最重要的一个原因就是:一方面拥有广袤的
国土,在这片国土上贸易能够自由流动,同时政府权力更分
散 、基层化、更接近于社区,这样政府就必须对它的决策负
责。关于自由贸易和分权的好处,更好的例子也许是瑞士。瑞
士只有七百万人口,但瑞士有三种主要的语言,各个民族之间
有着明显的文化差异。它通过分权程度很高的政治制度解决了
文化冲突问题 --- 全国分成二十个州和六个准州,州对绝大部
分公共事务负责,还有一个权限较小的中央政府,负责处理外
交事务、货币政策和执行权利法案。
\switchcolumn*
One of the key insights offered by the Swiss system is that
cultural conflicts can be minimized when they don't become political conflicts. Thus, the more of life that is kept in the private
sphere or at the local level, the less need there is for cultural groups to go to war over religion, education, language, and the
like. Separation of church and state and a free market both limit
the number of decisions made in the public sector, thus reducing
the incentive for groups to vie for political control.
\switchcolumn
瑞士体制的关键启示是文化冲突如果不变成政治冲突,它
就可以最小化。因此,越是让生活保持在私人领域或者地方层
面,不同文化族群之间就越是不会因宗教、教育、语言以及个
人好恶而对抗。政教分离以及自由市场限制了依靠公共部门作
出决策的数量,并由此减少了各个利益集团争夺政治控制权的
激励。
\switchcolumn*
People around the world are coming to understand the benefits of limited government and devolution of power. Even a student from faraway Azerbaijan recently said at a conference,
``My friends and I have been thinking, couldn't we solve the
conflict between Armenians and Azerbaijanis not by moving
the borders but by making them unimportant---by abolishing
internal passports and allowing property ownership and the
right to work on both sides of the border?''
\switchcolumn
世界各地的人们也逐渐理解了有限政府和分权的好处。一
个来自遥远的阿塞拜疆的学生最近在一次会议上说:“我和我
的朋友们正在思考,我们是否能够通过不改变边界,而是让边
界变得无足轻重的办法来解决亚美尼亚人和阿塞拜疆人之间的
冲突 --- 取消内部护照、承认财产所有权以及在边界两边自由
工作的权利?”
\switchcolumn*
Still, the centralists will not give up easily. The impulse to
eliminate ``inequities'' among regions is strong. President Clinton said in 1995, ``As president, I have to make laws that fit not
only my folks back home in Arkansas and the people in Mon-
tana, but the whole of this country. And the great thing about
this country is its diversity, its differences, and trying to harmonize those is our great challenge.'' A \textit{Washington Post} columnist
says that America ``needs badly $\ldots$ a single education standard
set by---who else?---the federal government.'' Kentucky governor Paul Patton says that if an innovative education program is
working, all schools should have it, and if it isn't, none should.
\switchcolumn
当然,中央集权主义者是不会轻易放弃的。消灭地方间
“不平等” 的冲动仍然很强烈。克林顿总统在1995年说:“作
为总统,我必须让法律不仅适合于我的阿肯色州老乡们,以及
蒙大拿的人们,而且适合整个国家。而这个国家的伟大之处就在于它的多样性,它的千差万别,而调和这些差别和多样性是对我们的重大挑战。” 《华盛顿邮报》 的一名记者说,美国
“极度需要一个统一的教育标准,而这个标准除了联邦政府之
外还有谁能够确定呢?” 肯塔基州长保罗$\cdot$ 帕顿(Paul Patton)说,如果一项教育革新计划能够奏效,那么所有学校都应该采用,如果不能奏效,那么就都不应该采用。
\switchcolumn
But why? Why not let local school districts observe other
districts, copy what seems to work, and adapt it to their own
circumstances? And why does President Clinton feel that his
challenge is to ``harmonize'' America's great diversity? Why not
enjoy the diversity? The problem for centralizers is that appreciating diversity means accepting that different people and different places will have different situations and different results.
The bottom-line question is whether centralized systems or
competitive systems produce better results---that is, arrive at
more solutions that although not perfect, are better than they
might have been. Libertarians argue that our experience with
competitive systems---whether that means democracy, federal-
ism, free markets, or the vigorously competitive Western intellectual system---shows that they find better answers than
imposed, centralized, one-size-fits-all systems.
\switchcolumn
为什么要这样做呢?为什么不能让地方学区到其他学区进
行观察,复制那些看上去有效的计划,然后再根据自己的环境
进行调整呢?为什么克林顿总统感到他面临的挑战是“调和”
美国伟大的多样性呢?为什么不是欣赏这种多样性呢?对中央
集权主义者来说,问题在于,欣赏多样性就意味着承认不同的
人、不同的地方会有不同的情况和不同的结果。最基本的问题
是,中央集权体制还是竞争性的体制能产生更好的结果 --- 也
就是说,能够得到更多尽管不完美的、但比以前的方案要好的
解决方案。古典自由主义者认为,对竞争体制的经验 --- 无论
是民主、联邦主义、自由市场还是富有活力的竞争的西方知识
体系 --- 都显示出竞争能够比强制的、中央集权的、一刀切的
体制发现更好的答案。
\switchcolumn*
Two large companies---ITT and AT\&T---both announced in 1995 that they would split themselves into three parts because
they had become too large and diverse to be managed efficiently. ITT had sales of about \$25 billion a year, AT\&T about
\$75 billion. If corporate managers and investors with their own
money at stake can't run businesses that size effectively, can it
really be possible for Congress and 2 million federal bureaucrats
to manage a \$1.6 trillion government---to say nothing of a \$6
trillion economy?
\switchcolumn
两家大公司---ITT和 AT\&T---都在1995年宣布他们将
把公司拆分成三个公司,因为它们规模太大、内部差别也很
大 ,以至于很难有效进行管理。ITT每年的销售额大约是250
亿美元,AT\&T大约是750亿美元。如果说公司管理者和那些
用自己的钱购买公司股份的投资者都不能让如此规模的企业有
效运转,那么国会和200万政府官僚真的能够管理好一个1. 6
万亿美元规模的政府吗---更不用说一个6 万亿美元规模的经
济体了?

\switchcolumn*[\section{The Information Age\\信息时代}]
One big reason that the future will be libertarian is the arrival
of the Information Age. Information is getting cheaper and
cheaper and thus more widespread; increasingly, our problem is
not a dearth but a glut of information. The Information Age is
bad news for centralized bureaucracies. First, as information
gets cheaper and more widely available, people will have less
need for experts and authorities to make decisions for them.
That doesn't mean we won't consult experts---in a complex
world, none of us can be expert in everything---but it does
mean we can choose our experts and make our own decisions.
Governments will find it more difficult to keep their citizens in
the dark about world affairs and about government malfeasance. Second, as information and commerce move faster, it will
be increasingly difficult for sluggish governments to keep up.
The chief effect of regulation on communications and financial
services is to slow down the pace of change and keep consumers
from receiving the full benefits that companies are striving to
offer us. Third, privacy is going to be easier to maintain. Governments will try to block encryption technology and demand
that every computer come with a government key---like the
``Clipper Chip''---but those efforts will fail. Governments will
find it increasingly difficult to pry into citizens' economic lives.
Finally, as techno-entrepreneur Bill Frezza puts it, ``coercive
force cannot be projected across a network.'' As digital bits become more valuable than coal mines and factories, it will be
more difficult for governments to exert their control.
\switchcolumn
认为未来将会是古典自由主义时代的一个重大理由是信息
时代的到来。信息的获得变得越来越便宜,因此它的传播也会
变得越来越广泛;渐渐地,问题就不是信息匮乏,而是信息过
剩了。信息时代的到来对中央集权的官僚们是一个坏消息。首
先 ,随着信息变得越来越便宜和传播更广泛,人们将更不需要
专家和权威来为他们作决定。这当然不是说我们不需要请教专
家 --- 在一个复杂的世界里,没有人能够是所有事情的专家
--- 但是这的确意味着我们能够选择专家,并且自己做出决
定。政府将发现更难让公民对世界事务和政府做的坏事一无所
知。其次,随着信息和商业运转速度加快,本来就动作迟缓的
政府将越来越难以跟上发展的速度。通信和金融服务管制的最
大影响就是减缓了变革的脚步,让消费者不能获得经营者努力
提供的服务的全部好处。第三,隐私更容易得到保护。政府会
阻止信息加密技术的发展,并且要求每台电脑都必须有一个政
府密钥 --- 类 似 于 “加密芯片” (clipper chip)  --- 但这些努
力都将失败。政府会发现它越来越难以窥探公民的经济生活
了。最后一点,正如科技企业家弗雷萨(Bill Frezza)所说:
“强制暴力不可能通过网络传播。”随着数字字节变得比煤矿
和工厂更有价值,政府就越来越难以实施对经济的控制了。
\switchcolumn*
Some people worry that the cost of computers and Internet access creates a new divide between the haves and the have-nots, but in fact an adequate used computer and on-line access
for a year can be had for the cost of a year's subscription to the
\textit{New York Times}---and nobody worries about the newspaper
have-nots. In any case, the cost of computers is falling and will
continue to fall, as did that of telephones and televisions, once
the playthings of the rich. By mid-1996 entrepreneurs were offering free e-mail to any customer willing to put up with advertisements on the computer screen. There will be no haves and
have-nots, says Louis Rossetto, editor of \textit{Wired}, the libertarian
bible of the Information Age: ``Better to think of the haves and
the have-laters. And the haves may be the ones who are really
disadvantaged, since they are the guinea pigs for new technology, paying an arm and a leg for stuff that in a couple of years
will be widely available for a fraction of its original price.'' Attempts to force companies to supply their technology to everyone at once or at a below-market cost will just reduce every
entrepreneur's incentive to come up with a new product and
thus slow down the pace of change.
\switchcolumn
有人担心电脑和互联网接入会在信息富人和信息穷人之间
制造出新的鸿沟,但实际上一年当中充分使用电脑和上网接入
所需要的费用并不比订阅一年《纽约时报》 的费用更高 ---
从来没人为没报纸的人担心过。不管怎样,电脑的价格正在下
降,而且将会继续下降,就像在电话和电视上曾经发生过的那
样 ,这两种东西曾经是富人的玩具。企业家们一直在提供免费
电子邮件,任何客户只要愿意让广告同时出现在计算机屏幕
上,他就可以申请使用免费邮箱。“没有信息富人和信息穷人
之分”,信息时代的圣经---《连线》 杂志的主编罗塞托(Louis Rossetto)说, “把他们看成信息的拥有者和将要拥有者应该更好一点。而且所谓的信息富人或者信息拥有者也许才
是弱势群体呢,因为他们是新技术做实验的小白鼠,为那个贵
得要死的东西付一大笔钱,而一两年之后,那东西满大街都
是 ,价钱只是原来价钱的一个零头。”强迫公司把它们的技术
产品立即提供给所有人使用,或者以低于市场价出售,只会打
击企业家提供新产品的积极性并因此减缓技术变革的速度。
\switchcolumn*
As more of the value in our world reflects the products of our
minds embedded in digital bits, traditional natural resources
will become less relevant. Institutional structures and human
capital will become far more important to wealth creation than
oil or iron ore. States will find it more difficult to regulate capital and entrepreneurship as it becomes easier for people and
wealth to move across borders. Countries will prosper by reducing taxes and regulation in order to keep innovators and investors at home and attract them from abroad.
\switchcolumn
随着我们这个世界更多的价值体现在头脑的产品之中,.并
且存储在数码字节当中,传统的自然资源将会变得不那么至关
重要。制度结构和人力资本对于财富的创造将比石油或铁矿更
加重要。国家将发现对资本和企业家精神更难进行管制,因为
人力资源和财富穿越边境更加容易了。为了让革新者和投资者
留在国内,以及把他们从国外吸引过来,国家将会减税和减少
管制,并因此获得繁荣。
\switchcolumn*
Some visionaries of the Information Age have overstressed its
differences from the industrial age. Many of the wealthiest
countries of the seventeenth through the twentieth centuries---
the Netherlands, Switzerland, Great Britain, Japan, Singapore---have been notably lacking in natural resources. They got
rich the old-fashioned way---actually the new-fashioned, capitalist way---through the rule of law, economic freedom, and a
hard-working and well-educated populace.
\switchcolumn
一些幻想家过分夸大了信息时代和工业时代的区别。从
17$\sim$20世纪的富国,荷兰、瑞士、英国、日本、新加坡都明
显缺乏自然资源。他们都是以非常老套的方式致富的---永不
过时的方式,资本主义的方式---法治、经济自由以及勤奋和
受到良好教育的人民。
\switchcolumn*
Still, the importance of free markets and individual effort
will indeed be enhanced by the more open, participatory economy made possible by cyberspace. Peter Pitsch of the Hudson Institute writes that ``Hayek and Schumpeter are prophets for
the Innovation Age,'' his term for the new economy. Hayek's
analysis of spontaneous order and the immense dangers of coercive tampering with its complex workings is more relevant than
ever in an era of unbounded opportunity and fast-paced
change. And Schumpeter's point that ``creative destruction is
the essential fact about capitalism'' will be more true than ever,
as entrepreneurs have learned and will continue to learn to their
chagrin. The overthrow of the mainframe by the personal computer, which cost IBM 70 percent of its market value in just five
years, was a dramatic example of creative destruction. Will the
PC itself be overthrown by the network? Will Microsoft be
rocked as IBM was? As Hayek and Schumpeter would tell us,
no one knows.
\switchcolumn
经济因网络而更加开放、参与机会更多,而在一个更加开
放的经济体中,自由市场和个人努力的重要性就更加凸现了出
来。哈 得 逊 研 究 所 (Hudson  Institute)的 彼 得 $\cdot$ 皮 茨 (Peter Pitsch)写道:“哈耶克和熊彼特是革新时代的先知。”革新时代是他对新经济的称呼。哈耶克对自发秩序以及强制干预其复
杂进程导致将巨大危险的分析,在一个机会无限和变化迅速的
时代尤其显得贴切。而熊彼特指出的“创造性的毁灭是资本
主义的基本事实” 也比以前任何时候都显得更为真实。这一
点企业家们已经认识到的,并且将通过他们的遗憾而进一步认
识。PC在 5 年内取代了大型计算机,IBM 的市值为此下降了
70\% , 这是一个创造性毁灭的戏剧性例子。PC会不会被网络
所替代呢?微软会不会像IBM 那样遭受重大损失呢?哈耶克
和熊彼特会告诉我们:没人知道。
\switchcolumn*
People have always had trouble seeing the order in the apparently chaotic market. Even as the price system constantly
moves resources toward their best use, on the surface the market seems the very opposite of order---businesses failing, jobs
being lost, people prospering at an uneven pace, investments
revealed to have been wasted. The fast-paced Innovation Age
will seem even more chaotic, with huge businesses rising and
falling more rapidly than ever, and fewer people having longterm jobs. But the increased efficiency of transportation, communications, and capital markets will in fact mean even more
order than the market could achieve in the industrial age. The
point is to avoid using coercive government to ``smooth out the
excesses'' or ``channel'' the market toward someone's desired result. Let the market work---let billions of people seek happiness
in their own ways---and the second edition of this book will
probably be composed on technology undreamed of in 1997.
\switchcolumn
人们看到表面上混乱的市场会感到困惑。当价格系统持续
地将资源分配到能够发挥其最大作用的地方去的时候,市场看
上去根本没有秩序 --- 企业失败、失业、人们没有共同富裕、
投资被证明是浪费。迅速发展的变革时代甚至将更加混乱,大
公司上升和下降的速度远超过从前,拥有长期工作的人减少
了。但是交通、通信和资本市场效率的提高,实际上意味着当
前的市场比工业时代所能达到的秩序更加有序。关键的一点
是 ,避免用强制性的政府来“平衡超额的收入” ,或 者 “引
导” 市场朝某些人想要的结果发展。让市场自己运转 --- 让
数十亿人用自己的方式来追求幸福 --- 那么这本书的第二版也
许就会用在1997年想都想不到的技术来写了。

\switchcolumn*[\section{Toward a Framework for Utopia\\一个为乌托邦准备的框架}]
Lots of political movements promise Utopia: Just implement
our program, and we'll usher in an ideal world. Libertarians
offer something less, and more: a framework for Utopia, as
Robert Nozick put it.
\switchcolumn
很多政治运动都会许诺一个乌托邦:只要实施我们的计
划 ,就会得到一个理想世界。古典自由主义提供的没有那么
多,但实际上却更多。就像诺齐克所说的,我们提供的是一个
乌托邦的框架。
\switchcolumn*
My ideal community would probably not be your Utopia. The attempt to create heaven on earth is doomed to fail, be-
cause we have different ideas of what heaven would be like. As
our society becomes more diverse, the possibility of our agreeing on one plan for the whole nation becomes even more remote. And in any case, we can't possibly anticipate the changes
that progress will bring. Utopian plans always involve a static
and rigid vision of the ideal community, a vision that can't accommodate a dynamic world. We can no more imagine what
civilization will be like a century from now than the people of
1900 could have imagined today's civilization. What we need is
not Utopia but a free society in which people can design their
own communities.
\switchcolumn
我的理想社会也许不是你的乌托邦。企图在地球上创造天
堂的努力必然会失败,因为每个人对天堂是什么样子都有着不
同的看法。社会变得越来越多元化,整个国家的计划达成一致
的可能性也就越来越遥远。而且无论如何,我们都不可能事先
预知进步所带来的变化。乌托邦的理想社会的幻想常常是静止
和不变的。这种幻想是不可能适合一个充满活力的世界的。我
们不可能想像到一个世纪之后的文明社会将是什么样子,同
样,1900年的人也不可能想像到今天的文明社会是什么模样。
我们所需要的不是乌托邦,而是一个自由社会,在这里人们能
设计自己生活的社区。
\switchcolumn*
A libertarian society is only a framework for Utopia. In such a
society, government would respect people's right to make their
own choices in accord with the knowledge available to them. As
long as each person respected the rights of others, he would be
free to live as he chose. His choice might well involve voluntarily agreeing with others to live in a particular kind of community. Individuals could come together to form communities in
which they would agree to abide by certain rules, which might
forbid or require particular actions. Since people would individually and voluntarily agree to such rules, they would not be giving up their rights but simply agreeing to the rules of a
community that they would be free to leave. We already have
such a framework, of course; in the market process we can
choose from many different goods and services, and many people already choose to live in a particular kind of community. A
libertarian society would offer \textit{more} scope for such choices by
leaving most decisions about living arrangements to the individual and the chosen community, rather than government's
imposing everything from an exorbitant tax rate to rules about
religious expression and health care.
\switchcolumn
一个古典自由主义的社会仅仅是为乌托邦准备了一个框
架。在这样一个社会里,政府会尊重人们运用所能得到的知识
作出自己选择的权利。只要尊重别人的权利,人们就可以按照
自己选择的方式来生活。他的选择也许正好包括自愿同意与他
人一起住在某种社区里面。人们可以在一起建立共同的社区,
在这个社区里他们同意遵守某些规则,也许会禁止某些行为,
或者要求某些行为。由于人们是以个人名义同意这些规则,并
且这种同意是自愿的,他们并没有放弃自己的权利,他只不过
是同意了一个社区的规则,而他随时都可以自由地离开这个社
区。当然,已经有了这样一个框架;在市场中我们可以从很多
不同商品和服务中进行选择,并且很多人也已经选择生活在某
种社区当中。一个古典自由主义的社会将会通过给个人和个人
选择的社区留下\textbf{更多}对自己的生活作决策的机会,而不是政府
强加从过高的税收到宗教表达规则以及医疗保险等所有东西,
从而为上述选择提供更广的范围。
\switchcolumn*
Such a framework might offer thousands of versions of
Utopia, which might appeal to different kinds of people. One
community might offer a high level of services and amenities,
with correspondingly high prices and fees. Another might be
more spartan, for those who prefer to save their money. One
might be organized around a particular religious observance.
Those who entered one community might forswear alcohol, tobacco, nonmarital sex, and pornography. Other people might
prefer something like Copenhagen's Free City of Christiana,
where cars, guns, and hard drugs are banned but soft drugs are
tolerated and all decisions are at least theoretically made in
communal meetings.
\switchcolumn
在这样一个框架下可能会出现成千上万种乌托邦,将吸引
各种不同的人。有的社区也许能提供高品质的服务和舒适的环境 ,但相伴而来的会是髙价格和费用。有的社区也许更斯巴达
一些,给那些愿意节俭的人居住。有的社区也许是按照某种宗
教教义建立起来的。那些进人这个社区的人也许必须发誓戒
酒 、戒烟、不得有婚前性行为、不得看色情图片书籍等。另一
个社区的人也许喜欢像哥本哈根式的克里斯蒂安娜自由市那
样 ,汽车、枪和重度毒品被禁止,而轻度毒品可以吸食,所有
的决策至少在理论上都必须通过社区会议来作出。
\switchcolumn*
One difference between libertarianism and socialism is that a
socialist society can't tolerate groups of people practicing freedom, but a libertarian society can comfortably allow people to
choose voluntary socialism. If a group of people---even a very
large group---wanted to purchase land and own it in common,
they would be free to do so. The libertarian legal order would
require only that no one be coerced into joining or giving up his
property. Many people might choose a ``utopia'' very similar to
today's small-town, suburban, or center-city environment, but
we would all profit from the opportunity to choose other alternatives and to observe and emulate valuable innovations.
\switchcolumn
古典自由主义和社会主义的一个不同之处是,社会主义社
会不能容忍人们结合在一起实践自由,但是古典自由主义社会
能够很愉快地允许人们选择自愿的社会主义。如果一群人 ---
甚至是很大的一群人 --- 想买一块地并且共同拥有,他们完全
有自由这么做。古典自由主义的法律秩序将仅仅要求不得有人
被强迫放弃自己的财产或者被强迫把财产加入到共同体当中。
很多人也许会选择一种非常接近于今天的小镇、郊区或者是市
中心环境的“乌托邦”,但是所有人将会因为有机会选择各种
生活方式以及观察和模仿有价值的革新而获益。
\switchcolumn*
In such a society, government would tolerate, as Leonard
Read put it, ``anything that's peaceful.'' Voluntary communities
could make stricter rules, but the legal order of the whole society would punish only violations of the rights of others. By radically downsizing and decentralizing government---by fully
respecting the rights of each individual---we can create a society based on individual freedom and characterized by peace,
tolerance, community, prosperity, responsibility, and progress.
\switchcolumn
在这样一个社会中,政府将会如里德所说的那样宽容
“任何和平的东西”。 自愿的社区可能会制定更严格的规则,
但是整个社会的法律秩序将只会惩罚那些侵犯别人权利的人。
通过根本性地裁减政府规模和让政府权力分散化 --- 按照完全
尊重个人权利的原则 --- 就能够创造一个建立在个人自由之上
的,以和平、宽容、合作、繁荣、责任和进步为特征的社会。
\switchcolumn*
Can we achieve such a world? It is hard to predict the short-term course of any society, but in the long run, the world will
recognize the repressive and backward nature of coercion and
the unlimited possibilities that freedom allows. The spread of
commerce, industry, and information has undermined the age-old ways in which governments held men in thrall and is even
now liberating humanity from the new forms of coercion and
control developed by twentieth-century governments.
\switchcolumn
我们能够到达这样的世界吗?对任何社会都很难预测其短
期趋势,但是就长期来说,整个世界将会认识到强制的压迫和
反动本质,并认识到自由所包含的无限可能性。而工业、商业
和信息的扩张,已经从根本上破坏了政府奴役人们的古老方
式 ,甚至现在就已经幵始将人类从20世纪的政府发明出来的新形式的强制和控制中解放出来。
\switchcolumn*
As we enter a new century and a new millenium, we encounter a world of endless possibility. The very premise of the
world of global markets and new technologies is libertarianism.
Neither stultifying socialism nor rigid conservatism could produce the free, technologically advanced society that we anticipate in the twenty-first century. If we want a dynamic world of
prosperity and opportunity, we must make it a libertarian world. The simple and timeless principles of the American Rev-
olution---individual liberty, limited government, and free markets---turn out to be even more powerful in today's world of
instant communication, global markets, and unprecedented access to information than Jefferson or Madison could have imagined. Libertarianism is not just a framework for Utopia, it is the
essential framework for the future.
\switchcolumn
随着进入新世纪和新千年,我们面对着一个有着无限可能
性的世界。全球化市场和新技术世界的前提正是古典自由主
义。无论是社会主义还是僵硬的保守主义都不可能产生出我们
正在参与的21世纪的自由和技术先进的社会。如果我们想要
的是一个充满活力和机会的繁荣世界,这个世界就必须在我们
手里变成一个古典自由主义的世界。美国革命的最朴素却又永
恒的原则 --- 个人自由、有限政府和自由市场 --- 在今天这个
即时通讯、全球市场和空前信息发达的世界里显示出的力量甚
至已经超出了杰斐逊和麦迪逊的想像。古典自由主义不仅仅是
乌托邦的框架,而且是未来的基本框架。

\end{paracol}