\chapter{WHAT RIGHTS DO WE HAVE?\\我们有哪些权利?}
\begin{paracol}{2}
\hbadness5000

Critics on both left and right have complained that America in the 1990s is awash in talk about rights. No political debate proceeds for very long without one side, or both,
resting its argument on rights---property rights, welfare rights,
women's rights, nonsmokers' rights, the right to life, abortion
rights, gay rights, gun rights, you name it.
\switchcolumn
左派和右派的批评家都在抱怨美国在1990年代被无休止
的、五花八门的权利主张淹没了。任何一次政治争论都是这
样 ,进行不了多久,就有一方或者双方开始争论权利 --- 财产
权 、福利权、女权、不抽烟者的权利、生命权、堕胎权、同性
恋权利、拥枪权,等等,名目繁多。
\switchcolumn*
A journalist asked me recently what I thought of a proposal
by self-proclaimed communitarians to ``suspend for a while the
minting of new rights.'' Communitarians in late twentieth-century America are people who believe that ``the community''
should in some way take precedence over the individual, so naturally they would respond to rights-talk overload by saying,
``Let's just stop doing it.'' How many ways, I mused, does that
get it wrong? Communitarians seem to see rights as little boxes;
when you have too many, the room won't hold them all. In the
libertarian view, we have an infinite number of rights contained
in one natural right. That one fundamental human right is the
right to live your life as you choose so long as you don't infringe
on the equal rights of others.
\switchcolumn
有位新闻记者最近问我,对自称社群主义者的人提出的
“暂停一下生造新的权利” 的主张有什么看法。20世纪晚期的
美国社群主义者是指那些相信“社群”有时候应当优先于个
人的人,因此,对于各种个人的权利主张,他们会很自然地反
应说: “让我们停止讨论这些东西吧。” 我仔细地想了一下,
这些社群主义者到底错在什么地方?社群主义者仿佛把权利看
成了一个个小盒子;如果小盒子太多,屋子里面就装不下了。
而在古典自由主义者着来,我们有数量无限的权利,只不过这
些权利都装在了一个权利里面,那就是自然权利。这个唯一的
基本的人的权利就是按照你自己的选择来生活,同时不能干预
别人同等的权利。
\switchcolumn*
That one right has infinite implications. As James Wilson, a
signer of the Constitution, said in response to a proposal that a
Bill of Rights be added to the Constitution: ``Enumerate all the
rights of man! I am sure, sirs, that no gentleman in the late Convention would have attempted such a thing.'' After all, a person has a right to wear a hat, or not; to marry, or not; to
grow beans, or apples; or to open a haberdashery. Indeed, to
cite a specific example, a person has a right to sell an orange to
a willing buyer even though the orange is only $2\frac{3}{8}$ inches in diameter (although under current federal law, that is illegal).
\switchcolumn
这个唯一的权利有包容无限的内涵。美国宪法的签署人之
一詹姆斯$\cdot$ 威尔逊(James Wilson) 曾经对一个要求把列举权
利的权利法案加入宪法的提案回应道: “列举出人的所有权利!我敢肯定,先生们,后面的会议当中没有人会希望做这样
的事情。” 毫无疑问,人们有权戴帽子,也有权不戴;有权结婚,也有权不结婚;有权种大豆,也有权种苹果;也有权开裁
缝店。举个特别的例子:人们有权把橘子卖给愿意买的买主,尽管橘子的直径只有2. 375英 寸 (按照现在的联邦法律,那是非法的)。
\switchcolumn*
It is impossible to enumerate in advance all the rights we
have; we usually go to the trouble of identifying them only
when someone proposes to limit one or another. Treating rights
as tangible claims that must be limited in number gets the
whole concept wrong.
\switchcolumn
事先列举出我们拥有的所有权利是不可能的;只有当有人
提议限制其中的一条或若干条权利的时候,我们才常常不得不
把这些权利甄别出来。把权利当作实际的主张,认为权利的数
量是有限的,这就把整个概念搞错了。
\switchcolumn*
But the complaint about ``the proliferation of rights'' is not
all wrong. There is indeed a problem in modern America with
the proliferation of phony ``rights.'' When rights become
merely legal claims attached to interests and preferences, the
stage is set for political and social conflict. Interests and preferences may conflict, but \textit{rights} cannot. There is no conflict of
genuine human rights in a free society. There are, however,
many conflicts among the holders of so-called welfare rights,
which require someone else to provide us with things we want,
whether that is education, health care, social security, welfare,
farm subsidies, or unobstructed views across someone else's
land. This is a fundamental problem of interest-group democracy and the interventionist state. In a liberal society, people \textit{assume} risks and obligations through contract; an interventionist
state \textit{imposes} obligations on people through the political process,
obligations that conflict with their natural rights.
\switchcolumn
但是,抱怨“权利的数量在不断增加”,这并不完全错。事实上,在现代美国的确存在着虚假的“权利”数量不断增
多的问题。当权利变成只是跟利益和偏好有关的法律主张的时候,就为政治冲突和社会冲突搭好了舞台。不同的利益和偏好
之间可能会冲突,\textbf{但权利与权利之间不会}。在自由社会,真正的人的权利是不会互相冲突的。然而,不同人的所谓的福利权
之间就会冲突不断,因为所谓福利权需要别人为我们提供我们所需要的东西,包括教育、医疗保障、社会保障、福利、农业
补贴,或者要求别人土地上的东西不能挡住自己视野的“视野不受阻挡权” 之类。这是利益集团组成的民主政体和国家
干预主义政权的一个基本问题。在自由社会,人们会\textbf{用契约来解决风险和责任问题},而国家干预主义政权则通过政治程序把
责任\textbf{强加给人民},使他们被强加的责任和他们的自然权利之间产生冲突。
\switchcolumn*
So what rights \textit{do} we have, and how can we tell a real right
from a phony one? Let's start by returning to one of the basicdocuments in the history of human rights, the Declaration of
Independence. In the second paragraph of the Declaration,
Thomas Jefferson laid out a statement of rights and their meaning that has rarely been equaled for grace and brevity. As noted
in chapter 2, Jefferson's task in writing the Declaration was to
express the common sentiments of the American colonists, and
he was chosen for the job not because he had new ideas but because of his ``peculiar felicity of expression.'' Introducing the
American cause to the world, Jefferson explained:
\switchcolumn
那么我们\textbf{到底有}哪些权利?我们如何区分真正的权利和虚假的权利?让我们从回顾人权史上最基础的文献之一《独立宣言》开始来回答这个问题。在 《独立宣言》第二段中,杰斐逊以少有的简洁和优雅的语言对权利及其意义进行了阐述。
我们在第二章中提到过,杰 斐 逊 起 草 《独立宣言》 是在表达北美殖民地人民的共同感受。他被选中做这个工作并不是因为
他有新的思想,而是因为他 那“极其得体的文字表述能力”。
在介绍美国人独立的理由的时候,杰斐逊解释道:
\switchcolumn*
\begin{quote}
	We hold these truths to be self-evident, that all men are created
equal, that they are endowed by their Creator with certain unalienable Rights, that among these are life, liberty, and the pursuit of happiness. That to secure these rights, governments are
instituted among men, deriving their just powers from the consent of the governed. That whenever any form of government
becomes destructive of these ends, it is the right of the people to
alter or abolish it.
\end{quote}
\switchcolumn
\begin{quote}
	我们认为下面这些真理是不证自明的:人人生而平
等,造物者赋予他们若千不可剥夺的权利,其中包括生命
权、自由以及追求幸福的权利。为了保陣这些权利,人类
才在他们之间建立政府,而政府之正当权力,是经被治理
者的同意而产生的。当任何形式的政府对这些目标具破坏
作用时,人民便有权力改变或废除它。
\end{quote}
\switchcolumn*
Let's try to draw out the implications of America's founding
document.
\switchcolumn
让我们来详细描述一下美国建国文献之中的结论


\switchcolumn*[\section{Basic Rights\\基本权利}]
Any theory of rights has to begin somewhere. Most libertarian
philosophers would begin the argument earlier than Jefferson
did. Humans, unlike animals, come into the world without an
instinctive knowledge of what their needs are and how to fulfill
them. As Aristotle said, man is a reasoning and deliberating animal; humans use the power of reason to understand their own
needs, the world around them, and how to use the world to satisfy their needs. So they need a social system that allows them
to use their reason, to act in the world, and to cooperate with
others to achieve purposes that no one individual could accomplish.
\switchcolumn
任何一种关于权利的理论都必须有个起点。大多数古典自
由主义理论家的论证起点都比杰斐逊的论述更靠前。人类和动
物不同,他们并没有与生而来的关于他们需要什么以及如何满
足这些需要的本能知识。亚里士多德说,人是一种理性和善于
思考的动物;人类运用理性来了解他们自己的需要,了解周围
的世界,以及如何利用周围的世界来满足他们的需要。因此,
他们需要一个社会系统,让他们运用自己的理性,在世界上进
行活动,与别人进行合作来完成任何个人都无法完成的目标。
\switchcolumn*
Every person is a unique individual. Humans are social animals---we like interacting with others, and we profit from it---but we think and act individually. Each individual owns himself
or herself. What other possibilities besides self-ownership are
there?
\switchcolumn
每个人都是一个独立的个体,虽然人类是社会动物。我们
喜欢与别人进行交往,并且从中获益,但是我们认为自己的思
想和行为是属于我们个人的。每个个人都对他自己有所有权。
那么除了自我所有权之外,有没有其他种类的所有权呢?
\switchcolumn*
\begin{itemize}
	\item \textit{Someone---a king or a master race---could own others}. Plato and
	Aristotle did argue that there were different kinds of humans,
	some more competent than others and thus endowed with the
	right and responsibility to rule, just as adults guide children.
	Some forms of socialism and collectivism are---explicitly or implicitly-based on the notion that many people are not competent to make decisions about their own lives, so that the more talented should make decisions for them. But that would mean
	there were no universal human rights, only rights that some
	have and others do not, denying the essential humanity of those
	who are deemed to be owned.
\end{itemize}
\switchcolumn
\begin{itemize}
	\item \textbf{有的人对其他人有所有权,即国王或者部族首领对其他人的所有权}。柏拉图和亚里士多德曾经讨论过,世界上有各种不
	同的人,其中有的人比其他人更有竞争力,于是拥有了统治的
	权利和责任,就像成年人引导孩子一样。一些类型的社会主义
	和集体主义建立的理论基础是:许多人没有能力对自己的生活
	作出判断和决策,因而天才的领导人应当替他们作决策。他们
	或者公幵宣传这样的理论,或者在自己的主张中隐含了这种理
	论。但是这就意味着没有普世的人权,只能是有的人有权利,
	另一些人没有权利,从而否定了那些人身所有权被别人占有的
	人的基本人格。
\end{itemize}
\switchcolumn*
\begin{itemize}
	\item \textit{Everyone owns everyone, a full-fledged communist system}. In such
	a system, before anyone could take an action, he would need to
	get permission from everyone else. But how could each other
	person grant permission without consulting everyone else?
	You'd have an infinite regress, making any action at all logically
	impossible. In practice, since such mutual ownership is impossible, this system would break down into the previous one: some one, or some group, would own everyone else. That is what
	happened in the communist states: the party became a dictatorial ruling elite.
\end{itemize}
\switchcolumn
\begin{itemize}
	\item \textbf{每个人都对其他人拥有所有权}。即一种全面的共产主义体
	制。在这种体制下,每个人在做任何事情之前都必须得到其他
	所有人的同意。但每个人同意之间,又都必须得到其他所有人
	的同意这样的话,你将陷入无限的循环之中,想做任何事情在
	逻辑上都是不可能的。实际上,由于这样的相互所有权是不可
	能存在的,这种体制就会破产,并且倒退到前面的那个体制:
	某个人或某个集团对其他所有人拥有所有权。
\end{itemize}
\switchcolumn*
Thus, either communism or aristocratic rule would divide the
world into factions or classes. The only possibility that is humane,
logical, and suited to the nature of human beings is self-ownership.
Obviously, this discussion has only scratched the surface of the
question of self-ownership; in any event, I rather like Jefferson's
simple declaration: Natural rights are self-evident.
\switchcolumn
因此,无论是共产主义还是贵族统治,都将把世界分成不
同的派别和阶级。而唯一的符合人性、符合逻辑、符合人类天
性的只能是自我所有权。当然,这部分讨论仅仅谈到了 “自
我所有权” 问题的皮毛;无论怎样,我更喜欢杰斐逊的简洁
宣言:自然权利是不证自明的。
\switchcolumn*
Conquerors and oppressors told people for millennia that
men were not created equal, that some were destined to rule and
others to be ruled. By the eighteenth century, people had
thrown off such ancient superstition; Jefferson denounced it
with his usual felicity of expression: ``The mass of mankind has
not been born with saddles on their backs, nor a favored few
booted and spurred ready to ride them legitimately by the grace
of God.'' As we enter the twenty-first century, the idea of equality is almost universally accepted. Of course, people are not
equally tall, equally beautiful, equally smart, equally kind,
equally graceful, or equally successful. But they have equal
rights, so they should be equally free. As the Stoic lawyer Cicero
wrote, ``While it is undesirable to equalize wealth, and everyone
cannot have the same talents, legal rights at least should be
equal among citizens of the same commonwealth.''
\switchcolumn
几千年来,征服者和压迫者们告诉人们:人们并不是生而
平等的,有的人注定是统治者,而另一些人注定是被统治者。
18世纪,人们推翻了这个古老迷信。杰斐逊用他一贯简洁的
语言宣告:“人类的大多数不是生来就应该在背上套上马鞍,
也不应该由得天独厚的少数人穿着皮靴,套着靴刺,堂而皇之
地骑在他们背上。” 当进入21世纪的时候,平等的理念已经被全世界人民普遍接受。当然,人们并没有同样的身高,没有同
等漂亮的相貌,并不同样聪明,不是同样善良,也不是同样优
雅 ,不是同样成功,但是他们有平等的权利,因此他们应当同
等自由。斯多噶学派的律师西塞罗写道:“均贫富是不值得追
求的,每个人也不可能拥有同等的才能,但至少同一个城邦的
公民之间的法律权利是平等的。”
\switchcolumn*
In our own time we've seen much confusion on this point.
People have advocated public policies both mild and repressive
to bring about equality of outcomes. Advocates of material
equality apparently don't feel the need to defend it as a principle; ironically, they seem to take it as self-evident. In defending
equality, they typically confuse three concepts:
\switchcolumn
但是在我们这个时代,我们对这一点却常常感到疑惑。人
们很多时候拥护一些导致结果平等的或温和或严厉的公共政策。
支持物质平等的拥护者们显然并没有感到需要将这一点作为原
则来加以辩护,可笑的是,他们好像把这当作了不证自明的事
情。关于平等,他们在为之辩护的时候常常混淆了三个概念:
\switchcolumn*
\begin{itemize}
	\item A right to equality before the law, which is the kind of equality Jefferson had in mind.
\end{itemize}
\switchcolumn
\begin{itemize}
	\item 法律面前平等的权利。这是杰斐逊头脑中所构想的那种平等。
\end{itemize}
\switchcolumn*
\begin{itemize}	
	\item A right to equality of results or outcomes, meaning that
	everyone has the same amount of---of what? Usually egalitarians mean the same amount of money, but why is money
	the only test? Why not equality of beauty, or of hair, or of
	work? The fact is, equality of outcomes requires a political
	decision about measurement and allocation, a decision no society can make without some group forcing its view on others. True equality of results is logically impossible in a diverse
	world, and the attempt to achieve it leads to nightmarish results. Producing equal outcomes would require treating people unequally.
\end{itemize}
\switchcolumn
\begin{itemize}	
	\item 结果平等的权利。意思是每个人都应当拥有同等数量
	的......什么呢?通常平等主义者指的是同等数量的钱,但是为
	什么唯一的标准是钱呢?为什么不是相貌平等、头发平等或者
	工作平等呢?实际上,这种结果平等要求对物品的衡量和分配
	进行政治抉择,而这种政治抉择在任何一种社会当中都不可能
	不通过把一群人的意见强加给其他人来实现。在多样化的世界
	中,真正的结果平等在逻辑上是不可能的,而企图达到这个目
	标的努力会导致噩梦般的结果。为了结果平等,就必须不平等
	地对待人。
\end{itemize}
\switchcolumn*
\begin{itemize}
	\item A right to equality of opportunity, meaning an equal chance
	to succeed in life. People who use ``equality'' this way usually
	mean equal rights, but an attempt to create true equality of
	opportunity could be as dictatorial as equality of results.
	Children raised in different households will never be equally
	prepared for the adult world, yet any alternative to family
	freedom would mean a nanny state of the worst order. Full
	equality of opportunity might indeed lead to the solution
	posed in Kurt Vonnegut's short story ``Harrison Bergeron,''
	in which the beautiful are scarred, the graceful are shackled,
	and the smart have their brain patterns continuously disrupted.
\end{itemize}
\switchcolumn
\begin{itemize}
	\item  机会平等的权利。意思是生活当中获得成功的机会的
	平等。人们在这个意义上说“平等” 的时候,通常是指平等
	的权利,但是企图创造一种真正的机会平等的努力可能会像追
	求结果平等一样独裁专断。在不同家庭成长的孩干决不会在进
	入成年世界时有同样的起跑线,而任何一种替代家庭自由选择的方案都意味着一个奶妈式的国家和更糟的秩序。完全的机会
	平等将会导致库尔特$\cdot$ 冯内古特\footnote{库尔特$\cdot$冯内古特(Kurt  Vonnegut, 1922$\sim$2007),美国作家,被誉为美国黑色幽默文学的代表人物。其代表作为《五号屠宰场》《猫的摇篮》 以及2005年创作的《没有国家的人》。冯内古特抓住了他生活时代的情绪,激发了一代人的想像。}的小说《哈里森$\cdot$伯杰隆》\footnote{《哈里森$\cdot$伯杰隆》(Harrison Bergeron)是一部.科幻小说,说的是2081年,终于人人平等。人们不仅在上帝和法律面前平等,而且在方方面面都一律平		等。没有哪个人比别人聪明些,没有哪个人比别人漂亮些,也没有哪个人比别人	强壮些或者灵巧些。因为美国通过了几条宪法修正案,规定人人必须平等,并且设置了设障上将这个职务,管理了一大群官员,专门通过各种科技手段给比别人漂亮、聪明、身髙比别人离的人设置障碍,让他们变得和别人相貌一样普通、智商一样,身高一样。}那样的结果,长相漂亮人脸上刻上伤疤,举止优雅的人身上则挂上沉重的障碍包,聪明的人则被戴上耳机,被迫接收噪音以干扰思维。
\end{itemize}
\switchcolumn*
The kind of equality suitable for a free society is \textit{equal rights}.
As the Declaration stated clearly, rights are not a gift from government. They are natural and unchanging, inherent in the nature of mankind and possessed by people by virtue of their
humanity, specifically their ability to take responsibility for
their actions. Whether rights come from God or from nature is
not essential in this context. Remember, the first paragraph of
the Declaration referred to ``the laws of nature and of nature's
God.'' What is important is that rights are imprescriptible, that
is, not granted by any other human. In particular, they are not
granted by government; people form governments in order to
protect the rights they already possess.
\switchcolumn
其中适合自由社会的平等是\textbf{权利的平等}。正如《独立宣言》里清晰论述的那样,权利不是政府赐予的礼物。权利是
自然的和不容改变的,是人类与生俱来的,是来自人本身的人的特性,尤其来自人拥有对自己行为负责的能力。在这种情况下 ,无论权利是来自上帝还是来自自然都不重要。不要忘了,《独立宣言》第一段说的是“ 自然的法以及自然的上帝的法”。
重要的是,权利是不可剥夺的。也就是说,权利不是任何其他人所授予的,尤其不是政府所授予的;人们建立政府是为了保护他们业已存在的权利。

\switchcolumn*[\subsection{Self-Ownership\\自我所有权}]
Because every person owns himself, his body and his mind, he
has the right to life. To unjustifiably take another person's
life---to murder him---is the greatest possible violation of his
rights.
\switchcolumn
因为每个人拥有他自己,拥有他自己的身体和头脑,因此他拥有生命权。不公正地剥夺其他人的生命 --- 杀害他 --- 是对他的权利最严重的侵犯。
\switchcolumn*
Unfortunately, the term ``right to life'' is used in two confusing ways in our time. We might do better to stick to ``right to self-ownership.'' Some people, mostly on the political right, use
``right to life'' to defend the rights of fetuses (or unborn children) against abortion. Obviously, that is not the sense in which Jefferson used the term.
\switchcolumn
不幸的是,“生命权”这个词在当前有两种容易让人混淆
的用法。我们也许最好坚持使用“ 自我所有权” 的意思。有
的 人 (大多是政治上的右派)用 “生命权” 的概念来保卫胎
儿 (也就是未出生的婴儿)不被堕胎的权利。这显然不是杰
斐逊使用这个概念时的意思。
\switchcolumn
Other people, mostly on the political left, would argue that
the ``right to life'' means that everyone has a fundamental right
to the necessities of life: food, clothing, shelter, medical care,
maybe even an eight-hour day and two weeks of vacation. But
if the right to life means this, then it means that one person has
a right to force other people to give him things, violating their
equal rights. The philosopher Judith Jarvis Thomson writes, ``If
I am sick unto death, and the only thing that will save my life is
the touch of Henry Fonda's cool hand on my fevered brow, then
all the same, I have no right to be given the touch of Henry
Fonda's cool hand on my fevered brow.'' And if not the right to
Henry Fonda's touch, then why would she have the right to a
room in Henry Fonda's house, or a portion of his money with
which to buy food? That would mean forcing him to serve her,
taking the product of his labor without his consent. No, the
right to life means that each person has the right to take action
in the furtherance of his life and flourishing, not to force others
to serve his needs.
\switchcolumn
另 一 些 人 (大多是政治上的左派)则 主 张 “生命权” 的
意思是每个人都有拥有生活必需品的基本权利:食物、衣服、
住房、医疗,甚至还有每天工作八小时,每年两周休假。但
是,如果生命权是这个意思的话,也就意味着一个人有权强迫
其他人给他提供这些东西,这就侵犯了别人的平等权利。哲学
家朱迪思$\cdot$ 贾维丝$\cdot$ 汤姆森\footnote{朱迪思$\cdot$贾维丝$\cdot$汤姆森(Judith  Jarvis Thomson), 麻省理工学院哲学教授。她在1971年发表了一篇为妇女争取堕胎选择权的经典文章。在这篇名为\textit{A Defence of Abortion}的论文里,汤姆森尝试证明,任何人都应该有权控制自己的身体 ,故此,妇女亦有权决定应否终止怀孕。}说过: “假如我生病快死了,唯一能够救我命的是亨利$\cdot$方达把他的凉手放在我发烧的额头上,尽管如此,我并没有权利让亨利$\cdot$方达把手放在我的额头
上。” 既然她没有权利让亨利$\cdot$ 方达摸她,那么怎么就会有权
利把方达的房子的一部分或者他的钱的一部分拿来买食物呢?
如果这样的话,也就意味着强迫方达为她服务,意味着未经同
意就把他的劳动成果拿走。不 ,生存权的意思是每个人有权为
了生命的存续和发展而采取行动,而不是说有权强迫别人为他
的需求而服务。
\switchcolumn*
Ethical univeralism, the most common framework for moral
theory, holds that a valid ethical theory must be applicable for
all men and women, at whatever time and place we find them.
The natural rights to life, liberty, and property can be enjoyed
by people under any normal circumstances. But so-called rights
to housing, education, medical care, cable television, or the ``periodic holidays with pay'' generously proclaimed in the United
Nations' Universal Declaration of Human Rights, cannot be
enjoyed everywhere. Some societies are too poor to provide
everyone with leisure or housing or even food. And remember
that there is no collective entity known as ``education'' or ``medical care''; there are only specific, particular goods, such as a seat
for a year in the Hudson Street School or an operation performed by kindly Dr. Johnson on Tuesday. Some person or
group of people would have to provide each particular unit of
``housing'' or ``education,'' and providing it to one person necessarily means denying it to other people. Therefore, it is logically
impossible to make such desirable things ``universal human
rights.''
\switchcolumn
道德普遍主义(ethical univeralism) 作为道德理论共同的
框架,认为正确的道德理论适用于任何时间、地点,适用于所有的男人和女人。人的自然权利,生命、自由和财产权应当被
人们在任何正常情况下所享有。但是所谓的住房权、教育权、
医疗权、有线电视权,或 者 “定期带薪休假权” 这 些 《联合
国人权宣言》 中所提出的权利,是不可能在每个地方都能享
受到的。.有的地区太穷,不可能给所有人提供休闲、住房,甚
至不能提供食品。而且并没有一个统一的实体叫做“教育”
或 者 “医疗” ,只有个别的、特定的商品,如在哈得逊街学校
的一年的座位,或者是星期二在好心的约翰逊大夫那里的一次
手术。一个人或者一群人将不得不提供一定数量的“住房”
和 “教育”,而把它提供给某一个人则必然意味着拒绝提供给
其他人。因此,在逻辑上把这些可欲的东西变成“普遍的人
权” 是不可能的。
\switchcolumn*
The right to self-ownership leads immediately to the right to
liberty; indeed, we may say that ``right to life'' and ``right to liberty'' are just two ways of expressing the same point. If people
own themselves, and have both the right and the obligation to
take the actions necessary for their survival and flourishing,
then they must enjoy freedom of thought and action. Freedom
of thought is an obvious implication of self-ownership; in a
sense, though, it's difficult to deny freedom of thought. Who
can regulate the content of someone else's mind? Freedom of
speech is also logically implied by self-ownership. Many governments have tried to outlaw or restrict freedom of speech, but
speech is inherently fleeting, so control is difficult. Freedom of
the press---including, in modern times, broadcasting, cable,
electronic mail, and other new forms of communications---is
the aspect of intellectual freedom that oppressive governments
usually target. And when we defend freedom of the press, we
are necessarily talking about property rights, because ideas are
expressed \textit{through property}---printing presses, auditoriums,
sound trucks, billboards, radio equipment, broadcast frequencies, computer networks, and so on.
\switchcolumn
自我所有权很自然包含自由的权利;事实上,我们可以说
“生存权” 和 “ 自由权”只不过是同一个东西的两种不同表
达。如果人们拥有他自己,并且有权为了他们生命的延续和发
展而采取必要的行动,并且为自己的行为承担责任,那么他们
必须拥有思想和行动的自由。自我拥有权的概念显而易见包含
了思想自由;换句话说,思想自由是很难否认的。谁能规定其
他人头脑中想什么?同样,言论自由在逻辑上也是蕴含在自我
所有权的概念之中。很多国家的政府曾经视言论自由为非法,
或者限制言论自由,但是言论有短暂、易逝的特点,因此控制
言论是很难做到的。新闻和出版自由,在现代还包括广播、电
视、电子邮件以及其他各种形式的新的通讯方式,即知识的自
由,是专制政府所压制的目标。当我们捍卫新闻出版自由的时
候 ,我们一定说的是财产权,因为思想是\textbf{通过财产}来表达的
--- 印刷品、教室、讲堂、广播车、公告牌、广播设备、广播
频率、计算机网络等。

\switchcolumn*[\subsection{Property Rights\\财产权}]
In fact, the ownership of property is a necessary implication of
self-ownership because all human action involves property. How
else could happiness be pursued? If nothing else, we need a place
to stand. We need the right to use land and other property to
produce new goods and services. We shall see that all rights can be understood as property rights. But this is a contentious point,
not always easily understood. Many people wonder why we
couldn't voluntarily share our goods and property.
\switchcolumn
实际上,对财产的拥有也必然是蕴含在自我所有权之中,
因为所有的人类行为都与财产有关。如果没有财产的话,我们
如何追求幸福呢?即使其他什么都不要,我们也必须要一片立
锥之地。我们需要用土地和其他财产来制造新的产品,提供新
的服务。我们将会看到所有的权利都可以被理解为财产权。但
是这是一个容易引起争议的观点,不太容易理解。很多人会问
为什么我们不能自愿共享东西和财产。
\switchcolumn*
Property is a\textit{ necessity}. ``Property'' doesn't mean simply land,
or any other physical good. Property is anything that people
can use, control, or dispose of. A property right means the freedom to use, control, or dispose of an object or entity. Is this a
bad, exploitative necessity? Not at all.
\switchcolumn
财产的存在是一种\textbf{必然}。“财产”并不仅仅指的是土地或
其他物理上的东西。财产是人们能够使用、控制或者出让的东
西。财产权意味着使用、控制或出让一个东西或实体的自由。
那么财产是一种坏的;剥削性的必然吗?绝对不是。
\switchcolumn*
If our world were not characterized by scarcity, we wouldn't
need property rights. That is, if we had infinite amounts of
everything people wanted, we would need no theory of how to
allocate such things. But of course scarcity is a basic characteristic of our world. Note that scarcity doesn't imply poverty or a
lack of basic subsistence. Scarcity simply means that human
wants are essentially unlimited, so we never have enough productive resources to supply all of them. Even an ascetic who had
transcended the desire for material goods beyond bare subsistence would face the most basic scarcity of all: the scarcity of
one's own body and life and time. Whatever time he devoted to
prayer would not be available for manual labor, for reading the
sacred texts, or for performing good works. No matter how rich
our society gets---nor how indifferent to material goods we become---we will always have to make choices, which means that
we need a system for deciding who gets to use productive resources.
\switchcolumn
如果世界不是以稀缺为特征的话,我们就不需要财产权
了。也就是说,如果人们需要的每一种东西的数量都是无限的
话 ,我们就不需要如何分配这些东西的理论了。但是,稀缺当
然是我们这个世界的基本特征。请注意,稀缺的意思并不是贫
穷和缺乏基本生存所需的东西。稀缺仅仅是指人们的欲求必然
是无限的,因此我们永远不可能有足够的资源来满足他们的所
有需要。甚至一个将自己的物欲降低到刚刚达到满足生存所需
的禁欲主义者,他也将面临着最基本的稀缺:他自己的身体和
生命、时间的稀缺。如果他把一段时间用于祷告,则不可能同
时用来做手工劳动、读圣贤书,或者做好事。不论社会富裕到
什么程度,或者说,不论我们对物质财富多么不在乎,我们都
总是会不得不进行选择,也就是说,我们需要一个体制来决定
谁来使用生产资源。
\switchcolumn*
We can never abolish property rights, as socialist visionaries
promise to do. As long as things exist, someone will have the
power to use them. In a civilized society, we don't want that
power to be exercised simply by the strongest or most violent
person; we want a theory of justice in property titles. When socialist governments ``abolish'' property, what they promise is
that the entire community will own all property. But since---visionary theory or no---only one person can eat a particular
apple, or sleep in a particular bed, or stand on a particular spot,
someone will have to decide who. That someone---the party official, or the bureaucrat, or the czar---is the real possessor of the property right.
\switchcolumn
永远都不可能像空想社会主义者所承诺的那样消灭财产权。只要东西存在,就必然有人有权力使用它们。在文明社
会 ,我们不希望那种权力只是操纵在最强壮或者最暴力的人手
里; 我们希望有一种公平的财产所有权理论。当社会主义政府
“消灭” 财产的时候,他们承诺整个社会拥有所有的财产。但
是无论理论是空想还是不空想,一个人只能吃某个苹果,只能
睡某张床,只能站在某个地点。必须由某个人来决定谁吃苹
果、睡床、站在某个地方。那 个 “某个人”就是财产权的真
正拥有者,无论他是党的官员、政府官僚还是沙皇。
\switchcolumn*
Libertarians believe that the right to self-ownership means
that individuals must have the right to acquire and exchange
property in order to fulfill their needs and desires. To feed ourselves, or provide shelter for our families, or open a business, we
must make use of property. And for people to be willing to save
and invest, we need to be confident that our property rights are
legally secure, that someone else can't come and confiscate the
wealth we've created, whether that means the crop we've
planted, the house we've built, the car we've bought, or the
complex corporation we've created through a network of contracts with many other people.
\switchcolumn
古典自由主义者相信自我拥有权意味着个人必须有权获得
和交换财产,以满足他们的需求和欲望。为了给我们自己准备
食物,或者给我们的家庭提供住房,或者做生意,我们必须使
用我们的财产。而且,为使人们有储蓄和投资的愿望,我们需
要确信我们的财产权得到了法律的保护,其他人不能抢走我们
所创造的财富,不论这些财富是我们种下的粮食、建造的房
子、购买的车,还是我们通过一系列契约和其他许多人一起建
立的合作企业。
\switchcolumn*
\textit{Original Acquisition of Property}. How do men and women come
to acquire property in the first place? Perhaps if a spaceship full
of men and women landed on Mars, there would seem to be no
need for conflict over land. Just pick a spot and start building or
planting. A cartoonist once depicted one caveman saying to another, ``Let's cut the earth into little squares and sell them.'' Put
like that, it sounds absurd. Why do that? And who would buy
the little squares? And with what? But as population increases,
it becomes necessary to decide what land---or water or frequency spectrum---belongs to whom. John Locke described
one way to acquire property: Whoever first ``mixed his labor
with'' a piece of land acquired title to it. By mixing his own
labor with a piece of previously unowned land, he made it his
own. He then had the right to build a house on it, put a fence
around it, sell it, or otherwise dispose of it.
\switchcolumn
\textbf{财产的最初获得}。人们最初是如何获得财产的呢?如果一
艘装满男人和女人的太空船降落在火星上,也许他们不会因为
土地发生冲突。他们只需要挑一个地点,开始建造房子或者种
植粮食。有个漫画家曾经画了幅漫画,里面的一个洞穴人对另
一个洞穴人说: “我们把土地分成小块,然后卖掉吧”之类
的,听上去很荒唐。为什么要这么做?谁会买这些小块的土
地?用什么买呢?但是随着人口的增长,就有必要确定哪块土
地归谁,或者确定哪块水域、哪个频率波段属于谁。洛克描述
了一种获得产权的方法:谁 第 一 次 “将自己的劳动与一块当
时无主的土地结合在一起” ,他就获得了土地所有权。通过将
自己的劳动与一块无主的土地结合,他将这块土地变成了自己的。于是他就有权在上面建造房屋,有权在周围竖立篱笆,有
权卖掉这块地,或者用其他的方式处置它。
\switchcolumn*
For each entity there is in fact a bundle of property rights,
which can be disaggregated. There can be as many property
rights attached to one entity as there are aspects of that entity.
For instance, you might purchase or lease the right to drill for
oil on a piece of land, but not the right to farm or build on it.
You might own the land but not the water under it. You might
donate your house to a charity but retain the right to live there
for your lifetime. As Roy Childs wrote in \textit{Liberty Against Power},
``Before there was a technology available to broadcast through
the airwaves, certain kinds of things $\ldots$ could not have been
property, because they could not have been specified by any
technological means.'' But once we understand the physics of
broadcasting, we can create property rights in the frequency
spectrum. Childs went on, ``As a society gets more complicated $\ldots$ and technology advances, the kinds of ownership that are possible to people become more and more complex.''
\switchcolumn
对每一个实体来说,实际上有一连串的产权,而且这些产权可以彼此不相关联。与一个实体相关的可能有许多产权,就像那个实体会有不同的方面一样。例如,你可以购买或者租赁一块土地来挖石油,但是无权在上面种地或者建造房屋。你可以拥有一片土地但不拥有土地下的水。你可以把你的房子捐献给慈善机构,但是保留住在那里的权利,直到你生命结束。柴尔兹(Roy Childs)在《以自由对抗权力》(\textit{Leberty Against Power})一书中说:“在通过无线电波进行广播的技术产生之前,这类东西……不可能有产权,因为没有任何技术手段来界定。” 但是一旦我们了解了广播的物理规律,我们就能够在无线频率波段中创造产权。柴尔兹继续说:“随着社会变得越来越复杂……随着技术进步,对人们来说可能出现的所有权种类将会变得越来越复杂。”
\switchcolumn*
The homesteading principle---initially acquiring a property
title by being the first to use or transform the property---may
operate differently with different kinds of property. For instance,
in a state of nature, when most land is unowned (as if men
landed on a new planet), we might say that simply camping on a
piece of land and remaining there is sufficient to acquire the
property right. Surely laying out the foundation for a house and
then beginning to build it would establish a property right.
Rights to water---whether in lakes, rivers, or underground
pools---have traditionally been acquired in ways different from
land acquisition. When people began to use the frequency spectrum to broadcast in the 1920s, they generally adopted a homestead principle: start broadcasting on a particular frequency, and
you acquire a right to continue using that frequency. (The role of
government in all these cases is simply to \textit{protect}, largely through
the courts, the rights that individuals acquire on their own.) The
important thing, as I'll discuss later, is that we have some way of
establishing property rights and then that we allow people to
transfer them to others by mutual consent.
\switchcolumn
先占原则,即通过首先使用或改造一片土地或财产而获得
最初产权的原则,可能会通过各种不同形式的产权而有不同的
实现方式。例如, 在自然状态下,大多数土地是无主的(就
像人们降落在一个新的星球上一样),我们可以说只需要在一
片土地上扎营并且住下来,就充分证明获得了这片土地的产
权。只要确实打下地基,并且开始建造房子就将确立产权。水
域 ,包括湖泊、河流和地下水的产权传统上的获得方式与土地
产权获得的方式不同。人们在20世纪20年代开始使用无线电
频率进行广播,他们总体上采用先占原则:谁首先使用某个频
率进行广播,谁就拥有继续使用那个频率的权利。而政府在所
有这些场合当中的角色,仅仅是对各人自己已经获得的权利进
行\textbf{保护}(大部分是通过司法审判来进行)。我后面将讨论到,重要的是我们有建立产权的途径,以及允许人们通过双方同意
把产权转让给他人。  
\switchcolumn*
\textit{Property Rights Are Human Rights}. What exactly does it mean to
own property? We might cite Jan Narveson's definition: ``'x is
A's property' means 'A has the right to determine the disposition of x.''' Note that a property right is not a right of property, or a right \textit{belonging to} a piece of property, as opponents of property rights often suggest. Rather, a property right is a human right \textit{to} property, the right of an individual to use and dispose of property that he has justly acquired. Property rights are human rights.
\switchcolumn
\textbf{财产权即人权}。拥有财产到底意味着什么呢?我们可以引
用纳维森(Jan Narveson)的定义: “某物是甲的财产的意思
是,甲有权决定如何支配某物。” 需要注意的是,财产权并不
等于 “财产的权利”,或者一种从属于一份财产的权利,就像
财产权的反对者经常说的那样。相反,\textbf{财产权是一种人对财产拥有的权利,是一种个人使用和支配他公平地获得财产的权
利。财产权即人权}。
\switchcolumn*
Indeed, as argued above, all human rights can be seen as
property rights, stemming from the one fundamental right of
self-ownership, our ownership of our own bodies. As Murray
Rothbard put it in \textit{Power and Market},
\switchcolumn
实际上,就像上面我们说过的那样,所有的人权都可以被
看作财产权,来源于最基本的自我所有权,即我们对自己身体
的所有权。如罗斯巴德在《权力与市场》(\textit{Power and Market})一书中所说的:

\switchcolumn*
\begin{quotation}
In the profoundest sense there are no rights but property
rights$\ldots$ There are several senses in which this is true. In the
first place, each individual, as a natural fact, is the owner of \textit{himself}, the ruler of his own person. The ``human'' rights of the person that are defended in the purely free-market society are, in effect,
each man's \textit{property right} in his own being, and from this property
right stems his right to the material goods that he has produced.


In the second place, alleged ``human rights'' can be boiled
down to property rights ... for example, the ``human right'' of
free speech. Freedom of speech is supposed to mean the right of
everyone to say whatever he likes. But the neglected question is:
Where? Where does a man have this right? He certainly does not
have it on property on which he is trespassing. In short, he has
this right only either on his \textit{own} property or on the property of
someone who has agreed, as a gift or in a rental contract, to allow
him on the premises. In fact, then, there is no such thing as a separate ``right to free speech''; there is only a \textit{man's property} right: the right to do as he wills with his own or to make voluntary
agreements with other property owners [including those whose
property may consist only of their own labor].
\end{quotation}
\switchcolumn
\begin{quotation}
最深地探究下去,除了财产权之外就没有别的权利了
……从不同的几个角度看,这是真理。首先一点,每个个
人作为自然的存在,是他自己的所有者,是\textbf{他自己本人}的
统治者。在纯粹的自由市场社会当中,人们所捍卫的人的“人”权,从实践意义上,实际上就是\textbf{每个人对自己的存在所拥有的产权},并且从这个产权开始引申出他对他生产的物品拥有的权利。


第二点,列举出来的 “人权” 可以归结为财产权……比如,言论自由的“人权”。言论自由被认为意味着
每个人说他们想说的话的权利。但是被忽略的问题是:在
哪儿?他的这项权利在哪里实现?他当然无权在他侵占的
财产上实现这项权利。简单地说,\textbf{他只能在自己的财产
上},或者在别人的财产上获得别人使用其场地的同意的情
况下,实现言论自由的权利,而获得别人使用场地的同意可以通过对方免费的许可或签订租赁合同的形式。那么实
际上,天下并没有一种独立的叫做“言论自由” 的东西;
只有\textbf{个人的产权}:按照自己的意愿做事的权利或者和其他
的财产所有者达成自愿协议的权利,包括和那些财产仅仅
只包括他们自己劳动的人。	
\end{quotation}
\switchcolumn*
When we understand free speech this way, we see what's wrong
with Justice Oliver Wendell Holmes's famous statement that
free speech rights cannot be absolute because there is no right
to falsely shout ``Fire!'' in a crowded theater. Who would be
shouting ``Fire''? Possibly the owner, or one of his agents, in
which case the owner has defrauded his customers: he sold
them tickets to a play or movie and then disrupted the show,
not to mention endangered their lives. If not the owner, then
one of the customers, who is violating the terms of his contract;
his ticket entitles him to enjoy the show, not to disrupt it. The
falsely-shouting-fire-in-a-crowded-theater argument is no reason to limit the right of free speech; it's an illustration of the
way that property rights solve problems and of the need to protect and enforce them.
\switchcolumn
当我们用这种方式来理解言论自由的时候,就会发现霍尔
姆 斯 (Oliver  Wendell Holmes) 大法官的说法错在什么地方。
他说:言论自由的权利不是绝对的,因为人们没有权利在拥挤
的剧场妄称起火。谁会妄称起火呢?也许是剧院老板或者他的
雇员,那么在这种情况下是剧院老板对观众进行了欺诈:他卖
票给他们看戏或者看电影,然后扰乱了这场戏,还让观众面临
生命危险。如果不是老板干的,而是一位观众干的,他就违反
了合同的条款;他购票的合同是让他欣赏这场戏的,而不是来
捣乱的。关于在拥挤的剧院妄称起火的争论并不能推导出应当
限制言论自由的权利;相反,这恰恰是一个证明产权可以解决
问题的例证,恰恰说明我们需要保护和强化产权。
\switchcolumn*
The same analysis applies to the much-debated right to privacy. In the 1965 case \textit{Griswold v. Connecticut}, the Supreme
Court struck down a Connecticut law prohibiting the use of
contraceptives. Justice William O. Douglas found a right to privacy for married couples in ``penumbras, formed by emanations'' from various parts of the Constitution. Conservatives
such as Judge Robert Bork have ridiculed such vague, rootless
reasoning for thirty years. The penumbras kept on emanating
to take in an unmarried couple's right to contraception and a
woman's right to terminate a pregnancy, but suddenly in 1986
they were found not to emanate far enough to cover consensual
homosexual acts in a private bedroom. A theory of privacy
rooted in property rights wouldn't have needed penumbras and
emanations---which, penumbras being imperfect shadows, are
necessarily pretty vague---to find that a person has a right to
purchase contraceptives from willing sellers or to engage in sexual relations with consenting partners in one's own home. ``A
man's home is his castle'' provides a stronger foundation for privacy than ``penumbras, formed by emanations.''
\switchcolumn
同样的分析也适用于反复讨论的隐私权问题。在1965年的“格里斯沃尔德诉康涅狄格州案”(\textit{Griswold v. Connecticut})
中,最高法院废除了康涅狄格州的一条法律,该法律禁止使用避孕药品。道 格 拉 斯 (William  O. Douglas) 大法官在宪法的不同部分的条款中通过“权利半影”\footnote{“权利半影”(Penumbras, formed by emanations),意指由一些已知权利推论出来的权利。这些权利就像已知权利的“半影” 并由已知权利賦予其效力。}的推理,找到了对已婚夫妇隐私权保护的模糊依据。保 守 主 义 者 如 博 尔 克(Robert Boric)大法官对这种含糊不清的、没有根据的推理嘲笑了30年。而 这 个 “半影”继续扩展到未婚伴侣的避孕权以及妇女
中止怀孕的权利,但 是 在 1986年 ,他 们 突 然 发 现 这 个“半
影”并没有扩展到足够保护在私人卧室里双方同意的同性恋
行为。一种建立在财产权基础之上的隐私权理论并不“权利
半影” 推理的帮助,因为半明半暗的“半影”必然是模糊不
清的。隐私权理论认为,人们有权从愿意出售的买方那里购买
避孕药品,只要取得对方同意,任何人都有权在自己家里和任
何伙伴发生性关系。“一个人的家就是他的城堡”为隐私权提
供了更坚强的基础,而 不 是 “通过法律条款中寻找模糊的依
据 ,并对其进行扩展而建立起来的保护伞”。
\switchcolumn*
Those who reject the libertarian principle of property rights
need to do more than criticize. They need to offer an alternative
system that would as effectively define who may use each particular resource and in what ways, ensure that land and other
property is adequately cared for, provide a framework for economic development, and avoid the war of all against all that can
ensue when control over valuable goods is not clearly defined.
\switchcolumn
那些反对古典自由主义的财产权原则的人,需要做的不仅
仅是批评。他们需要提出一种替代财产权的制度,能够和财产
权制度同样有效地界定谁来使用每份资源,以及以何种方式来
使用。替代制度要能够确保土地和其他财产权得到合理的使
用 ,能够提供一种经济发展的制度框架,以及避免由于对有价
值的物品没有进行清晰的产权界定而导致的所有人对所有人的
战争。

\switchcolumn*[\section{Nozick's Entitlement Theory of Justice\\诺齐克的权利正义理论}]
In his 1974 book \textit{Anarchy, State, and Utopia}, the Harvard
philosopher Robert Nozick discussed alternative conceptions of
property rights in a very illuminating way. This subject is frequently called ``distributive justice,'' but that term biases the discussion. As Nozick points out, the term as often used implies
that there is some process of distribution, which may have gone
awry and which we may want to correct. But in a free society
there is no central distribution of resources. Milton Friedman
tells of visiting China in the 1980s and being asked by a government minister, ``Who in the United States is in charge of materials distribution?'' Friedman was left almost speechless by the
question but had to explain that in a market economy there is no
person or committee ``in charge of materials distribution.'' Millions of people produce goods---through a complex network of
contracts in an advanced economy---and then exchange them.
As Nozick says, ``What each person gets, he gets from others
who give it to him in exchange for something, or as a gift.''
\switchcolumn
1974年出版的《无政府、国家与乌托邦》中,哈佛大学哲学家诺齐克对替代财产权的各种概念进行了清晰明了的讨
论。这个讨论主题常常被称作“分配公平”, 但是这个词组是先入为主地偏向了一方。诺齐克指出,这个常用的词组隐含的
意思是的确存在某种分配的程序,只不过被扭曲了,我们希望
纠正它。但是在自由社会当中并不存在对资源的中央分配。米
尔 顿 $\cdot$ 弗里德曼1980年代访问中国时,中国的一位部长曾经
问他:“在美国谁负责物资分配?” 对这个问题,弗里德曼几
乎无言以对。他不得不解释,在市场经济当中并没有某个人或
者某个委员会“负责物资分配”。在一个发达的经济体当中,
数以百万计的人们通过一个复杂的由契约结成的网络来生产产
品,然后拿来交换。正如诺齐克所说: “每个人所得到的东
西,或者是由其他人为了与他交换某些东西而给他的,或者是
作为礼物赠送给他的。”
\switchcolumn*
Nozick suggests that there are two ways of looking at the
question of justice in property rights. The first is historical: if
people acquired their property justly, then they are entitled to
it, and it would be wrong to interfere by force to redistribute
property. The other view is based on patterns or end results or
what he calls ``current time-slice principles.'' That is, ``the justice of a distribution is determined by how things are distributed (who has what) as judged by some structural principle of
just distribution.'' Advocates of a patterned distribution ask not
whether property was justly acquired but whether today's pattern of distribution fits what they consider the correct pattern.
There are many kinds of patterns people might prefer: whites
should have more property (or money or whatever) than blacks,
Christians should have more than Jews, smart people should
have more, good people should have more, people should have
what they need. Some of those views are abhorrent. Others may
well be held by your friends and other decent people. But what
they all have in common is this: They assume that a just distribution is determined by who has what, without any reference to
how it was obtained. The view most likely to be held by critics
of capitalism today, however, is that everyone should have equal
property, or no one should have more than twice as much as
anyone else, or some other variant of equality. So that is the alternative to libertarianism we'll consider.
\switchcolumn
诺齐克指出,对财产权的公平问题有两种不同的看法。第
一种是历史原则:如果人们获得财产的方式是正当的,那么他
们就对其拥有产权,而通过强制干预来重新分配财产是错误
的。而另一种看法则建立在某些模式化的分配原则和最终结果
之上,或者他称 之 为 “即时时间片断原则”(current time-slice principles) , 也 就 是 说 “分配正义决定于东西如何分配
(谁得到什么),而这又是通过某种结构性的分配正义原则来
判断的”。模式化分配的拥护者不会问财产是否是正当获得,
而是关心现有的分配模式是否符合他们所认为合适的模式。有
很多种人们喜欢的分配模式:白人应当比黑人拥有更多财产
(或者钱和其他东西)、基督徒应当比犹太人得到更多、聪明
人应当得到更多、好人应当得到更多、按需分配等。这些观点
有的是令人憎恶的,但除了那些可厌的观点之外,持其他一些
观点的也许正是你的朋友或者其他的正派好人。但是他们的共
同点是:他们认为,分配正义是由谁应得到什么来决定的,而
不关心这些东西是如何获得的。然而,今天资本主义的批评者
最主要的观点是,每个人应当拥有相等的财产,或者不应有人
拥有超过任何其他人两倍的财产,或者是其他各种花样的平
等。这就是我们应当考虑的替代古典自由主义的理论。
\switchcolumn*
Nozick lays out his entitlement theory of justice this way:
First, people have a right to acquire unowned property. That's
the principle of justice in acquisition. Second, people have a
right to give their property to others, or to voluntarily exchange
it with others. That's the principle of justice in transfer. Thus,
\switchcolumn
诺齐克是这样论述他的权利正义理论的:首先,人们有权
获得无主的财产,即获得的正义原则。其次,人们有权把财产转给其他人,或者与其他人进行自愿交换,即转让的正义原
则。因此:
\switchcolumn*
\begin{quotation}
	If the world were wholly just, the following inductive definition
	would exhaustively cover the subject of justice in holdings:
	\begin{enumerate}
		\item A person who acquires a holding in accordance with the
		principle of justice in acquisition is entitled to that holding.
		\item A person who acquires a holding in accordance with the
		principle of justice in transfer, from someone else entitled to
		the holding, is entitled to the holding.
		\item No one is entitled to a holding except by (repeated) applications of 1 and 2. 
	\end{enumerate}
	The complete principle of distributive justice would say simply that a distribution is just if everyone is entitled to the holdings they possess under the distribution.
	
	
	A distribution is just if it arises from another just distribution
	by legitimate means.
\end{quotation}
\switchcolumn
\begin{quotation}
	如果这个世界是完全正义的,那么以下归纳的定义就完全覆盖了财产的正义:
	\begin{enumerate}
		\item 一个人依照获得的正义原则而取得财产,即有权拥有那份财产。
		\item 一个人依照转让的正义原则获得财产,并且因此从所有人那里获得持有的资格,他就有权拥有那份财产。
		\item 除非是通过上述1和 2 的 (重复),无人应当对财产拥有权利。
	\end{enumerate}
	完整的分配正义原则简单说就是:如果每个人对分配在其名下的财产都是有权利的,那么这个分配是正义的。
	
	
	如果分配是通过正当途径的公平分配,则分配是正义的。
\end{quotation}
\switchcolumn*
Once people have property (including the labor of their own
minds and bodies, which they have inherently), they may legitimately exchange it with someone else for any property that
person has legitimately acquired. They may also give it away.
What people may not do is take another person's property without his consent.
\switchcolumn
一旦人们拥有财产(包括他们自己与生俱来的头脑和身
体),他们就可以与其他人合法取得的财产进行合法交换。他
们也可以放弃财产。人们不可以做的就是未经他人同意就拿走
他的财产。
\switchcolumn*
Nozick goes on to discuss the question of equality in a famous section of his book called ``How Liberty Upsets Patterns.''
Suppose we begin with a society in which wealth is distributed
in the way that you think is best. It could be that all the Christians have more than all the Jews, or that the members of the
Communist Party own all the property (except for our individual bodies), or whatever. But let's assume that your favorite
pattern is that everyone have an equal amount of wealth, and
that's what we see in our hypothetical society. Now consider
just one intervening event.
\switchcolumn
诺齐克在他书中的著名章节“ 自由如何搅乱模式化分配”
当中继续讨论平等的问题。假设我们从一个财富已经按照你认
为的最好方式分配完毕的社会开始。也许是所有的基督徒拥有
比犹太人多的财产,或者是党员拥有所有财产(除了我们个
人的身体之外),或者其他模式。假设你最喜欢的模式是每个
人都拥有相同数量的财富,而这正是我们假想的这个社会的实
际情况。那么,现在我们来看一下一个假设的事件。
\switchcolumn*
Suppose that the rock group Pearl Jam goes on a concert
tour. They charge people \$10 to see them play. During the tour
a million people come to their concerts. At the end of the tour a
million people are \$10 poorer than they were, and the members
of Pearl Jam are \$10 million richer than everyone else. Here's
the question: The distribution of wealth is now unequal. Is it
unjust? If so, why? We agreed that the distribution of wealth at
the beginning was just, because we stipulated that it was your
preferred distribution. Each person at the beginning was presumably entitled to the money he had, and thus entitled to
spend it as he chose. Many people exercised their rights, and
now the Pearl Jam musicians are richer than everybody else. Is
that wrong?
\switchcolumn
假设摇滚乐队“珍珠酱” 进行巡回演出。他们对每个看演出的人收取10美元的票价。在整个巡回演出当中,有 100
万观众看了他们的演出。巡回演出结束之后,有 100万人的财
产比演出之前少了 10美元,而 “珍珠酱”乐队则比其他每一
个人财产多了 1000万美元。问题来了:现在财产的分配是不
平等的了。但这是不正义的吗?如果是不正义的,为什么?我
们都同意最初的财富分配是正义的,因为我们已经假定这是你
喜欢的分配模式。假设每个人在最开始的时候都对自己的钱拥
有权利,也就是说有权按照自己的选择来把钱花掉。人们实践
了他们的权利,现 在 “珍珠酱” 的音乐家们比其他每个人都
有钱。这错了吗?
\switchcolumn*
All those people chose to spend their money that way. They
could have bought Michael Jackson albums or granola or copies
of the \textit{New York Review of Books}. They could have given money to the Salvation Army or to Habitat for Humanity. If they were
entitled to the money they had in the beginning, surely they are
entitled to spend it, in which case the pattern of wealth distribution will change.
\switchcolumn
所有人都选择以那种方式花他们的钱。他们本来可以购买
迈克尔$\cdot$ 杰克逊的专辑,或者购买格兰诺拉麦片,或者购买
《纽约书评》。他们本来可以把钱捐赠给救世军或者人类家园
组织。如果在最初他们有权拥有这些钱,当然意味着他们有权
花这些钱,于是在这个案例当中财产的模式化分配发生了
改变。
\switchcolumn*
Whatever the pattern is, as different people choose to spend
their money, and choose to offer goods or services to other people in order to get more money to spend, the pattern will be
constantly changing. Someone will go to Pearl Jam and offer to
promote their concerts in return for some of the gate receipts,
or to produce albums and sell them. Someone else will start a
print shop to produce the tickets for their concerts. As Nozick
says, to prevent inequality in wealth, you would have to ``forbid
capitalist acts between consenting adults.'' He goes on to point
out that no pattern of distribution can be maintained ``without
continuous interference with people's lives.'' Either you have to
continuously stop people from spending money as they choose,
or you have to continuously---or at regular intervals---take
from people money that other people chose to give them.
\switchcolumn
无论这个模式是什么,只要不同的人对如何花他们的钱有
权选择,或者有权选择为其他人提供商品和服务来获得金钱,
模式化分配就会不断地变化。有的人会找到“珍珠酱” 乐队,
为他们提供宣传演唱会的服务,回报是得到他们的部分门票收
人,或者为他们制作并销售专辑。有人会开一家印刷店为演唱
会印制门票。就像诺齐克说的,为了防止财富的不平等,你将
不 得 不 “禁止成年人之间的双方自愿合意的资本主义行为”。
他继续指出,没有一种模式化分配能 够 “不通过对人们的生
活进行不断的干预” 而维持。你或者不得不不断地制止人们
按照他们的选择花钱,或 者 不 断 地 (或者定期的干预)通过
把其他人选择给他们的钱从人们手里拿走。
\switchcolumn*
Now it's easy to say that we don't mind rock musicians get-
ting rich. But of course the same principle applies to capitalists,
even billionaires. If Henry Ford invents a car that people want
to buy, or Bill Gates a computer operating system, or Sam Walton a cheap and efficient way to distribute consumer goods, and
we're allowed to spend our money as we choose, then they will
get rich. To stop that, we would have to stop consenting adults
from spending their money as they choose.
\switchcolumn
现在我们可以轻松地说我们不在乎摇滚音乐家们有钱。但
是当然同样的原则也适用于资本家,甚至亿万富翁。如果亨
利 $\cdot$ 福特发明了一种人们愿意买的汽车,或者比尔$\cdot$盖茨发明
了一种计算机操作系统,或者山姆$\cdot$华生发明了一种便宜而高
效的消费品物流方式,同时我们被允许按照自己的选择支配我
们的钱,那么他们必然会变得有钱。如果要阻止这种情况发
生,我们就不得不禁止成年人双方自愿合意地按照自己的选择
来支配他们的钱。
\switchcolumn*
But what about their children? Is it fair that the mogul's
children will be born to greater wealth, probably leading to better education, than you or me? The question misunderstands
the nature of a complex society. In a primitive village, comprising a few people who were probably an extended family, it was
appropriate to distribute the tribe's goods on the basis of ``fairness.'' But a diverse society will never agree on a ``fair'' distribution of goods. What we can agree on is justice---that people
should be able to keep what they produce. That means not that
Henry Ford's son had a ``right'' to inherit wealth, but that
Henry Ford had a right to acquire wealth and then to give it to
anyone he chose, including his children. A distribution by a central authority---how your father doles out allowances, or
how a teacher assigns grades---may be deemed fair or unfair.
The complex process by which millions of people produce
things and sell or give them to others is a different kind of
process, and it makes no sense to judge it by the rules of fairness
that apply to a small group under central direction.
\switchcolumn
但是,他们的孩子呢?富豪的孩子一生下来就拥有巨额财
富,也许比你我受到更好的教育,这公平吗?提出这个问题说
明提问者没有理解复杂社会的实质。在一个原始村落,只住着
很少的人,也许他们还属于同一个家族,以 “公平”为基础
分配部落物品是适当的方式。但是在一个多样化的社会,我们
将 无 法 就“公平” 分配达成一致。我们能够达成一致的是公
正 --- 人们应当拥有他们生产的东西。也就是说不是亨利$\cdot$福
特 的 儿 子 “有权” 继承财富,而 是 亨 利 $\cdot$福特有权获得财富
以及把财富交给任何他选择的人,包括他的孩子。由中央政府
进行分配,例如决定你父亲如何进行慈善捐赠,决定一个教师
如何安排学分,可能会被认为公平或不公平。而数以百万计的
人们生产东西、卖东西或者把东西给其他人的复杂过程是一个
完全不同的过程。用那种适用于中央计划的小团体的公平原则
来判断它是不恰当的。
\switchcolumn*
According to the entitlement theory of justice, people have a
right to exchange their justly acquired property. Some ideologies have a principle of ``to each according to his------.'' For
Marx it was ``from each according to his ability; to each according to his need.'' Note that Marx separates production and distribution; in between those two clauses there's some authority
deciding what your ability and my need are. Nozick offers a libertarian prescription, integrating production and distribution in a just system:
\switchcolumn
按照权利正义理论,人们有权用他们正当获得的财产进行
交换。一些意识形态理论有一个“按什么分配” 的原则。例
如,马克思说的“按劳分配” “按需分配”。我们注意到马克
思将生产和分配分割开来,在这两个阶段之间由某种权威来决
定你的能力和需要是什么。诺齐克提出了一种古典自由主义的规则,将生产和分配整合在同一个公正的体制下:
\switchcolumn*
\begin{quote}
From each according to what he chooses to do, to each according
to what he makes for himself (perhaps with the contracted aid of
others) and what others choose to do for him and choose to give
him of what they've been given previously (under this maxim)
and haven't yet expended or transferred.
\end{quote}
\switchcolumn
\begin{quote}
按每个人所选择去做的事来生产,按每个人通过自己
或者通过契约在其他人的帮助下所生产的东西,以及其他
人选择为他做的事情,以及选择把自己拥有的并且尚未花
掉和转让掉的东西给他来分配。
\end{quote}
\switchcolumn*
That lacks the vigor of a good slogan. So, to paraphrase Nozick,
we can sum it up as

\textit{\textit{From each as he chooses, to each as he is chosen.}}
\switchcolumn
当然,这段话缺少一个响亮的口号。为了更好地阐释诺齐
克的理论,我们可以把这段话总结为:

\textbf{\textbf{按选择生产,按选择分配}}。

\switchcolumn*[\section{The Nonaggression Axiom\\互不侵犯原则}]
What are the limits of freedom? The corollary of the libertarian
principle that ``every person has the right to live his life as he
chooses, so long as he does not interfere with the equal rights of
others'' is this:
\switchcolumn
那么自由的限度是什么?古典自由主义原则“每个人有
权按照自己的选择来生活,同时尊重别人的同等权利” 的自
然推论是:
\switchcolumn*
\begin{quote}
\textit{\textit{No one has the right to initiate aggression against the person or property of anyone else.}}
\end{quote}
\switchcolumn
\begin{quote}
	\textbf{\textbf{任何人都无权首先侵犯别人的人身和财产}}
\end{quote}
\switchcolumn*
This is what libertarians call the nonaggression axiom, and it is
a central principle of libertarianism. Note that the nonaggression axiom does not forbid the retaliatory use of force, that is, to regain stolen property, to punish those who have violated the
rights of others, to rectify an injury, or even to prevent imminent injury from another person. What it does state is that it is
wrong to use or threaten physical violence against the person or
property of another who has not himself used or threatened
force. Justice therefore forbids murder, rape, assault, robbery,
kidnapping, and fraud. (Why fraud? Is fraud really an initiation
of force? Yes, because fraud is a form of theft. If I promise to sell
you a Heineken for a dollar, but I actually give you Bud Light,
I have stolen your dollar.)
\switchcolumn
这就是古典自由主义所说的互不侵犯原则,这也是古典自
由主义的中心原则。需要注意的是,互不侵犯原则并不禁止报
复性地使用暴力,例如拿回被偷走的财产,惩罚那些侵犯了他
人权利的人,纠正伤害,甚至阻止对另一个人的正在发生的侵
害。需要说明的是,如果对方并没有使用暴力或者威胁使用暴
力 ,那么对对方的人身和财产使用暴力或者威胁使用暴力就是
错误的。因此,司法禁止谋杀、强奸、殴打、抢劫、绑架和欺
诈。那么为什么有欺诈?难道欺诈是使用暴力吗?是的,因为
欺诈是一种盗窃。如果我答应以一美元的价格卖给你一瓶喜力
啤酒,但我实际上给了你一瓶百威低度啤酒,我就是偷了你
的钱。
\switchcolumn*
As noted in chapter 1, most people habitually believe in and
live by this code of ethics. Libertarians believe this code should
be applied consistently, to actions by governments as well as by
individuals. Rights are not cumulative; you can't say that six
people's rights outweigh three people's rights, so the six can
take the property of the three. Nor can a million people ''combine'' their rights into some cumulative right to take the property of a thousand. Thus libertarians condemn government
actions that take our persons or our property, or threaten us
with fines or jail for the way we live our personal lives or the
way we engage in voluntary interactions with others (including
commercial transactions).
\switchcolumn
我在第一章中提到过,大多数人是基于习惯相信这种道德
准则,并且按照这种道德准则来生活。古典自由主义者则认为
这种准则应当一以贯之地应用,不管是对个人还是政府行为。
权利不是累加的,你不能说6 个人的权利比3 个人的权利更
重。同样,你不能说100万人把他们的权利联合起来,就可以
拿 走 1000个人的财产。因此,古典自由主义者谴责政府剥夺
我们人身和财产的行为,谴责因我们按照个人生活方式来生
活、或者与他人之间的自愿互动(包括商业交换)而被威胁
罚款和监禁。
\switchcolumn*
Freedom, in the libertarian view, is a condition in which the
individual's self-ownership right and property right are not invaded or aggressed against. Philosophers sometimes call the libertarian conception of rights ``negative liberty,'' in the sense that
it imposes only negative obligations on others---the duty not to aggress against anyone else. But for each individual, as Ayn
Rand puts it, a right is a moral claim to a positive---``his freedom to act on his own judgment, for his own goals, by his own \textit{voluntary, uncoerced} choice.''
\switchcolumn
在古典自由主义者看来,自由是一种状态,在这种状态下
个人的自我所有权和财产权不受侵占和侵犯。理论家有时把古
典自由主义的权利概念称作“消极权利”,这种权利只要求人
们对其他人负有消极责任,即不侵犯其他任何人的责任。但是
正如安$\cdot$ 兰德所说,对每个个人来说,权利的主张其实上是积
极的,即 “为了他自己的目标,通过\textbf{他自己自愿的、非强制的}选择,按照自己的判断行动的自由”。
\switchcolumn*
Communitarians sometimes say that ``the language of rights
is morally incomplete.'' That's true; rights pertain only to a certain domain of morality---a narrow domain in fact---not to all
of morality. Rights establish certain minimal standards for how
we must treat each other: we must not kill, rape, rob, or otherwise initiate force against each other. In Ayn Rand's words,
``The precondition of a civilized society is the barring of physical
force from social relationships---thus establishing the principle that if men wish to deal with one another, they may do so only
by means of \textit{reason}: by discussion, persuasion and voluntary, uncoerced agreement.'' But the protection of rights and the establishment of a peaceful society is only a \textit{precondition} for
civilization. Most of the important questions about how we
should deal with our fellow men must be answered with other
moral maxims. That doesn't mean that the idea of rights is invalid or incomplete \textit{in the domain where it applies}; it just means
that most of the decisions we make every day involve choices
that are only broadly circumscribed by the obligation to respect
each other's rights.
\switchcolumn
社群主义者有时说“用权利来描述整个道德是不完整的”,这是对的;权利仅仅适合于某个范围内的道德 --- 实际
上是很窄的范围 --- 而不适合于所有的道德。权利为我们如何
对待他人建立了一种最低标准:我们不能杀害、强奸、抢劫他
人或者对他人使用暴力。用安$\cdot$兰德的话说就是:“文明社会
的前提条件是在社会关系当中禁止使用暴力 --- 由此建立的原
则是:如果人们希望与其他人相处,他就必须只能通过讲理的
方式:通过讨论、说服和自愿的非强制性合约。”但是保护权
利和建立一个和平社会仅仅是文明的前提。关于我们如何对待
别人的绝大多数重要问题还需要通过其他道德准则来回答。但
这并不是说权利的理念在其适用的范围内是无效的和不完整的;它只是意味着我们每天所做的大多数决定和选择的边界是
我们尊重彼此的权利。

\switchcolumn*[\section{Implications of Natural Rights\\自然权利理论的推论}]

The basic principles of self-ownership, the law of equal freedom, and the nonaggression axiom have infinite implications.
As many ways as the state can think of to regulate and expropriate people's lives, that's how many rights libertarians can
identify.
\switchcolumn
从自我所有权的基本原则、同等自由的法则以及互不侵犯
原则可以推导出无限的推论。政府能够想出多少办法来管制和
剥夺人们的生活,古典自由主义就能够提出多少权利。
\switchcolumn*
The most obvious and most outrageous attempt to violate
the right of self-ownership is involuntary servitude. From time
immemorial, people claimed the right to hold others as slaves.
Slavery wasn't always racial; it generally began as the spoils of
victory. The conquerors had the power to enslave the conquered. The greatest libertarian crusade in history was the effort  to  abolish  chattel  slavery,  culminating  in  the
nineteenth-century abolitionist movement and the heroic Underground Railroad. But despite the Thirteenth Amendment to
the Constitution, which abolished involuntary servitude, we
still see vestiges of it to this day. What is conscription---the military draft---if not temporary slavery (with permanent consequences, for those draftees who don't come home alive)? No
issue today more clearly separates libertarians from those who
put the collective ahead of the individual. The libertarian believes that people will voluntarily defend a country worth defending, and that no group of people has the right to force
another group to give up a year or two of their lives---and possibly life itself---without their consent. The basic liberal principie of the dignity of the individual is violated when individuals
are treated as national resources. Some conservatives (such as
Senator John McCain and William F. Buckley, Jr.) and some of
today's so-called liberals (such as Senator Edward M. Kennedy
and Ford Foundation president Franklin Thomas) advocate a
system of compulsory national service in which all young people would be required to spend a year or two working for the
government. Such a system would be an abominable violation
of the human right of self-ownership, and we can only hope
that the Supreme Court would find it unconstitutional under
the Thirteenth Amendment.
\switchcolumn
最明显的和最恶劣的侵犯人们自我所有权的方法是非自愿
的奴役。从远古以来,就有很多人宣布有权把别人当作奴隶。
奴隶制并不一直是种族之间的,它总的来说开始于战争的掠
夺。征服者拥有把被征服者当作奴隶的权力。历史上最伟大的
古典自由主义圣战就是废除奴隶制的斗争,这场圣战在19世
纪废奴运动和英雄史诗般的地下铁路\footnote{“地下铁路”(Underground  Railroad),指 19世纪千千万万美国人自发组织的帮助南方黑奴逃到北方和加拿大并获得自由的运动。据统计,1810 $\sim$1850年,超过10万的南方黑奴通过“地下铁路”这个庞大的自发网络逃到了北方和加拿大。}时期达到了顶峰。但是尽管美国宪法第十三修正案废除了非自愿的奴役,但是直到今
天我们仍然还能发现它的存在。美国的军事征兵制如果不是短
期奴隶制的话那又是什么呢?而且,对没有活着回家的被征士
兵来说,这种奴隶制是永久的。没有什么比这个问题能更加清
晰地区分古典自由±义者和那些主张把集体放在个人之前的人
了。古典自由主义者相信,人们将会自愿保卫他们认为值得保
卫的国家,没有任何一个集团有权在未经本人同意的情况下,
让其他人放弃一年或两年的生命(甚至可能是整个生命)。当个
人被当作国家的资源来对待的时候,他的个人尊严这个基本的原则就受到了侵犯。有的保守主义者,如参议员约翰$\cdot$麦凯恩
(John McCain)和参议员威廉$\cdot$ 小 巴 克 利 (William  F. Buckley,Jr.) , 以及今天的所谓的“ 自由派” ,如参议员爱德华$\cdot$肯尼迪(Edward M. Kennedy)和福特基金会总裁弗兰克林$\cdot$托马
斯 (Fraklin  Thomas),都支持强制的国家服役制度,这种制度
要求所有的年轻人都为政府工作一年到两年。这种制度是对自
我所有权的基本人权的恶劣侵犯。我们只能希望最高法院能够
判决这个法案违反了第十三修正案,是违宪的。

\switchcolumn*[\subsection{Freedom of Conscience\\倌仰自由}]
It's also easy for most people to see the implications of libertarianism for freedom of conscience, free speech, and personal freedom. The modern ideas of libertarianism began in the struggle
for religious toleration. What can be more inherent, more personal, to an individual than the thoughts in his mind? As religious dissidents developed their defense of toleration, the ideas
of natural rights and a sphere of privacy emerged. Freedom of
speech and freedom of the press are other aspects of the liberty
of conscience. No one has the right to prevent another person
from expressing his thoughts and trying to persuade others of
his opinions. That argument today must extend to radio and
television, cable, the Internet, and other forms of electronic
communications. People who don't want to read books by communists (or libertarians!), or watch gory movies, or download
pornographic pictures, don't have to; but they have no right to
prevent others from making their own choices.
\switchcolumn
大多数人都很容易发现信仰自由、言论自由和个人自由是
古典自由主义的自然推论。现代的古典自由主义理念开始于宗
$\cdot$ 教自由的斗争。对个人来说,还有什么比头脑中的思想更加内
在 、更加个人?随着宗教上的持不同意见者捍卫宗教自由的斗
争的发展,关于自然权利和部分领域的隐私权的观念开始出
现。言论自由和新闻出版自由是信仰自由的其他几个组成部
分。没有人有权阻止别人表达他的思想以及用他的观点说服别
人。这个观点在今天必然扩展到广播电视、有线电视、互联网
以及其他形式的电子通信手段。有的人不想读共产主义书籍
(或者古典自由主义书籍),或者不想看血腥的电影,不想下
载色情图片,没人强迫他们去看;但是他们也无权阻止别人做
出自己的选择。
\switchcolumn*
The ways that governments interfere with freedom of speech
are legion. American governments have constantly tried to ban
or regulate allegedly indecent, obscene, or pornographic literature and movies, despite the clear wording of the First Amendment: ``Congress shall make no law $\ldots$ abridging the freedom
of speech or of the press.'' As a headline in \textit{Wired} magazine put
it, ``What part of 'no law' don't you understand?''
\switchcolumn
政府干预言论的方式有很多。美国政府曾经不断地试图禁
止或者管制有猥亵下流和色情嫌疑的小说和电影,尽管在第一
修正案当中清楚地写着“ 国会不得制定法律……剥夺言论或
新闻出版自由。” 正如《连线》杂志的标题写的:“哪些 “不得”,你不明白呢?”

\switchcolumn*
Libertarians see dozens of violations of free speech in American law. Information about abortion has been banned, most recently in the 1996 law regulating communication over the Internet. The federal government has often used its monopoly
post office to prevent the delivery of morally or politically offensive material. Radio and television broadcasters must get
federal licenses and then comply with various federal regulations on the content of broadcasts. The Bureau of Alcohol, Tobacco, and Firearms forbids the producers of wine and other
alcoholic beverages from noting on their labels that medical
studies indicate that moderate consumption of alcohol reduces
the risk of heart disease and increases longevity---even though
the latest dietary guidelines from the Department of Health
and Human Services note the benefits of moderate alcohol use.
In the 1990s, more than a dozen states have passed laws making it illegal to publicly disparage the quality of perishable
items---that is, fruits and vegetables---without having ``sound
scientific inquiry, facts, or data'' to back you up.
\switchcolumn
古典自由主义者在美国的法律当中可以发现一打侵犯言论
自由的条款。例如,1996年通过的管制互联网通信的法律规
定 ,堕胎的信息不得传播。联邦政府常常使用其垄断的邮局来
阻止道德和政治上的攻击性材料的传递。广播电视必须得到联
邦颁发的执照,而且必须遵守联邦政府的各种内容管制规定。
美国烟酒枪支与爆炸物管理局(ATF) 禁止红酒和其他酒精
饮料厂商在包装上说明医学报告显示适度饮用酒精会减少心脏
病的危险并且延长寿命,尽管美国卫生及公共服务部在最近发
布的饮食指南中已经证明了适度饮酒的好处。1990年代,超
过一打的州通过了法律,规 定 如 果 “没有科学的调查、事实
或数据” 的支持就公幵指责易腐食品如水果、蔬菜的质量是非法的。
\switchcolumn*
Landlords can't advertise that an apartment is ``within walking distance to synagogue''---an effective marketing point for
Orthodox Jews, who aren't supposed to drive on the Sabbath---
because it allegedly implies an intent to discriminate. Colleges
try to ban politically incorrect speech; the University of Connecticut ordered students not to engage in ``inappropriately directed laughter, inconsiderate jokes, and conspicuous exclusion
[of another student] from conversation.'' (To be precise here, I
believe that private colleges have the right to set rules for how
their faculty and students will interact, including speech
codes---which is not to say that such codes would be wise. But
state colleges are bound by the First Amendment.)
\switchcolumn
由于涉嫌包含歧视性内容,地产所有者不能在广告中说一
个 公 寓 “步行距离即可到犹太教堂”,尽管这对于不想开车去
做安息日礼拜的正统犹太教徒来说是有效的卖点。大学也;禁止
政治不正确的言论,如康涅狄格大学命令学生不得“在交谈
当中对其他学生有不适当的直接嘲笑,不得开不合时宜的玩
笑 ,不得明显排斥其他学生”。准确地说,我认为私立大学有
权制定规章,对其全体师生间的交往包括言论规则进行规定,’
虽然这并不表明这些规定是明智的。但是公立大学是受到第一
修正案约束的。
\switchcolumn*
And of course every new technology brings with it new demands for censorship from those who don't understand it, or
who understand all too well that new forms of communication
may shake up established orders. The 1996 telecommunica-
tions reform act, which admirably deregulated much of the industry, nevertheless included a Communications Decency Act
that would prevent adults from seeing material that might be
inappropriate for children. A 1996 law in France requires that
at least 40 percent of the music broadcast by radio stations be
French. It also requires that every second French song come
from an artist who has never had a hit. ``We're forcing listeners to listen to music they don't want to hear,'' says a radio programmer.
\switchcolumn
当每一种新科技出现的时候,都很自然会有一些不懂这些
新科技的人呼吁对其进行审查监管,而那些非常清楚新的通信
方式将动摇现有秩序的人也加入到要求审查监管的呼吁者行
列。尽 管 1996年的通讯改革法案令人欣赏地取消了对这个行
业的许多管制,但是加入了 “通讯庄重法案”,这将阻止成年
人看到儿童不宜的信息。而法国1996年通过的一项法律则规定电台广播的音乐当中至少要有40\%是法国音乐,同时还要
求每次播放的第二首法国歌曲必须是从未上过榜的音乐家写
的。一位电台节目制作人称,“我们是在强迫听众听他们不想
听的音乐”。
\switchcolumn*
Most important, people who want to spend money to support the political candidates of their choice are limited to contributions of \$1,000---sort of like telling the \textit{New York Times}
that it can write an editorial endorsing Bill Clinton but it can
only print 1,000 copies of the paper. That's how the political establishment, while proclaiming its devotion to free speech, hobbles the kind of speech that might actually threaten its power.
\switchcolumn
最重要的是,在美国,人们如果想花钱支持他们选的政治
候选人的话,他会被限制最多只能捐1000美元。这就像告诉
《纽约时报》,它可以写一篇赞扬比尔$\cdot$ 盖茨的评论,但这期
报纸只能印刷1000份。这就是政治机器,一方面宣称支持言
论自由,另一方面阻止那些可能实际威胁到它的权力的言论。
\switchcolumn*
There's a utilitarian argument for freedom of expression, of
course: out of the clash of different opinions, truth will emerge.
As John Milton put it, ``Who ever knew Truth put to the worse
in a free and open encounter?'' But for most libertarians, the
\textit{primary} reason to defend free expression is individual rights.
\switchcolumn
关于言论自由,功利主义者的辩护是:真理是在与不同观
点的碰撞过程中自然显现出来的。这当然是正确的。就像弥尔
顿说的:“谁如果掌握真理,就应与谬误自由公开交锋。”但
是对绝大多数古典自由主义者来说,捍卫言论自由的\textbf{根本理由}
是个人权利。
\switchcolumn*
The right of self-ownership certainly implies the right to decide for ourselves what food, drink, or drugs we will put into
our own bodies; with whom we will make love (assuming our
chosen partner agrees); and what kind of medical treatment we
want (assuming a doctor agrees to provide it). These decisions
are surely as personal and intimate as the choice of what to believe. We may make mistakes (at least in the eyes of others), but
our ownership of our own lives means that others must confine
their interference to advice and moral suasion, not coercion.
And in a free society, such advice should come from private parties, not from government, which is at least potentially coercive
(and in our own society is indeed quite coercive). The role of
government is to protect our rights, not to poke its nose into
our personal lives. Yet a few state governments as recently as
1980 banned alcohol in restaurants, and some twenty states
today outlaw homosexual relations. The federal government
currently prohibits the use of certain lifesaving and pain-relieving drugs that are available in Europe. It threatens us with
prison if we choose to use such drugs as marijuana or cocaine.
Even when it doesn't ban something, the government intrudes
into our personal choices. It hectors us about smoking, nags us
to eat a proper diet---all our daily foods organized into a neat
pyramid chart---and advises us on how to have safe and happy
sex. Libertarians don't mind advice, but we don't think the government should forcibly take our tax money and then use it to
advise everyone in society on how to live.
\switchcolumn
自我所有权自然包含自我决定把什么食品、饮料或者药
品、毒品吃进自己身体的权利;包括支配自己的身体和别人做
爱 的 权 利 (假定性伙伴是同意的);包括选择治疗方案的权利
(假定医生同意提供该治疗方案)。这些决策的做出当然就像
选择信仰什么一样是个人的、私密的事务。我们也许会犯错
误,至少在别人眼里是在犯错误,但是我们对自己生命的所有
权决定了,其他人只能将他们的干预限定在提供建议和道义劝
告上,而不能采取强制手段。而且在自由社会,这种建议应当
来自个人,而不是政府,因为政府多少会产生强制,而我们社
会中的强制已经够多的了。政府的角色是保护我们的权利,而
不是把它的手伸进我们的私人生活。然而,80年代,一些州
的政府禁止在餐馆饮酒,有大约20个州在今天仍将同性恋视
为非法。联邦政府到现在仍然在禁止一些能够拯救生命和减轻痛苦的药物的使用,而这些药物在欧洲是合法的。政府威胁我
们说,如果我们选择使用大麻或者可卡因作为药物,它就会把
我们投入监狱。甚至即使当它不禁止某些东西的时候,政府也
要侵犯我们的个人选择。政府在吸烟问题上吓唬我们,把日常
食物编成一张金字塔型的营养表,不断地絮叨告诫我们要有适
当的饮食,还告诉我们什么才是快乐和安全的性关系。古典自
由主义者当然不介意别人的建议,但是我们认为政府不应该强
制性地收我们的税用来给每个人建议怎样生活。

\switchcolumn*[\subsection{Freedom of Contract\\契约自由}]
The right to enter into contracts is crucially important to libertarianism and to civilization itself. The British scholar Henry
Sumner Maine wrote that the history of civilization was a
movement from a society of status to a society of contract---
that is, from a society in which each person was born into his
place and was defined by his status to one in which the relationships among individuals are determined by free consent
and agreement.
\switchcolumn
签订契约的权利对古典自由主义以及文明社会本身来说都
是至关重要的。英国学者亨利$\cdot$ 梅因说过,文明的历史是从等
级社会向契约社会转变的历史 --- 也就是说,从一个每个人地
位由出生决定、人的发展由社会地位决定的社会,转变到一个
个 人之 间的 关系 由 自 由 的合意(consent) 和契约所确定的
社会。
\switchcolumn*
Libertarianism is neither libertinism nor chaos. People in a
libertarian society may be bound by many rules and restrictions. But only the most general of these is unchosen: the mini-
mal duty to respect everyone else's natural rights. Most of the
rules that bind us in a free society we assume by contract, that
is, by \textit{choice}. We may, for instance, assume an obligation by signing a rental agreement. In that case, the owner of the house assumes the obligation to allow a tenant to live in the house for,
say, a year and to maintain the house in an agreed-upon condition. The tenant assumes the obligation to pay the rent every
month and avoid unnecessary damage to the house. The contract may spell out other obligations assumed by either party---
thirty days' notice to terminate the agreement, a guarantee of
heat and hot water (probably taken for granted in modern
America, but by no means assumed in America fifty years ago
or in many parts of the world today), no loud parties, and so on.
Once the contract is signed, both parties are bound by its
terms. Both can also be said to have acquired new rights by
signing the contract---not natural rights, but special rights.
The owner now has a right to a payment from the tenant every
month, and the tenant has a right to live in the house for an
agreed-upon term. This is not a general right to an income or to
housing, but a particular right created by voluntary agreement.
\switchcolumn
古典自由主义既不是放纵,也不是无秩序。生活在自由社
会的人会被无数的规则和法律所限制。但是这些规则当中只有
一条最普遍的规则不是自由选择的:作为最低程度的义务,尊
重其他每一个人的自然权利。自由社会中绝大多数限制我们的
规则都是经由契约而形成的,也就是说是通过我们的\textbf{选择}而形
成的。例如,我们会通过签订一份房屋租赁合同而承担责任。
房屋的主人通过允许承租人住进他的房子(假设是一年)承
担了按照合同约定的标准维护房屋的责任。而承租人则承担了
每月支付房租和避免对房屋进行不必要损坏的责任。而这份合
同也许还规定了双方都必须承担的责任:中止合同必须提前
30天通知,确保暖气和热水供应(这在现代美国也许被认为是理所当然的,但是在50年前的美国以及今天世界的其他国
家就不是),聚会的时候不得有噪音,等等。一旦签订了合
约,双方都受到了合约条款的限制。可以说双方都通过签订合
约而获得了新的权利 --- 不是自然权利,而是特别权利。房主
现在有了从承租人那里每月得到报酬的权利,而承租人则有了
在合约规定的期限内在房子里居住的权利。这并不是一种普遍
的权利,而是由自愿的合约所创造出的特定的权利。
\switchcolumn*
Other contracts, of course, can apply to virtually anything in
a free society: mortgages, marriage, employment, sales, cooperative agreements, insurance, club or association membership,
and so on. Why do people sign contracts? Largely to remove some of the uncertainty from life and enable us to pursue projects that require some assurance of others' continuing cooperation. You could call your employer early every morning and ask
if he had work for you and what he'd be willing to pay, but both
of you prefer to make a long-term agreement (even if most
American employment contracts allow either party to cancel
the arrangement at will). You could pay your landlady every
morning for a night's lodging, but obviously you both prefer to
eliminate the uncertainty of that arrangement. And for people
who can't make long-term arrangements, there are short-term
options as well, such as hotels for travelers, where the contract
is frequently for one night's lodging.
\switchcolumn
当然,在自由社会中,契约实际上适用于任何事务:抵
押 、婚姻、雇佣、销售、合作合同、保险、俱乐部或者各种协
会会员资格,等等。人们为什么签订契约?很大程度上是为了
消除生活中的不确定性,让我们能够进行那些要求确保其他人
合作的连续性计划。你可以每天早上给你的老板打电话问他是
不是有工作给你做,以及他希望付给你多少钱,但是你们双方
更倾向于一份长期的合同(尽管绝大多数美国的雇佣合同允
许双方随意解除合同)。你可以每天早上把昨晚过夜的房钱付
给房东太太,但是显然你们双方都更倾向于消除其中的不确定
性。那些不能签订长期合约的人,可以选择短期合约,例如给
旅行者居住的宾馆,这里的契约常常只牵涉一晚上的居住。
\switchcolumn*
What is the nature of a contract? Is it just a promise? No, a
contract is a mutual exchange of title to property. For a contract
to be valid, both parties must have legitimate title to the property that they propose to exchange. If they do, then they can
agree to transfer their title to another person in return for title
to some property that he owns. Remember, every object has a
bundle of property rights attached to it; the owner can transfer
the whole bundle of rights or only some of them. When you sell
an apple or a house, you generally transfer the entire bundle of
rights in return for some consideration, probably money, from
the other party. But when you rent a house, you transfer only
the right to live in the house for a specified period of time under
specified rules. When you lend money, you transfer the title to
a certain sum of money now in return for title to a certain sum
at some point in the future. Since it's always better to have
money now than later, the borrower generally agrees to pay
back a larger sum than the one borrowed. Thus, ``interest'' is
the inducement that persuades a lender to give up money now
and get it back only later.
\switchcolumn
那么契约的本质是什么?仅仅是承诺吗?不 ,契约是双方
财产权的交换。为使契约有效,双方都必须对他们要交换的财
产拥有合法产权。如果有合法产权,他们就有权同意将他们财
产的产权转移给另一个人,以交换对方所拥有产权的财产。不
要忘了,与每个物体相关的财产权利都有许多种;所有者可以
将所有权利都转让出去,也可以只转让部分权利。当你卖出一
个苹果或者一座房子的时候,你是把所有权利都转让出去,以
从对方获得回报(也许是钱)。但是当你出租房子的时候,你
只是通过规定条款在一段时间内将房屋的居住权转让出去。当你借出钱的时候,你是在当前将某一笔钱的所有权转让出去,以在未来的某个时间换回某一笔钱的所有权。而由于现在拥有金钱常常比将来拥有金钱要好,借债的人通常都会同意还回来比所借的钱更多的一笔钱。因此,是“利息”说服了出借人现在放弃这笔钱、将来拿回这笔钱。
\switchcolumn*
Failure to live up to a contract is a form of theft. If Smith borrows \$1,000 from Jones, agrees to pay back \$1,100 a year later,
and doesn't do so, he is in effect a thief. He has stolen \$1,100
that belongs to Jones. If Jones sells Smith a car, guaranteeing
that it has a working radio, and it doesn't, then Jones is a thief:
he has taken Smith's money and not delivered what he contracted to deliver.
\switchcolumn
不遵守合约则是一种盗窃。如果甲从乙那里借走1000美
元,同意一年之后还回1100美元,但他没做到,那么他实际
上就是一个贼。他偷走了属于乙的1100美元。如果乙卖给甲
一辆车,保证车上的收音机能用,但实际上不能用,那么乙就
是一个贼:他拿走了甲的钱但是并没有把契约承诺的东西交
给他。
\switchcolumn*
Without contracts, it would be difficult for an economy to move beyond the subsistence level. Contracts enable us to make
long-term plans and to carry on business over a wide geographical area and with people we don't know.
\switchcolumn
没有契约,一个经济体就很难摆脱仅仅维持基本生存的阶
段。契约使得我们能够做长期的计划以及在更大的地理范围内
和我们不认识的人做生意。
\switchcolumn*
For an extended society to work, it is essential that people
meet the obligations they have assumed and that contracts be
enforced. If people are not generally trustworthy, none of us will
want to enter into contracts with people we don't know, and
the market economy will not be able to expand and flourish. If
specific individuals renege on their contracts, people won't
want to do business with them and they may find limited opportunities in the market system. But when people do live up to
their contracts, and especially when most people do, vast and
complex networks of contracts can make possible long chains of
production over time and distance, allowing us to create the
amazing technological achievements and the previously
unimaginable standard of living of modern capitalism.
\switchcolumn
一个大范围的社会要正常运转,人们履行其承担的责任和
契约责任是必需的。如果人们基本上缺乏信用,就没有人会和
不认识的人签订合同,市场经济就不可能发展和繁荣。如果某
个人违背契约,人们就不会和他做生意,他就会发现自己在市
场体系中的机会受到了限制。但是如果人们遵守契约,尤其是
绝大多数人都这样做的时候,大量契约组成的复杂网络就会使
得生产的链条最大限度地延伸到遥远的地方,最大限度地延伸
到未来,使得我们能够创造出神奇的技术进步,达到从前不可
想像的现代资本主义的生活标准。

\switchcolumn*[\section{Do You Have to Believe in Natural Rights to Be a Libertarian?\\只有相信自然权利才算古典自由主义者吗}]

Most intellectuals who call themselves libertarians believe in
the concept of natural individual rights and agree, more or less,
with the above outline. The case for rights presented here reflects the arguments of John Locke, David Hume, Thomas Jefferson, William Lloyd Garrison, and Herbert Spencer;
twentieth century libertarians such as Ayn Rand, Murray Rothbard, Robert Nozick, and Roy Childs; and contemporary
philosophers such as Jan Narveson, Douglas Rasmussen, Douglas Den Uyl, Tibor Machan, and David Kelley.
\switchcolumn
绝大多数自称古典自由主义者的知识分子都相信个人的自然权利的概念,或多或少都同意上面这个标题。我们在这里提
出的关于权利的观点反映了洛克、休谟、杰斐逊、威 廉 $\cdot$洛伊
德 $\cdot$ 加里森以及斯宾塞的观点,以及20世纪的古典自由主义
者如安$\cdot$ 兰德、罗斯巴德、诺齐克、柴尔兹等的观点,以及当
代思想家如纳维逊、拉 斯 姆 森 (Douglas  Rasmussen)、邓友利
(Douglas Den Uyl), 麦 肯 (Tibor  Machan) 以 及 凯 利 (David
Kelly)的观点。
\switchcolumn*
However, some libertarians, especially economists, do not accept the theory of natural individual rights. Jeremy Bentham, a
generally libertarian British philosopher of the early nineteenth
century, derided natural rights as ``nonsense upon stilts.'' Such
modern economists as Ludwig von Mises, Milton Friedman,
and Milton's son David Friedman reject natural rights and
argue for libertarian policy conclusions on the basis of their beneficial consequences.
\switchcolumn
然而,有的古典自由主义者,特别是经济学家不接受个人
的自然权利的理论。边沁这位基本上算是古典自由主义者的
19世纪早期的英国哲学家就嘲笑自然权利是“大言不惭的胡
说八道”。现代经济学家如米瑟斯、米 尔 顿 $\cdot$弗里德曼和他的
儿子大卫$\cdot$ 弗 里 德 曼 (David  Friedman)也反对自然权利。他
们把古典自由主义的政策结论建立在政策将带来的好处之上。
\switchcolumn*
Such a position is often called utilitarianism. The classic for mulation of utilitarianism is to take as a standard for ethics and
political philosophy ``the greatest good for the greatest number.'' That sounds unobjectionable, but it has some problems.
How do we know what is good for millions of people? And
what if the overwhelming majority in some society want something truly reprehensible---to expropriate the Russian kulaks,
genitally mutilate teenage girls, or murder the Jews? Surely a
utilitarian faced with the claim that the greatest number
thought that such a policy would do the greatest good would
fall back on some other principle---probably an innate sense
that certain fundamental rights are self-evident.
\switchcolumn
这样的立场通常被称作功利主义。功 利主 义 的 公 式 “最
大多数人的最大利益” 被当作了道德和政治哲学的标准。这
个公式看上去无可反驳,但是它也有问题。我们如何知道什么
是对数以百万计的人们有利的?如果在有的社会当中绝大多数
人的意愿其实是值得谴责的呢?例如,如果俄罗斯大多数人支
持没收富农的土地,如果大多数人支持对十几岁的女孩实行割
礼 ,或者支持屠杀犹太人呢? 一个功利主义者面对这种“最
大多数人认为这些举措会带来最大利益”的提法必然会求助
于其他原则,或是一种直觉:一些基本权利是不证自明的。

\switchcolumn*[\subsection{Mises's Utilitarianism\\米瑟斯的功利主义}]

The economist Ludwig von Mises was both a firm utilitarian
and an uncompromising advocate of laissez-faire economics.
How did he justify his rejection of all coercive interference into
market processes if not by a doctrine of individual rights? He
said that as an economic scientist he could demonstrate that interventionist policies would bring about results that even the
advocates of those policies would consider undesirable. But, as
Mises's student Murray Rothbard asks, how does Mises know
what the interventionists want? Mises can demonstrate that
price controls will produce shortages, but maybe the advocates
of price controls are socialists who want the controls as a step
toward total government control of the economy, or extreme
environmentalists who deplore excessive consumption and
think fewer goods are a great idea, or egalitarians who figure
that at least if there are shortages the rich won't be able to buy
more than the poor.
\switchcolumn
经济学家米瑟斯既是一位坚定的功利主义者,也是一位不
妥协的自由市场经济的拥护者。那么他如何证明对强制干预市
场过程的反对的正当性呢?如果不是通过个人权利的原则又是通过什么呢?他说,经济学家能够证明,干预主义政策所带来
的结果甚至是那些政策的拥护者们自己都不愿意看到的。但
是 ,就像他的学生罗斯巴德所问的那样,米瑟斯怎么知道干预
主义者要的是什么?米瑟斯能够证明干预主义政策将带来短
缺,但是,也许价格管制的拥护者是社会主义者,把价格管制
当作通向全面政府控制经济的一个步骤;或者他们是极端环保
主义者,痛恨过度消费,认为减少一些商品是件好事;或者他
们是平等主义者,认为短缺至少可以让富人不能比穷人买到更
多的东西。
\switchcolumn*
Mises explains that he ``presupposes that people prefer life to
death, health to sickness, nourishment to starvation, abundance
to poverty.'' If so, the economist can demonstrate that private
property and free markets are the best way to achieve that goal.
He's right, as we'll discuss further in chapter 8, but he's still
making a big assumption. People may well prefer some less
abundance in exchange for more equality, or preserving the
family farm, or simply hurting the rich out of envy. How can a
utilitarian object to taking people's property if a majority have
determined that they don't mind the reduced economic growth that such a policy will generate? Thus, most libertarians con-
clude that liberty is better protected by a system of individual
rights than by simple utilitarianism or economic analysis.
\switchcolumn
米瑟斯解释道,他 的 “预设前提是人们喜欢生而不是死,
喜欢健康而不是疾病,喜欢食物而不是饥饿,喜欢富裕而不是
贫穷”。如果是这样的话,经济学家就能够证明私人产权和自
由市场是达到上述目标的最佳途径。他 是 对 的 (这个问题我
们将在第八章中进一步讨论),但是他仍然是在进行假设。人
们也许恰好喜欢用不那么富裕来换取更大的平等,或者对家庭
农场的保护,或者他们仅仅是因为嫉妒富人而想伤害他们。如
果多数人认为他们不在乎拿走人们的财产会引起经济增长减
速,那么功利主义的目标是不是就变成剥夺人们的财产呢?因
此 ,绝大多数古典自由主义者得出结论,通过个人权利为基础
的理论体系比简单的功利主义或者经济学分析能够更好地捍卫
自由。
\switchcolumn*
This is not to say, Let justice be done though the heavens fall.
Of course consequences matter, and few of us would be libertarians if we thought a strict adherence to individual rights would
lead to a society of conflict and poverty. Because individual
rights are rooted in the nature of man, it is natural that societies
that respect rights are characterized by a greater degree of harmony and abundance. Laissez-faire economic policy, based on a
strict respect for rights, \textit{will} lead to the greatest prosperity for
the greatest number. But the root of our social rules must be
the protection of each individual's right to life, liberty, and
property.
\switchcolumn
这并不是说, “即使天塌下来,也要申行正义”。结果当
然重要,如果我们认为绝对坚持个人权利会导致一个冲突和贫
穷的社会,那么我们当中就很少有人会是古典自由主义者了。
由于个人权利是植根于人的本性,很自然地,一个尊重权利的
社会将是一个富裕和谐的社会。建立在绝对尊重个人权利基础
之上的自由市场经济\textbf{必然}会带来最大多数人的最大富裕。但是我们社会的根基必须是对每个个人的生命、 自由和财产的
保护。

\switchcolumn*[\subsection{Emergencies\\紧急状态}]

In his book \textit{The Machinery of Freedom}, after making a powerful
case for the benefits of libertarian policies, David Friedman
poses several objections to libertarian principles as embodied in
the law of equal freedom and the nonaggression axiom. Many
of them involve emergency or ``lifeboat'' situations. The classic
lifeboat example is, suppose you're in a shipwreck, and there's
only one lifeboat that will hold four people, but there are eight
people trying to cling to it. How do you decide? And---directed
at libertarians or other natural-rights advocates---how does
your rights theory answer this question? David Friedman says,
suppose only by stealing a gun or a piece of scientific equipment
can you stop a madman from shooting a dozen innocent people
or an asteroid from crashing into Baltimore. Would you do it,
and what about property rights?
\switchcolumn
大 卫 $\cdot$ 弗里德曼在他的著作《自由的机制》(\textit{The  Machinery of Freedom}) 中,在强有力地证明了古典自由主义政策的好处之后,对古典自由主义的同等自由原则和互不侵犯原则提
出了几条反对意见。其中很多条都提到紧急状态和“救生艇”
问题。典型的救生艇问题是:假设你在一艘失事的海船上,只
有一艘救生艇,这艘艇只能装四个人,但是有八个人要上去。
你如何决策?你如何用权利理论来回答这个问题?(这个问题直接针对古典自由主义者和其他的自然权利的拥护者)大
卫 $\cdot$ 弗里德曼还举例说,假设一个疯子在向一群无辜的人射
击,而你只有偷一把枪才能制止他;或者一颗陨石将要撞击巴
尔的摩,而你只有偷一台科学仪器才能阻止,你会去做吗?财
产权呢? 
\switchcolumn*
Such questions can be valuable for testing the limits of a theory of rights. In some emergencies, considerations of rights go
out the window. But those questions are not the first that students of ethics should examine, and they don't tell us much
about the ethical systems humans need, because such questions
involve situations that humans are likely never to encounter in
the course of a life. The first task of an ethical system is to enable men and women to live peaceful, productive, cooperative
lives in the normal course of events. We don't live in lifeboats;
we live in a world of scarce resources in which we all seek to im-
prove our lives and the lives of those we love.
\switchcolumn
这些问题对于拷问权利理论的限度是很有价值的。在某些
紧急状态下,对权利的考虑会被置之一旁。但是这些问题不是
道德伦理课的学生首先应该考试的題目,它们并不会告诉我们
关于人类需要的道德体系的更多东西,因为这些问題当中的情
形 ,人们在整个一生几乎都碰不到。道德体系的首要任务是让
男人们和女人们在日常生活环境中过上和平的、富有成果的、
相互合作的生活。我们并没有生活在救生艇当中,我们生活在
一个资源稀缺的世界,在这里我们都在寻求改善我们以及我们
所爱的人的生活。

\switchcolumn*[\subsection{The Limits of Rights\\权利的边界}]
We can imagine other, less outlandish, challenges to the notion
that natural rights are absolute, that is, in the words of the
philosophers Douglas Rasmussen and Douglas Den Uyl, that
they {``}`trump' all other moral considerations in constitutionally
determining what matters of morality will be matters of legality.'' Must a starving man respect the rights of others and not
steal a piece of bread? Must the victims of flood or famine die of
starvation or exposure while others have plenty of food and
shelter?
\switchcolumn
我们可以设想还有其他一些对自然权利是绝对的这一观点
的挑战,虽然可能不像上面的例子那么离奇,用思想家粹斯姆
森和邓友利的话说就是,他们超越了其他所有的道德问题,不
断地让你判断各种道德困境下人是否应遵守法律义务。一个快
饿死的人是不是必须尊重他人的权利,不去偷一片面包?在其
他人有充足的食物和房子的情况下,陷于洪水或饥荒灾害中的
人应该饿死或者露宿街头吗?
\switchcolumn*
Conditions of flood and famine are not normal. When they
occur, according to Rasmussen and Den Uyl in \textit{Liberty and Nature}, we may have to acknowledge that the conditions for social
and political life no longer exist, at least temporarily. Libertarian rules enable social and political life to exist and provide a
context in which people can pursue their own ends. In an emergency situation---two men fighting for one lifeboat, many people made homeless by disaster---social and political life may be
impossible. Each person's moral obligation is to ensure at least
the minimum conditions of his own survival. Rasmussen and
Den Uyl write, ``When social and political life is not possible,
when it is in principle `impossible' for human beings to live
among each other and pursue their well-being, consideration of
individual rights is out of place; they do not apply.''
\switchcolumn
洪水和饥荒并不是常态。按照拉斯姆森和邓友利在《自由和自然》(\textit{Liberty and Nature}) 一书中所说,当这些情况发
生的时候,我们不得不承认社会生活和政治生活运行的前提条
件已经不复存在,至少是暂时不存在了。古典自由主义的规则
确保社会和政治生活的存在,提供了一种情境,在其中人们能
够追求他们自己的目标。在紧急状态下,例如在两个人为了一
艘救生艇而搏斗,或者很多人因为灾难而无家可归的状态下,
社会生活和政治生活也许是不可能的,保证自己生存的起码条
件就成了每个人的道德准则。拉斯姆森和邓友利写道:“当社
会和政治生活不可能存在的时候,当人们共同生活追求他们的
利益基本成为不可能的时候,个人权利的考虑就退场了;它此
时并不适用。”
\switchcolumn*
For a man, \textit{through no fault of his own}, to find himself unable to get work or assistance and on the verge of starvation is
extremely rare in a functioning society. There is almost always
work available at a wage sufficient to sustain life (though minimum-wage laws, taxes, and other government interventions
may reduce the number of jobs). For those who really can't find
work, there are relatives and friends available to help. For those
without friends, there are shelters, missions, and other forms of
charity. But for the sake of the theoretical analysis, let us assume that an individual has failed to find work or assistance and
faces imminent starvation. He is presumably living in a world
where social and political life is still possible; yet we may say that he is in an emergency situation and must take the action
necessary for his own survival, even if that means stealing a loaf
of bread. However, if his victim, on hearing his story, is unpersuaded, it may be appropriate to take the starving man to court
and charge him with theft. A legal order still exists, though a
judge or jury might decide to acquit the man after hearing the
circumstances---without throwing out the general rules of justice and property.
\switchcolumn
一个人发现自己不是\textbf{因为自己的错误}而得不到工作,得不
到救助,生活在饿死的边缘,这在运作有效的社会当中是极端
罕见的。尽管最低工资法、税收和其他的政府干预减少了很多
工作机会,但总会有工作能够提供薪水让人们维持生存。那些
的确找不到工作的人,还有亲戚朋友可以提供救助。那些确实
没有朋友的,有庇护所、教会和其他形式的慈善。但是,为了
进行理论分析方便,我们假设一个人找不到工作、得不到救
助,正在面临饥饿,假设他生活在一个社会和政治生活仍然存
在的地方,那么我们可以说他本人是处在一种紧急状态下,必
须采取必要行动保证他自己的生存,尽管其手段是偷一条面
包。然而,他的受害者听了他的故事之后,并没有被打动,那
么把这个饥饿的人送到法院告他盗窃就是恰当的方式。法律秩
序仍然存在,尽管法官和陪审团在听取了案件全部经过情形之
后也许会判决这个人无罪,这样就没有放弃产权和正义的普遍
规则。
\switchcolumn*
Note that this analysis does not suggest that the starving
man or the flood victim has a right to someone else's assistance
or property; it merely says that rights cannot apply where social
and political life is not possible. But have we discarded rights
entirely, opening the door to redistribution of wealth to all
those who find themselves in dire straits? No. We stress that
these exceptions apply only in emergency situations. A key part
of the situation must be that a person finds himself in a desperate situation through no fault of his own. It cannot be enough that
he simply has less than others, or even that he has too little to
survive. Rasmussen and Den Uyl write, ``Poverty, ignorance,
and illness are not metaphysical emergencies. Wealth and
knowledge are not automatically given, like manna from
heaven. The nature of human life and existence is such that
every person has to use his own reason and intelligence to create
wealth and knowledge.''
\switchcolumn
请注意,这种分析并不表明这个饥饿的人或者洪水中的灾
民有权要求别人的帮助和侵犯别人的财产;这仅仅是说权利的
原则不能适用于社会和政治生活不复存在的地方。但是我们完
全放弃权利的概念了吗?我们赞同把财富重新分配给所有处在
悲惨境地的人吗?没有。我们强调这种例外只是用于紧急状
态。这种状态的关键一点必须是一个人非因本人错误而处在绝
望的境地。如果仅仅是他比别人穷一些,或者甚至他太穷了搏
不下去了,理由都是不充分的。拉斯姆森和邓友利写道:“贫
困、缺乏教育和疾病并不是抽象的紧急状态。财富和知识不是
像来自天堂的吗哪一样可以不劳而获的。人类的生命和存在的
本质是每一个人运用自己的理性和智慧创造财富和知识。”
\switchcolumn*
If a person declines to get necessary education or training, refuses to work at uninteresting or poorly paid jobs, or destroys
his own health, he can't claim to be in desperate straits through
no fault of his own. A woman wrote to Ann Landers to ask
whether she should feel obliged to give a kidney to a sister
who---despite repeated warnings and offers of assistance from
her family---had used alcohol and drugs to excess and ignored
medical advice. Rights theory can't tell us what moral obligations we ought to feel toward family members, whatever responsibility they bear for their own condition; but it can tell us
that such a person is not the moral equivalent of a shipwreck or
famine victim.
\switchcolumn
如果一个人拒绝接受必要的教育或培训,拒绝接受不感兴
趣的工作或者收人低的工作,或者故意损害自己的健康,他就
不能宣称自己非因本人错误而处于悲惨境地。一个女人曾经写
信问爱恩$\cdot$ 兰德丝\footnote{爱恩$\cdot$兰德丝(Ann  Landers, 1918$\sim$2002),美国著名女专栏作家。在专栏中,她向人们提供生活的建议和忠告,在美国拥有极为广泛的读者。},如果她的妹妹过量饮酒和吃药,无视医院的建议,多次拒绝家人的警告和帮助,她是否有责任把肾脏捐献给她妹妹?权利理论并不能告诉我们对家庭成员应该有什么样的道德责任,他们应当为自己的境况负什么样的责任,但是它能告诉我们,他们并不是沉船和大饥荒的受害者。
\switchcolumn*
We arrive at these extreme exceptions to rights protection
only after several conditions have been satisfied: that one or
more persons are in imminent danger of death from exposure, starvation, or illness; that they were put in such straits through
no fault of their own; that there is no time or opportunity for
any other solution; that despite all efforts they have been unable to find either remunerative work or private charity; and
that they recognize that they have incurred an obligation to
someone whose property they take, that is, that as soon as they
are on their feet again they will endeavor to repay whatever
property they took.
\switchcolumn
因此我们得出结论,对权利保护的特殊例外必须满足几个
条件:其一,一个人或者更多的人面临寒冷、饥饿、疾病而处
于即刻的生命危险之中;其二,他们是非因本人错误而陷入窘
境;其三,已经没有时间和机会想其他办法;其四,尽管尽了
最大的努力仍然没有找到有报酬的工作和私人慈善救助;其
五,他们认识到他们对被偷走财产的人欠下债务,也就是说,
只要能够生存下来并站稳脚跟,他们会努力偿还偷走的财产。
\switchcolumn*
The possibility that rights may not apply to conditions where
social and political life is impossible does not undermine the
moral status and social benefits of rights in normal situations.
We live virtually all of our lives in normal situations. Our ethics
should be designed for our survival and flourishing in normal
conditions.
\switchcolumn
而权利不适用于社会政治生活不复存在的地方,这并不会
破坏正常状态下权利的道德地位和对社会的益处。我们实际上
整个一生都生活在正常状态下。我们的伦理学应该是为在正常
状态下生存和发展而设计的。
\switchcolumn*
A final word on utilitarian libertarianism: Libertarians who
reject natural rights as a basis for their views nevertheless arrive at virtually the same policy conclusions as rights-based libertarians. Some even say that government should operate as if
people had natural rights---that is, that government should
protect each individual's person and property from aggression
by others and otherwise leave people free to make their own
decisions. The legal scholar Richard Epstein, after offering in
his book \textit{Simple Rules for a Complex World} an essentially utilitarian case for self-ownership and private property, concludes by
arguing that ``the consequences for human happiness and productivity'' of the principle of self-ownership ``are so powerful
that it should be treated as a moral imperative, even though
the most powerful justification for the rule is empirical, not deductive.''
\switchcolumn
对功利主义的古典自由主义观点最后一句话是:虽然反对
将自然权利作为观点的基础,但实际上他们得出了和以权利为
基础的古典自由主义者同样的政策结论。他们中的一些人甚至
还说政府运作的时候应该假定人们拥有自然权利,也就是说,
政府应当保护每个个人的人身和财产不受其他人的侵犯,除此
之外,让人们自由地做出自己的决策。法学家爱泼斯坦(Richard Epstein)在他的著作《复杂世界的简单规则》(\textit{Simple Rules for a Complex World}) 中为自我所有权和财产权提供
了一个完全功利主义的解释,论证说自我所有权原则“产生
人类幸福和富裕的结果” “是如此强大,它应当被认为是一个
道德律令,尽管这项原则最强有力的辩护是经验主义的,而不
是逻辑演绎的”。

\switchcolumn*[\section{What Rights Aren't\\权利不是什么}]

As the complaints about a proliferation of rights indicate, political debate in modern America is indeed driven by claims of
rights. To some extent this reflects the overwhelming triumph
of (classical) rights-based liberalism in the United States.
Locke, Jefferson, Madison, and the abolitionists laid down as a
fundamental rule of both law and public opinion that the function of government is to protect rights. Thus, any rights
claim effectively trumps any other consideration in public
policy.
\switchcolumn
对权利种类的扩展的抱怨显示,现代美国的政治争论实际
上被权利的主张所左右。从某种意义上说,这反映了以权利为
基础的古典自由主义在美国的压倒性胜利。洛克、杰斐逊、麦
迪逊以及废奴主义者们为法律和公共观点建立了基本的准则,
那就是,政府的职能是保护权利。因此,所有的权利主张在公
共政策当中实际上超越了其他任何考虑。
\switchcolumn*
Unfortunately, academic and popular understanding of natural rights has declined over the years. Too many Americans
now believe that any desirable thing is a right. They fail to distinguish between a right and a value. Some claim a right to a
job, others a right to be protected from the existence of pornography somewhere in town. Some claim a right not to be bothered by cigarette smoke in restaurants, others a right not to be
fired if they are smokers. Gay activists claim a right not to be
discriminated against; their opponents---echoing Mencken's
jibe that Puritanism is ``the haunting fear that someone, somewhere may be happy''---claim a right to know that no one is engaging in homosexual relationships. Thousands of lobbyists
roam the halls of Congress claiming for their clients a right to
welfare, housing, education, Social Security, farm subsidies,
protection from imports, and so on.
\switchcolumn
不幸的是,随着时间的推移,学院和公众对自然权利的理
解下降了。太多的美国人现在认为任何希望得到的东西都是权
利。他们不去区分权利和价值。有的人要求工作的权利,另一
些人则要求保护他们不受到城里面某个地方的色情书籍污染的
权利。有人要求在餐馆不被吸烟者吞云吐雾打扰的权利,另一
些人则要求不因为吸烟而被开除的权利。同性恋积极分子要求
不被歧视的权利,而他们的反对者则要求被告知没有人搞同性
恋 的 权 利 (这不由得让我想起门肯嘲笑清教徒的话“他们无
法控制地担心别人可能在什么地方很快乐”)。数以千计的走
廊 游 说 家 (lobbyists)游荡在国会的大厅里面,为他们的客户
要求权利:福利、住房、教育、社会保险、农业补贴、贸易保
护等。
\switchcolumn*
As courts and legislatures recognize more and more such
``rights,'' rights claims become ever more audacious. A woman
in Boston claims ``my constitutional right to work out with (the
heaviest) weights I can lift,'' even if the heaviest weights at her
gym are in the men's weight room, which is off limits to
women. A man in Annapolis, Maryland, demands that the city
council require pizza and other food-delivery companies to deliver to his neighborhood, which the companies say is too dangerous, and the council is receptive to his request. He says, ``I
want the same rights any other Annapolitan has.'' But no Annapolitan has the right to force anyone else to do business with
him, especially when the company feels it would be putting its
employees in danger. A deaf man is suing the YMCA, which
won't certify him for lifeguard duty because, according to the
YMCA, a lifeguard needs to be able to hear cries of distress. An
unmarried couple in California claim a right to rent an apartment from a woman who says their relationship offends her religious beliefs.
\switchcolumn
随着法院和立法机构承认越来越多的这类所谓“权利”,
对权利的要求变得越来越肆无忌惮。一位波士顿妇女要求
“我举我能举的最重的杠铃的宪法权利”,因为她所在的健身
中心最重的杠铃在男士举重室,不允许女士去举。马里兰州安
那波利斯有位男士则要求市参议会命令比萨店和其他送餐公司
给他的邻居送餐,而这些公司说那个街区太危险了,参议会则接受了他的请求。他说:“我想得到和其他安那波利斯人同样
的权利。”但是没有一个安那波利斯人有权强迫别人与他做生
意,尤其是当公司感到这会让自己的雇员面临危险的时候。一
个耳聋的人状告YMCA\footnote{YMCA, Young Men's Christian Association的英文缩写,中文名“基督教青年会”,是一个遍布全世界的基督教青年组织,发源于英国伦敦,创立于 1844年。},因为YMCA拒绝证明他适合救生员的职务,YMCA说救生员必须能够听到落水者的呼救。 一对未婚的情侣在加利福尼亚要求从一个女人那里租一套公寓的权利,那个女人说他们的关系冒犯了她的宗教信仰。
\switchcolumn*
How do we sort out all these rights claims? There are two
basic approaches. First, we can decide on the basis of political power. Anyone who can persuade a majority of Congress, or a
state legislature, or the Supreme Court, will have a ``right'' to
whatever he desires. In that case, we will have a plethora of conflicting rights claims, and the demands on the public treasury
will be limitless, but we'll have no theory to deal with them;
when conflicts occur, the courts and legislatures will sort them
out on an ad hoc basis. Whoever seems most sympathetic, or
has the most political power, wins.
\switchcolumn
那么如何安排这些权利主张呢?有两个基本途径。第一个
是通过政治权力的基本组织来决定。任何人只要能够说服国会
的大多数,或者州议会,或最高法院,他就会拥有任何想要的
“权利”。在这种情况下,将出现大量的互相冲突的权利主张,
而对公共财政的需求也将会是无限的,但是我们没有任何理论
来安排它们; 当冲突出现的时候,法院和议员们将采取一些特
别的原则来进行安排。谁看上去最值得同情,或者谁有最强的
政治影响力,他就会贏。
\switchcolumn*
The other approach is to go back to first principles, to assess
each rights claim in the light of each individual's right to life,
liberty, and property. Fundamental rights \textit{cannot} conflict. Any
claim of conflicting rights must represent a misinterpretation of
fundamental rights. That's one of the premises, and the virtues,
of rights theory: because rights are universal, they can be enjoyed by every person at the same time in any society. Adherence to first principles may require us, in any given instance, to
reject a rights claim by a sympathetic petitioner or to acknowledge someone else's right to engage in actions that most of us
find offensive. What does it mean to have a right, after all, if it
doesn't include the right to do wrong?
\switchcolumn
另一个途径是回到基本的原则,用每个个人的生命、自由
和财产权来衡量每一种权利主张。基本权利是\textbf{不可能}冲突的。
任何冲突的权利主张一定是因为对基本权利的错误理解。这是
权利理论的前提之一,也是优点之一:权利是普适的,因此能
够被任何人在任何社会所同时享有。坚持基本原则也许要求我
们在任何情况下反对由同情请求者(帮助别人提出请求的人)
所提出的权利请求,或者承认别人从事我们大多数人都感到不
高兴的事情的权利。毕竟,如果拥有权利不包括做错事的权利的话,那又怎么能说拥有权利?
\switchcolumn*
To acknowledge people's ability to take responsibility for
their actions, the very essence of a rights-bearing entity, is to accept each person's right to be ``irresponsible'' in his exercise of
those rights, subject to the minimal condition that he not violate the rights of others. David Hume recognized that justice
frequently required us to make decisions that seem unfortunate
in a given context: ``However single acts of justice may be contrary, either to public or private interest, 'tis certain, that the
whole plan or scheme is highly conducive, or indeed absolutely
requisite, both to the support of society, and the well-being of
every individual.'' Thus, he says, we may sometimes have to
``restore a great fortune to a miser or a seditious bigot,'' but
``every individual person must find himself a gainer'' from the
peace, order, and prosperity that a system of property rights establishes in society.
\switchcolumn
承认人们有对他们的行为承担责任的能力,这是权利最基
本的属性,意味着接受每个人在他实践这些权利时有“不负
责任” 的权利,对此的最低条件是他没有损害到别人的权利。
休谟在文章中承认,正义常常要求我们做出一些看上去不幸的
决定,“尽管在个别的具体案例当中,无论是对公众还是个人
利益来说,结果都是不正义的,但 是 ‘ 整 个 (对财产权的保
护)计划或体系对社会和个人利益来说都是有益的,而且事
实上是绝对必要的。” 因此他说,我们也许有时候不得不“把
一大笔财富还给守财奴或作乱的顽固派” ,但 是 “每个个人必
然会发现他自己是赢家” ,获得了和平、秩序和繁荣,而这正
是财产权制度给这个社会所带来的。
\switchcolumn*
If we accept the libertarian view of individual rights, we have
a standard by which to sort out all these conflicting rights
claims. We can see that a person has a right to acquire property, either by homesteading unowned property or---in almost all
cases in modern society---by persuading someone who owns
property to give or sell it to him. The new property owner then
has a right to use it as he chooses. If he wants to rent an apartment to a black person, or to a grandmother with two grandchildren, then it is a violation of property rights for zoning laws
to forbid that. If a Christian landlady refuses to rent a room to
unmarried couples, it would be unjust to use the power of government to force her to do so. (Of course, other people have
every right to consider her prejudiced and to express their opinions, on their own property or in newspapers that choose to
publish their criticisms.)
\switchcolumn
如果我们接受古典自由主义的个人权利理念,我们就有了
将所有互相冲突的权利主张进行安排的标准。一个人有权获得
财产,无论是通过对无主财产的先占原则,还是通过说服拥有
财产的人把财产赠给他或者卖给他(这在现代社会几乎是唯
一的方式)。新的财产所有人于是就有权按照自己的选择来使
用这份财产。假如他想将房屋出租给一个黑人,或者出租给一
个带着两个孙子的奶奶,如果地方法律禁止,则是侵犯了他的
财产权。假如一位信基督教的女房东拒绝把房屋租给未婚情
侣,那么运用政府权力强迫她出租则是不公正的。当然,其他
人有权认为她歧视,也有权在自己的产业上表达自己的观点,
也有权在报纸上刊登自己的批评意见,如果报纸愿意登的话。
\switchcolumn*
People have a right to take up any line of business for which
they can find a willing employer or customers---thus the classical liberal rallying cry of ``\textit{la carriere ouverte aux talents}'' (``opportunity to the talented'') not protected by guilds and
monopolies---but they don't have a right to force anyone to hire
them or do business with them. Farmers have a right to plant
crops on their own property and sell them, but they don't have
``a right to a living wage.'' People have a right not to read information about midwifery; they have a right not to sell it in their
own bookstores or allow it to be transmitted over their own online service; but they don't have a right to prevent other people
from entering into various contracts to produce, sell, and buy
such information. Here again, we see, the right to a free press
comes back to freedom of property and contract.
\switchcolumn
人们有权加人到商业链条的任何一个环节,在这里他可以
找到愿意雇用他的雇主,或者找到顾客 --- 因此传统的自由主
义者的战斗口号是为那些没有受到行会和垄断者保护的“有才能的人提供机会”(\textit{la carrier ouverte dux talents})--- 但是他们无权强迫任何人雇用他们或者和他们做生意。农民有权在
他们自己的土地上种植粮食并且出售它们,但是他们没有权利
要 求 得 到 “保证生活的收入”。人们有权不看关于助产术的信
息,他们也有权不在自己的书店出售,或者不允许通过自己的
网站交换这类信息,但是他们没有权利阻止别人通过各种契约
生产、出售和购买这样的信息。这里,我们再次看到,新闻出
版自由归根结蒂是财产和契约自由。
\switchcolumn*
One of the benefits of the system of private property---or
several property, as Hayek and others have called it---is pluralism and the decentralization of decision making. There are 6
million businesses in the United States; rather than having one
set of rules for all of them, a system of pluralism and property
rights means that each business can make its own decisions.
Some employers will offer higher wages and less pleasant working conditions; others will offer a different package, and potential employees can choose. Some employers will no doubt be
prejudiced against blacks, or Jews, or women---or even men, as
a 1995 lawsuit against the Jenny Craig Company complained
---and will pay the costs associated with that, and others will profit by hiring the best workers regardless of race, gender, religion, sexual orientation, or any other non-work-related characteristic. There are 400,000 restaurants in the United States;
why should they all have the same rules about smoking, as
more and more governments are mandating? Why not let different restaurants experiment with different ways to attract
customers? The board of directors of the Cato Institute has
banned smoking in our building. That is a real imposition on
one of my colleagues, who slips off to the garage for a desperate
puff on the vile weed every hour or so. His attitude is, ``I'd like
to have an interesting job, with congenial colleagues, at a great
salary, in an office that allowed smoking. But a really interesting job, with congenial colleagues, at an adequate salary, in a
nonsmoking office, is better than the other alternatives available to me.''
\switchcolumn
私 人 产 权 (或者哈耶克所称的分立产权)制度的一个好
处是多元主义和非中央决策。美国有600万家公司。尽管有一
系列的法律对所有的公司进行管制,但是一个多元主义和财产
权的体制意味着每家公司都能够做出自己的决策。有的雇主愿
意提供高工资,但工作条件不那么令人愉快;其他一些雇主愿
意提供不同组合的工资福利。找工作的人能够从中进行选择。
有的雇主无疑歧视黑人、犹太人、女人或者甚至歧视男人
(1995年起诉杰妮$\cdot$克莱格公司的原告就指责该公司歧视男
人),但他们会为他们的歧视付出相应成本,其他公司则会因
为不考虑种族、性别、宗教、性取向或者其他任何与工作无关
的特征而雇了最好的员工从而从中获益。美国有40万家餐馆,
为什么他们必须要对吸烟有同样的规定呢?而越来越多的政府
正在这么规定。为什么不让不同餐馆尝试用不同的方法来吸引
顾客呢?我们 Cato研究所的理事会规定,办公大楼里禁止吸
烟。对我的某个同事来说,这的确很难受。他几乎每个小时都
要溜到院子里拼命地吞云吐雾,抽的还是劣质的香烟。他的想
法是;“我希望有一份有趣的工作,有一群意气相投的同事,
很不错的薪水,办公室允许抽烟。但是一份真正有趣的工作,
一群意气相投的同事,过得去的薪水,办公室禁烟,比其他可
供我选择的工作要好。”
\switchcolumn*
The \textit{Wall Street Journal }reported recently that ``employers will
increasingly be asked to juggle the demands of workers who
want to express their faith during the workday and those who
don't want to hear it.'' Some employees are demanding the
``right'' to practice their religion in the workplace---with on-the-job Bible study and prayer sessions, wearing large antiabortion buttons with a color photo of a fetus, and the like---while
other employees are suing to demand a ``right'' not to hear
about religion in the workplace. Government, either through
Congress or the courts, could make a rule on how employers
and employees must deal with religion and other controversial
ideas in the workplace. But if we relied on the system of property rights and pluralism, we would let millions of businesses
make their own decisions, each owner weighing his own religious convictions, the concerns of his employees, and whatever
other factors seem important to him. Potential employees could
negotiate with employers, or make their own decisions about
which workplace environment they preferred, while also taking
into account such other considerations as salary, fringe benefits,
convenience to home, hours of work, how interesting the work
is, and so on. Life is full of trade-offs; better to let those tradeoffs be made on a localized and decentralized basis than by a
central authority.
\switchcolumn
《华尔街日报》最近报道说:“雇主将越来越多地像玩扔
球接球的杂技一样,在要求在工作日表达信仰的雇员和那些不
想听到这些布道的人当中玩平衡。”有的雇员要求在工作场地
实践他们信仰的“权利” ,比如要求工作时间进行《圣经》学
习和祷告会,要求上班时间戴着印着婴儿彩色照片的反堕胎标
志的大徽章,诸如此类,而其他一些雇员则起诉要求不在工作
场所听到与宗教有关的东西的“权利”。而政府,不论是通过
国会还是法院,会制定一部法律,规定雇主和雇员在工作场所
必须如何处理宗教和其他有争议的理念。但是如果我们依靠财
产权和多元主义的体制,我们就会让数百万的公司自己做决
定 ,每位公司所有者将会在他自己的宗教信仰与雇员所关心的
事情以及其他对他来说重要的因素之间进行衡量。应聘者可以
和雇主进行谈判,或者对他更喜欢什么样的工作环境作出决
策,同时再把其他的因素如工资、福利、交通的方便程度、工
作时间、工作是否有意思等考虑进去。生活充满了各种各样的
权衡,把这种权衡建立在本地化和非中央决策的基础之上,比
通过一个中央政府进行决策更好一些。

\switchcolumn*[\section{How the Government Complicates Rights\\政府是如何把权利复杂化的}]

I've argued that conflicts over rights can be settled by relying
on a consistent definition of natural rights, especially private
property, on which all our rights depend. Many of the most
contentious conflicts over rights in our society occur when we
transfer decisions from the private sector to the government,
where there is no private property. Should prayers be said in
school? Should residents of an apartment complex be allowed to
own guns? Should theaters present sexually explicit productions? None of these questions would be political if the schools,
apartments, and theaters were private. The proper stance
would be to let the owners make their own decisions, and then
potential customers could decide whether they wanted to patronize the establishments.
\switchcolumn
前面论证了权利之间的冲突能够依靠对自然权利的前后一
致的界定而进行合理安排,尤其是财产权,是我们所有权利的
基础。在我们的社会中,绝大多数有争议的权利冲突,是在我
们把决策权利从私人转让给政府(在这里是没有私人产权的)
之后出现的。能在学校里做祷告吗?可以允许公寓大楼的居民
持枪吗?剧院可以进行明显与性有关的演出吗?如果学校、公
寓和剧院是属于私人的话,那么这些问题没有一个是政治性的。正确的立场是让产权所有人自己作决定,而潜在的顾客会
决定是否光顾这些地方。
\switchcolumn*
But make these institutions public, and suddenly there is no
owner with a clear property right. Some political body decides,
and the whole society may get drawn into the argument. Some
parents don't want the government forcing their children to listen to a prayer; but if school prayer is banned in public schools,
then other parents feel that they are being denied the right to
raise their children as they see fit. If Congress tells the National
Endowment for the Arts not to fund allegedly obscene art,
artists may feel that their liberty is restricted; but what about
the liberty of the taxpayers who elected those members of Congress to spend their tax dollars wisely? Should the government
be able to tell a doctor at a government-funded pregnancy
clinic not to recommend abortion?
\switchcolumn
但是一旦把这些机构变成公立,突然间这些地方就没有了
拥有清晰产权的所有者。由政治实体来决策,整个社会就会被
拉进来进行争论。有的父母不希望政府强迫他们的孩子听祷
告,但是如果公立学校禁止祷告的话.,其他父母就会感到他们
按照自己认为适合的方式养育孩子的权利被侵犯了。如果国会
告诉国家艺术基金会不要资助被认为是淫秽作品的艺术,艺术
家们也许会感到他们的自由被限制了;但是那些纳税人的自由
怎么办,他们选国会议员可是为了让他们明智地支配所交的税
款的。政府能告诉一家政府资助的妇产医院的医生不要做流产
手术吗?
\switchcolumn*
Duke University law professor Walter Dellinger, a top legal
official in the Clinton administration, warned that such rules
are ``especially alarming in light of the growing role of government as subsidizes landlord, employer and patron of the arts.''
He's right. Such rules extend the government's reach into more
and more aspects of our lives. But as long as government is the
biggest landlord and employer, we can't expect citizens and
their representatives to be indifferent to how their money is
spent.
\switchcolumn
杜克大学法学教授、克林顿政府高级法律官员沃尔特$\cdot$德
林 格 (Walter Dellinger)警告说: “由于政府扮演补贴者、地
主、雇主和艺术赞助人的角色,这样的规定才尤其值得警
惕。” 他说得很对。这些规定使政府的触角越来越多地深入到
我们生活的各个领域。但是由于政府是最大的地主和雇主,我
们不能指望公民及其代表会对他们的钱被如何花掉无动于衷。
\switchcolumn*
Government money always comes with strings attached.
And government must make rules for the property it controls, rules that will almost certainly offend some citizen-taxpayers.
That's why it would be best to privatize as much property as
possible, to depoliticize decision making about the use of
property.
\switchcolumn
政府的资助常常带着一连串的附加条件。政府也会为它控
制的产业制定法律,而这些法律将几乎肯定会侵犯到一些公民
和纳税人。这就是为什么财产的私有化范围应该尽可能的广,
因为这样就会使如何使用财产的决策尽可能非政治化。

\switchcolumn*
We should recognize and protect natural rights because justice demands it, and also because a system of individual rights
and widely dispersed property leads to a free, tolerant, and civil
society.
\switchcolumn
我们应当认和保护自然权利,因为这是正义的要求,同
时也是因为,一个个人权利和广泛分散的产权体制将会带来一
个自由、宽容和文明的社会。

\end{paracol}