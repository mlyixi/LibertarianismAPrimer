\chapter{THE DIGNITY OF THE INDIVIDUAL\\个人尊严}
\begin{paracol}{2}
\hbadness5000

Not long ago, on a Saturday morning in a small city in
France, I walked up to an automatic teller machine
set into the massive stone wall of a bank that was closed for the
weekend. I stuck a piece of plastic into the machine, punched
some buttons, waited a few seconds, and collected about \$200,
all without contact with any human being, much less anyone
who knew me. I then took a taxi to the airport, where I approached a clerk at a rental-car counter, showed him a different
piece of plastic, signed a form, and walked out with the keys to
a \$20,000 automobile, which I promised to return to someone
else at a different location in a few days.
\switchcolumn
不久前的一个星期六的早晨,在法国的一个小城市,我步
行到一个自动柜员机前取钱。这台柜员机装在银行巨大结实的
石墙上,而银行周末是不开门的。我把一张塑料卡片塞进机
器,按了一些按钮,等了几秒钟,取出了大约200美元。整个
过程没有和任何人接触,也没有任何人认识我。然后我坐了一
辆出租车去机场,走到一个租车公司的店员跟前,填了一张表
格 ,然后就拿着一把价值两万美元的汽车的钥匙走了出去。我
答应几天之后在另一个地方把车还给另外一个人。
\switchcolumn*
These transactions are so routine that the reader wonders
why I bother to mention them. But stop for a moment and reflect on the wonders of the modern world: A man I had never
seen before, who would never see me again, with whom I could
barely communicate, trusted me with a car. A bank set up an
automatic system that would give me cash on request thousands of miles from my home. A generation ago such things
weren't possible; a couple of generations ago they would have
been unimaginable; today they are the commonplace infrastructure of our economy. How did such a worldwide network
of trust come about? We'll discuss the strictly economic aspects
of this system in a later chapter. In this and the next few chapters, I want to explore how we get from the lone individual to
the complex network of associations and connections that make
up the modern world.
\switchcolumn
读者可能会感到疑惑,这些交易都是很普通的,为什么要
不厌其烦地提出来呢?但是请先暂停一下,静想一下稞代社会
的奇迹:一个人我从没见过,我也不会再见到他,我也几乎不
用跟他交流,他就相信我,把车交给我。银行安装了自动取款
系统,愿意接受我从几千英里之外的要求把钱给我。仅仅一代
人之前,这样的事情还是不可能的;而在几代人之前,这甚至
连想都想不到的;而到了今天,这巳经成了我们经济中平凡得
不值一提的基础设施。那么这样一个全球范围内的信用网络是
如何出现的呢?我们将在后面的章节当中讨论这个精密的经济
体制的各个方面。在本章和随后的章节中,我想探讨一下我们
是如何从孤独的个人变成各种组织和合作方式所组成的复杂网
络 ,从而组成现代社会的。

\switchcolumn*[\section{Individualism\\个人主义}]

For libertarians, the basic unit of social analysis is the individual. It's hard to imagine how it could be anything else. Individuals are, in all cases, the source and foundation of creativity,
activity, and society. Only individuals can think, love, pursue
projects, act. Groups don't have plans or intentions. Only individuals are capable of choice, in the sense of anticipating the
outcomes of alternative courses of action and weighing the consequences. Individuals, of course, often create and deliberate in
groups, but it is the individual mind that ultimately makes
choices. Most important, only individuals can take responsibility for their actions. As Thomas Aquinas wrote in \textit{On the Unity of the Intellect}, the concept of a group mind or will would mean
that an individual would ``not be the master of his act, nor will
any act of his be praiseworthy or blameworthy.'' Every individual is responsible for his actions; that's what gives him rights
and obligates him to respect the rights of others.
\switchcolumn
按照古典自由主义的观点,社会分析的基础单元是个
人 ,很难想像会是别的东西。在所有情形下,个人是创造
力 、行动和社会的基础和源泉。只有个人才能思考、爱 、进
行计划和行动。集体不会有计划和意图。只有个人才能进行
选择,才能预测各种行动计划的结果以及对这些结果进行衡
量。当然,个人常常建立集体并且在集体当中思考,但是最
终作出决策的只能是个人的头脑。最重要的是,只有个人才
能对他们的行为负责。就 像阿奎那在《论肉体与灵魂的统一》一书中说:集体思考和意愿的概念意味着个人“不是他
行为的主人,他的任何行为都不应受到赞扬或谴责”。每个
个人都对自己的行为负责,这是他权利的来源,也是他尊重
别人权利的义务的来源。
\switchcolumn*
But what about society? Doesn't society have rights? Isn't
society responsible for lots of problems? Society is vitally important to individuals, as we'll discuss in the next few chapters. It is
to achieve the benefits of interaction with others, as Locke and
Hume explained, that individuals enter into society and establish a system of rights. But at the conceptual level, we must understand that society is composed of individuals. It has no
independent existence. If ten people form a society, there are
still ten people, not eleven. It's also hard to define the boundaries or a society; where does one ``society'' end and another
begin? By contrast, it's easy to see where one individual ends
and another begins, an important advantage for social analysis
and for allocating rights and duties.
\switchcolumn
但是社会呢?社会有权利吗?难道社会不对大量的问题
负责吗?社会对个人来说是极端重要的,这一点我们将在后
面的几章中进行讨论。就像洛克和休谟曾经说过的那样,正
是为了在相互交往过程中获益,个人才组成社会,建立一套
权利的体系。但是在观念的层面,我们必须明白,社会是由
个人所组成。社会不是独立于个人而存在的。如果十个人建
立了一个社会,就仍然只是十个人,而不是十一个人。一个
“社会” 的边界也是很难界定的;请 问 一 个 “社会” 的边界
在哪里结束,另一个社会的边界在哪里开始?但与此相对,
我们很容易知道一个人的边界从哪里结束,另一个人的边界
从哪里开始,这对于社会分析和权利责任的界定有很大的
好处。
\switchcolumn*
The libertarian writer Frank Chodorov wrote in \textit{The Rise and Fall of Society} that ``Society Are People'':
\switchcolumn
古典自由主义作家肖多罗夫在《社会的兴衰》(\textit{The Rise and Fall of Society})一书中说“社会就是人”:
\switchcolumn*
\begin{quote}
Society is a collective concept and nothing else; it is a convenience for designating a number of people$\ldots$ The concept of Society as a metaphysical concept falls flat when we observe that
Society disappears when the component parts disperse; as in the
case of a ``ghost town'' or of a civilization we learn about by the
artifacts they left behind. When the individuals disappear so does
the whole. The whole has no separate existence.
\end{quote}
\switchcolumn
\begin{quote}
社会是一个集合概念,除此之外什么都不是;这是为
了给一定数量的人命名方便而造出来的概念……社会的概
念是一个抽象的概念,当我们发现它的组成部分 --- 人消
失的时候,例如我们所知道的被人为遗弃的 “ 鬼城”和
古代文明,于是社会就消失了,社会这个概念就变得毫无
意义了。当个体消失的时候,整体也随之消失。整体并不
会孤立存在。
\end{quote}
\switchcolumn*
We cannot escape responsibility for our actions by blaming society. Others cannot impose obligations on us by appealing to
the alleged rights of society, or of the community. In a free society we have our natural rights and our general obligation to respect the rights of other individuals. Our other obligations are
those we choose to assume by contract.
\switchcolumn
我们不能通过谴责社会而逃脱对自己行为所负的责任。其
他人不能通过要求所谓的社会权利或者社区权利而把责任强加
给我们。在自由社会,我们拥有我们的自然权利和尊重其他个
人的自然权利的普遍责任。我们其他的责任都是来自我们的选
择 ,通过与别人签订的契约而承担。
\switchcolumn*
Yet none of this is to defend the sort of ``atomistic individualism'' that philosophers and professors like to deride. We \textit{do} live together and work in groups. How one could be an atomistic
individual in our complex modern society is not clear: would
that mean eating only what you grow, wearing what you make,
living in a house you build for yourself, restricting yourself to
natural medicines you extract from plants? Some critics of capitalism or advocates of ``back to nature'' might endorse such a
plan, but few libertarians would want to move to a desert island
and renounce the benefits of what Adam Smith called the Great
Society, the complex and productive society made possible by
social interaction.
\switchcolumn
然而这绝对不是为所谓的“原子化的个人主义”进行辩
护,虽然很多理论家和教授喜欢对这一点进行嘲笑。我们\textbf{当然}
是生活在一起,在各种集体当中工作。在我们这个复杂的现代
社会当中,一个人怎么可能是原子化的个人?是不是说一个人
只能吃他自己种的东西,穿他自己做的衣服,住他自己建的房
子 ,局限于只吃自己从植物当中榨取的天然药物?批评资本主
义的人或者“回归自然” 的拥护者也许会赞成这么做,但是
很少会有古典自由主义者会希望搬到一个荒岛上,放弃亚当*
斯密所称的“伟大社会” (是指通过社会互动而建立起来的复
杂而生产力水平高的社会)的种种好处。
\switchcolumn*
Individuals benefit greatly from their interactions with other
individuals, a point usually summed up by traditional philosophers as ``cooperation'' and by modern texts in sociology and
management as ``synergy.'' Life would indeed be nasty, brutish,
and short if it were solitary.
\switchcolumn
个人通过与其他个人之间的互动而受益匪浅,这一点常常
被传统的理论家概括为“合作”,而被现代社会学和管理学称为 “协作”。生命如果孤独,那么它的确是艰难、痛苦和短
暂的。

\switchcolumn*[\section{The Dignity of the Individual\\个人尊严}]

Indeed, the dignity of the individual under libertarianism is a
dignity that \textit{enhances} social well-being. Libertarianism is good
not just for individuals but for societies. The positive basis of
libertarian social analysis is methodological individualism, the
recognition that only individuals act. The ethical or normative
basis of libertarianism is respect for the dignity and worth of
every (other) individual. This is expressed in the philosopher
Immanuel Kant's dictum that each person is to be treated not
merely as a means but as an end in himself.
\switchcolumn
实际上,古典自由主义之下的个人尊严是一种能够\textbf{增强}社
会和谐的尊严。古典自由主义不仅对个人有好处,而且对社会
有好处。古典自由主义社会分析的实证基础是方法论上的个人
主义,也就是说承认只有个人才能采取行动。古典自由主义的
道德或普通法的基础是尊重每一个个人(以及其他人)的尊
严和价值。这一点在康德的一句名言当中表达了出来:人应被
看作自身的目的,而不是手段。
\switchcolumn*
Of course, as late as Jefferson's time and beyond, the concept
of the individual with full rights did not include all people. Astute observers noted that problem at the time and began to
apply the ringing phrases of Locke's \textit{Second Treatise of Government} and the Declaration of Independence more fully. The equality
and individualism that underlay the emergence of capitalism
naturally led people to start thinking about the rights of
women and of slaves, especially African American slaves in the
United States. It's no accident that feminism and abolitionism
emerged out of the ferment of the Industrial Revolution and the
American and French revolutions. Just as a better understanding of natural rights was developed during the American struggle against specific injustices suffered by the colonies, the
feminist and abolitionist Angelina Grimke noted in an 1837
letter to Catherine E. Beecher, ``I have found the Anti-Slavery
cause to be the high school of morals in our land---the school in
which human rights are more fully investigated, and better understood and taught, than in any other.''
\switchcolumn
当然,直到杰斐逊的时代之前,个人拥有全部权利的概念
并不包括所有人。那时,敏锐的观察者注意到了这个问题,开始将《政府论》和《独立宣言》 的响亮段落应用于更广泛的
范围。导致资本主义出现的平等和个人主义很自然地让人们开
始思考女性和奴隶的权利,尤其是美国黑人的权利。因此,在
工业革命、美国革命、法国革命的孕育下出现男女平权运动和
废奴运动绝非偶然。正是由于在北美殖民地为反对殖民者带来
的各种不公正而斗争的时期,自然权利得到了更好的理解和发
展,男女平权主义者和废奴主义者安吉利娜$\cdot$ 格 林 姆 柯 (An­gelina Grimke) 在 1837年 写 给 凯 瑟 琳 $\cdot$ 比 彻 (Catherine  E. Beecher)的一封信中说:“我发现反奴隶制运动在我们国家是一所教授道德常识的高中学校 --- 在这所学校里人权概念得到了比其他任何国家更好的研究、理解和传播。”

\switchcolumn*[\subsection{Feminism\\男女平权主义}]
The liberal writer Mary Wollstonecraft (wife of William Godwin and mother of Mary Wollstonecraft Shelley, the author of \textit{Frankenstein}) responded to Edmund Burke's \textit{Reflections on the
Revolution in France} by writing \textit{A Vindication of the Rights of Men},
in which she argued that ``the birthright of man ... is such a
degree of liberty, civil and religious, as is compatible with the
liberty of every other individual with whom he is united in a social compact.'' Just two years later she published \textit{A Vindication of the Rights of Woman}, which asked, ``Consider $\ldots$ whether, when
men contend for their freedom ... it be not inconsistent and unjust to subjugate women?''
\switchcolumn
\footnote{Feminism,通常的翻译是女权主义,但是译者认为在这里翻译成男女平权主义更恰当一些。因为本书提到这个词是从男人、女人在法律面前一律平等这个意义上说的。女权主义在19世纪中后期兴起之时,主要目标是为妇女争得政治上、经济上、法律上的合法权利。而从I960年代开始,现代的女权主义已经比这个走得更远,除了要求法律面前一律平等之外,还要求消灭性别上的差别,不管男女之间的生理和心理差别,强行要求男女在生活中扮演同等的角色,包括在工	作机会上要求对女性进行照顾,这就违背了男女平权的本意。}自由派作家玛丽$\cdot$ 沃尔斯通克莱夫特(威廉$\cdot$戈德温的
妻子,《弗兰肯斯坦》作者玛丽$\cdot$ 沃尔斯通克莱夫特$\cdot$雪莱的
母亲)在评论埃德蒙$\cdot$ 伯克\footnote{埃 德 蒙$\cdot$伯克(Edmund Buikc, 1729$\sim$1797),英国18世纪思想家、政治家,辉格党人,被誉为西方保守主义的奠基人。主要著作有 《法国革命论》《自由与传统》等。他的思想以“对传统的尊崇” 和 “对自由的保守”而著称于世。伯克强调自由与财产的关系:“财产是自由精神的载体,也是自由的保障”。}的《法国革命论》时,写了《为人的权利辩护》(\textit{A Vindication of the Rights of Men}) 一书,在书中她说“人的天赋权利……是一定程度的世俗或宗教自由。
这种自由以尊重与之通过社会契约相联系的其他人的自由为前提。” 两年之后,她出版了《为女人权利辩护》(\textit{A Vindication of the Rights of Woman}),其中写道,“想想吧……当男人为他们自己的自由而斗争的时候……却把女人当作附属,这在逻辑上是一致和公正的吗?”
\switchcolumn*
Women involved in the abolitionist movement also took up
the feminist banner, grounding their arguments in both cases in 
the idea of self-ownership, the fundamental right of property in
one's own person. Angelina Grimke based her work for abolition and women's rights explicitly on a Lockean libertarian
foundation: ``Human beings have rights, because they are moral
beings: the rights of all men grow out of their moral nature;
and as all men have the same moral nature, they have essentially the same rights. ... If rights are founded in the nature of
our moral being, then the \textit{mere circumstance of sex} does not give to
man higher rights and responsibilities, than to women.'' Her
sister, Sarah Grimke, also a campaigner for the rights of blacks
and women, criticized the Anglo-American legal principle that
a wife was not responsible for a crime committed at the direction or even in the presence of her husband in a letter to the
Boston Female Anti-Slavery Society: ``It would be difficult to
frame a law better calculated to destroy the responsibility of
woman as a moral being, or a free agent.'' In this argument she
emphasized the fundamental individualist point that every individual must, and only an individual can, take responsibility
for his or her actions.
\switchcolumn
那些加人到废奴运动当中的女人同时也举起了男女平权主
义的旗帜,将她们的观点建立在了自我所有权这个主张对自己
本身拥有基本所有权的理念之上。安吉利娜$\cdot$格林姆柯的废奴
主义和妇女权利的主张,基础很显然是洛克式的古典自由主
义. “人类拥有权利,因为他们是道德存在:所有人的权利来自其道德本性;所有人都有同样的道德本性,他们必然拥有同样的权利……如果权利是在我们道德存在的本性中被发现的,
那么仅仅由于性别原 因是不能给男人以比女人更髙的权利和责
任。” 她的姐姐萨拉$\cdot$ 格林姆柯(Sarah  Grimke) 也是一位争
取黑人和妇女权利的活动家。她对盎格鲁-美利坚的传统法律
原 则 “妻子对自己在丈夫命令下或者丈夫在场的情况下所犯
下的罪行都不必负责” 提出了批评。她在一封写给波士顿妇
女反奴隶制协会的信中说:“对一项摧毁女人作为道德存在和
道德主体的人的责任的法律,我很难给它以更高的评价。”在
信中,她强调了基本的个人主义观点:每 个 个 人 必 须 (而且
只有个人才能)对他或她的行为负责。”
\switchcolumn*
A libertarian must necessarily be a feminist, in the sense of
being an advocate of equality under the law for all men and
women, though unfortunately many contemporary feminists
are far from being libertarians. Libertarianism is a political philosophy, not a complete guide to life. A libertarian man and
woman might decide to enter into a traditional working-husband/nonworking-wife marriage, but that would be their voluntary agreement. The only thing libertarianism tells us is that
they are political equals with full rights to choose the living
arrangement they prefer. In their 1986 book \textit{Gender Justice},
David L. Kirp, Mark G. Yudof, and Marlene Strong Franks endorsed this libertarian concept of feminism: ``It is neither equality as sameness nor equality as differentness that adequately
comprehends the issue, but instead the very different concept of
equal liberty under the law, rooted in the idea of individual autonomy.''
\switchcolumn
一个古典自由主义者必然是男女平权主义者,当然这是
从支持所有的男人和女人在法律面前一律平等这个意义上来
说的,尽管很不幸,许多当代的女权主义者\footnote{女权主义者(feminist),在文中和男女平权主义者是同一个单词,但因为意义已经发生变化,这里翻译成女权主义者。}已经远远不是古典自由主义者了。古典自由主义是一种政治哲学,而不是
一种生活指南。一对古典自由主义者的男人和女人可能会决
定结成传统的“丈夫工作、妻子不工作” 模式的婚姻,但这
是他们自愿签订的契约。古典自由主义能够告诉我们的仅仅
是 :他们在政治上是平等的,都完全拥有选择其喜欢的生活
方式的权利。1986年出版的《性别的正义》( \textit{Gender Justice}) 一书中,大卫 $\cdot$ 克普(David L. Kirp)、马克 $\cdot$尤道夫(Mark G. Yudof )、马尔兰$\cdot$斯特朗$\cdot$弗兰克(Marlene Strong Franks)这样提出古典自由主义的男女平权主义概念:
“平等既不是相等,也不是不同到不能理解这个问题,而是一种完全不同于上面两点的概念.植根于个人自治理念的法律面前的平等的自由。”

\switchcolumn*[\subsection{Slavery and Racism\\奴隶制和种族主义}]
The abolitionist movement, too, grew logically out of the Lockean libertarianism of the American Revolution. How could Americans proclaim that ``all men are created equal $\ldots$ endowed by their Creator with certain unalienable rights,'' without noticing that they themselves were holding other men and
women in bondage? They could not, of course, and indeed the
world's first antislavery society was founded in Philadelphia the
year before Jefferson wrote those words. Jefferson himself
owned slaves, yet he included a passionate condemnation of
slavery in his draft of the Declaration of Independence: ``[King
George] has waged cruel war against human nature itself, violating its most sacred rights of life and liberty in the persons of
a distant people who never offended him.'' The Continental
Congress deleted that passage, but Americans lived uneasily
with the obvious contradiction between their commitment to
individual rights and the institution of slavery.
\switchcolumn
废奴运动同样在逻辑上也源于美国革命的洛克式古典自由
主义。美国人怎么会一边宣称“人人生而平等……被造物主
赐予某些不可剥夺的权利”,同时却看不到他们自己正在把别
的男人和女人当作奴隶呢?当然不会,实际上,世界上第一个
反奴隶制的社会就是在费城建立起来的,时间是杰斐逊写下上
面这些文字之前一年。杰斐逊自己拥有奴隶,而 他 在 《独立
宣言》 的草稿中对奴隶制进行了激烈谴责:“ (英王乔治)对
人性本身发动了残酷的战争,侵犯了遥远地方的从未冒犯过他
的人民最神圣的生命和财产权利。” 大陆会议删除了这一段,
但是美国人已经不能容忍在个人权利的实践和奴隶制度之间显
而易见的矛盾。
\switchcolumn*
Although they were intimately connected in American history, slavery and racism are not inherently bound together. In
the ancient world the act of enslaving another person did not
imply his moral or intellectual inferiority; it was just accepted
that conquerors could enslave their captives. Greek slaves were
often teachers in Roman households, their intellectual eminence acknowledged and exploited.
\switchcolumn
尽管奴隶制、种族主义与美国历史密不可分,但它们之间
并没有内在的联系。在古代社会,一个人成为奴隶并不意味着
他在道德和智力上低人一等,而只是因为他做了俘虏被迫接受
征服者把他变成奴隶。希腊奴隶常常在罗马人的家里做教师,
他们卓越的知识是被承认和得到利用的。
\switchcolumn*
In any case, racism in one form or another is an age-old problem, but it clearly clashes with the universal ethics of libertarianism and the equal natural rights of all men and women. As
Ayn Rand pointed out in her essay ``Racism,''
\switchcolumn
无论如何,形形色色的种族主义是一个古老的问题,但是
显然与古典自由主义的普世道德和所有男人女人的平等的自然
权利是互相冲突的。安 $\cdot$ 兰 德 在 “种族主义” ---文
中指出:
\switchcolumn*
\begin{quote}
Racism is the lowest, most crudely primitive form of collectivism.
It is the notion of ascribing moral, social or political significance
to a man's genetic lineage $\ldots$ which means, in practice, that a
man is to be judged, not by his own character and actions, but by
the characters and actions of a collective of ancestors.
\end{quote}
\switchcolumn
\begin{quote}
	种族主义是一种最低级、最粗鄙原始的集体主义。
	一种把道德、社会和政治的特征归因于人的遗传血统的
	观点 ...... 也就是说,在实际生活当中不以每个人的品格
	和行为来判断一个人,而是通过他祖先们的品格和行为来判断。
\end{quote}
\switchcolumn*
In her works Rand emphasized the importance of individual
productive achievement to a sense of efficacy and happiness.
She argued, ``Like every other form of collectivism, racism is a
quest for the unearned. It is a quest for automatic knowledge---
for an automatic evaluation of men's characters that bypasses
the responsibility of exercising rational or moral judgment---
and, above all, a quest for \textit{an automatic self-esteem} (or pseudo-self-esteem).'' That is, some people want to feel good about themselves because they have the same skin color as Leonardo
da Vinci or Thomas Edison, rather than because of their individual achievements; and some want to dismiss the achievements of people who are smarter, more productive, more
accomplished than themselves, just by uttering a racist epithet.
\switchcolumn
在她的著作当中,安 $\cdot$ 兰德强调个人成就的重要性,视其
为一种幸福和良药。她说: “就像其他类型的集体主义一样,
种族主义是一种不劳而获,是一种不动脑子的认识,企图逃避
对人进行理性和道德判断的责任,不经思考就对人的品格进行
评价,总之,这是一种未经思考的感觉良好(或者说是假感
觉良好)。” 也就是说,有的人感觉良好,因为他们认为自己
有和达$\cdot$ 芬奇或者爱迪生一样的肤色,而不是因为他们的个人
成就;而有的人则只是希望用种族主义的蔑称来否定那些比他
们更聪明、更有能力、更成功的人的成就。

\switchcolumn*[\section{Individualism Today\\今天的个人主义}]
How fares the individual in America today? Conservatives, liberals, and communitarians all complain at times about ``excessive individualism,'' generally meaning that Americans seem
more interested in their own jobs and families than in the
schemes of social planners, pundits, and Washington interest
groups. However, the real problem in America today is not an
excess of individual freedom but the myriad ways in which government infringes on the rights and dignity of individuals.
\switchcolumn
个人主义在今天的美国是什么样的?保守主义者、自由派
和社群主义者都在抱怨“个人主义过头了”,大体意思是美国
人似乎关心自己的工作和家庭胜过关心社会计划者们的宏伟蓝
图,胜过关心学者和华盛顿的利益集团谈论的话题。然而,今
天美国的真正问题不是个人自由太多了,而是政府在通过无数
途径侵犯个人权利和尊严。
\switchcolumn*
Through much of Western history, racism has been wielded
by whites against blacks and, to a lesser extent, people of other
races. From slavery to Jim Crow to the State Sovereignty Commission of Mississippi to the comprehensive racist system of
apartheid to the treatment of the native inhabitants of Australia, New Zealand, and America, some whites have used the
coercive mechanisms of the state to deny both the humanity
and the natural rights of people of color. Asian Americans have
also been subjected to such deprivations of liberty, though never
on the scale of slavery: the Chinese Exclusion Act of 1882, the
nineteenth-century law forbidding Chinese Americans to testify
in court, and most notoriously the incarceration of Japanese
Americans (and the theft of their property) during World War
II. European settlers in North America sometimes traded and
lived in peace with American Indians, but too often they stole
Indian lands and practiced policies of extermination, such as the
notorious uprooting of Indians from the southern states and
their forced march along the Trail of Tears in the 1830s.
\switchcolumn
在西方历史上很多时期,种族主义大多数是白人针对黑人,
也有一些是针对其他种族。从奴隶制到“吉姆 $\cdot$ 克劳法”\footnote{吉姆 $\cdot$克劳法(Jim Crow laws), 泛指美国南部各州自1870年开始制定的对黑人实行种族隔离或种族歧视的法律。杰姆$\cdot$克劳原是19世纪上半叶黑人歌舞剧中一首歌曲的名称,剧中扮演反叛者的黑人在歌词中曾用杰姆$\cdot$克劳的名字,以后就成为对黑人的蔑称。这种法律实质上是美国内战结束后南部各州所制定的“黑人法典” 的继续,主要内容是通过征收人头税、选举登记、文化测验等手段,剥夺黑人选举权;并在学校、住区、公共交通以及就业、军役等各方面实行种族歧视。},从“密西西比州主权委员会”\footnote{密西西比州主权委员会(the State Sovereignty Commission of Mississippi),存在于1956$\sim$1977年 ,是密西西比州政府成立的一个情报机构,主要任务是防止联邦权力借民权运动进入密西西比州,并阻止民权运动的薨延。解密的政府文件显示,这个情报机构对民权运动活跃分子进行了长期跟踪和监视,以恫吓、非法监禁、收买陪审团以及其他非法手段来挫败争取民权的积极分子的活动。}到澳大利亚、新西兰和美国对土著居民实施的全面种族隔离制度,部分白人用国家强制机器来剥夺有色人种的人权和自然权利。亚裔美国人也曾经类似地被剥夺过自由,尽管没有达到奴隶制那样的程度:1882年的《排华法案》,这 部 19世纪的法律禁止华裔美国人到法院作证。最臭名昭著的是第二次世界大战中对日裔美国人的监禁以及没收他们的财产。欧洲殖民者有时候会和美洲印第安人进行贸易、和平相处,但是他们经常抢夺印第安人的土地,采取消灭印第安人的政策,例 如 19世纪30年代臭名昭著的驱逐印第安人出南部各州并强迫他们沿着“泪水之路”\footnote{泪水之路(Trail of Tears),是指19世 纪 20$\sim$30年代的印第安人大规模西迁事件。 当时,南部各州发现金矿,白人采矿者大量涌入,加上当地政府许诺	资助印第安人西迁,因此印第安各部族陆续开始放弃原有家园往西迁移。其中以1838$\sim$1839年切诺基族西迁最为著名。切诺基人称这次西迁为Trail of Tears, 因为一路上因为饥饿、疾病和疲劳,许多人在迁移过程中死去。以前中文通常翻译成 “血泪之路” ,有些超出原有意思。译者在这里翻译成“泪水之路”。}迁移。
\switchcolumn*
Millions of Americans fought to overturn first slavery and
more recently Jim Crow and the other trappings of state-sponsored racism. However, the civil rights movement eventually
lost its moorings and undercut its libertarian goal of equal 
rights under the law with advocacy of a new form of state-sponsored discrimination. Instead of guaranteeing to every American equal rights to own property, make contracts, and
participate in public institutions, laws today require racial discrimination by both governments and private businesses. The
Congressional Research Service in 1995 found 160 federal programs employing explicit race and gender criteria. Throughout
the early 1990s it was the policy of the University of California
at Berkeley to choose half its freshman class on the basis of
grades and test scores, the other half according to racial quotas.
Other major colleges, despite a lot of rhetoric designed to confuse the issue, do the same.
\switchcolumn
数以百万计的美国人为推翻奴隶制和“吉 姆 $\cdot$ 克劳法”
及其他得到国家支持的种族主义而斗争。然而,民权运动逐渐
脱离了其初衷,切断了与古典自由主义争取法律面前平等权利
的目标的联系,而成了一种国家支持下的新形式歧视。现在的
美国法律不是保证每个美国人拥有财产、签订契约和加人公共
机构的平等权利,而是通过政府和私人公司进行种族歧视。美
国国会研究事务处(Congressional Research Service) 1995 年
发 现 160项联邦计划雇佣职员时明显考虑了种族和性别标准。从 1990年代前期幵始,加州大学伯克利分校的政策是. 一半
新生的招生按照学分和考试成绩来录取,另一半则按照种族划
分比例来录取。其他的主要大学,尽管使用了一大堆修辞手法
来掩饰这个问题,但他们的做法和加州大学伯克利分校是一
样的。
\switchcolumn*
If we hand out jobs and college admissions on the basis of
race, we can expect plenty of group conflict over which groups
will get how many places, just as we've seen in countries from
South Africa to Malaysia where goods are handed out by racial
quota. We'll get more cases like the Hispanic member of the
U.S. Postal Service Board of Governors who complained that
the Postal Service was hiring too many blacks and not enough
Hispanics. Just as some blacks tried to ``pass'' as white in order
to get the rights and opportunities reserved by law for whites
earlier in this century, we see people today---and we can expect
to see more---trying to claim membership in whatever racial
group has the highest quotas. In Montgomery County, Maryland, in 1995, two half-Caucasian, half-Asian five-year-old girls
were denied a place in a French-immersion school as Asian applicants but were told that they could reapply as whites. In San
Francisco hundreds of parents each year change their official
ethnicity to get their children into the schools they prefer, and
white firefighters conduct elaborate genealogical investigations
in hopes of turning up a long-lost Spanish ancestor who will
qualify them as Hispanic. One California contractor won a \$19
million contract from the Los Angeles rapid transit system because he was $\frac{1}{64}$ American Indian. Soon we may need to send
observers to South Africa to find out how their old Population
Registration Act worked, with its racial courts deciding who
was really white, black, ``colored,'' or Asian. Hardly a happy
prospect for a nation founded on the rights of the individual.
How much better off we might be today if the Census Bureau had accepted the proposal of the American Civil Liberties
Union to remove the ``race'' question from the census forms in
1960.
\switchcolumn
如果工作和大学入学都按照种族来录取,可以想像各个族
群之间将会为了每一个族群该得到多少位置而冲突不断,就像
我们从南非到马来西亚的许多国家所看到的那样,物资都按照
种族份额来安排。我们就会有很多像美国邮政局那样的例子。
美国邮政理事会的拉丁裔理事抱怨说邮政局雇用了太多的黑
人,而拉丁裔雇员不够。就像这个世纪早期一些黑人试图
“变成” 白人,以得到法律给白人保留的权利和机会,今天我
们也看到很多人(今后我们会看到越来越多的人)宣称是某
个种族的一员,只要这个种族的人数占据最大比例。1995年
在马里兰州的蒙哥马利县,两个半髙加索血统半亚洲血统的5
岁女孩作为亚裔申请人时被拒绝进入一所法语浸礼学校,但是
她们被告知,可以作为白人重新申请。在旧金山市,数以百计
的父母每年更改他们的登记种族身份,为的是让他们的孩子进
入心仪的学校,一些白人消防队员则在进行详细的家族谱系调
查,希望找出一个失落很久的西班牙裔祖先,以使他们的种族
身份变成拉丁裔。有一个加利福尼亚承包商从洛杉矶捷运系统
得到了一份1900万美元的合同,因为他有1/64的美洲印第安
人血统。也许我们很快就需要派观察员到南非,去学习他们旧
的人口登记法和它的种族法庭是如何运转的,以便用来判决谁
是真正的白人、黑人、有色人种、亚洲人。对一个建立在个人
权利基础之上的国家来说,这很难说是一个愉快的前景。如果
美国人口调查局在I960年就接受美国公民自由联盟的提案要求 ,从人口调查表上取消“种族” 一项,那么我们今天的状
况会好得多。
\switchcolumn*
Of course, official race and gender discrimination is not the
only way in which governments treat us as groups rather than
as individuals today. We're constantly exhorted to look at public policy in terms of its effect on groups, not whether it treats
individuals according to the principle of equal rights. Interest
groups from the American Association of Retired Persons to the
National Organization for Women to the National Gay and
Lesbian Task Force to the Veterans of Foreign Wars to the National Farmers Organization to the American Federation of
Government Employees encourage us to think of ourselves as
members of groups, not as individuals.
\switchcolumn
当然,官方的种族和性别歧视并不是美国政府把我们当成
集团而不是个人看待的唯一方式。长期以来我们被劝告说,看
待公共政策时要看它对各种集团的影响,而不是看它是否按照
平等权利的原则来对待个人。各种利益集团,从 “美国退休
者协会” 到 “全国妇女组织” ,一 直 到 “全国男女同性恋特别
小组” “海外战争退伍军人协会” “全国农场主组织” “美国
政府雇员联盟” 等,鼓励我们把自己当作集团的一员看待,
而不是当作个人。
\switchcolumn*
First Lady Hillary Rodham Clinton epitomizes some of the
problems individualism faces in contemporary America. Beginning with the sensible if often exaggerated proverb that ``it
takes a village to raise a child,'' she ends up, in her book \textit{It Takes
a Village}, calling on all 250 million Americans to raise each
child. We can't possibly all take responsibility for millions of
children, of course. She calls for ``a consensus of values and a
common vision of what we can do today, individually and collectively, to build strong families and communities.'' But there
can be no such collective consensus. In any free society, millions
of people will have different ideas about how to form families,
how to rear children, and how to associate voluntarily with others. Those differences are not just a result of a lack of understanding of each other; no matter how many Harvard seminars
and National Conversations funded by the National Endowment for the Humanities we have, we will never come to a national consensus on such intimate moral matters. Clinton
implicitly recognizes that when she insists that there will be
times when ``the village itself [read: the federal government]
must act in place of parents'' and accept ``those responsibilities
in all our names through the authority we vest in government.''
In the end, then, she reveals her antilibertarianism: \textit{Government}
must make decisions about how we raise our children.
\switchcolumn
前第一夫人希拉里曾概括了个人主义在当代美国所面临的
种种问题。她 用 一 个 感 性 的 (如果不是夸张的的话)谚语
“全村人养一个孩子” ,在她的 《全村人养一个孩子》(\textit{It Takes a Village}) 一书结尾,她呼吁2. 5 亿美国人来养育每一个孩子。我们所有人当然不可能对数以百万计的孩子负责。她呼吁:“无论是个人还是集体都应对我们今天能够做的事情的
价值达成共识,有共同的愿景,以建立强健的家庭和社区。”可是我们不可能拥有这样的集体共识。在任何自由社会,数以
百万计的人们对于如何组建家庭、如何抚养孩子、如何与别人
进行自愿的联合有着各不相同的理念。这些不同不仅仅是因为
相互之间缺乏理解。不管国家人文基金会每年资助召开多少次
哈佛研讨会和全国对话会,我们都永远不会就这些私人道德问
题达成全国性的共识。希拉里很显然认识到了这点(人们不
可能达成共识),因此她在很多次讲话中呼吁, “村子自己
(解读:指联邦政府)必须代替父母采取行动,通过我们选举
的政府以我们全体公民的名义承担起抚养孩子的责任。”最
后,她显现出反古典自由主义的面目:\textbf{政府}必须对我们如何抚
养孩子做出决策。
\switchcolumn*
Even when the government doesn't step in to take children
from their parents, Hillary Clinton sees it constantly advising, nagging, hectoring parents: ``Videos with scenes of commonsense baby care---how to burp an infant, what to do when soap
gets in his eyes, how to make a baby with an earache comfortable---could be running continuously in doctors' offices, clinics,
hospitals, motor vehicle offices, or any other place where people
gather and have to wait.'' The child-care videos could alternate
with videos on the food pyramid, the evils of smoking and
drugs, the need for recycling, the techniques of safe sex, the joys
of physical fitness, and all the other things the responsible adult
citizens of a complex modern society need to know. Sort of like
the telescreen in \textit{1984}.
\switchcolumn
甚至当政府没有介入进来把孩子从他们父母那里带走的时
候 ,希拉里就让它对父母们不断提供建议,不断地絮叨和吓
唬:“关于如何照顾婴儿的常识的录像应当不断地在医生办公
室、诊所、医院、汽车销售厅,以及其他任何人们聚集和不得
不等候的地方播放。”关于照顾婴儿的录像在这里可以换成关
于饮食金字塔的录像、关于抽烟和吸毒的坏处的录像、关于废
物利用的录像、关于安全性关系的技巧的录像、关于健身的快
乐的录像,以及所有其他成年公民在复杂的现代社会需要知道的事情的录像。这看起来很像《1984》\footnote{《1984》是英国作家乔治• 奥维尔写于1940年代的小说。小说描写了一个虚构的极权社会,在这个极权社会里,人们的所有生活,包括个人的思想都在国家极权机器的控制之下。在 1984的社会里,有一个无孔不人的现代化设备,叫做 “ 电屏”。每个房间墙上都装有一面长方形的金属镜子,可以视听两用,也可	以发号施令,室内一言一语,一举一动,无时无刻不受电屏的监视和支配。平时无事,电屏就没完没了地播送大军进行曲、政治运动 的口 号或“第九个三年计划” 超额胜利完成的消息。这些噪音由中央枢纽控制,个人无法关掉。} 里的电屏。
\switchcolumn*
When Bill Clinton announced that by his own authority he
was issuing new regulations on tobacco and smoking in the
name of ``the young people of the United States,'' he said,
``We're their parents, and it is up to us to protect them.'' And
Hillary Clinton told \textit{Newsweek} in 1996, ``There is no such thing
as other people's children.'' These are profoundly anti-individualist and anti-family ideas. Instead of recognizing individual parents as moral agents who can and must take responsibility for
their own decisions and actions, the Clintons would absorb
them into a giant mass of collective parenting directed by the
federal government.
\switchcolumn
当比尔$\cdot$ 克林顿宣布通过他自己的职权,以 “美国年轻
人” 的名义发布了关于吸烟的新规定的时候,他说:“我们是
他们的父母,我们有责任保护他们。” 希 拉 里 $\cdot$克林顿则在
19\%年 对 《新闻周刊》说:“没有所谓的别人的孩子。”这是
极度反个人主义和反家庭的观念。克林顿夫妇不承认作为个人
的父母是能够对他们自己的决定和行为承担责任的道德主体,
相反却打算把他们纳入联邦政府指导的数量庞大的集体抚养孩
子的系统当中。
\switchcolumn*
The growing state has increasingly treated adult citizens as
children. It takes more and more money from those who produce it and doles it back to us like an allowance, through a myriad of ``transfer programs'' ranging from Head Start and
student loans to farm subsidies, corporate welfare, unemployment programs, and Social Security. It doesn't trust us to decide
for ourselves (even in consultation with our doctors) what medicines to take, or where our children should go to school, or
what we can access through our computers. The state's all-encompassing embrace is particularly smothering for those who
fall into its much-touted safety net, which ends up trapping
people in a nightmare world of subsidy and dependence, taking
away their obligation as responsible adults to support themselves and sapping them of their self-respect. A caller to a talk
show on the government radio network complained recently,
``You can't cut the budget without causing the total economic---and in some cases physical---annihilation of millions of us who have nowhere to turn except to the federal government.'' What has the government done to make millions of
adult Americans afraid that they could not survive the loss of a
welfare check?
\switchcolumn
不断膨胀的国家越来越把成年公民也当作孩子对待。它从
创造财富的我们手里拿走越来越多的钱,然后通过数不清的“转移支付计划” 把这些钱像发零用钱一样施舍给我们,从
“先行计划”\footnote{先行计划(Head  Start),美国 1960年代开始实施的对学前和中小学的低收入家庭子女提供教育、卫生、营养等综合援助。}、学生贷款到农业补贴、公司福利、失业计划、社会保障等。它不相信我们能决定自己该吃什么药(甚至在和自己的医生商量的情况下也不行);也不相信我们能够决定自己的孩子是不是应该上学;不相信我们知道通过电脑上网什
么能看,什么不能看。国家全方位的包围尤其让那些落入了它
吹嘘的社会保障网络的人透不过气来,这个网络让人们陷入吃
救济和依赖的梦魇不能自拔,夺走他们作为负责任的成年人的
自立责任,削弱了他们的自尊。最近,在一次政府广播网的谈
话节目中,一 位 听 众 打 进 电 话 说 “你们砍掉预算必然会引起
整体数以百万计的我们的经济的毁灭(有些甚至是肉体上的
毁灭),我们没有别的办法,只能找联邦政府。”政府是怎样
让数以百万计的美国成年人担心没有福利支票就无法生存的?
\switchcolumn*
Libertarians sometimes say, ``Conservatives want to be your
daddy, telling you what to do and what not to do. Liberals want
to be your mommy, feeding you, tucking you in, and wiping
your nose. Libertarians want to treat you as an adult.'' Libertarianism is the kind of individualism that is appropriate to a free
society: treating adults as adults, letting them make their own
decisions even when they make mistakes, trusting them to find
the best solutions for their own lives.
\switchcolumn
古典自由主义者有时说: “保守主义者想成为你的爸爸,
告诉你做什么,不做什么。自由派想成为你的妈喂你,把
你放进被窝里,给你擦鼻涕。古典自由主义者则希望把你当成
成年人。” 古典自由主义是一种适合自由社会的个人主义:把
成年人当成年人看,让他们做出自己的决定,哪怕犯错误,相
信他们能够为自己的生活找到最好的方案。
\end{paracol}