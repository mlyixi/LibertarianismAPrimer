\chapter{CIVIL SOCIETY\\市民社会}
\begin{paracol}{2}
\hbadness5000

In the libertarian view, the role of government is to protect people's rights---that is all. But that is quite
enough of a task, and a government that does a good job of it
deserves our respect and congratulations. The protection of
rights, however, is only a minimal condition for the pursuit of
happiness. As Locke and Hume argued, we establish government so that we may be secure in our lives, liberties, and property as we go about the business of surviving and flourishing.
\switchcolumn
按照古典自由主义的观点,政府的角色就是保护人民的权
利,仅此而已。但是这个任务并不轻松,如果政府把这个工作
做好了,它就应该得到我们的尊敬和祝贺。然而,对于追求幸
福的目标来说,保护权利仅仅是一个最低的条件。正如洛克和
休谟所说:我们建立政府的目的是在我们为了生存和发展而从
事事业的时候,生命、自由和财产能够得到保护。
\switchcolumn*
We can barely survive, and hardly flourish, without interacting with other people. We want to associate with others to
achieve instrumental ends---producing more food, exchanging
goods, developing new technology---but also because we feel a
deep need for connectedness, for love and friendship and community. The associations we form with others make up what we
call civil society. Those associations can take an amazing variety
of forms---families, churches, schools, clubs, fraternal societies,
condominium associations, neighborhood groups, and the myriad forms of commercial society, such as partnerships, corporations, labor unions, and trade associations. All of these
associations serve human needs in different ways. Civil society
may be broadly defined as all the natural and voluntary associations in society. Some analysts distinguish between commercial
and nonprofit organizations, arguing that businesses are part of
the market, not of civil society; but I follow the tradition that the real distinction is between associations that are coercive (the
state) and those that are natural or voluntary (everything else).
Whether a particular association is established to make a profit
or to achieve some other purpose, the key characteristic is that
our participation in it is voluntary. The associations within civil
society are created to achieve particular purposes, but civil society as a whole has no purpose; it is the undesigned, spontaneously emerging result of all those purposive associations.
\switchcolumn
如果不与其他人进行交往,我们将几乎不能生存和发展。
希望与别人交往不仅是一种为达到目的而采用的手段,不仅是
为了生产更多的粮食、交换更多的商品、开发新技术等,而且
也是因为我们感觉到对交往、爱、友谊和集体生活有着很深的
需要。我们与别人一起建立的各种社会组织就构成了市民社
会。这些社会组织有令人惊异的各种不同形式:家庭、教会、
学校、俱乐部、共济会、公寓业主协会、街坊会以及金字塔结
构的商业组织,如合作社、公司、工会、商会等。所有这些社
会组织都是为满足人们的各种需要而建立起来的。市民社会也
许可以宽泛地定义为社会中的所有自然自发的社会组织。有的
分析家将商业组织和非营利组织区分开来,认为商业组织是市
场的一部分,而不是市民社会的一部分;但是在这里我遵循传
统的观点,认为组织之间真正的差别在于有的是强制性的,如
政府;有的是自然的或自发的,除了政府之外的其他组织都属
于此类。无论一个社会组织的建立是为了谋取利润还是达到其
他的什么目标,最关键的特征是我们参与社会组织是自愿的。
在市民社会当中,社会组织的创建是为了达到特定的目的,但
是市民社会作为一个整体是没有目的的;它是那些有目的的社
会组织的活动所形成的非故意的、自发的结果。
\switchcolumn*
Some people don't really like civil society. Karl Marx, for instance. Commenting on political freedom in an early essay, ``On
the Jewish Question,'' Marx wrote that ``the so-called rights of
man $\ldots$ are nothing but the rights of the member of civil society, i.e., egoistic man, man separated from other men and the
community.'' He argued that ``man as he is in civil society'' is ``an
individual withdrawn behind his private interests and whims
and separated from the community.'' Recall that Thomas Paine
distinguished society from government, civil society from political society. Marx revives that distinction, but with a twist: He
wants political society to squeeze out civil society. When people
are truly free, he says, they will see themselves as citizens of the
whole political community, not ``decomposed'' into different,
nonuniversal roles as a trader, a laborer, a Jew, a Protestant. Each
person will be ``a communal being'' united with all other citizens, and the state will no longer be seen as an instrument to
protect rights so that individuals can pursue their selfish ends
but as the entity through which everyone will achieve ``the
human essence [which] is the true collectivity of man.'' It was
never made clear just how this liberation would arrive, and the
actual experience of Marxist regimes was hardly liberating, but
the hostility to civil society is clear enough.
\switchcolumn
有的人是真的不喜欢市民社会,如马克思。在一篇早期文
章 “论犹太人问题” 中他表达了对政治自由的看法。马克思
说 :“所谓的人权……无非是作为市民社会成员的权利,即脱
离了人的本质和共同体的利己主义的人的权利。” 他论述道:
“一个人,当他进入市民社会时” 是 “孤独地藏在他的个人利
益和想法背后的单独个人,与这个共同体是脱离的”。此前,
潘恩划分了社会与政府、市民社会与政治社会。马克思呼应了
这种划分,但是进行了曲解:他想用政治社会把市民社会挤压
出去。他说,当人们获得真正自由的时候,他们将会把自己看
作整个政治社会的公民,而不再分解为五花八门的各种角色:
商人、工人、犹太人、新教徒等等。每个人都将 是 一 个 “社
会的存在” ,与其他公民联合在一起,而国家将不再被看作是
一个保护个人权利、让个人,能够追求他们自私目标的工具,而
是作为一个整体,在其中每 个 人 都 将 能 够 达 到 “人的本
质-----个真实的集体的人”。马克思并没有解释清楚,这样
的解放如何才能够达到,而马克思主义的实际情况是基本上没
有得到解放,但其对市民社会的敌意却很清楚。
\switchcolumn*
Marxism is a bad word these days (as it should be), but
Marx's powerful hold on so many people for so long indicates
that he was on to something when he wrote about people feeling alienated and atomized. People do want to feel at least some
connection to other people. In traditional, precapitalist communities they didn't have much choice about it; in a village,
people you had known all your life were all around you. Like it
or not, you couldn't avoid having a sense of community. As liberalism and the Industrial Revolution brought freedom, affluence, and mobility to more people, more and more of them
chose to leave the villages of their birth, often even the countries of their birth, and go off to make a better life elsewhere.
The decision to leave indicated that people expected to find a
better life; and continuing mobility and emigration, generation
after generation in modern society, would seem to indicate that
people \textit{do} find better opportunities in new places. But even a
person who is glad he left the village or the old country may feel
a loss of that sense of community, just as one's departure from
the family to become an adult may generate a profound sense of
loss even as one enjoys autonomy and independence. That's the
longing to which Marxism seemed, to many people, to provide
an answer.
\switchcolumn
马克思能够在相当长的时间里对如此多的人产生强烈吸引
力,说明他在描述人们对异化和人的原子化的感觉时,的确抓
住了一些实质性的东西。人们的确希望感觉自己和其他人至少
是有某种联系的。在传统的前资本主义社会里,人们在这方面
并没有多少选择。在一个村庄里,你一辈子所认识的人都在周
围。不管你喜不喜欢,你都无法摆脱作为社会一分子的感觉。当自由主义和工业革命给越来越多的人带来自由、富裕和流动性的时候,他们当中越来越多的人选择离开出生的村庄,甚至
常常是离开出生的国家,去另外一个地方创造更好的生活。决
定离开说明人们希望寻找更好的生活;而在现代社会中,一代
又一代人持续地流动和移民,似乎说明人们\textbf{的确}在新地方找到
了更好的机会。但是,即便一个很高兴自己离开村庄或者原来
国家的人,有时也会感到有些失落,就像一个人离开家变成成
年人的时候也许会激起深深的失落感,尽管他同时很喜欢自治
和自立。对很多人来说,马克思主义将能够为这种渴望提供
答案。
\switchcolumn*
Ironically, Marxism promised freedom and community but
delivered tyranny and atomization. The tyranny of the Marxist
countries is well known, but it may not be so well understood
that Marxism created a society far more atomized than anything in the capitalist world. The Marxist rulers in the Soviet
empire, in the first place, believed theoretically that men under
conditions of ``true freedom'' would have no need for organizations catering to their individual interests, and in the second
place, understood practically that independent associations
would threaten the power of the state. Thus, they not only
eliminated private economic activity, they sought to stamp out
churches, independent schools, political organizations, neighborhood associations, and everything else, down to the garden
clubs. After all, the theory went, such nonuniversal organizations contributed to atomization. What happened, of course,
was that people deprived of any form of community and connectedness between the family and the all-powerful state became atomistic individuals with a vengeance. As the
philosopher and anthropologist Ernest Gellner wrote, ``The system created isolated, amoral, cynical individualists-without-opportunity, skilled at double-talk and trimming.'' The normal
ways in which people were tied to their neighbors, their fellow
parishioners, the people with whom they did business were destroyed, leaving them suspicious and distrustful of one another,
seeing no reason to cooperate with others or even to treat them
with respect.
\switchcolumn
具有讽刺意味的是,马克思主义承诺的是自由和共同体,
但却提供了专制和原子化。有一点可能大家知道得不多,那就
是马克思主义创造了一个比资本主义世界要原子化得多的社
会。首先,苏维埃帝国的统治者们在理论上相信人在“真正
自由” 的条件下将不需要组织来满足他们的个人利益;其次,
他们在实践中认识到,独立的社会组织将会威胁到国家权力。
于是,他 们 不 但 消 灭 了 私 人 经 济 而 且 试 图 消 灭 教 会 、独
立的学校、政治组织、街坊会以及其他的社会组织,一直到花
园倶乐部。最后,因为马克思主义理论认为,此类的非统一的
组织造成了社会的原子化。于是很自然地,随后人们被剥夺了
在家庭和全能政府之间的任何形式的共同体和联系,彻底变成
了原子化的个人。正如哲学家和人类学者格尔纳(Ernest Gellner)所说: “这种制度创造了孤立的、无道德的、玩世不恭
的、毫无希望的个人,精于含糊其辞、随风转舵。”正常的社
会方式以及人们与邻里、教区群众、生意伙伴之间的关系已被
摧毁。他们互相猜忌和不信任,看不到任何理由与他人进行合
作,甚至连仅仅是互相尊重都做不到。
\switchcolumn*
The even greater irony, perhaps, was that Marxism eventually produced a renewed appreciation for civil society. As the
corruption of the Brezhnev years faded into liberalization under
Gorbachev, people began to look for an alternative to socialism,
and they found it in the concepts of civil society, pluralism, and
freedom of association. The billionaire investor George Soros,
eager to liberate the land of his birth (Hungary) and its neighbors, began by making large contributions not to bring about
political revolution but to rebuild civil society. He sought to
subsidize everything from chess clubs to independent newspapers, to get people once again working together in nonstate institutions. The burgeoning of civil society was not the only
factor in the restoration of freedom to Central and Eastern Europe, but a stronger civil society will help to protect the new
freedom, as well as supply all the other benefits that people can
achieve only in association.
\switchcolumn
更大的讽剌也许是马克思主义社会逐渐对市民社会产生了
新的正面评价。当腐败的勃涅日列夫时代逐渐演变为自由化的
戈尔巴乔夫时代以后,人们开始寻找替代模式。他们找到了市
民社会、多元主义和自由结社。亿万富翁、投资人乔治$\cdot$索罗
斯的出生地匈牙利及其邻国获得解放,他开始了大量捐助,希
望在不带来政治革命的情况下重建市民社会。他给各种各样的
事业提供资助,从国际象棋倶乐部到独立报纸,让人们能再次
在非国家的机构中一起工作。市民社会的萌芽和生长并不是中
欧、东欧重返自由的唯一因素,但是一个强大的市民社会将给
新的自由提供屏障,给人们提供所有其他只在社会组织当中才
能得到的好处。
\switchcolumn*
Even people who aren't Marxists share some of Marx's concerns about community and atomization. Communitarian
philosophers, who believe individuals must necessarily be seen
as part of a community, worry that people in the West, especially in the United States, overemphasize claims to individual
rights at the expense of the community. Their view of our relationship to others could be represented as a series of concentric
circles: an individual is part of a family, a neighborhood, a city, a
metropolitan area, a state, a nation. The implication of these arguments is that we sometimes forget to focus on all the circles
and that we should somehow be encouraged to do so.
\switchcolumn
甚至那些不是马克思主义者的人对共同体和原子化的问题
也持有与马克思有同样的观点。社群主义理论家认为个人必然
是共同体的一部分,因此担心在西方尤其是美国过度强调个人
权利将会使社会付出代价。他们关于人们之间相互关系的观点
可以被描述为一串同心圆:个人、家庭、街坊社区、城市、城
市带、州、国家。这些论点的推论是我们有时没有注意所有这
些圆圈,而我们应当被鼓励这么做。
\switchcolumn*
But are the circles merely concentric? A better way to understand community in the modern world is as a series of \textit{intersecting} circles, with myriad complex connections among them.
Each of us has many ways of relating to other people---precisely
what Marx complained of and libertarians celebrate. One person may be a wife, mother, daughter, sister, cousin; an employee of one business, an owner of another, a stockholder in
others; a renter and a landlord; an officer in a condominium association; active in the Little League and the Girl Scouts; a
member of the Presbyterian Church; a precinct worker for the
Democratic Party; a member of a professional association; a
member of a bridge club, a Jane Austen fan club, a feminist
consciousness-raising group, a neighborhood crimewatch, and
more. (True, this particular person probably feels pretty frazzled, but at least in principle, one can have an indefinite number of associations and connections.) Most of these associations
serve a particular purpose---to make money, to reduce crime, to
help one's children---but they also give people connections with
other people. No one of them, however, exhausts one's personality and defines one completely. (One can approximate such exhaustive definition by joining an all-embracing religious
community, say, a Roman Catholic order of contemplative
nuns, but such choices are voluntary and---because one can't
alienate one's right to make choices---reversible.)
\switchcolumn
但是这些圆圈是同心的吗?在现代世界当中,更好的办法
是把社会理解成一系列\textbf{互相交叉的圆},在他们之间有着大量的
复杂联系。我们每一个人和其他人之间都有很多种不同的联系
--- 这恰好是马克思所抱怨的,是古典自由主义者所应庆祝
的。一个人可能同时是妻子、母亲、女儿、姐妹、表亲;是一
家企业的员工、另一家企业的所有人、另外一些企业的股东;
是房主和出租人;是一名公寓业主协会的管理人员;可能参加
少年棒球联盟和女童子军的活动;是基督教长老会的成员;是
民主党的选区工作人员;职业协会的成员;桥梁俱乐部、简 $\cdot$奥斯汀爱好者倶乐部、女权意识提升组织的成员;是街坊犯罪
联防组织的成员等。当然这个人也许会感到有些晕头转向,但
总的来说,一个人能够拥有数量无限的社会关系和社会组织。
绝大多数的社会组织都有一个特定的目标 --- 赚钱、减少犯
罪、帮助儿童一但它们同时也给人们提供了与其他人的联
系。没有一个组织能够描述一个人的整体个性,能够完全定义
一个人的特征。一个人可能因为加入一个全面的宗教共同体而
接近于这样的完全定义,如罗马天主教会的修女,但是这样的
选择是自愿的和可逆的,因为这个人并没有放弃选择的权利。
\switchcolumn*
In this libertarian conception we connect to different people
in different ways by free and voluntary consent. Ernest Gellner
says that modern civil society requires ``modular man.'' Instead
of being entirely the product of, and absorbed by, a particular
culture, modular man ``can combine into specific-purpose, ad
hoc, limited associations, without binding himself by some
blood ritual.'' He can form links with others, ``which are effective even though they are flexible, specific, instrumental.''
\switchcolumn
在这种古典自由主义观念之下,我们通过自由的和自愿的
同意与不同的人产生不同形式的联系。欧内斯特$\cdot$格尔纳说现
代市民社会需要“模块人”。不是整个一个特定的文化的产物
或者被一种文化所吸收,模 块 人 “能够与特定目标,有限组
织 ,不把自己固定在某个血缘关系上”。他能够与其他人建立
联系,“更加有效,尽管他们很灵活、特定和工具化”。
\switchcolumn*
As individuals combine in countless ways, community
emerges: not the close community of the village, or the messianic community promised by Marxism, national socialism,
and all-fulfilling religions, but a community of free individuals
in voluntarily chosen associations. Individuals do not emerge
from community; community emerges from individuals. It
emerges not because anyone plans it, certainly not because the
state creates it, but because it must. To fulfill their needs and
desires, individuals must combine with others. Society is an association of individuals governed by legal rules, or perhaps an
association of associations, but not one large community, or one
family, in Mario Cuomo's and Pat Buchanan's utterly misguided conception. The rules of the family or small group are
not---cannot be---the rules of the extended society.
\switchcolumn
当无数个人以难以计数的方式联系起来的时候,社会就产
生了:不是小村庄那样的封闭社会,也不是马克思主义、国家
社会主义和那种心想事成的宗教所承诺的人间天堂,而是一种
自由的个人以各种自愿选择的方式结合起来的社会。不是个人
来自社会,而是社会来自个人。社会的出现不是因为某些人的
计划,当然也不是国家所创造的,而是一种必须。为满足自己
的需求和欲望,个人必须和别人相结合。社会是一种个人之间
的联合,通过法律来治理,也可以说社会是一个各种自由结社
的集合;但社会不是一个大的组织,也不是马里奥$\cdot$ 库尔莫\footnote{马里奥$\cdot$库尔莫(Mario Matthew Cuomo, 1932 $\sim$), 美国民主党政治家,1983$\sim$1994年任纽约州州长。}和帕特$\cdot$ 布坎南\footnote{帕特$\cdot$布坎南(Pat Buchanan, 1938$\sim$),美国共和党政治家,1969$\sim$1974年任尼克松总统髙级顾问,1985$\sim$1987年担任里根总统的联络主任,作为共和党人曾在1992年和1996年两次参加竞选。}所误导的那样是一个家庭。家庭或小型组织的规则不是也不应该是范围广大的社会的规则。
\switchcolumn*
The distinction between individual and community can be
misleading. Some critics say that community involves a surrender of one's individuality. But membership in a group need not
diminish people's individuality; it can amplify it, by freeing
people from the limits they face as lone individuals and increasing their opportunities to achieve their own goals. Such a view
of community requires that membership be chosen, not compulsory.
\switchcolumn
个人与组织之间的区别很容易让人产生误解。有的批评家
说,加入集体必然意味着个性的投降。但是成为组织的一员并
不必然消灭个性;相反它能够通过克服人们作为孤独个体时遇
到的限制来放大每个人的个性,增加达到个人目标的机会。前
提是这样的组织是人们自愿加入,而不是强制加入。

\switchcolumn*[\section{Cooperation\\合作}]

Because humans can't achieve much of what they want on their
own, they cooperate with other people in a variety of ways. The
government's protection of rights and freedom of action creates
an environment in which individuals can pursue their goals, secure in their person and property. The result is a complex network of free association in which people voluntarily assume and
fulfill obligations and contracts.
\switchcolumn
因为不能单凭自己的力量达到目标,于是,人们通过各种
方式与别人进行合作。政府对个人权利和自由的保护创造出一
种环境,个人能够在人身和财产安全得到保证的前提下追求自
己的目标。结果是形成了一个复杂的自由结社的网络,在这个
网络中,人们自愿承担和履行责任,签订合同和履行合同。
\switchcolumn*
Freedom of association helps to reduce social conflict. It allows members of society to link themselves together and build
intertwining networks of personal relationships. Many of these
relationships cross religious, political, and ethnic boundaries.
(Others, of course, such as religious and ethnic associations,
unite people within a particular group.) The result is that diverse and unfamiliar people come together in fellowship. Tensions that might otherwise divide people are countered by these
aspects of connectedness. A Catholic and a Protestant, who
might otherwise find themselves in conflict, meet as buyer and
seller in the marketplace, as members of the same parent-teacher association, or as participants in a softball league, where
they also meet and associate with Muslims, Jews, Hindus,
Taoists, and nonbelievers. They may disagree about religion,
may even believe one another engaged in mortal error, but civil
society provides spaces where they may cooperate peacefully. A
\textit{Washington Post} story on the growing popularity of noontime
worship services begins, ``On the street, these men and women
are clerks and lawyers, Democrats and Republicans, city
dwellers and suburbanites. Here, they are Catholics.'' A different story might begin, ``Outside, these men and women are
Catholics and Baptists, black and white, gay and straight, married and single. Here, they are employees of America Online.''
Or ``here, they are tutors for underprivileged children.'' In each
circumstance, people who may not see themselves as comfortable members of a tight community with the others in the
group can come together for a specific purpose, in the process
learning to coexist if not to embrace.
\switchcolumn
结社自由能够减少社会冲突。它允许社会成员之间自我联
合 ,建立交错复杂的人际关系网络。很多这样的关系超越了宗
教 、政治和种族的边界。当然,也有如宗教组织和种族组织那
样的组织,把人们联合在了一个特定的群体当中。结果是千差
万别、互不相识的个人走到了一起成为伙伴。人与人之间分裂
的各种紧张关系也将因此得到缓解。一个天主教徒和一个新教
徒 ,在其他时候也许会相互冲突,但他们会在市场上相遇,作
为买卖双方达成交易,或者同时成为某个教师家长联合会的成员 ,或同时加人某个垒球联盟,同时,他们也许会碰到穆斯
林 、犹太教徒、印度教徒、道教徒和无神论者并和他们建立关
系。他们也许在宗教上无法达成一致,甚至还会认为对方的宗
教在道德上是错误的,但是市民社会给他们提供了足够的空
间,让他们能够平和地进行合作。《华盛顿邮报》 的一篇关于
流行午间礼拜仪式的报道是这样幵始的:“走在街上,这些男
人和女人是店员和律师,是民主党和共和党,是城市居民和郊
区居民。在这里,他们都是天主教徒。”但另一篇不同的报道
也许会这样开始:“在外面,这些男人和女人是天主教徒和浸
礼会教徒,是白人和黑人,是同性恋和异性恋,已婚和未婚。
在这里,他们都是美国在线的员工。” 或者, “这里,他们都
是贫困家庭孩子的老师。”人们也许不会喜欢其所在的紧密共
同体里的所有人,但在每种环境下他们都可以为了一个共同的
目标而走到一起,并且在合作的过程中学会共存,即使还做不
到互相拥抱的话。
\switchcolumn*
No one person made the complex order that emerges. No
one designed it. It is the product of many human actions but of
no design.
\switchcolumn
这种复杂秩序不是哪一个人创造出来的,也不是哪一个人.
设计出来的。它是无数未经设计的人类行为的结果。

\switchcolumn*[\section{Personal Responsibility and Trust\\个人责任和信用}]
In a previous chapter I recounted the remarkable network of
trust that allows me to get cash and automobiles halfway
around the world. If critics of libertarianism were right, wouldn't
the ``atomistic'' commercial society tend to \textit{reduce} the levels of
trust and cooperation that allow bank machines to dispense
cash to strangers? This common criticism is belied by the evidence around us.
\switchcolumn
在前面的章节,我描述了神奇的信用网络,这个网络让我
能够在半个地球之外拿到现金和汽车。如果对古典自由主义的
那些批评是对的,那 么 “原子化” 的商业社会是不是会\textbf{减少}
人们之间的信任和合作程度呢?事实上,恰恰是人们发展起来
的高度信任和合作,才能让陌生人从自动提款机取到钱。这种
对自由主义的普遍批评被我们周围发生的一切证明是胡扯。
\switchcolumn*
If we are going to pursue happiness by entering into agreements with others, it's important that we be able to rely on each
other. Other than the minimal obligation not to violate the
rights of others, in a free society we have only the obligations
we voluntarily assume. But when we do assume obligations by
entering into contracts or joining associations, we are both
morally and legally bound to live up to our agreements. Several
factors help to ensure that we do: our own sense of right and
wrong; our desire to have the approval of others; moral exhortation; and, when necessary, various ways of enforcing those
obligations, including the refusal of others to do business with
people who default on their obligations.
\switchcolumn
如果我们想要通过与其他人之间达成契约来追求幸福,很重要的一点是我们能够信任别人。在自由社会,除了不侵犯别
人利益这个最起码的责任之外,其他的责任只能由我们自愿承
担。但是一旦通过签订契约或者加入社会组织而承担起责任,
我们在道德上和法律上就和契约绑在了一起。有几个因素有助
于确保我们履行承诺:我们自己的对错判断;我们希望得到别
人承认的愿望;道德戒律;在必要的时候还会有很多因素迫使
我们不得不履行责任,例如人们将不再与不履行责任的人做
生意。  
\switchcolumn*
As society develops and people want to take on larger tasks,
it becomes necessary to be able to trust more people. At first,
people may have trusted only their own family or the people in
their village or tribe. The extension of the circle of trust is one of
the great advances in civilization. Contracts and associations
play a major role in enabling us to trust each other.
\switchcolumn
当社会发展,人们希望做更大事业的时候,能够信任更多
的人就非常重要了。开始,人们只信任自己的家人或者本村、
本部落的人。信任圈子的扩大是文明社会的巨大进步。契约与
社会组织在加强我们的彼此信任上扮演了主要角色。
\switchcolumn*
Like the hero celebrated in a country song, my father was a
man ``who could borrow money at the bank simply on his
word.'' That kind of honor and trustworthiness is essential to
markets and to civilization. But it isn't enough in an extended society. My father's good reputation didn't extend much beyond the small town where we lived, and he would have had
trouble borrowing money in a hurry even a few towns over,
much less across the country or around the world. But as I
noted above, I have instant access to cash and credit virtually
anywhere in the world---not because I have a better reputation
than my father, but because the free market has developed
credit institutions that extend around the world. As long as I
pay my bills, the complex financial networks of American Express and Visa and MOST allow me to \textit{get} goods, services, or
cash wherever I go. These systems work so well that we take
them for granted, but they are truly a marvel. They work on a
much larger scale than my personal cash withdrawals and car
rentals, of course. The combination of institutions that vouch
for an individual's creditworthiness and legal institutions to
punish contract violations when necessary makes possible vast
economic undertakings, from the design and construction of
airplanes to building a tunnel under the English Channel to
worldwide computer networks such as CompuServe and America Online.
\switchcolumn
就像一首美国乡村歌曲的主人公那样,我父亲是一个
“说句话就可以从银行借到钱” 的人。这种荣誉和信任对市场
经济和文明本身都是至关重要的,但对于一个范围广大的社会
来说这还不够。父亲的好名声并不能延伸到我们生活的小镇之
外 ,在几个小镇之外,他要马上借到钱就会有些困难,更不要
说在全国范围乃至全世界范围内了。但是,就像前面提到过的
那样,我实际上可以在全世界范围内任何地方取到钱或者得到
信用贷款,并不是因为我比父亲的名声更好,而是因为自由市
场发明了信用制度并扩展到了全世界。当我支付账单的时候,
美国运通和VISA以及MOST的复杂金融网络让我无论在什么
地方都能够\textbf{得到}商品、服务或者现金。这些信用系统运作得如
此之好,以至于我们平时对之熟视无睹,但它们的确是一个奇
迹。当然,它们的功能远远超过我个人取款和租车。这些信用
服务体系结合在一起确保了个人的信用,再加上必要时用来惩
罚违反合同行为的法律制度,大型的经济事业才成为可能,比如设计和制造飞机,建造穿越英吉利海峡的海底隧道以及像
CompuServe和 AOL那样全球范围的电脑网络。
\switchcolumn*
As credit becomes so widespread and readily available, some
people come to think of it as a right. They get morally exercised
when people are denied credit. They demand regulation of
credit bureaus, suppression of bad credit information, limits on
interest rates, and so on. Such people don't understand the crucial importance of trust. They seem not to realize that people
don't want to lend their hard-earned money to unreliable credit
risks. If reliable credit information is unavailable, interest rates
will go up to cover the increased risk. If information is unreliable enough, the extension of credit will grind to a halt, or
credit will be available only through personal and family connections, surely the opposite of what critics of credit bureaus
want.
\switchcolumn
由于信用贷款服务变得如此广泛,随时可以得到,有人就
开始认为这是一项个人权利。当有人信用贷款被拒绝的时候,
他们会从道德角度对此担忧。他们要求信用机构出面进行干
预,要求把不良信用记录压下来,限制利率等。这些人不明白
信用极端重要的作用。他们似乎并没有认识到,人们不想把自
己辛苦挣来的钱冒险借给信用记录不值得信赖的人。如果没有
可信赖的信用记录,贷款利率将会上升以覆盖较高的风险。如
果信用记录根本不值得信赖,那么信用贷款服务的扩展就会被
粉碎或者中断,或者信用贷款就只会通过个人关系和家庭关系
来提供。这当然不是信用机构的批评者们所希望看到的。
\switchcolumn*
The network of trust and credit relies on all the institutions
of a free society: individual rights and responsibility, secure
property rights, freedom of contract, free markets, and the rule
of law. A complex order rests on a simple but secure foundation.
As in chaos theory, a simple nonlinear equation can produce
endless mathematical complexity, so the simple rules of a free society can produce infinitely complex social, economic, and
legal relationships.
\switchcolumn
信任和信用的网络依赖于自由社会的所有这些制度:个人
权利和责任、产权保护、契约自由、自由市场和法治。复杂的
秩序是建立在虽然简单却可靠的基础之上的。按照混沌理论,
一个简单的非线性方程能够产生无穷的数学复杂性,同样,自
由社会的简单规则能够产生无限复杂的社会、经济和法律
关系。

\switchcolumn*[\section{The Dimensions of Civil Society\\市民社会的维度}]

It would be difficult to describe all the forms that civil society
takes in a complex world. More than 100 years ago, Alexis de
Tocqueville wrote in \textit{Democracy in America}, ``Americans of all
ages, all conditions, and all dispositions constantly form associations $\ldots$ to give entertainments, to found seminaries, to build
inns, to construct churches, to diffuse books, to send missionaries to the antipodes; in this manner they found hospitals, prisons, and schools.'' Today you can pick up any daily newspaper
and take a look at the kinds of organizations described their businesses, trade associations, ethnic and religious associations,
neighborhood groups, music and theater groups, museums,
charities, schools, and more. On the day I started writing this
chapter, I picked up the \textit{Washington Post}. Besides all the usual
groups that form the background to each day's news, I found
three stories that stood out for me as examples of the diversity
of civil society.
\switchcolumn
要描述在复杂世界当中市民社会的所有形式是相当困难
的。一百多年前,托克维尔在《论美国的民主》一书中写道:
“各个年龄段、不同条件、各种气质的美国人不断地建立各种
社会组织……提供娱乐、召开学术研讨会、建旅馆、建教堂、
散发书籍、把传教士送到地球的另一面;并以这种精神建立了
医院、监狱和学校。”今天你可以拿起任何一份日报看一看上面提到的各种各样的组织 --- 企业、商会、种族组织、宗教组织、街坊组织、音乐和戏剧组织、博物馆、慈善机构、学校,
等等。就在我开始写这一章内容的时候,我 拿 起 了 《华盛顿邮报》。除了构成每天新闻背景的所有常见社会组织之外,我
还找到三个故事作为市民社会多样性的例子。
\switchcolumn*
On the front page was a story about three double-income
suburban families who have created a supper club in which each
family cooks one meal a week that the other two families pick
up and take home. That way the busy families get more home-cooked family meals than any of them could produce on their
own, in the hectic world of the two-career family. Not quite as
much community, perhaps, as if the three families sat down and
ate together, but the participants say they do feel a sense of extended family: ``We stand in each other's kitchens and talk
about each other's children.'' Another story discussed a devout
Baptist family who ``try to shelter their [six children] from the
temptations and trials of the secular world, creating a life populated largely by people with similar values and beliefs.'' The
mother schools her children at home, tries to provide whole-some books, videos, and games for them, and gets them involved with other children at their church, in their
home-schooling network, and through the oldest son's interest in piano. In some ways it might seem that this family is withdrawing from civil society, but I believe we should see the story
as an example of the diversity that civil society allows, even for
those who want to pursue a way of life different from that desired by most people in the larger society. Finally, another story
told of a children's play group that has connected five families
for ten years. Not only did the group provide playmates for the
children, but by taking turns babysitting, the mothers could
give each other ``a few precious moments of independence.'' The
author concluded, ``{My daughter} can't remember a time when
she didn't know her play group friends, and I can barely remember when I didn't know mine. Bonds between friends can
be like that. In the absence of nearby kin they can be the most
sustaining ones of all.''
\switchcolumn
第一版的一篇故事,讲的是三个郊区双职工家庭,夫妻都
有工作,他们建立了一个晚餐倶乐部,每个家庭一周做一次饭
供其他两个家庭带回家。这样,在紧张的双职工家庭里,这些
繁忙的家庭就比他们自己单独做饭时吃到了更多种类的家常
菜。也许,这三个家庭的关系并没有到像大家庭一样坐下来一
起吃饭的程度,但是这些参与者说他们的确有一种大家庭的感
觉 :“我们站在每家人的厨房里,聊聊孩子。”另一个故事讨
论了一个浸礼会教徒家庭为了 “保护他们的六个孩子不受世
俗社会的诱惑和考验,创造一种周围都是拥有相同价值观和信
仰的人的生活环境。” 母亲在家里教自己的孩子,试着让他们
看健康的书籍、录像,玩健康的游戏,让他们在家庭教育网络
里、通过最大的孩子的钢琴特长来与自己教会的孩子们交往。
从某种角度看,似乎这个家庭是在逃避市民社会,但是我认为
应该把这个故事看作一个市民社会允许多样性的例子,尤其是
对那些希望追求一种不同于这个社会绝大多数人生活的人来说
更是如此。最后一个故事讲的是一个亲子游乐坊\footnote{亲子游乐坊(playgroup),在美国、欧洲等地流行的一种自发组织。一些家庭组织起来,家长们让学龄前儿童在一起玩耍、学习认字和相互交往,通常由家长们轮流看护。playgroup除了给孩子们提供一起玩耍、社交、发展友谊的机会之外,也给成年家长们提供了彼此交流的机会。}让五个家庭保持了十年的联系。这个亲子游乐坊不仅让孩子们找到了玩伴 ,而且通过轮流照看孩子,母亲们也能够得到 “ 一些珍贵的独立时间”。作者最后总结道: “我女儿现在巳经记不得结识她的游乐坊朋友之前的事情了,我也几乎记不得结识现在游乐坊的那些家长朋友们之前的事情了。朋友之间的联系也就是这样。在最近的亲戚都不在的时候,他们就会是所有关系当中最长久的。”  

\switchcolumn*[\section{Charity and Mutual Aid\\慈善与互助}]
Charitable institutions are an important aspect of civil society.
They are the focus of the quotation above from Tocqueville.
People have a natural desire to help the less fortunate, and they
form associations with others to do so, ranging from local soup
kitchens and church charity bazaars to complex national and international enterprises like United Way, the Salvation Army,
Doctors without Borders, and Save the Children. Americans
spend some \$150 billion on charity every year.
\switchcolumn
慈善组织是市民社会当中最重要的一个组成部分,这也是
托克维尔著作关注的焦点。人天然有帮助不幸者的愿望,于是
他们和其他人一起建立组织来做这件事,从施粥场、教会慈善
义卖会到复杂的全国性甚至国际性的慈善组织,如联合劝募会
(United Way) 、救世军(the Salvation Army) 、无国界医生组织(Doctors without Borders)以及儿童救助会(Save the Children) 。 美国人每年花在慈善上的钱大约有1500亿 美 元 。
\switchcolumn*
Critics of libertarianism say, ``You want to abolish essential
government programs and put nothing in their place.'' But the
absence of coercive government programs is most decidedly not
nothing. It's a growing economy, the individual initiative and
creativity of millions of people, and thousands of associations
set up to achieve common purposes. What kind of social analysis is it that looks at a complex society like the United States
and sees ``nothing'' except what government does?
\switchcolumn
批评古典自由主义的人说: “你们想取消基本的政府项
目,但是没有拿来什么代替政府。” 然而,即便没有强制性的
政府项目,也绝对不是什么都没有。正是不断增长的经济规
模 、数以百万计的人们的个人慈善动机和创造力以及数以千计
的慈善组织结合在一起才达到了共同的慈善目标。一边观察着
像美国这样的复杂社会,一边却说除了政府做的工作之外什么
都没有,这是什么样的社会分析啊!
\switchcolumn*
Charity plays an important role in a free society. But it is not
\textit{the answer} to the question of how a free society will help the
poor. The first answer to that question is that by dramatically
increasing and spreading wealth, a free economy eases and even
eliminates poverty. By the standards of history, even poor people in the United States and Europe are enormously wealthy.
The fabulous palace of Versailles had no plumbing facilities; the
orange trees on the grounds were an attempt to cover up the
stench. Gorman Beauchamp of the University of Michigan
wrote in the \textit{American Scholar} in 1995 about the abundance that
free markets and modern technology have produced:
\switchcolumn
慈善在自由社会当中扮演了重要的角色。但是这并不是对“ 自由社会是如何帮助穷人的” \textbf{这个问题的答案}。对这个问题
最重要的答案是,通过财富戏剧性的增长和扩散,一个自由的
经济体能够减少甚至消灭贫困。在美国和欧洲,即便是穷人,
如果按历史的标准来看都可以算是相当富有。神话般的凡尔赛宫连抽水马桶都没有,在周围种上橘子树是为了掩盖臭味。密
歇 根大 学 的 高 曼 $\cdot$ 比坎普(Gorman  Beauchamp) 1995年在《美国学者》(\textit{American Scholar}) 一书中谈到自由市场和现代科技创造的财富:
\switchcolumn*
\begin{quotation}
[A film] on the life of Empress Wu, China's equivalent of Catherine the Great$\ldots$ opened with a scene of a mounted courier riding furiously to pass some obviously precious packet to another
courier who tears off to the next station to pass the packet on to
another courier---and so on across North China to Peking and ultimately to the Imperial Palace. The content of the packet,
brought so effortfully from the distant mountain peaks, was then
revealed to be---ice. Ice to chill the emperor's drinks.


What struck me so forcefully about this scene, I remember,
was the realization that I could have, any time I chose, all the ice
I wanted simply by opening my refrigerator door. In this respect,
as in countless others, the material level of my life---a young person of no consequence, living on a modest stipend---was
markedly superior to that of a powerful emperor of China$\ldots$


I am warmer in the winter (central heating) and cooler in the
summer (air conditioning) than he was; I get more and better information faster and more reliably than he did; I can get to any
destination more quickly and comfortably; I am (most likely) in
less pain less of the time and get better medical care; I see better
longer (bifocals) and have better teeth (fluoride) and a dentist
who uses Novocain; and while he may have had a golden bird to
sing for him---okay, okay, that was the Byzantine emperor---I
have Rosa Ponselle or Ezio Pinza or Billie Holiday or Edith Piaf
[for younger readers, we might add, or the Rolling Stones, or the
Grateful Dead, or Alanis Morrisette] or any one of literally hundreds of performers whose voices I have on my shelves and can
summon up with the flick of a couple of switches.
\end{quotation}
\switchcolumn
\begin{quotation}
一部关于中国武则天女皇的电影,她在中国历史中的
地位相当于俄罗斯的叶卡捷琳娜女皇……最开始的镜头是
一名信使骑着马疾驰而过,把一个显然很贵重的邮包交给
另一个信使,另一个信使也拍马飞奔,到下一个驿站把包
裹交给下一个信使 --- 如此这般穿过整个北中国到达北京\footnote{原文如此},最后送进了皇宫。这个包裹里面的东西,如此兴师动众从遥远山区带过来的东西,打开一看是 --- 冰。这些冰是为了加在皇上的饮料当中的。
	
	
我记得,这个镜头最让我印象深刻的是,当时我意识
到我能够在任何愿意的时间得到所有我想要的冰块,只要
打开我的冰箱门就可以了。从这个例子和其他无数的例子
来看,我的物质生活水平很显然比中国最强大的皇帝还要
高得多,尽管我只是一个没有任何权势的年轻人,拿着不
兩不低的析水......
	
	
在冬天我能够比中国皇帝感觉更温暖,因为我有暖
气;在夏天则比皇帝感觉更凉爽,因为我有空调;我能够
得到比皇帝更多更好的信息,而且更快、更可靠;我能够
更加迅速和安全地达到任何目的地;我 (很可能)生病
的机会更少,得病的持续时间更短,能得到比他更好的医
疗服务;我的眼睛能看得更清楚,看得清楚的年龄更长
( 借助老花镋),我的牙齿更好(通过氟化物牙骨),一位牙医用局部麻醉剂给我看牙;他也许有一只金色鸟儿给他
唱歌 --- 好 吧 , 我说错了,有金色鸟儿的是拜占庭皇帝
--- 但我有罗莎 $\cdot$ 邦丝莱、埃 齐 奥 $\cdot$ 平扎、比 莉 $\cdot$哈乐
黛、迪瑟雅芙\footnote{罗莎$\cdot$邦丝莱(Rosa Ponsdle,1897$\sim$1981),美国1930年代红极一时的女高音,号称歌剧皇后中的皇后,拥有 “纯金色的嗓音”。埃齐奥$\cdot$平扎(Ezio Pinza, 1892$\sim$1957),意大利男低音歌唱家。比莉$\cdot$哈乐黛(Billie Holiday),美国1930$\sim$1940年代爵土乐女天后。迪琵雅芙(Edith Piaf),法国国宝级女歌手。}(对年轻读者来说,也许可以加上滚石乐队,或者感恩而死乐队、艾拉妮丝 $\cdot$ 莫莉塞特\footnote{滚石乐队(Rolling Stone)组建于1965年的英国摇滚乐队,是摇滚音乐史上最成功、最长寿的乐队之一。感恩而死乐队(Gratcftil Dead),组建于1964年 ,迷幻摇滚的代表。艾拉妮丝 $\cdot$莫莉塞特(Alanis Morrisette),生于1974年,加拿大人,美国红极一时的女歌手,90年代中后期长期保持唱片销景冠军。})或者差不多数百名歌手中的任何一个。他们的声音就藏在我的书架上,按几个键就可以召集他们出来歌唱。
\end{quotation}
\switchcolumn*
We should not lose sight of the universal poverty and back-breaking labor that free markets have eliminated. But by contemporary standards, of course, millions of Americans do live in
a poverty that is less material than spiritually deadening---
marked by a feeling of hopelessness. So the second answer to the question is that government should stop trapping people in
poverty and making it difficult for them to escape. Taxes and
regulations eliminate jobs, especially for the least skilled, and
the welfare system makes possible unwed motherhood and
long-term dependency. A third answer is mutual aid: people
banding together not to help the less fortunate but to help
themselves through times of trouble. I will deal with economic
growth, welfare, and charity in subsequent chapters, but here I
want to focus on mutual aid.
\switchcolumn
我们不应该对自由市场已经消除普遍贫困和艰辛劳动的成
果视而不见。当然,用当代的标准,数百万美国人的确生活在
贫困当中,不过与其说是物质上的贫困,不如说是精神上的隔
离 --- 标志是感觉不到希望。因此对这个问题的第二个答案
是:政府应该停止实施那些让人们陷入贫困,并难以自拔的政
策。征税和管制消灭了工作机会,尤其是非熟练工作的机会,
福利体制让政府成了妈妈,助长了长期依赖。第三个答案是互
助:人们联系在一起不是为了帮助不幸的人,而是帮助他们自
己解决所遇到的麻烦。在后面的几章里我将讨论经济增长、福
利和慈善的问题,但是这里我想集中谈谈互助。
\switchcolumn*
Mutual aid has a long history---and not just in the West, by
any means. The early craft guilds, before they became the stultified monopolies known to every student of medieval history,
were mutual-aid associations of people in the same trade. In the
African custom of susu, people would contribute a certain
amount into a pot, and when the fund reached a certain
amount, members took turns collecting it. As the Ghanaian
economist George Ayittey writes, ``Were the 'primitive' susu
system introduced in America it would be called a \textit{credit union}.''
Or if introduced by Korean Americans it might be called the
\textit{keh}, a group of people who get together once a month for dinner, socializing, advice, and the contribution of money to a
common pot to be given each month to one participant.
\switchcolumn
互助有很长的历史,而且从任何角度看,也不仅仅存在于西方。早期的手工业行会,在他们变成每个学生学中世纪历史时所熟悉的愚蠢的垄断组织之前,就是同行业的人们组成的互助组织。在非洲苏苏(susu)人的习俗当中,人们会将一定数量的捐献放在一起。当这笔钱累积到一定数量的时候,每个人轮流把钱拿走。加 纳 经 济 学 家 阿 伊 特 (George  Ayittey)说 :“如果原始的苏苏人体制引进到美国,现在它应该被叫 做信用联盟 。” 或者,如果被韩裔美国人引进,它就会被称为 keh,指的是一群人每个月聚会一次,吃饭、社交、建议以及捐钱给一个公用的罐子,并且每个月把钱给一位成员。
\switchcolumn*
The historian Judith M. Bennett wrote in the February 1992
issue of \textit{Past and Present} about the ``ales'' of medieval and early
modern England at which people would gather for drinking,
dancing, and games, paying above-market prices to help out a
neighbor: church-ales, to raise money for the parish; bride-ales,
to get a marrying couple started; and help-ales, to assist those
who had fallen on hard times. Bennett calls the ales an example
of how ordinary people ``looked not only to the 'better sort' for
relief, but also to each other,'' a ``social institution through
which neighbours and friends assisted each other in times of crisis or need.'' The ales reaffirmed social solidarity among working people. They usually required the active efforts of the
person in need, and contributions depended on the degree to
which the person was judged deserving. Unlike charity, ales involved a relationship among equals: ``By merging alms-giving
with both conviviality and commerce, charity-ales minimized the potential social divisiveness of poverty and charity.'' There
was also a sense of reciprocity among ``people who could reasonably expect that they would both contribute to and benefit
from charity ales during the course of their lives.''
\switchcolumn
历史学家贝内特(Judith M. Bennett) 1992年 2 月写的文
章 “过去与现在” 中提到中世纪和现代早期英国的“啤酒宴”(ales)。在宴会上人们聚在一起喝酒、跳舞、玩游戏,用超过
市场价的钱支付啤酒宴的费用、现场购买商品,以此来帮助一
个邻居脱离困境:教会啤酒宴是为了给教区穷人筹集善款;婚
礼啤酒宴,则是为了凑钱让新婚夫妇开始家庭生活;互助啤酒
宴,是为了帮助那些陷入困境的人。贝内特认为, “啤酒宴”
是一个很好的例子,说明普通人是如何“不仅仅看重救济的
正面作用,而且也看重相互之间的社交关系”。这 是 一 种 “社
会惯例,通过这种惯例在发生危机和需要的时候,邻居和朋友
之间能够互相帮助”。啤酒宴再次证明了人们之间存在的社会
团结。这种啤酒宴通常要求需要救助的人进行主动积极的活
动,例如筹备宴会等,这种制度依赖的是人们之间的相互信
任。与慈善不同的是,啤酒宴包含的是一种平等的关系。 “把
募捐活动和宴会游乐以及商业活动结合在一起,这种慈善宴会
把需要救助的人因贫穷和慈善活动可能造成的尴尬和不快降到
最低限度。” 而且人们认为这种活动是互惠的, “人们会合理
地相信在付出捐款之后,一旦自己今后有需要,也能从这种慈
善宴会当中得到救助。”
\switchcolumn*
A more modern example of mutual aid---which until recently has gone virtually unnoticed by historians who study
poverty, charity, and welfare---is the role of fraternal and
friendly societies. David Green of London's Institute of Economic Affairs describes the ways that British manual workers
formed ``friendly societies,'' which were self-governing mutual-benefit associations. Individuals joined and contributed to the
group, pledging to help each other in times of trouble. Because
they were mutual associations, the payments received---sick
pay, medical care, burial expenses, and survivor's benefits---
were ``not a matter of largesse but entitlement, earned by the
regular contributions paid into the common fund by every
member and justified by the obligation to do the same for other
members.'' Some societies were just neighborhood clubs, but
others evolved into national federations with hundreds of thousands of members and extensive investments. By 1801 it is estimated that there were 7,200 societies in Britain with 648,000
adult male members, out of a total population of 9 million. By
1911 there were 9 million people covered by voluntary insurance associations, more than two-thirds of them in friendly societies. They had names such as the Manchester Unity of
Oddfellows, the Ancient Order of Foresters, and the Working-
men's Conservative Friendly Society.
\switchcolumn
关于互助的另一个更晚近的例子是美国的共济会(fraternal societies) 和英国的互助会(friendly societies) --- 而直到
最近这些组织也没有进入研究贫困、慈善和福利的历史学家们
的视野。伦敦经济学院的大卫$\cdot$ 格 林(David Green) 对英国
体力劳动工人的自治互惠组织“互助会” 的建立方式进行了
介绍。个人加入并且把钱捐给互助会,以便会员在遇到困难的
时候从中得到帮助。由于这是互助组织,接 受 的 钱 就 “不是
一种慈善捐款,而是他的权利。这种权利是每个会员通过平时
对共同基金的捐献而获得的,是通过对其他会员承担同样的责
任而获得的正当权利”。有的互助会只不过是邻居间的俱乐
部,但是有的组织扩大成了全国性的联盟,拥有数十万的成
员,并且有数额庞大的对外投资。1801年,据估计英国有
7200个互助会,在当时的900万总人口当中有648000名成年
男性是互助会员。而到了 1911年,巳经有900万英国人参加
了这种自愿保险组织,其中互助会会员超过三分之二。互助会
都有自己的名字,例如曼彻斯特共济联合会、森林人旧秩序互
助会和从业者保守互助会等。
\switchcolumn*
The friendly societies had an important economic purpose---
to jointly insure against sickness, old age, and death. But they
served other purposes as well, such as fellowship, entertainment, and enlargement of one's network of contacts. More important, members of the society felt bound together by
common ideals. A central purpose was the promotion of good
character. They understood that developing good habits is not
easy; many of us find it helpful to have external support for our
good intentions. Churches and synagogues provide that for
many people; Alcoholics Anonymous provides it for a particular
aspect of good character, sobriety. Another benefit of the friendly societies was that working people got experience in
running an organization, a rare opportunity in Britain's class-based society.
\switchcolumn
互助会都有一个重要的经济目的 --- 对疾病、衰老和死亡
进行联合保险。但是他们同时也有其他的功能,如发展友谊、
娱乐和扩大个人的社交圈等。更重要的是,互助会的成员感觉
自己和其他人通过共同的理念联系在了一起。互助会的一个中
心目标就是促进会员优良品质的形成和发展。他们明白,好习
惯的形成并不容易;很多人都发现,来自外部的支持对于好品
质的形成是相当有帮助的。基督教会和犹太教会给很多人提供
了这样的支持,而戒酒互助协会对培养某个方面的好品质给人
们提供了支持,这就是戒酒。互助会的另一个好处是劳动者能
够从中获得运作一个组织的经验,在英国这样以等级为基础的社会当中,这种机会是很少的。
\switchcolumn*
The historian David Beito has done similar pioneering research on American fraternal societies such as the Masons, Elks,
Odd Fellows, and Knights of Pythias. Beito writes, ``Only
churches rivaled fraternal societies as institutional providers of
social welfare before the advent of the welfare state. In 1920,
about eighteen million Americans belonged to fraternal societies, i.e. nearly 30 percent of all adults.'' A 1910 article in
\textit{Everybody's Magazine} explained, ``Rich men insure in the big
companies to create an estate, poor men insure in the fraternal
orders to create bread and meat. It is an insurance against want,
the poorhouse, charity and degradation.'' Note the aversion to
charity: people joined fraternal societies so that they could mutually provide for their own needs in time of misfortune and not
be forced to the indignity of taking chanty from others.
\switchcolumn
历史学家贝托(David Beito) 则对美国共济会组织进行了类似的探索研究,如梅森共济会、埃尔克斯共济会、怪人共济会 、皮西厄斯骑士共济会等。贝托说: “在福利国家出现之前,作为社会福利的制度性提供者,只有教会才能和共济会相提并论。到 1920年 ,大约1800万美国人,相当于近30\% 的成年 人 加 入 了 各 种 共 济 会 。” 1910年 的 《人人杂志》(\textit{Everybodys Magezine})中有一篇文章对此进行了解释: “富人把保险投进大公司是为了留下财产,穷人把保险投进共济会是为了留下面包和肉。这是一种为生存需要而投的保险,是为了避免降低尊严找救济所、寻求施舍而投的保险。”这里我们注意到人们对慈善的反感。人们加人共济会是为了在发生不幸的时候能够彼此提供帮助,好让自己不至于被迫羞耻地从别人那里得到施舍。
\switchcolumn*
At first, fraternal insurance protection centered around the
death benefit. By the early twentieth century, many orders were
also offering sickness or accident insurance. An interesting aspect of fraternal insurance is how it overcomes the problem of
moral hazard, the risk that people will take advantage of the insurance system. When dealing with a government agency or
distant insurance company, an individual may be tempted to
malinger, to claim exaggerated benefits for minor or nonexistent problems. But the feeling of community with other members of the fraternal order and the desire to have the approval of
one's peers reduce the temptation to cheat. Beito suggests that
that is why fraternal societies ``continued to dominate the sickness insurance market long after they had lost their competitive
edge in life insurance''---where malingering is a bit more problematic. By 1910 fraternal health insurance often included
treatment by a ``lodge doctor,'' who contracted to provide medical care to all the members for a fixed price.
\switchcolumn
最开始,共济会的互助保险主要是针对死亡而提供的保
险。到了 20世纪早期,很多共济会开始提供疾病或意外事故
保险。关于共济会互助保险最有意思的一个问题是,它们是如
何克服道德风险问题的?所谓道德风险就是人们可能会从保险
系统当中骗取保费。人们在面对政府福利机构或者一个遥远的
保险公司的时候,可能会有动机通过装病、夸大一个小问题或
者提出一个根本不存在的问题而要求得到超额的受益金,但是
共济会成员之间共同体的感觉,以及希望得到别人承认的愿望
降低了欺诈骗保的诱惑。贝托认为,这也是为什么共济会互助
保 险 “在人寿保险市场丧失了竞争优势之后很长时间,仍然
继续占据疾病保险市场的最大份额”。 --- 因为在共济会里面
装病有些困难。到 1910年,共济会互助健康保险通常还把
“共济会签约医生”的治疗费用包括在内。所谓共济会签约医生,就是与共济会签约,以固定的价格为共济会成员提供医疗
服务的医生。
\switchcolumn*
Immigrants formed many fraternal societies, such as the National Slovak Society, the Croatian Fraternal Union, the Polish
Falcons of America, and the United Societies of the U.S.A. for
Russian Slovaks. Jewish groups included the Arbeiter Ring
(Workmen's Circle), the American-Hebrew Alliance, the National Council of Jewish Women, the Hebrew Immigrants Aid
Society, and more. By 1918 there were more than 150,000
members in the largest Czech-American associations. Spring-field, Illinois, with a total Italian population of 3,000 in 1910,
had a dozen Italian societies.
\switchcolumn
移民建立了很多共济会,如全美斯洛伐克人共济会、克罗
地亚人共济联盟、美国波兰人猎鹰会,以及全美俄罗斯斯洛伐
克人共济会联盟等。犹太人的社团则包括阿尔贝特圈子、美国
希伯来人联盟、全美犹太妇女公会、希伯来移民互助会等。
1918年,最大的捷克裔美国人组织有15万会员。伊利诺伊州
的春田市在1910年尽管只有3000意大利人,却 拥 有 10多个
意大利人共济会。
\switchcolumn*
In his landmark 1944 study \textit{An American Dilemma}, the
Swedish economist Gunnar Myrdal asserted that African Americans of all classes were even more likely than whites to join fraternal orders such as the Prince Hall Masons, the True
Reformers, the Grand United Order of Galilean Fishermen, and
parallel versions of the Elks, the Odd Fellows, and the Knights
of Pythias. He estimated that over 4,000 associations in
Chicago were formed by the city's 275,000 blacks. In 1910 the
sociologist Howard W. Odum estimated that in the South, the
``total membership of the negro societies, paying and nonpaying, is nearly equal to the total church membership.'' Fraternal
societies, he said, were ``a vital part'' of black ``community life,
often its center.''
\switchcolumn
在其代表作1944年的研究报告《美国的困境》(\textit{Anmerican Dilemma})中,瑞典经济学家缪尔达尔\footnote{缪尔达尔(Gunnar Myrdal,1898$\sim$1987),瑞典经济学家,由于在货币和经济波动理论方面的贡献以及对经济社会和制度现象的内在依赖性进行的分析,1974年和哈耶克一起荣获诺贝尔经济学奖。}认为,美国各个阶层的黑人都比白人更倾向加入共济会,如王子堂梅森共济会 、真正宗教改革者共济会、加利利渔夫共济会大联盟,以及与埃尔克斯共济会、怪人共济会、皮西厄斯骑士共济会类似的一些共济会。他估计,芝加哥有超过4000个共济会是由黑人建立的,有 275000黑人会员。1910年,社会学家奥顿(How­ard W.Odmn)估计在美国南部, “黑人共济会的成员总数,包括交费的和不交费的,几乎与基督徒的人数相等。” 他说,共济 会是 “黑人社区生活至关重要的部分,而且常常是黑人社区生活的中心”。
\switchcolumn*
Like the British societies, American fraternal societies emphasized a code of ethics and each member's mutual obligations
to the other members. The historian Don H. Doyle, in \textit{The Social Order of a Frontier Community}, found that the small town of
Jacksonville, Illinois, had ``dozens $\ldots$ of fraternal lodges, reform societies, literary clubs, and fire companies'' and that the
lodges enforced ``a broad moral discipline affecting personal behavior in general and temperance in particular, matters closely
tied to the all-important problem of obtaining credit.''
\switchcolumn
和英国互助会一样,美国共济会强调会员之间的相互责
任,以此作为共济会的道德准则。历 史 学 家 唐 $\cdot$ 都乐(Don H. Doyle)在 《一个新建社区的社会秩序》(\textit{The Social Order of a Frontier Community}) 一书中指出,在伊利诺伊州小镇杰克逊维尔有“数十个共济会、革新会、文学倶乐部以及消防
组织”,而共济会则起到了 “总体上促进广泛的道德准则,影
响人们的行为,让他们生活有所节制,使他们在与个人信誉有
关的所有行为上都更加检点”。
\switchcolumn*
Fellowship and solidarity discouraged members from claiming benefits without good cause, but the societies also had rules
and practices to ensure adherence to them. The rules of the socialist-oriented Western Miners' Federation denied benefits to
members when ``the sickness or accident was caused by intemperance, imprudence or immoral conduct.'' The Sojourna Lodge
of the House of Ruth, the largest black women's voluntary organization in the early part of the century, required members to
present a notarized medical certificate from a doctor in order to
claim sickness benefits and also had a committee on sickness to
both support and investigate sick members.
\switchcolumn
虽然友谊和团结会让会员们打消没有正当理由就要求共济
会保险受益的念头,但是这些共济会也有规章和实践来确保他
们遵守这些原则。社会主义倾向的西部矿工联合会的规章规
定: “因为酗酒、鲁莽的行为以及不道德行为而造成的疾病和
事故” 不能得到受益金。露丝房子寄居者共济会是20世纪早
期最大的黑人妇女自愿者组织,要求会员在申请医疗补偿时提
供一份从医生那里得到的经过公证的医疗证明,还有一个疾病
保险委员会对患病的会员提供医疗支持或者对患病会员进行
调查。
\switchcolumn*
Fraternal associations also helped people cope with the increasing mobility of society. Some of the multi-branch British societies provided members with places to stay when they went to
other towns to look for work. Doyle found that ``for the transient member, a transfer card from the Odd Fellows or Masons
was more than a ticket of readmission to another lodge. It was
also portable certification of the status and reputation he had
established in his former community, and it gave him access to
a whole new network of business and social contacts.''
\switchcolumn
共济会也会帮助人们适应流动性不断增强的社会。在各地
都有分支机构的一些英国互助会给那些到外地找工作的会员提
供住宿。都乐发现,“对短期成员来说,一张来自怪人共济会
或者梅森共济会的转移卡不仅是一张重新入会的门票,它也是
一个证明,证明他在以前居住的社区所获得的地位与声望,它
也是他进入一个全新的商务和社交网络的门票。”
\switchcolumn*
Critics frequently assert that libertarian solutions to social
problems are fantastical. ``Eliminate the government's safety
net, and just \textit{hope} that churches, charities, and mutual-aid
groups will expand to fill the gap?'' The answer is twofold. Yes,
these groups will step up to the plate; they always have. But
more important, the \textit{existence} of the government's safety net and
the massive taxes that support it have squeezed out those efforts. The forms that mutual aid takes are countless, from play
groups and supper clubs to trade associations to neighborhood
crime watches. They have declined dramatically not because of
women entering the workforce, or because of television's hold
on our free time, but because of government's expansion.
\switchcolumn
批评者们常常断言,古典自由主义的社会问题解决路径是
一个\textbf{空想}。“废除政府的社会保障网,仅仅寄希望于教会、慈
善和互助组织,能够填上社会贫富的鸿沟吗?”要从两个方面
来回答这个问题。是的,这些社会组织将会达到这个目标,他
们以前就做到过。但更重要的是,政府的社会保障网的\textbf{存在}以
及支持这个网络的庞大税收阻碍了这些社会组织的努力。民间
互助的形式是无穷的,从亲子游乐坊、晚餐倶乐部到商会、街坊犯罪联防组织,不一而足。这些社会组织的戏剧性衰落不是因为妇女进入劳动力市场,也不是因为电视占领了我们的空闲
时间,而是因为政府的扩张。

\switchcolumn*[\section{Government and Civil Society\\政府与市民社会}]
Government's protection of individual rights is vital for creating a space in which people can pursue their many and varied
interests in voluntary association with others. When government expands beyond that role, however, it pushes into the
realm of civil society. As government borrowing ``crowds out''
private borrowing, government activity in any field crowds out
voluntary (including commercial) activity.
\switchcolumn
政府对个人权利的保护对于创造一个空间,在其中人们能
够通过与其他人自愿结成的组织追求彼此互不相同的利益是至
关重要的。然而,当政府的扩张超出了这个目标,它就强行闯
入了市民社会的领域。当 国 债 “挤出” 私人借贷的时候,政
府在任何领域的行为都会对自愿的行为包括商业行为产生挤出
效应\footnote{挤出效应(crowds-out),是指政府支出增加所引起的降低私人消费或投资的作用。}。
\switchcolumn*
From the Progressive Era on, the state has increasingly disrupted natural communities and mediating institutions in
America. Public schools replaced private community schools,
and large, distant, unmanageable school districts replaced
smaller districts. Social Security not only took away the need to
save for one's own retirement but weakened family bonds by reducing parents' reliance on their children. Zoning laws reduced
the availability of affordable housing, limited opportunities for extended families to live together, and removed retail stores
from residential neighborhoods, reducing community interaction. Day-care regulations limited home day care. In all these
ways, civil society was crowded out by the state.
\switchcolumn
从 “进步时代”\footnote{进步时代(Progressive Era),是指 1890$\sim$1913年 ,美国自由派引人了集体主义的 “进步主义” 理念,开始大幅度扩张政府领域,开展政府工程和福利工程 ,侵人市民社会和商业社会的领域。}以来,政府就对美国自发形成的社会组织和中间机构不断进行瓦解。公共学校逐渐代替私立社区学校,规模巨大、距离遥远、无法管理的学区划分代替了小的学区。社会保障系统不仅代替了个人对自己的退休生活进行储蓄的需要,而且通过降低父母(退休后)对子女的依靠而削弱了家庭的联系纽带。区域法(zoning  law)减少了房屋供应,
限制了大家庭共同生活的可能性,让零售商店从居民区中消
失 ,减少了社区的社交生活。托儿条例则限制了家庭的婴儿照
看。通过这些政策,市民社会就被政府“挤出” 了。
\switchcolumn*
What happens to communities as the state expands? The
welfare state takes over the responsibilities of individuals and
communities and in the process takes away much of what
brings satisfaction to life: If government is supposed to feed the
poor, then local charities aren't needed. If a central bureaucracy
downtown manages the schools, then parents' organizations are
less important. If government agencies manage the community
center, teach children about sex, and care for the elderly, then
families and neighborhood associations feel less needed.
\switchcolumn
那么政府的扩张对社会组织都产生了什么样的影响?福利国家替代了个人责任和社会组织的作用,在这个过程中还拿走
了很多给生活带来满足的东西:如果政府被认为理应养活穷
人 ,那么当地的慈善机构就不再需要。如果一个在城里的中央
官僚机构管理学校,那么家长间的组织就不那么重要了。如果
政府机构管理社区中心,教孩子们性知识,照顾老人,那么家
庭和邻居街坊间的组织显然就不需要了。

\switchcolumn*[\subsection{From Charity and Mutual Aid to Welfare State\\从慈善互助到福利国家}]
Charity and mutual aid have particularly been squeezed by the
expansion of the state. Judith Bennett notes that as early as the
thirteenth century, ``ecclesiastical and royal officials had ordered
the elimination of scot-ales.'' By the seventeenth century, the
opposition was more serious because of a general campaign
against traditional culture, a move toward more centralized
control of charity, and the development of tax-funded support
for the poor.
\switchcolumn
慈善和互助已经在很大程度上被政府的扩张所排挤出人们
的生活。贝内特提到,早在13世纪,“教会和王室的官员们就
明令禁止了苏格兰啤酒募捐会”。到 17世纪,情况更加严重,
因为发生了针对传统文化的运动,这场运动的目的是建立起对
慈善事业的中央控制,试图用强制性税收来资助穷人。
\switchcolumn*
During the above discussion on fraternal societies, readers
may have wondered: if they were so great, where are they now?
Many of them are still around, of course, but they have fewer
members and less stature in society, at least partly because the
state took over their functions. David Green writes, ``It was at
the height of their expansion that the state intervened and
transformed the friendly societies by introducing compulsory
national [health] insurance.'' Their major function nationalized,
the societies atrophied. Beito found that American fraternal insurance was impeded by medical licensing laws that undermined the lodge-doctor arrangement, by legal prohibitions on
certain forms of insurance, and by the rise of the welfare state.
As the states and the federal government created workers' compensation, mothers' pensions, and Social Security, the need for
mutual aid societies diminished. Some of that impact may have
been unintentional, but President Theodore Roosevelt objected to the immigrant fraternal societies, saying, ``The American
people should itself [note the collective pronoun] do these
things for the immigrants.'' Even the historian Michael Katz, a
supporter of the welfare state, concedes that federal welfare initiatives ``may have weakened [these] networks of support
within inner cities, transforming the experience of poverty and
fueling the rise of homelessness.''
\switchcolumn
通过前面对共济会组织的讨论,读者也许会有疑问:如果
这些组织境况不佳,那么现在它们在哪里呢?当然,很多共济
会组织仍然存在于我们周围,只不过他们的成员减少了很多,
在社会中的地位也大大下降,至少一部分是因为政府代替了它
们的功能。大 卫 $\cdot$ 格林提到:“就在它们扩张的黄金时期,政
府介入了进来,通过引入强制的国家保险改变了这些互助
会。” 它们的主要功能被国有化,组织大大萎缩。贝托发现,
美国共济保险的停止发展,一是因为医疗执照法破坏了共济会
签约医生制度,二是因为某些保险形式被明令禁止,三是因为
国家福利的崛起。因为州和联邦政府建立了工人退休金制度、
母亲生育补贴制度和社会保障制度,对互助组织的需求就消失
了。这些影响也许不是有意为之的,除了西奥多$\cdot$罗斯福总统
对移民共济会的反对之外,他说:“美国人民自己(请注意这
里用的是集体名词‘ 人民’ )可以为移民做这些事。”甚至福利国家的支持者历史学家迈克尔$\cdot$ 卡 兹 (Michael Katz) 也承
认,联邦福利的启动“也许会在城市中心削弱这些支持网络,
改变贫困的体验,使无家可归者人数增加”。
\switchcolumn*
The government is still squeezing out charitable institutions.
The Salvation Army operates twenty homeless shelters in Detroit, but in 1995 the city of Detroit passed a law to license and
regulate homeless shelters. The law required that all staffers be
trained, that all menus be approved by a registered dietitian,
that all medication be kept in a locked storage area, that the
shelter ascertain the ages of the people in the shelter and make
sure that the children attend school. All fine ideas, but the Salvation Army official in charge of the shelters says, ``All these requirements cost money, and our budget is \$10 a day per
person.'' What will happen? Some shelters will probably close,
and either the homeless will live in abandoned buildings and
cardboard boxes or there will be pressure for Detroit to spend
even more money to build city-run shelters. And Salvation
Army volunteers will have one less opportunity to help.
\switchcolumn
现在,政府仍然在继续排挤慈善组织。救世军在底特律有
20个无家可归者救助站,但 是 1995年底特律市通过了一部法
律,要求无家可归者救助站必须获得执照,并且对救助站进行
了管制。这部法律要求救助站的所有工作人员都必须经过培
训 ,所有用餐菜单必须经过注册营养师的批准,所有药品都应
该保存在封闭的储藏区,救助站必须查明站内被救助者的年
龄,并确保每个儿童都能上学。这都是很好的想法,但是管理
这些救助站的救世军管理人员说:“所有这些要求都是要花钱
的,而我们的预算是每人每天10美元。”那么将会发生什么
呢? 一些救助站可能会关门,然后或者是无家可归者住进废弃
的建筑和纸壳箱子当中,或者是底特律市政府迫于压力,花更
多的钱来建造由市政府运营的救助站。而救世军的志愿者们对
此几乎无能为力。
\switchcolumn*
Texas bureaucrats demand that a successful drug-treatment
program called Teen Challenge comply with state regulations
on record keeping, shelter-maintenance standards, and especially the use of licensed counselors instead of its religiously
based program, often run by ex-alcoholics and reformed addicts. Teen Challenge does not take government grants, and a
Department of Health and Human Services study found it to
be both the best and the cheapest of drug-treatment programs
examined. But in 1995 the state of Texas ordered the South
Texas program to shut down or pay a fine of \$4,000 a day. Teen
Challenge took the bureaucrats to court, at the very least diverting some of its scarce time and money to fighting for permission to stay open.
\switchcolumn
德克萨斯州的官员们要求一个成功的戒毒治疗项目“少年挑战”(Teen  Challenge)遵守州政府的规定,如保存治疗记
录、严格遵守救助站维护标准,尤其是要求必须使用有执照的
辅导员来代替它以宗教为基础的戒毒计划,而这个计划原本是
使用已戒酒的原酗酒者和已戒毒的原吸毒上瘾者。 “少年挑
战” 计划并没有得到政府的资助,而一份卫生与人类服务部
的报告发现,这个项目是现有戒毒项目中效果最好的,而且成
本最低。但是德克萨斯州在1995年 要 求 “少年挑战”在德州
南部的项目关闭,否则就将课以每天4000美元的罚款。“少年
挑战” 打算将这些德州官员告上了法庭,想要至少用它的宝
贵时间和金钱为争取这个项目继续存在进行而抗争。
\switchcolumn*
What is the cost to our society of having government take
over more and more roles that individuals and communities
used to serve? Tocqueville warned us of what might happen:
\switchcolumn
政府越来越多地取代个人和社会组织的作用,我们的社会
将会为此付出什么样的代价?托克维尔曾经警告过我们将会发
生什么:
\switchcolumn*
\begin{quote}
After having thus successively taken each member of the community in its powerful grasp and fashioned them at will, the
supreme power then extends its arm over the whole community.
It covers the surface of society with a network of small complicated rules, minute and uniform, through which the most original minds and the most energetic characters cannot penetrate,
to rise above the crowd. The will of man is not shattered, but
softened, bent, and guided: men are seldom forced by it to act,
but they are constantly restrained from acting: such a power
does not destroy, but it prevents existence; it does not tyrannize,
but it compresses, enervates, extinguishes, and stupefies a people, till each nation is reduced to nothing better than a flock of
timid and industrious animals, of which the government is the
shepherd.
\end{quote}
\switchcolumn
\begin{quote}
在相继从每个成员那里把共同体拿走,紧紧攥在手
里,并且按照他的意愿对其进行改造之后,最高权力就会
把他的手伸到整个共同体。它会用小而复杂的规则网络覆
盖整个社会,细致而统一,即便是最有创造力的头脑和最
有活力的人物也不能穿透这个网络不能从芸芸众生之中脱
颖而出。人们的意愿并没有被驱散,而是被软化、弯曲和
制式化:人们很少被强迫去做什么事情,但是权力不断地
对他们的行动进行限制:这样的权力不会去摧毁什么东
西,但是它能够阻止一些东西的出现;它并不专制,但是
它压缩、削弱人们,麻醉人们,消灭人们的希望,直到每
个国家的人民变成一群胆小怕事的工业化动物。而对这些
人来说,政府就是牧羊人。
\end{quote}
\switchcolumn*
As Charles Murray puts it, ``When the government takes
away a core function [of communities], it depletes not only the
source of vitality pertaining to that particular function, but also
the vitality of a much larger family of responses.'' The attitude
of ``let the government take care of it'' becomes a habit.
\switchcolumn
正如查尔斯$\cdot$ 穆 雷(Charles  Murray)所说:“当政府夺走
社会组织的核心功能时,它耗尽的不仅仅是与这些功能有关的
生命力之源,而且是更大范围的生命力。” “让政府来管这事”
就成了人们的习惯。
\switchcolumn*
In his book \textit{In Pursuit: Of Happiness and Good Government},
Murray reported some evidence that relying on government
does indeed substitute for private action. He found that from
the 1940s to 1964, the percentage of American income given to
philanthropic causes rose---as we might expect, given that in-
comes were rising and people probably felt able to do more for
others. ``Then, suddenly, sometime during 1964---65, in the
middle of an economic boom, this consistent trend was reversed.'' Although incomes continued to grow (the great slow-down in economic growth didn't begin until about 1973), the
percentage of income given to philanthropy fell. Then in 1981,
during a recession, the trend suddenly reversed itself, and contributions as a percentage of income rose sharply. What happened? Murray suggests that when the Great Society began in
1964-65, with President Lyndon Johnson proclaiming that the
federal government would launch a War on Poverty, maybe
people figured their own contributions weren't needed so
much. Then in 1981 President Ronald Reagan came into office,
promising to cut back government spending; maybe then people figured that if the government wasn't going to help the
poor, they'd better.
\switchcolumn
在 《追求:幸福和好政府》(\textit{In Pursuit: Of Happiness and Good Government}) 一书当中,穆雷提出了一些证据证明对政府的依赖的确替代了个人行为。他发现,1940$\sim$1964年,美
国人收入用于慈善事业的比例上升了 --- 正如我们所预料的那
样 ,如果收入增加了,人们会感到能够为他人做更多事情了。
“然后突然地,大约在1964$\sim$1965年,在一次经济增长周期的
中间,这个持续上升的趋势转向了。”尽管总收入在继续增长(直到 1973年之前,美国经济并没有发生增长速度的显著下
滑),但收入用于慈善事业的比例下降了。然后到了 1981年,
在经济衰退期,这个趋势突然又发生了转向,美国人慈善奉献
占收入的比例突然直线上升。发生了什么事情?穆雷解释道,
1964$\sim$1965年 是 美 国 “伟大社会” (Great  Society) 计划的开
始,当时林登$\cdot$ 约翰逊总统宣布,联邦政府将发动一场对贫穷
的战争,也许人们把这理解为他们不需要再做那么多的慈善奉
献了。而 1981年则是里根总统开始其总统任期,他承诺削减
政府开支。也许是当时人们的理解是:如果政府不打算帮助穷
人 ,那最好他们自己来做。

\switchcolumn*[\subsection{The Formation of Character\\个人品格的构成}]
Expansive government destroys more than institutions and
charitable contributions; it also undermines the moral character
necessary to both civil society and liberty under law. The ``bourgeois virtues'' of work, thrift, sobriety, prudence, fidelity, self-reliance, and a concern for one's reputation developed and
endured because they are the virtues necessary for advancement
in a world where food and shelter must be produced and people
are responsible for their own flourishing. Government can't do
much to instill those virtues in people, but it can do much to
undermine them. As David Frum writes in \textit{Dead Right},
\switchcolumn
政府扩张摧毁的不仅仅是社会组织和慈善事业,它还破坏
了对市民社会和法治下的自由都必不可少的道德品格。中产阶
级的美德包括:工作、节俭、节制、谨慎、忠实、自力更生、
注重个人名誉。这些美德之所以应该发展和保持,是因为在一
个必须通过生产才能提供食物和住房、人们必须对自己的发展
负责的世界里,它们是世界取得进步所必需的。政府对把这些
美德灌输给人们的工作是无能为力的,但是它能够很轻易地毁
坏它们。大卫$\cdot$ 弗拉姆(David From)在《绝对正确》(\textit{Dead Right})中这么说:
\switchcolumn*
\begin{quote}
Why be thrifty when your old age and health care are provided
for, no matter how profligately you acted in your youth? Why be
prudent when the state insures your bank deposits, replaces your
flooded-out house, buys all the wheat you can grow, and rescues
you when you stray into a foreign battle zone? Why be diligent
when half your earnings are taken from you and given to the
idle? Why be sober when the taxpayers run clinics to cure you of
your drug habit as soon as it no longer amuses you?
\end{quote}
\switchcolumn
\begin{quote}
如果无论你年轻时多么挥霍无度,你年老时的保障和
医疗保障都有人提供,你为什么要节俭呢?如果国家确保
你的银行存款,给你重建因洪水冲垮的房屋,买下你种出
来的所有麦子,以及当你误入外国战争地带时一定会去营
救你,你为什么要谨慎呢?如果你一半的收入被拿走交给
一些懒虫的时候,你为什么要勤奋呢?如果纳税人支持的人开诊所,当毒品一旦不再让你快乐的时候就迅速赶来治
疗你的吸毒习惯,你为什么要节制呢?
\end{quote}
\switchcolumn*
Frum sums up government's impact on individual character as
``the emancipation of the individual from the restrictions imposed on it by limited resources, or religious dread, or community disapproval, or the risk of disease or personal catastrophe.''
Now one might suppose that the very aim of libertarianism is
the emancipation of the individual, and so it is---but the emancipation of the individual from artificial, coercive restraints on
his actions. Libertarians never suggested that people be ``emancipated'' from the reality of the world, from the obligation to
pay one's own way and to take responsibility for the consequences of one's own actions. As a moral matter, individuals
must be free to make their own decisions and to succeed or fail
according to their own choices. As a practical matter, as Frum
points out, when we shield people from the consequences of
their actions, we get a society characterized not by thrift, sobriety, diligence, self-reliance, and prudence but by profligacy, intemperance, indolence, dependency, and indifference to consequences.
\switchcolumn
弗拉姆总结了政府对个人品格的影响,“将个人从因资源
有限、宗教敬畏、共同体的排斥、疾病或个人灾难的风险而产
生的限制当中解放出来。” 现在,你也许认为古典自由主义的
目标不正是个人的解放吗?的确如此,但是古典自由主义只是
认为,个人应当从人为强制的限制中解放出来。从来没有说人
们应当从真实世界中、从为他自己的选择所承担的责任、从承
担他行为的后果中“解放” 出来。在道德层面,个人应当能
够自由地做出自己的决定,并接受因其自由选择而成功或失败
的结果。在实际操作层面,如弗拉姆所指出的那样,当我们让
人们不再承担起行为后果的时候,我们就会得到一个不以人们
节制、节俭、勤奋、自力更生和谨慎为特征的社会,而是一个
以挥霍无度、放纵、懒惰、依赖以及对结果毫不在乎为特征的
社会。
\switchcolumn*
To return to the image with which we began chapter 4---
being able to get cash and rent cars around the world---the
human need for cooperation has helped to create vast and complex networks of trust, credit, and exchange. For such networks
to function, we need several things: a willingness on the part of
most people to cooperate with others and to keep their
promises, the freedom to refuse to do business with those who
refuse to live up to their commitments, a legal system that enforces the fulfillment of contracts, and a market economy that
allows us to produce and exchange goods and services on the
basis of secure property rights and individual consent. Such a
framework lets people develop a diverse and complicated civil
society that serves an incredible variety of needs.
\switchcolumn
回到第四章开头的场景 --- 人们能够在全世界范围内得到
现金和租车,人们对合作的需要有助于创造一个巨大而复杂的
信任、信用和交换网络。为了保证这样的网络能够有效运转,
我们需要这样几件东西:人们与他人合作和遵守自己承诺的愿
望,拒绝与那些不依靠自己所作所为的信用而生活的人做生意
的自由,一个保障合同执行的法律系统,一个让我们在保护财
产权和个人承诺的基础之上生产和交换产品与服务的市场经济
体系。这样的一个框架就能够让人们发展出一个多样化的复杂
的市民社会,满足人们五花八门的各种需要。

\end{paracol}