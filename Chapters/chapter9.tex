\chapter{WHAT BIG GOVERNMENT IS ALL ABOUT\\大政府到底是什么}
\begin{paracol}{2}
\hbadness5000

Government has an important role to play in a free society. It is supposed to protect our rights, creating a society in which people can live their lives and undertake projects
reasonably secure from the threat of murder, assault, theft, or
foreign invasion. By the standards of most governments in history, this is an extremely modest role. That's what made the
American Revolution so revolutionary. The Declaration of Independence proclaimed, ``To secure these rights, governments
are instituted among men.'' Not ``to make men moral.'' Not ``to
boost economic growth.'' Not ``to ensure everyone a decent
standard of living.'' Just the simple, revolutionary idea that
government's role was limited to securing our rights. But
imagine how much better off we would all be if our government did an adequate job at this simple, limited task.
\switchcolumn
政府在自由社会中扮演着重要的角色。它应当保护我们的
权利,建立起一种社会,在这种社会中,人们可以在不受谋
杀 、强奸、盗窃或者外国入侵的情况下生活和从事自己的事
业。用历史上绝大多数政府作为标准来衡量,这是一个相当内
敛的角色。这也正是美国革命的革命性所在。《独立宣言》宣
告:“为保护上述权利,政府在人民之中建立” ,而 不 是 “为
促进人民道德水平” ,更 不 是 “为促进经济增长”,也不是
“为确保人民生活达到小康水平”。就这么简单,但却是革命
性的理念:政府的角色仅限于保护我们的权利。但是请设想一
下,如果政府真的把这个简单而有限的任务不折不扣地做好
了,我们现在的状况该有多好。
\switchcolumn*
Unfortunately, most governments fail to live up to Thomas
Jefferson's vision in two ways. First, they don't do a good job of
swiftly and surely apprehending and punishing those who violate our rights. Second, they seek to aggrandize themselves by
taking on more and more power, intruding themselves into
more aspects of our lives, demanding more of our money, and
depriving us of our liberty.
\switchcolumn
很不幸,绝大多数政府都在两个方面背离了杰斐逊的教
诲。其一,在保护我们权利的工作上,他们并没有做得很好,
并不总是能及时而无遗漏地发现和惩罚那些侵犯我们权利的
人。其二,他们不断攫取越来越多的权力,寻求扩大自己的权
势,越来越多地侵人我们生活的各个领域,从我们手里拿走越
来越多的钱,不断剥夺我们的自由。
\switchcolumn*
The most revolutionary aspect of the American Revolution
was that it sought to create from scratch a national government limited to very little more than protecting individual rights. During the Middle Ages, in England and other European countries, the idea of limits on government had grown.
Cities had written their own constitutional charters, and representative assemblies had sought to control kings through documents such as Magna Carta and the Golden Bull of Hungary.
Many of the American colonists---and some of their British
supporters such as Edmund Burke---saw the Revolution as a reclaiming of their rights as Englishmen. But the soaring words
of the Declaration and the strict rules of the Constitution went
further than any previous effort in declaring the natural rights
of life, liberty, and property and delegating to the new government only the powers necessary to protect those rights.
\switchcolumn
美国革命最具革命性的部分是它寻求建立一个把权力限制
在仅仅略多于保护个人权利的范围的全国性政府。中世纪,在
英国和其他欧洲国家,有限政府的理念就已经发展了出来。很
多城市与领主们签订了具有宪法性质的自治城市特许状,贵族或平民代表会议也在寻求通过诸如英国《大宪章》 和匈牙利
《金玺诏书》这样的文件来限制王权。很多北美殖民地人民
--- 以及一些英国国内的支持者,如埃德蒙$\cdot$ 伯克 --- 起初是
把北美殖民地的革命看成是对他们作为英国人的权利的重申。
但 是 《独立宣言》雄辩动人的语句和《美国宪法》制定的严
格规则远远超过了以前的目标,开始伸张生命、自由和财产的
自然权利,致力于建立一个权力仅限于必要的保护权利的新型
政府。
\switchcolumn*
We should distinguish at this point between ``government''
and ``state.'' Those two terms are sometimes used interchangeably, especially in American English, but they actually refer to
two very important but easily confused kinds of institutions. A
government is the consensual organization by which we adjudicate disputes, defend our rights, and provide for certain common needs. A condominium association, for example, has a
government to adjudicate disputes among owners, regulate the
use of common areas, make the residents secure from outside
intruders, and provide for other common needs. We can readily
see why people seek to have a government in this sense. In
every case, the residents agree to the terms of the government
(its constitution or charter or bylaws) and give their consent to
be governed by it. A state, on the other hand, is a coercive organization asserting or enjoying a monopoly over the use of
physical force in some geographic area and exercising power
over its subjects. The audacity and the genius of the American
Founders was to attempt to create a government that would
not be a state.
\switchcolumn
在这里我们应当分辨一下“政府” 和 “国家”之间的区
别。这两个词常常互相混用,尤其是在美国英语当中,但是它
们实际上指的是两种非常重要但是很容易混淆的制度。政府是
在人们同意下成立的一种组织,通过这个组织我们裁决争端、
捍卫权利,以及满足某些公共的需求。一个公寓业主协会就是
一个政府,裁决业主间的争端,规定公共区域的使用方式,保
护居民不受到外来侵犯,以及满足某些公共需求。通过这样的
分析,我们就很容易看到人们为什么要建立政府。在这两种情
况下,居 民 都 同 意 政 府 的 条 款 (宪法、特许状或者协会章
程),并且都同意被政府统治。国家则是一个强制性的组织,
是拥有或者声称拥有在某个地理区域内拥有唯一暴力,并有权
对区域内的对象使用暴力的垄断力量。美国建国先贤们的开创
性和天才之处就在于试图建立一个政府,而不是国家。
\switchcolumn*
Historically, the real origins of the state lie in conquest and
economic exploitation. The sociologist Franz Oppenheimer
pointed out that there are two basic ways to acquire the means
to satisfy our human needs. ``These are work and robbery, one's
own labor and the forcible appropriation of the labor of others.''
He called work and free exchange the ``economic means'' of acquiring wealth, and the appropriation of the work of others the
``political means.''
\switchcolumn
从历史的角度来看,国家的真正起源是军事征服和经济剥
削。社会学家奥本海默\footnote{弗兰茨$\cdot$奥本海默(Franz  Oppenheimer, 1864$\sim$1943),德国社会学家和政治经济学家。代表作《论国家》(The State)论述了国家的起源。奥本海默要批评的对象是卢梭。他不承认国家的形成是社会契约或者和平商议的结果。他认为国家是战争的结果,而不是避免战争的结果。}指出,有两种基本的方式被用来获得财富,满足人们需要。“它们是工作和抢劫,一个人自己进行劳动和强行夺取别人的劳动。”他把工作和自由交换称作获取财富的 “经济手段” ,而把夺取别人的工作果实称作“政治手段”。
\switchcolumn*
From this basic insight, Oppenheimer said, we can discern
the origins of the state. Banditry and robbery and fraud are the
usual ways in which people seek to forcibly appropriate what
others have produced. But how much more efficient it would be
to organize and regularize robbery! According to Oppenheimer,
``The State is the organization of the political means.'' States
arose when one group conquered another and settled in to rule
them. Instead of looting the conquered group and moving on,
the conquerors settled down and switched from looting to taxing. This regularization had some advantages for the conquered
society, which is one reason it endured: rather than planting
crops or building houses and then being subject to unpredictable looting by marauders, the peaceful and productive
people may prefer simply to be forced to give up, say, 25 percent of their crop to their rulers, secure in the knowledge that
that will---usually---be the full extent of the depredation and
that they will be protected from marauders.
\switchcolumn
奥本海默说,从这个基本观点出发,我们就能够分辨出国
家的起源。匪帮、抢劫和诈骗是人们强行夺取别人劳动成果的
常用方式。但是他们组织起来进行正规化抢劫则要有效得多!
按照奥本海默的说法就是: “ 国家是政治手段的组织。” 当一
群人征服了另一群人并且安顿下来统治他们的时候,国家就诞
生了。征服者开始停止抢了就走的掠夺方式,转而安顿下来用
收税来进行掠夺。这种抢劫的正规化对于被征服的社会来说是
有好处的。它之所以能够忍受是因为:相对于种植粮食、建房
子并且不得不遭受难以预料的匪帮掠夺来说,这些和平的和有
生产能力的人们宁可把他们粮食的一部分,比方说四分之一给
统治者,他们知道这常常就是全部损失了,而他们将因此受到
保护,不被其他的掠夺者所抢劫。
\switchcolumn*
This basic understanding of the distinction between society
and the state, between the people and the rulers, has deep roots
in Western civilization, going back to Samuel's warning to the
people of Israel that a king would ``take your sons, and your
daughters, and your fields'' and to the Christian concept that
the state is conceived in sin. The Levellers, the great fighters for
English liberty in the time of Charles I and Cromwell, understood that the origins of the English state lay in the conquest of
England by the Normans, who imposed on free Englishmen a
``Norman yoke.'' A century later, when Thomas Paine sought to
undermine the legitimacy of the British monarchy, he pointed
out, ``A French bastard, landing with an armed banditti, and establishing himself king of England against the consent of the
natives, is in plain terms a very paltry rascally original.''
\switchcolumn
对社会与国家之间、人民与统治者之间区别的基本理解 ,
深深植根于西方文明当中,最早可以追溯到撒母耳对以色列人
的警告:国 王 将 “夺走你的儿子、女儿和土地 ”, 以及基督教
的 “ 国家即罪恶” 的概念。查理一世和克伦威尔时期捍卫英
国人自由的伟大战士平权派人士就指出,英国国家起源于诺曼
底人对英格兰的征服,他们在自由的英格兰人脖子上套上了
“诺曼桎梏”(Norman  yoke)。一个世纪之后 , 潘恩对英国君
主制的合法性基础进行了否定。他说:“一个法国混蛋,带着
一帮武装匪徒登陆英格兰,在没有得到当地居民同意的情况下
自我加冕为英格兰国王,概括来说就是:源于一个无比卑鄙无
耻的流氓。”
\switchcolumn*
In a 1925 essay, ``More of the Same,'' the journalist H. L.
Mencken agreed:
\switchcolumn
在 1925年 的 文 章 “见微知著”(More  of the Same)中,
记者门肯对此表示同意:
\switchcolumn*
\begin{quote}
The average man $\ldots$ sees clearly that government is something
lying outside him and outside the generality of his fellow men---
that it is a separate, independent, and hostile power, only partly
under his control, and capable of doing him great harm$\ldots$
[Government] is apprehended, not as a committee of citizens chosen to carry on the communal business of the whole population, but as a separate and autonomous corporation, mainly devoted to exploiting the population for the benefit of its own
members$\ldots$ When a private citizen is robbed, a worthy man is
deprived of the fruits of his industry and thrift; when the government is robbed, the worst that happens is that certain rogues and
loafers have less money to play with than they had before.
\end{quote}
\switchcolumn
\begin{quote}
普通人……看得很清楚,政府是在他和他周围的普通
人之外的某种东西,是一种异己的、独立的、不友善的力
量,只有部分在他掌握当中,并能够对他造成巨大的伤
害……政府被理解为不是一个全体人民选择开展共同事业
的公民委员会,而是一个异己的、自我运转的组织,主要
目标是夺取所管轄的人民的利益……如果一个公民被抢
劫,一个高尚的人的商业和劳动果实就被剥夺了;如果政
府 被 剥 夺 (权力),最坏的事情不过是某些恶棍和游手好
闲的人不能像从前那样有那么多钱来抢了。”
\end{quote}
\switchcolumn*[\section{The Democratic State\\民主国家}]
It is usually argued in the United States that all this may have
been true in ancient times, or even in the countries our forefathers fled, but that in a democratic country ``\textit{we} are the government.'' The Founders themselves hoped that a democratic---or,
as they would have said, a republican---form of government
would never violate people's rights or do anything against the
interests of the people. The unfortunate reality is that we can't
all be the government. Most of us are too busy working, producing wealth, taking care of our families to watch what the
rulers are doing. What normal, productive person can read a
single one of the 1,000-page budget bills that Congress passes
each year to find out what's really in it? Not one American in a
hundred knows how much he really pays in taxes, given the
many ways that politicians hide the real costs.
\switchcolumn
在美国经常听到的一个观点是:(国家是一种抢劫)也许
在古代是事实,甚至在我们祖辈逃出来的那些国家是事实,但
是在一个民主国家里, “\textbf{我们}就是政府”。建国先贤们希望一个民主 --- 或者用他们的话来说 --- 共和政府将不再侵犯人民
权利或者做任何违背人民利益的事情。不幸的是,不可能我们
所有人都是政府。我们当中的绝大多数人忙于工作、创造财
富、照顾家庭,不能监督统治者在做什么。一个普通的、从事
生产的人能够读完国会每年通过的任何一个厚达1000页的预
算案,并知道里面到底有什么猫儿腻吗? 100个美国人当中不
会有超过1个人知道自己到底交了多少税,因为政客们有无数
办法隐藏真实的开支。
\switchcolumn*
Yes, we have the power every four years or so to turn the rascals out and put in a new set of rascals. But many factors limit
the value of that power:
\switchcolumn
是的,我们的确有权每四年一次把这些流氓赶下台并且换
上一批新流氓。但是很多因素限制了这项权力的作用:
\switchcolumn*
\begin{itemize}
	\item There aren't many fundamentally different alternatives on
	the ballot. The choice between Bush and Clinton, or Clinton
	and Dole, is hardly worth getting excited about. Even the supposedly revolutionary Congress of 1994 barely slowed the rate
	at which the federal government got bigger.
\end{itemize}
\switchcolumn
\begin{itemize}
	\item 首先,投票的时候并没有很多观点和政策本质不同的
	候选人供选择。在布什和克林顿之间,或者在克林顿和多尔之
	间进行选择,并没有太多的不同值得大家兴奋。甚 至 1994年
	经选举产生的被认为是革命性的国会,也几乎没有减缓联邦政
	府规模膨胀的速度。
\end{itemize}
\switchcolumn*
\begin{itemize}
	\item We have to choose a package deal. Sesame Street recently
	gave us an example of what that means. In an election special,
	the Muppets and their human friends have \$3 to spend, and
	they learn about voting by deciding whether to buy crayons or
	juice.
	\begin{quotation}
		\textit{Rosita}: You count the people who want crayons. Then you count
		the people who want juice. If more people want juice, it's juice
		for everyone. If more people want crayons, it's crayons.
		
		\textit{Telly}: Sounds crazy but it might just work!
	\end{quotation}
	But why not let each child buy what \textit{he} wants? Who needs
	democracy for such decisions? There \textit{may} be some public goods,
	but surely juice and crayons don't count. In the real world, one
	candidate offers higher taxes, legalized abortion, and getting
	out of the War in Vietnam; another promises a balanced budget, school prayer, and escalation of the war. What if you want
	a balanced budget \textit{and} withdrawal from Vietnam? In the marketplace, you \textit{get} lots of choices; politics forces you to choose
	among only a few.
\end{itemize}
\switchcolumn
\begin{itemize}
	\item 其次,我们不得不选择一个打包的政策组合。动画系
	列片芝麻街(Sesame Street) 最近提供了一个例子告诉我们这
	意味着什么。在一次特别投票当中,布娃娃和他们的人类朋友
	有 3 美元可以花,于是他们学着通过投票来决定是买蜡笔还是
	果汁。
	\begin{quotation}
		罗斯塔:你来点点想要蜡笔的人数。你来点点想要果
		汁的人数。如果要果汁的人更多,那么就给每个人买果
		汁。如果要培笔的人更多,那就买蜡笔。
		
		特利- 听上去很疯狂,不过也许可行!
	\end{quotation}
	但是为什么不让每个孩子买\textbf{他}想要的东西呢?在做这种决
	定的时候,谁需要民主?\textbf{也许}是有一些东西属于公共物品,但
	是果汁和蜡笔肯定不算。真实世界里通常是,一个候选人提出
	高税收、堕胎合法化和停止越南战争的竞选纲领;而另一个则
	提出平衡政府预算、允许在学校祷告、升级战争的纲领。那么
	如果你想要平衡政府预算\textbf{和}从越南撤军呢?在市场上,你\textbf{会有
	很多}的选择;而政治则迫使你只在有限的几个选项中进行
	选择。
\end{itemize}
\switchcolumn*
\begin{itemize}
	\item People employ what economists call ``rational ignorance.''
	That is, we all spend our time learning about things we can actually do something about, not political issues that we can't really affect. That's why more than half of us can't name either of
	our U.S. senators. (I'm sure the readers of this book can, but 54
	percent of those polled by the \textit{Washington Post} couldn't.) And
	why most of us have no clue about how much of the federal
	budget goes to Medicare, foreign aid, or any other program. As
	an Alabama businessman told the \textit{Post}, ``Politics doesn't interest
	me. I don't follow it. $\ldots$ Always had to make a living.'' Ellen
	Goodman, a sensitive, good-government liberal columnist,
	complains about a friend who has spent months researching
	new cars, and of her own efforts to study the sugar, fiber, fat,
	and price of various cereals. ``Would my car-buying friend use
	the hours he spent comparing fuel-injection systems to compare
	national health plans?'' Goodman asked. ``Maybe not. Will the
	moments I spend studying cereals be devoted to studying the
	greenhouse effect on grain? Maybe not.'' \textit{Certainly} not---and
	why should they? Goodman and her friend will get the cars and
	the cereal they want, but what good would it do to study national health plans? After a great deal of research on medicine,
	economics, and bureaucracy, her friend may decide which
	health-care plan he prefers. He then turns to studying the presidential candidates, only to discover that they offer only vague
	indications of which health-care plan they would implement. But after diligent investigation, our well-informed voter
	chooses a candidate. Unfortunately, the voter doesn't like that
	candidate's stand on anything else---the package-deal problem---but he decides to vote on the issue of health care. He has
	a one in a hundred million chance of influencing the outcome of
	the presidential election, after which, if his candidate is successful, he faces a Congress with different ideas, and in any case, it
	turns out the candidate was dissembling in the first place. Instinctively realizing all this, most voters don't spend much time
	studying public policy. Give that same man three health insurance plans that he can choose from, though, and chances are
	that he \textit{will} spend some time studying them.
\end{itemize}
\switchcolumn
\begin{itemize}
	\item 人们都有一种经济学家所称的“理性的无知”(ration­al ignorance) 。 意思是,我们花时间学习那些实际上用得到的
	知识,而不会把时间花在那些我们不能真的对其有影响的政治
	问题上。这就是为什么我们当中超过一半的人叫不出美国国会两院名字的原因(我敢肯定这本书的读者能叫得出来,但是
	根 据 《华盛顿邮报》 的调查,54\%的美国人叫不出来)。这也
	是我们当中的大多数人对联邦预算有多少用在医疗、对外援助
	和其他项目上几乎一无所知的原因。阿拉巴马州一位商人对
	《华盛顿邮报》说 :“我对政治不感兴趣,我不关心它……毕竟我还得谋生。” 艾伦 $\cdot$ 古德曼(Ellen Goodman) 是一个敏感
	的、支持 “好政府” 的自由派专栏作家。他抱怨说,他的一个朋友可以花数月时间去搜寻新车,乐此不疲地研究糖、食物
	纤维、脂肪、各种麦片的价格之类。古德曼问道:“我那个买
	车的朋友会把比较各种汽车喷油系统的时间抽出几个小时来,
	比较一下候选人提出的几个国家医疗计划吗?也许不会。那
	么,我会用研究麦片的时间来研究温室效应吗?也许不会。”
	\textbf{当然不会} --- 怎么可能会呢?古德曼和他的朋友会得到他们想
	要的汽车和麦片,但是研究国家医疗计划对他们有什么用呢?
	在对药品、经济学和官僚机构进行了大量研究之后,她的朋友
	终于可以根据自己的偏好对医疗计划进行选择了。然后,他转
	而对总统候选人进行研究,却发现候选人只对将要实施的医疗
	计划提出了一些模糊概念。不管怎样,在辛苦调研之后,这位
	得到充分信息的选民选了一个候选人。不幸的是,这位选民并
	不喜欢这个候选人的其他立场 --- 施政纲领打包带来的问题
	--- 但是他还是决定为了医疗计划而投票。然而,他只有一亿
	分之一的机会能够影响总统选举结果,这之后,如果他看中的
	候选人获胜,他也会面对一个意见五花八门的国会。很显然,
	这个候选人从一开始就在自欺欺人,一场儿戏。由于直觉上意
	识到了这点,大多数选民不会花太多时间来研究公共政策。然
	而 ,如果给同样这些人三个医疗个人保险方案供他选择,他多半\textbf{愿意}花时间来研究。
\end{itemize}
\switchcolumn*
\begin{itemize}
	\item Finally, as noted above, the candidates are likely to be kidding themselves or the voters anyway. One could argue that in
	every presidential election since 1968, the American people
	have tried to vote for smaller government, but in that time the
	federal budget has risen from \$178 billion to \$1.6 trillion.
	George Bush made one promise that every voter noticed in the
	1988 campaign: ``Read my lips, no new taxes.'' Then he raised
	them. If we are the government, why do we get so many policies we don't want, from school busing and the war in Vietnam
	to huge deficits, tax rates higher than almost any American approves, and the war in Bosnia?
\end{itemize}
\switchcolumn
\begin{itemize}
	\item 最后,就像上面提到的那样,候选人总是喜欢自欺欺
	人。有人也许会争辩说,1968年以来,每一次总统选举美国
	人都会投票给主张小政府的候选人,但是从那时起,联邦预算
	从 1780亿美元增长到了16000亿美元。老布什在1988年竞选
	当中没有像其他候选人那样做出一大堆许诺,他只做出一个许
	诺 ,他说:“看我的嘴唇:不加税。” 然而上台以后,他加税
	了。如果民主政体下,我们就是政府的话,为什么会有这么多
	我们不想要的政策,从校车、越南战争到巨额财政赤字?税率
	超过几乎每个美国人所承认的水平,还有波斯尼亚战争。
\end{itemize}
\switchcolumn*
No, even in a democracy there is a fundamental difference
between the rulers and the ruled. Mark Twain once said, ``It
could probably be shown by facts and figures that there is no
distinctly native American criminal class except Congress.'' Of
course, Congress is no worse than its counterparts in other
countries.
\switchcolumn
不,甚至民主政体下,统治者和被统治者之间仍然存在着
根本不同。马 克 $\cdot$ 吐温曾经说过: “事实和数字也许会显示,
除了国会之外再没有一个明显的美国本地犯罪阶级了 。” 当
然,美国国会并不比其他国家的国会更糟。
\switchcolumn*
One of the most charming and honest descriptions of politics
ever penned came from a letter written by Lord Bolingbroke,
an English Tory leader in the eighteenth century.
\switchcolumn
在史上用笔写就的文章当中 , 对政治的描述最有吸引力和
最忠实的莫过于18世 纪 的 博 林 布 鲁 克 勋 爵(Lord Bolingbroke) 写的一封信。他当时任英属特洛伊的高级官员:
\switchcolumn*
\begin{quote}
I am afraid that we came to Court in the same dispositions as all
parties have done; that the principal spring of our actions was to
have the government of the state in our hands; that our principal views were the conservation of this power, great employments to ourselves, and great opportunities of rewarding those who had helped to raise us and of hurting those who stood in opposition to us.
\end{quote}
\switchcolumn
\begin{quote}
我担心的是当我们进入宫廷的时候,所有各方的分赃
都已经完毕;我担心的是我们行为的最高动机只不过是让
国家政府掌握在自己手里;我担心我们最关心只是如何保
住权力 , 如何把好的职位留给自己,如何把有巨大回报的
机会给那些支持我们的人,以及如何伤害那些站在我们相
反立场的人。
\end{quote}
\switchcolumn*
Libertarians recognize that power tends to corrupt its holders. How many politicians, no matter how well intentioned, can
avoid abusing the considerable power of today's expansive governments? Look at Senator Robert Byrd's constant exertions to
move the entire federal payroll to West Virginia, or Senator Bob
Dole's long record of generous contributions from the Archer-Daniels-Midland Corporation and his championing of huge
federal subsidies for ADM. Or note the clear echo of Bolingbroke's letter in a White House aide's notes about Hillary Clinton's instructions to fire the career civil servants at the White
House Travel Office: ``We need those people out---We need our
people in---We need the slots.''
\switchcolumn
古典自由主义者认为,权力总是会腐蚀掌权者。在今天这
个庞大的政府当中,很少有政治人物能够避免滥用职权,不管
他目标是多么纯洁。看一下这些案例吧:参议员罗伯特$\cdot$博德
(Robert Byrd)挪用发给西弗吉尼亚联邦雇员的工资款,参议员鲍勃$\cdot$ 多尔(Bob Dole)长期接受ADM公司的礼金为他们
争取巨额政府拨款。还有,就像博林布鲁克勋爵写的情形一
样,一位白宫助手透露:希拉里曾经命令开除白宫差旅办公室
的文职服务人员,她说:“我们需要这些人出去,以便让我们
的人进来。我们需要这些职位。”
\switchcolumn*
A particularly striking illustration of what we might call Bolingbroke's Law is the record of Maryland governor Parris Glendening. Elected in 1994, Glendening seemed a clean, honest,
moderate, technocratic former professor. He might give Maryland big government, but at least it would be clean govern-
ment. So what did he do when he took office? Well, here's how
the \textit{Washington Post} described his first budget: ``In his first major
act as Maryland governor, Parris N. Glendening unveiled a nonew-taxes budget that unabashedly steers the biggest share of
spending to the three areas that voted most strongly for him:
Montgomery and Prince George's counties and Baltimore.''
Lord Bolingbroke, call your office. A few days later it turned
out that Glendening and his top aide were collecting tens of
thousands of dollars in early pension payments from Prince
George's County---where Glendening had served as county ex-
ecutive until his election as governor---thanks to his creative interpretation of rules that gave early pension benefits to
government employees who suffered ``involuntary separation''
from their jobs. Glendening decided that he had been ``involuntarily separated'' because of the two-term limit on the county
executive. And he ``demanded'' the resignations of his top aides
a month before he left his county job---making them also victims of ``involuntary separation''---whereupon he hired them as
his top aides in the governor's mansion.
\switchcolumn
对我们可以称之为“博林布鲁克法则”(Bolingbroke’s Law)现象的一个惊人证明是马里兰州州长格林邓林(Pairis Glendening)的记录。1994年当选时,格林邓林看上去干净、
诚实、谦虚,像个技术专家或教授。他虽然会给马里兰带来一
个大政府,但那看来至少会是一个干净的政府。那么他上台之
后干了些什么呢?这 是 《华盛顿邮报》 描述他的第一份预算
时说的话:“在他作为州长颁布的第一个重大法令当中,格林
邓林提出了一份不需加税的预算,但是毫不脸红地把预算的最
大份额花在选举中他的支持票数最多的三个地区:蒙哥马利、
乔治王子县、巴尔的摩。” 博林布鲁克勋爵显灵! 一些天之
后 ,格林邓林和他的高级助手从乔治王子县(在当州长之前
他在当地任县执行官)那里收到了数万美元提前支付的退休
金。这得益于他对法规的创造性解释,按照规定,政府雇员在
“非自愿离职” 的时候,可以得到提前支付的退休金。格林邓
林认为他算是“非自愿离职”,因为县执行官被限制只能有两
届任期。而 且 他 “要求” 自己的高级助手在他离开县里的职
位前一个月离职,于是,他们也成了 “非自愿离职” 的 “受
害者”。然后,格林邓林雇用他们成为州长官邸的高级助手。
\switchcolumn*
Like the Energizer bunny, the Glendening money train just
kept on going. In May 1995, the governor asked the legislature
to spend \$1.5 million in taxpayer funds to rescue a struggling
high-tech firm in Prince George's County headed by one of his
political supporters. Then in August, Frank W. Stegman, the
state secretary of labor, licensing, and regulation, hired the wife
of Theodore J. Knapp, the state personnel secretary and a colleague of Stegman's from the Prince George's government, for
a job in his agency. No ingrate, Knapp then returned the favor
by recommending a \$10,000 raise in Stegman's meager
\$100,542 salary.
\switchcolumn
就像劲量电池广告里的邦尼兔一样,格林邓林的金钱列车
仍然滚滚向前。1995年 5 月,州长向州参议会要求从纳税人
基金里拨出150万美元拯救乔治王子县一家濒临破产的高科技
公司,而那家公司的头儿是他的政治支持者。而到了8 月 ,州劳工、执照与法规局长斯德格曼(Frank W. Stegman) 雇用了
州人力资源局长耐普(Theodor J. Knapp) 的妻子在他的局里
工作,耐普是斯德格曼在乔治王子县工作时的同事。耐普知恩
图报,很快就还了这个人情,立刻批准将斯德格曼的100542
美元的薪水上调1 万美元。
\switchcolumn*
If this is what the apparently honest politicians do, just
imagine what the others are up to.
\switchcolumn
如果这是公认的诚实政治家的行径,那么想想那些不那么
诚实的家伙会干些什么吧。

\switchcolumn*[\section{Why Government Gets Too Big\\为什么政府会变得过于庞大}]
Thomas Jefferson wrote, ``The natural progress of things is for
liberty to yield and government to gain ground.'' Two hundred
years later, James M. Buchanan won a Nobel Prize in economics for a lifetime of scholarly research confirming Jefferson's in-
sights. Buchanan's theory, developed along with Gordon
Tullock, is called Public Choice. It's based on one fundamental
point: Bureaucrats and politicians are just as self-interested as
the rest of us. But lots of scholars did---and do---believe otherwise, and that's why textbooks tell us that people in the private
economy are self-interested but the government acts in the
public interest. Notice the little sleight of hand in that last sentence? I said ``people in the private economy,'' but then I said
``government acts.'' Switching from the individual to the collective confuses the issue. Because actually, the \textit{government} doesn't
act.\textit{ Some people in the government} act. And why should the guy
who graduates from college and goes to work for Microsoft be
self-interested, while his roommate who goes to work for the
Department of Housing and Urban Development is suddenly
inspired by altruism and starts acting in the public interest?
\switchcolumn
杰斐逊曾经说过:“事物的自然发展趋势就是,自由逐渐
退让,政府步步紧逼。” 200年后,詹姆斯$\cdot$布坎南\footnote{詹姆斯$\cdot$布坎南(James M. Buchanan, 1919$\sim$) ,美国著名经济学家,	公共选择学派创始人,1986年诺贝尔经济学奖获得者。现任乔治$\cdot$梅森大学教授。布坎南突出的理论贡献是创立了公共选择理论。布坎南的主要著作有:《赞同的计算:宪法民主的逻辑基础》 (与塔洛克合著)《成本与选择:一个经济理论的探讨》 《公共产品的需求与供应》 《自由的限度》 《宪法契约中的自由》 《自由、市场和国家:80年代的政治经济学》等。}因终身致力于以学术研究证明杰斐逊的洞见而获得了诺贝尔经济学奖。布坎南与戈登$\cdot$ 塔洛克\footnote{戈登$\cdot$塔洛克(Gordot Tullock),美国著名经济学家,公共选择理论的创始人之一,“寻租理论” 创始人。现任 乔治$\cdot$梅森大学法和经济学教授。主要著作有:《官僚政治学》《关税、垄断和偷窃的福利成本》《寻租》《论投票》。}对该理论的发展一起被称为公共选择理论(Public Choice)。这个理论建立在一个基本论点之上:官僚和政治家与所有其他人一样自利。但是很多学者的确不相
信这一点,因此教科书告诉我们说,私人经济当中的人是自利
的,而政府行为则是为了公共利益。注意到这个句子中的文字
游戏了吗?前面说的是“私人经济当中的人”,而后面说的是“政府行为”。从说个人转换到说集体搅乱了这个问题。因为
实际上 , \textbf{政府}不会做任何事,是\textbf{政府中的某些人}在做事。为什
么一个大学毕业之后进入微软的人是自利的,而他那进入住房
与城市发展部当公务员的寝室室友就突然被利他主义所激励,
开始为公共利益而做事呢?
\switchcolumn*
As it turns out, making the simple economic assumption
that politicians and bureaucrats act just like everyone else,
namely, in their own interest, has enormous explanatory power. Far better than the simplistic civics-book model that assumes
public officials act in the public interest, the Public Choice
model explains voting patterns, lobbying efforts, deficit spending, corruption, the expansion of government, and the opposition of lobbyists and members of Congress to term limits. In
addition, the Public Choice model explains why self-interested
behavior has positive effects in a competitive marketplace but
does such harm in the political process.
\switchcolumn
事实证明, 这个简单的经济学假设:与其他人一样 ,政治家和官员的行为也在追求他们自己的利益,有很强的解释力。
相比公民教材中假设公务员追求公共利益的过于简单化的解释
模式,公共选择理论对投票、院外活动、赤字开支、腐败、政
府扩张以及议员和院外活动家在提案限制\footnote{案限制(term limit), 指在国会中对一个人可以提出的提案数量的限制。}上的对抗的解释要有力得多。另外,公共选择理论也解释了为什么自利行为在竞争性的市场上有正面的影响,而在政治过程中却是有害的。
\switchcolumn*
Of course politicians and bureaucrats act in their own inter-
est. One of the key concepts of Public Choice is \textit{concentrated benefits and diffuse costs}. That means that the benefits of any
government program are concentrated on a few people, while
the costs are diffused among many people. Take ADM's
ethanol subsidy, for instance. If ADM makes \$200 million a
year from it, it costs each American about a dollar. Did you
know about it? Probably not. Now that you do, are you going
to write your congressman and complain? Probably not. Are
you going to fly to Washington, take your senator out to dinner, give him a \$1,000 contribution, and ask him not to vote
for the ethanol subsidy? Of course not. But you can bet that
ADM chairman Dwayne Andreas is doing all that and more.
Think about it: How much would you spend to get a \$200
million subsidy from the federal government? About \$199
million if you had to, I'll bet. So who will members of Congress
listen to? The average Americans who don't know that they're
paying a dollar each for Dwayne Andreas's profits? Or Andreas, who's making a list and checking it twice to see who's
voting for his subsidy?
\switchcolumn
政治家和官员们的行为当然是在追求他们自己的利益。公
共选择理论的一个关键概念是 “\textbf{ 收益集中而成本分散} ” (con­centrated benefits and diffuse costs)。意思是任何政府计划的收
益都是集中在一部分人身上,而成本却分散到所有人身上。对
ADM公司的乙醇项目补贴就是一个例子。如 果 ADM每年从
这个项目中获得2 亿美元补贴,那么每个美国人就会承担大约
1美元的成本。你知道这件事吗?也许不知道。现在你知道
了,你会写信给你的众议员抱怨吗?也许不会。你会飞到华盛
顿 ,把你的参议员叫出来吃个便饭,给 他 1000美元捐款,让
他不要给乙醇补贴项目投票吗?当然不会。但 是 ADM主席安
德 烈 斯(Dwayne Andreas)会做这些事,而且会做得更多。想
一想吧:为了从联邦政府得到2 亿美元的补贴你会花多少钱?我敢打赌,如果不得不花1.99亿美元的话,你也会花的。那
么国会议员会听谁的?是那些甚至不知道自己正在给安德烈斯
的利润支付1美元的普通美国人,还是那个正在反复研究补贴
议案投票人名单的安德烈斯?
\switchcolumn*
If it were just ethanol, of course, it wouldn't matter very
much. But most federal programs work the same way. Take the
farm program. A few billion dollars for subsidized farmers, who
make up about 1 percent of the U.S. population; a few dollars a
year for each taxpayer. The farm program is even more tricky
than that. Many of its costs involve raising food prices, so consumers are paying for it without realizing it.
\switchcolumn
当然,如果只是乙醇的话,就无关紧要了。但是绝大多数
联邦计划都是以这种方式来运作的。几十亿美元的补贴给了农
民,而这些农民只占美国人口的1% ; 相当于每位纳税人要支
付好几十美金。农业补贴的成本问题还有微妙之处。它的成本
还要包括食品价格的上升,而消费者在不知不觉中为此付出了
代价。
\switchcolumn*
Billions of dollars are spent every year in Washington to get
a piece of the trillion dollars of taxpayers' money that Congress
spends every year. Consider this ad from the \textit{Washington Post}:
\switchcolumn
每年有数以十亿计的美元花在华盛顿,就是为了争取得到
国会每年要花掉的上万亿的纳税人的钱。请 看 看 《华盛顿邮报》上的这个广告:
\switchcolumn*
\begin{quote}
\textit{Infrastructure} $\ldots$ is a new Washington buzzword for: A. America's crumbling physical plant? \$3 trillion is needed to repair
highways, bridges, sewers, etc. B. Billions of federal reconstruction dollars? The 5c per gallon gasoline tax is only the beginning. C. \textit{Your bible for infrastructure spending---where the money is
going and how to get your share}---in a concise biweekly newsletter?
ANSWER: All of the above. Subscribe today.
\end{quote}
\switchcolumn
\begin{quote}
基础设施是一个新的华盛顿时髦词:人美国损坏的
基础设施?需要3 万亿美元来维修公路、桥梁、下水道等
等。B .数十亿的联邦重建资金?每加仑5 美分的税收只
是一个开始。C .你的基础设施开支圣经 --- 你的钱去了
哪里以及怎样得到你的份额----在一份双周新闻通讯都能
看到吗?答案:都能看到。赶快订阅吧。
\end{quote}
\switchcolumn*
Countless such newsletters tell people what kind of money the
government is handing out and how to get their hands on it.
\switchcolumn
无数这样的新闻通讯报告诉人们,政府正在支出什么样的
钱以及怎样把你的手伸进去。
\switchcolumn*
In 1987 an advertisement in the Durango, Colorado, \textit{Herald}
touting the Animas-La Plata dam and irrigation project made
explicit the usual hidden calculations of those trying to get their
hands on federal dollars: ``Why we should support the AnimasLa Plata Project: Because someone else is paying the tab! We
get the water. We get the reservoir. They get the bill.''
\switchcolumn
1987年科罗拉多州《杜兰戈导报》刊登了一篇广告,极力兜售安尼玛斯一拉普拉塔(Animas-La Plata) 大坝水利工程 ,让很多企图把自己的手伸进联邦预算的隐藏算计一下子暴露了出来:“我们为什么应该支持安尼玛斯一拉普拉塔工程:因为是别人买单!我们得到水,我们得到水库,别人得到账单。”
\switchcolumn*
Economists call this process rent-seeking, or transfer-seeking. It's another illustration of Oppenheimer's distinction between the economic and the political means. Some individuals
and businesses produce wealth. They grow food or build things
people want to buy or perform useful services. Others find it
easier to go to Washington, a state capital, or a city hall and get
a subsidy, tariff, quota, or restriction on their competitors.
That's the political means to wealth, and, sadly, it's been growing faster than the economic means.
\switchcolumn
经济学家称这种过程为寻租(rent-seeking) , 或者寻求
转 移 支 付 (transfer-seeking)。这也是对奥本海默的区分经济
手段和政治手段的理论的另一种描述。有些个人和企业创造财
富。他们生产粮食或者制造别人愿意购买的东西,或者提供有
用的服务。其他一些人则发现,跑到华盛顿或者州首府、市政
厅去寻求补贴、关税、配额,或者对竞争对手进行限制要更容
易一些。这就是用政治手段得到财富,而且可悲的是,它的增
长比经济手段的增长更快。
\switchcolumn*
Of course, in the modern world of trillion-dollar governments
handing out favors like Santa Claus, it becomes harder to distinguish between the producers and the transfer-seekers, the predators and the prey. The state tries to confuse us, like the three-card
monte dealer, by taking our money as quietly as possible and
then handing some of it back to us with great ceremony. We all
end up railing against taxes but then demanding our Medicare,
our subsidized mass transit, our farm programs, our free national
parks, and on and on and on. Frederic Bastiat explained it in the
nineteenth century: ``The State is that great fiction by which
everyone tries to live at the expense of everyone else.'' In the aggregate, we all lose, but it's hard to know who is a net loser and
who is a net winner in the immediate circumstance.
\switchcolumn
当然,在万亿美元规模的政府像圣诞老人一样普遍派发礼
物的现代世界里,越来越难区分生产者和转移寻求者、捕食动
物和被捕食动物了。国家试图迷惑我们,就像墨西哥的三张牌
戏 (tree-card monte)的发牌员一样,悄无声息地拿走我们
的钱,然后敲锣打鼓地把其中的一部分还给我们。我们所有人
都一方面在不停地咒骂征税,另一方面却要求医疗保障,要求
补贴公共交通、补贴农业、补贴免费国家公园,等等。巴斯夏
在 19世纪对此进行了解释:“国家就是一个每个人都想靠其他
人的开支生活的神话。” 从整体来说,我们所有人都是输家,
但是很难知道在当前环境下谁是净输家谁是净贏家。
\switchcolumn*
In his book\textit{ Demosclerosis}, the journalist Jonathan Rauch described the process of transfer-seeking:
\switchcolumn
在他的书《民主硬化症》(\textit{Demosclerosis}) 当中,新闻记者 乔 纳 森 $\cdot$ 罗施(Jonathan Rauch) 描述了寻求转移支付的
过程:
\switchcolumn*
\begin{quotation}
In America, only a few classes of people have the power to take
your money if you don't fend them off. One is the criminal class.
People who break into your car or rob your house (or punch holes
in your roof) are members of the parasite economy in the classic
sense: they take your wealth if you don't actively fight them off.
Such people are costly to society, not only for what they take, but
for the high cost of fending them off. They make us buy locks,
alarms, iron gates, security guards, policemen, insurance, and on
and on$\ldots$


Criminals, however, aren't the only ones who play the distributive game. Legal, noncriminal transfer-seeking is perfectly possible---on one condition. You need the law's help. That is, you
need to persuade politicians or courts to intervene on your behalf.
\end{quotation}
\switchcolumn
\begin{quotation}
在美国,只有几个阶级的人有权力拿走你的钱,如果
你没有保护好的话。第一个是犯罪阶级,就是那些打碎玻
璃进入你的汽车或者抢劫你的房子(或者在你的天花板
上打洞)的人,他们是经济体的寄生虫中很古老意义上
的成员:他们只有在你没有积极反抗的时候才拿走你的财富。这一类人对社会来说成本很高,不仅是因为他们拿走
的东西, 而且是因为为了防备他們我们不得不付出的高昂
成本。我们被迫购买门锁、警报器、铁门、保安、警察、
保险等服务和商品……

然而,罪犯不是唯一玩这个再分配游戏的群体。在某
种条件下,合法的、非犯罪的转移支付很可能拿走你的
钱。你需要法律的帮助,也就是说,你需要说服政治家或
者法院千预你的生活。
\end{quotation}
\switchcolumn*
Thus, he goes on, every group in society comes up with a way
for the government to help it or penalize its competitors: businesses seek tariffs, unions call for minimum-wage laws (which
make high-priced skilled workers more economical than
cheaper, low-skilled workers), postal workers get Congress to
outlaw private competition, businesses seek subtle twists in
regulations that hurt their competitors more than themselves.
And because the benefits of every such rule are concentrated on
a few people, while the costs are spread out over many consumers or taxpayers, the few profit at the expense of the many,
and they reward the politicians who made it happen.
\switchcolumn
他继续论述道,于是,社会当中的每个集团都到政府那里
要求帮助惩罚它的竞争对手:企业寻求关税保护,工会要求最低工资法(该法律让雇用高工资的熟练工人比雇用非熟练工
人更加合算),邮政工人要求国会禁止私营邮递的竞争,公司
寻求更巧妙、更绕弯子的管制来让竞争对手受到的伤害超过自
己。而且由于每一项这样规定的受益都集中在一部分人身上,
当其成本传递出来让大多数消费者或纳税人承担的时候,少数
人就因为大多数人买单而获利,这些获利者会回报那些促成转
移支付的政治家。
\switchcolumn*
Another reason that government grows too big is what Milton and Rose Friedman have called ``the tyranny of the status quo.'' That is, when a new government program is proposed,
it's the subject of heated debate. (At least if we're talking about
big programs like farm subsidies or Medicare. Plenty of smaller
programs get slipped into the budget with little or no debate,
and some of them get pretty big after a few years.) But once it
has passed, debate over the program virtually ceases. After that,
Congress just considers every year how much to increase its
budget. There's no longer any debate about whether the program should exist. Reforms like zero-based budgeting and sunset laws are supposed to counter this problem, but they haven't had much effect. When the federal government moved to shut
down the Civil Aeronautics Board in 1979, they found that
there were no guidelines for terminating a government
agency---it just never happens. Even President Clinton's own
National Performance Review---the much-touted ``reinventing
government'' project---said, ``The federal government seems
unable to abandon the obsolete. It knows how to add, but not
to subtract.'' But you could search a Clinton budget for a long
time and not find a proposal to eliminate a program.
\switchcolumn
另一个政府变得过于庞大的原因是米尔顿$\cdot$弗里德曼和萝
丝 $\cdot$ 弗里德曼所称的“现状的暴政”(the  tyranny of the status quo)。意思是说,当一个新的政府计划被提出的时候,它必
然导致激烈的争论(至少大计划如农业补贴或者医疗保障是
这样。大量小的项目都是在没有争论和极少争论的情况下就悄
悄被塞进了预算,而其中的一些项目在几年之后变得非常庞
大)。但是一旦得到通过,对这个计划的争论就会偃旗息鼓。
从那以后,国会只是每年审议一下需要对它增加多少预算,而
不会有任何关于这个计划是不是应该继续存在的争论。一些改革 如 “零基预算”\footnote{零基预算(zero-based  budgeting),是指不考虑过去的预算项目和收支水平,以零为基点编制的预算。零基预算的基本特征是不受以往预算安排和预算执行情况的影响,一切预箅收支都建立在成本效益分析的基础上,根据需要和可能来编制预算。} 和 “ 日落立法”\footnote{日落立法(Sunset Laws),政府机构在设立之后,或者政策已经实施之后 ,往往很难撤消,这是政府机构肿胀和财政膨胀的重要原因之一。 日落立法针对的就是这种现象,认为太阳有升有落,政府机构和政策就应像太阳一样,有始有终,在期限将到之际,行政部门与立法部门应进行效果评估,以决定该机构或政策是否有继续存在的必要。}被认为可以解决这个问题 ,但是它们并没有多少效果。1979年,当联邦政府准备关
掉 “民用太空版”(the  Aeronautics Board)项目时,他们发现没有任何法律依据可以停止一个政府部门的运转 --- 因为这从
来没有发生过。甚至克林顿总统的“国家绩效审查” (Nation­al Performance Review)  --- 这 个 被 宣 传 为 “重新设计政府”
的项目 --- 都说: “联邦政府看上去不能放弃已经无效的部分。它知道如何增加机构,但是不知道如何减少机构。”但是
你无论花多长时间来搜索克林顿的预算,也找不到一项削减某个计划的动议。
\switchcolumn*
One element of the tyranny of the status quo is what Washingtonians call the iron triangle, which protects every agency
and program. The Iron Triangle consists of the congressional
committee or subcommittee that oversees the program, the bureaucrats who administer it, and the special interests that benefit from it. There's a revolving door between these groups: a
congressional staffer writes a regulation, then she goes over to
the executive branch to administer it, then she moves to the private sector and makes big bucks lobbying her former colleagues
on behalf of the regulated interest group. Or a corporate lobbyist makes contributions to members of Congress in order to get
a new regulatory agency created, after which he's appointed to
the board of the agency---because who else understands the
problem so well?
\switchcolumn
现状的暴政的一个要素是华盛顿人所说的“铁三角”,正
是这铁三角保护了每一个政府部门和项目。铁三角包括审查项
目的国会委员会或小组委员会,执行这些项目的官员,以及从
中获利的特殊利益集团。这些集团就像一个旋转门:一名国会
工作人员拟了一份提案,然后找到行政部门让他们执行,随后
找到私人企业或机构得到一大笔美元,然后代表与管制相关的
利益集团找国会的同僚们游说。或者一个企业的院外游说家给
国会议员献金,成立一个新的管制机构,然后他就被任命为这
个机构的成员 --- 还有谁比他更了解这个问题呢?
\switchcolumn*
If bureaucrats and politicians are self-interested, like the rest
of us, how will they act in government? Well, no doubt they
will sometimes seek to serve the public interest. Most people
believe in trying to do the right thing. But the incentives in
government are not good. To make more money in the private
economy, you have to offer people something they want. If you
do, you'll attract customers; if you don't, you may go out of
business, or lose your job, or lose your investment. That keeps
businesses on their toes, trying to find ways to better serve consumers. But bureaucrats don't have customers. They don't
make more money by satisfying more consumers. Instead, they
amass money and power by enlarging their agencies. What do
bureaucrats ``maximize''? Bureaucrats! Their incentive, then, is
to find ways to hire more people, expand their authority, and
spend more taxpayers' dollars. Discover a new problem that
your agency could work on, and Congress may give you another billion dollars, another deputy, and another whole bureau under
your control. Even if you don't discover a new problem, just advertise that the problem you were commissioned to handle is
getting a lot worse, and you may get more money and power.
Solve a problem, on the other hand---improve children's test
scores or get all the welfare recipients into jobs---and Congress
or your state legislature is likely to decide you don't need more
money. (It could even decide to shut your agency down, though
this is largely an idle threat.) What an incentive system! How
many problems are likely to get solved when the system punishes problem solving?
\switchcolumn
如果官僚和政治家们像其他人一样是自利的,他们在政府
中会怎么做呢?当然,毫无疑问他们有时会为了公共利益而服
务。大多数人都认为自己应该去做正当的事情。但政府的激励
机制可并不那么好。在私人经济中,为了得到更多的钱,你必
须吸引客户;否则你也许会破产、丢掉工作或者让投资打水漂
儿。企业因此如履薄冰,竭力寻找更好的服务消费者的方式。
但是政府官僚并没有客户。他们并不是通过更多地满足消费者
来挣更多的钱,而是通过扩大自己的部门来争取更多的钱和权
力。什么能让官僚利益“最大化”?是官僚!他们的激励是想
办法雇用更多的人、扩大权力,从而花更多的纳税人的钱。如
果你发现了一个新问题而你的部门能够去处理,那么国会也许
就会再给你10亿美元、一个副手和另一个部门供你支配。即
使没发现新问题,你也可以宣传说,你在处理的问题正在恶
化,因此你需要更多的拨款和权力。另一方面,如果你把问题
解决了,例如改善了孩子们的考试成绩或者让所有享受福利的
人得到了工作,那么国会或者你所在的州参议会就可能会认为
你不再需要更多的拨款了。甚至还会决定裁掉你的部门,尽管
这常常是无法做到的。这是一个什么样的激励机制!如果对解
决问题进行惩罚的话,还能指望有多少问题会被解决呢?
\switchcolumn*
The obvious answer would seem to be to change the incentive
system. But that's easier said than done. Government doesn't
have customers, who can use its products or try a competitor instead, so it's difficult to decide when government is doing a
good job. If more people send letters every year, is the U.S.
Postal Service doing a good job of serving its customers? Not
necessarily, because its customers are captive. If they want to
mail a letter, they have to do it through the Postal Service (unless they're willing to pay at least ten times as much money for
overnight service). As long as any institution gets its money coercively, through legally required payments, it is difficult if not
impossible to measure its success at serving customers. Meanwhile, special interests within the system---politicians, administrators, unions---fight over the spoils and resist any attempts to
measure their productivity or efficiency.
\switchcolumn
看上去解决办法很明显,应该改变激励机制,但说起来容
易做起来难。政府的服务并没有竞争者,因此很难判定政府工
作完成得好坏。假如每年有更多的人寄信,是不是就说明联邦
邮局服务客户的工作做得好呢?不一定,因为它的客户是被拴
住的,并没有其他选择。如果想寄一封信,他们就必须通过联
邦 邮 局 (除非他们愿意付至少10倍的价钱给私营快递服务)。
只要任何机构通过强制,即通过法定价格来挣钱,就很难
(如果不是不可能的话)衡量它在服务客户上是否成功。与此同时,体制内的特殊利益集团,如政客、政府管理人员、工会
等会为了这些战利品而争夺不休,并对任何企图评估他们生产
能力和效率的努力都坚决抵制。
\switchcolumn*
To see the self-interested nature of those in the state, just look
at any day's newspapers. Check out how much better the federal
employees' pension system is than Social Security. Look at the \$2
million pensions that will be collected by retiring members of
Congress. Note that when Congress and the president temporarily shut down the federal government, they kept on getting their paychecks while rank-and-file employees had to wait.
\switchcolumn
想知道在国家政权当中的人的自利本能是什么样的,只需
要打开任何一天的报纸就可以了。你可以看看联邦雇员的退休
金系统要比国民社保系统好上多少,看看国会议员退休可以拿
到 200万美元退休金的报道,请注意,如果国会和总统暂时停
止联邦政府的运作,他们自己仍然可以继续拿到薪水,而普通
的联邦雇员则不得不等待。
\switchcolumn*
Political scientist James L. Payne examined the record of 14
separate appropriations hearings, committee meetings where
members of Congress decide which programs to fund and by
how much. He found that a total of 1,060 witnesses testified, of
which 1,014 testified in favor of the proposed spending and
only 7 against (the remainder were not clearly for or against). In other words, in only half the hearings was there even \textit{one} witness
against the program. Congressional staff members confirmed
that the same was true in each member's office: The ratio of
people coming in to ask the congressman to spend money versus those who opposed any particular program was ``several
thousand to one.''
\switchcolumn
政治学家詹姆斯$\cdot$ 佩 恩(James L. Payne) 查阅了 14个不
同的财政拨款听证会和委员会会议记录,这些听证会和委员会
是国会议员们为审议对哪个项目进行拨款以及拨款多少而召开
的。他发现,总共1060名听证证人中,有 1014人赞成拨款提案 ,仅有7人反对(其他人没有清楚表明赞成还是反对)。换言之,一半的听证会上\textbf{只有一个人}反对计划,另一半则无人反对。议员办公室的国会工作人员也证实了这个结论:向议员要
求拨款的人与反对拨款计划的人的比例是 “ 几千比一” 。
\switchcolumn*
No matter how opposed to spending a new legislator may
be, the constant, day-in-and-day-out, year-in-and-year-out requests for money have an effect. He would increasingly say,
We've got to get spending down, but \textit{this} program is necessary.
Studies indeed show that the longer a person stays in Congress,
the more spending he votes for. That's why Payne called Washington a Culture of Spending, in which it takes almost superhuman effort to remember the general interest and vote against
programs that will benefit some particular person who visited
your office or testified before your committee.
\switchcolumn
不管一个议员新上任的时候是多么反对花钱,只要拨款要
求被日复一日、年复一年、不停地送到他面前,最后总会奏
效。他会说,我们必须让开支降下来,但是\textbf{这个}项目是必要
的。研究的确表明,一个议员在国会里呆得越久 , 他就越倾向
于支持花钱。这就是为什么佩恩称华盛顿为“拨款文化”。在
这里,你需要付出超人的努力才能想起大众的利益,投票反对
那些拨款计划,这些计划将让拜访过你办公室或者在你的委员
会上听证的人受益。
\switchcolumn*
About a century ago a group of brilliant Italian scholars set
out to study the nature of the state and its monetary affairs.
One of them, Amilcare Puviani, tried to answer this question: If
a government were trying to squeeze as much money as possible out of its population, what would it do? He came up with
eleven strategies that such a government would employ.
They're worth examining:
\switchcolumn
大约一个世纪以前,一群窨智的意大利学者开始对政府及
其财政系统的规律进行了研究。其中的一位,阿米卡列$\cdot$布维阿尼(Amicare Puviani)尝试着回答了这个问题:如果一个政
府试图从人民那里最大限度地把钱拿走,它会怎么做呢?他概
括了政府会采用的十一种策略,都很值得研究:
\switchcolumn*
\begin{itemize}
	\item The use of indirect rather than direct taxes, so that the tax
	is hidden in the price of goods
\end{itemize}
\switchcolumn
\begin{itemize}
	\item 使用间接税而不是直接税。把税收藏在商品价格背后。
\end{itemize}
\switchcolumn*
\begin{itemize}
	\item Inflation, by which the state reduces the value of everyone
	else's currency
\end{itemize}
\switchcolumn
\begin{itemize}
	\item 通货膨胀。国家通过通胀减少每个人现金的价值。
\end{itemize}
\switchcolumn*
\begin{itemize}
	\item Borrowing, so as to postpone the necessary taxation
\end{itemize}
\switchcolumn
\begin{itemize}
	\item 借款。通过政府债来推迟必要的征税。
\end{itemize}
\switchcolumn*
\begin{itemize}
	\item Gift and luxury taxes, where the tax accompanies the receipt or purchase of something special, lessening the annoyance of the tax
\end{itemize}
\switchcolumn
\begin{itemize}
	\item 礼品与奢侈品税。这是对接受礼品或者购买一些特殊的商品征收的税目,是为了减轻人们对征税的怒气。
\end{itemize}
\switchcolumn*
\begin{itemize}
	\item ``Temporary'' taxes, which somehow never get repealed when the emergency passes
\end{itemize}
\switchcolumn
\begin{itemize}
	\item “临时” 税。不知为何,当紧急状况消失的时候它也从来不被取消。
\end{itemize}
\switchcolumn*
\begin{itemize}
	\item Taxes that exploit social conflict, by placing higher taxes
	on unpopular groups (such as the rich, or cigarette smokers, or windfall profit makers)
\end{itemize}
\switchcolumn
\begin{itemize}
	\item 为缓和社会冲突而征税。主要通过对那些不受欢
	迎的群体征高额税收,例如对富人、吸烟者或者获得暴利
	和意外收入的人。
\end{itemize}
\switchcolumn*
\begin{itemize}
	\item The threat of social collapse or withholding monopoly
	government services if taxes are reduced
\end{itemize}
\switchcolumn
\begin{itemize}
	\item 威胁如果减税就会产生社会崩溃或者减少垄断的政府服务。
\end{itemize}
\switchcolumn*
\begin{itemize}
	\item Collection of the total tax burden in relatively small increments (a sales tax, or income tax withholding) over time,
	rather than in a yearly lump sum
\end{itemize}
\switchcolumn
\begin{itemize}
	\item 提高总体税收负担的时候,故意让营业税或所得
	税以较小的累进额度逐渐递增,而不是在一年当中突然
	增加。
\end{itemize}
\switchcolumn*
\begin{itemize}
	\item Taxes whose exact incidence cannot be predicted in advance, thus keeping the taxpayer unaware of just how
	much he is paying
\end{itemize}
\switchcolumn
\begin{itemize}
	\item 对那些事先无法预测自己税负的人征税,这样就让纳税人无法知道他付了多少税。
\end{itemize}
\switchcolumn*
\begin{itemize}
	\item Extraordinary budget complexity to hide the budget
	process from public understanding
\end{itemize}
\switchcolumn
\begin{itemize}
	\item 让预算变得复杂无比,超出公众的理解能力,以
	掩盖预算的炮制过程。
\end{itemize}
\switchcolumn*
\begin{itemize}
	\item The use of generalized expenditure categories, such as ``education'' or ``defense,'' to make it difficult for outsiders to
	assess the individual components of the budget
\end{itemize}
\switchcolumn
\begin{itemize}
	\item 使用通用的开支名目,如 “教育”、 “ 国防”等
	等,让外面的人难以评估预算的各个部分。
\end{itemize}
\switchcolumn*
Notice anything about this list? The United States government uses every one of those strategies---and so do most foreign
governments. That just might lead a cynical observer to conclude that the government was actually \textit{trying} to soak the taxpayers for as much money as it could get, rather than, say,
raising just enough for its essential functions.
\switchcolumn
从这个单子当中发现了什么吗?美国政府使用了上面所有的这些诡计 --- 其他国家的政府也是一样。这就难免让冷嘲热
讽的观察者得出结论:政府实际上是在\textbf{竭力}榨取纳税人的钱,
能榨取多少就榨取多少,而不是征收刚好够它的基本职能需要
的钱。
\switchcolumn*
In all these ways, government's constant instinct is to grow,
to take on more tasks, to arrogate more power to itself, to extract more money from the citizenry. It seems that Jefferson was
right: ``The natural progress of things is for liberty to yield and
government to gain ground.''
\switchcolumn
所有这些都说明,政府的本能就是扩张,获得更多的职
能,把更多的权力攥在手中,从公民手里抽取更多的钱。看起
来杰斐逊是对的:“ 自然发展的趋势就是自由不断让步,政府
步步进逼。”

\switchcolumn*[\subsection{Big Government and Its Court Intellectuals\\大政府及其御用知识分子}]
The power of the state has always rested on more than just laws
and the might to back them up, of course. It's much more efficient to persuade than to force people to accept their rulers.
Rulers have always employed priests, magicians, and intellectuals to keep the people content. In ancient times, priests assured
the people that the king was himself divine; as recently as
World War II the Japanese people were told that their emperor
was directly descended from the sun.
\switchcolumn
当然,国家的权力不仅依靠法律和其能力来支撑。用武力
迫使人们接受他们的统治不如用说服的方式更有效率。统治者
常常用祭司、魔法师和知识分子来让人们同意他们的统治。古
代 ,祭司让人们相信国王本身就是神;最近,直到第二次世界
大战日本人还被告知天皇是太阳的后裔。
\switchcolumn*
Rulers have often given money and privilege to intellectuals
who would contribute to their rule. Sometimes these court intellectuals actually lived at court, participating in the luxurious
life that was otherwise denied to commoners. Others were appointed to high office, ensconced at state universities, or funded
by the National Endowment for the Humanities.
\switchcolumn
统治者常常给那些愿意为他们的统治作贡献的知识分子赏
赐金钱和授予特权。有时候这些御用知识分子实际上就住在宫
廷里,享受着平民无缘参与的奢华生活。另一些则被任命为高
级官员,或在国立大学当中任职,或者接受国家人文学科基金
的资助。
\switchcolumn*
In the post-Enlightenment world, ruling classes have realized that divine ordinance would not be sufficient to maintain
their hold on popular loyalty. They have thus tried to ally themselves with secular intellectuals from painters and scriptwriters
to historians, sociologists, city planners, economists, and technocrats. Sometimes the intellectuals had to be wooed; sometimes they were positively eager to glorify the state, as did the
professors at the University of Berlin in the nineteenth century,
who proclaimed themselves ``the intellectual bodyguard of the
House of Hohenzollern'' (that is, the rulers of Prussia).
\switchcolumn
启蒙运动之后,统治阶级意识到,依靠君权神授的理论已
不足以保持被统治臣民的忠诚。于是他们就开始与世俗知识分
子结盟,从画家、剧作家到历史学家、社会学家、城市规划
师、经济学家和高级技术人员。有的知识分子需要统治者来拉
拢 ,有的则热血沸腾地主动要求为国效力,就 像 19世纪桕林
大学的教授,他们宣称:“我们都是霍亨索伦王室(普鲁士的
统治者)的知识保镖。”
\switchcolumn*
In modern America, for at least two generations, the majority of intellectuals have told the populace that an ever bigger
state was needed---to deal with the complexity of modern life,
and to help the poor, and to stabilize the business cycle, and to
enhance economic growth, and to bring about racial justice,
and to protect the environment, and to build mass transit, and
for numerous other purposes. Coincidentally, that ever bigger
state has meant ever more jobs for intellectuals. A minimal
government, one that would, in Jefferson's words, ``restrain
men from injuring one another {and} leave them otherwise free
to regulate their own pursuits of industry and improvement,''
would have little use for planners and model builders; and a
free society might not evidence much demand for sociologists
and urban planners. Thus, many intellectuals are simply acting
in their class interest when they churn out books and studies
and movies and newspaper articles on the need for bigger government.
\switchcolumn
在现代美国,至少两代人的时间里,大多数知识分子告诉
大众必须有一个更大的政府来应付现代社会的复杂性、帮助穷
人 、稳定经济周期、促进经济增长、促进种族平等、保护环
境 、建立大规模转移支付系统等,还有无数其他的目标。凑巧
的是,更大的政府意味着有更多的职位需要知识分子来做。而
一个最小程度的政府,用杰斐逊的话来说就是,“它会制止人
们相互伤害,让他们自由管理自己的实业和改善自己的状
况”。计划大师和造模大师\footnote{造模大师(model builder) , 指善于为各种社会问题构造数学模型的经济学家、政治学家、社会学家等。}们在这样的小政府里基本上没有用武之地。因此,当许多知识分子为了大政府的需要而炮制出一大堆书籍、研究报告、电影和媒体文章的时候,他们只是为了自己阶层的利益。
\switchcolumn*
Don't be fooled, by the way, by the supposedly ``irreverent''
and ``antiestablishment'' and even ``antigovernment'' stances of
many modern intellectuals, even some of those funded by the
state itself. Look closely, and you'll see that the ``establishment''
they oppose is the capitalist system of productive enterprise,
not the leviathan in Washington. And in their brave criticisms
of government, they generally chide the state for doing too little or mock the elected officials who are trying halfheartedly to
respond to public demand for less government. The provocative
documentaries on the State (oops, ``Public'') Broadcasting System's \textit{Frontline} and \textit{P.O.V.} usually indict the American state for
its inaction. What ruling class wouldn't be glad to subsidize dissident intellectuals who constantly demand that the ruling
class expand its scope and power?
\switchcolumn
顺便提一下,大家千万不要被一些现代知识分子的所谓
“批评的” “反主流的” 甚 至 是 “反政府的” 姿态所迷惑。他
们中的一些人甚至是得到政府直接资助的。仔细看看他们的观
点 ,你会发现,他 们 所 反 对 的 “主流”是资本主义制度及其
生产企业,而不是华盛顿的利维坦\footnote{利维坦(leviathan),指权力巨大的大政府。原意是《圣经》中象征邪恶的海中怪兽。英国17世纪哲学家霍布斯在其代表作《利维坦》 中以这个名词比喻权力集中的国家政权。.从此,利维坦就成了大政府、中央集权国家的代名词。}。而他们对政府勇敢无畏的批评,常常是怒斥国家\textbf{做得太少}了,或者是挖苦那些民选官员竟然半推半就地迎合公众要求小政府的呼声。在 “国家广播电视台” (哦,搞错了,应 该 是 “公共广播电视台”\footnote{公共广播电视台(Public Broadcasting System, PBS) 这里作者故意把“公共”(Public)一词换成 “国家” (State) , 就成了 “ 国家广播电视台”。})的“前线” 和 “观点”节目播放的煽动性纪录片就常常指责美国政府不作为。统治阶层对这些不断要求统治者扩大权力和范围的 “异议知识分子” 怎么会不热心资助呢?  
\switchcolumn*
Court intellectuals are not simply corrupt, of course. Many of
them genuinely believe that a permanently growing state is in
the public interest. Why is that? Why did European and American intellectuals turn from the courageous and visionary libertarianism of Milton and Locke and Smith and Mill to a crabbed
and reactionary statism---of Marx, of course, but also of T. H.
Green, John Maynard Keynes, John Rawls, and Catharine
MacKinnon? One answer we've already examined: The state
moved to co-opt them and make them its handmaidens, with
access to some of the perks of power. But that's not the whole
answer. Many distinguished scholars have tried to fathom the
great attraction statism and planning holds for intellectuals.
\switchcolumn
当然,不能简单地说这些御用知识分子就是道德败坏。他
们当中很多人真诚地相信一个不断扩张的国家政权是符合公众
利益的。为什么会这样?为什么欧洲和美国的知识分子会从弥
尔顿、洛克、斯密和密尔那勇敢的而富有远见卓识的自由主义
转向晦涩的、倒退的国家主义 --- 不但有马克思的,还有格
林\footnote{格林(Thomas Hill Green, 1836$\sim$1882),英国哲学家、政治思想家、伦理学家,“新自由主义” 政治思想的先驱。他的政治思想主要体现在《伦理学绪论》和 《政治义务原理讲演录》 中。格林反对以国家权力的减弱程度作为判断个人自由增长与否的标准。主张国家发挥积极、主动的作用。认为只有增强国家权力 ,为全体成员的共同善提供保证,扩大国家干预范围,压制可能侵害个人自由的行为,才能促进个人能力的发挥和自由的增长。因此提出,人是道德的存在物,国家干预十分必要。人们对国家限制的忍受是“真正自由的第一步”。格林把人的权利视为国家与社会对其成员的一种承认和让步。权利不是天陚的。}、凯恩斯、罗尔斯以及凯瑟琳$\cdot$ 麦金农\footnote{凯瑟琳$\cdot$麦金农(Catharine Mackinnon, 1949$\sim$), “激进主义女权主义” 的代表人物,著名的律师和女权主义活动家,长期致力于有关性别平等和妇女权的诉讼、立法和政策发展。她在 1979年的著作中最早提出了 “性骚扰”概念和对此进行法律诉讼的主张,为这一领域的立法奠定了基础。麦金农等人曾在美国展开声势浩大的反色情出版物运动,以性别歧视为由要求立法判定色情出版物不受第一修正案的“言论自由”保护,但最后反色情出版法仍被最髙法院裁定违宪。}的?其中一个原因我们前面提过:有的人是国家选择了他们,让他们成为国家的婢女,让他们能够被恩准获得一些权力。但这并不是全部答案。很多卓越的学者的确是在探索将有强大吸引力的国家主义和计划掌握在知识分子手里的可能性。
\switchcolumn*
Let me suggest at least a few reasons. First, the idea of planning has great appeal for intellectuals because they like to analyze and to put things in order. They are enthusiastic builders of
systems and models, models by which the builder can measure
reality against an ideal system. And if an individual or a business profits by planning a course of action, shouldn't the same
be true for a whole society? Planning, the intellectual believes,
is the application of human intelligence and rationality to the
social system. What could be more appealing to an intellectual,
whose stock in trade is his intelligence and rationality?
\switchcolumn
这里我认为至少有以下几个原因: 一 、 计划的理念对知识
分子有强烈的吸引力,他们喜欢分析和让事情有秩序。他们对
建构制度和模型很有热情,通过构建模型他们就能用理想制度来衡量真实世界。如果个人或者公司的利润是通过对行动程序
的计划而得来的,为什么计划对整个社会来说就不行呢?知识
分子相信,计划是人类智慧和理性在社会制度上的应用。当他
可用于交易的资本就是智慧和理性的时候,还有什么比计划更
能吸引一个知识分子呢?
\switchcolumn*
Intellectuals have devised all sorts of planning systems for
states, especially in the twentieth century, with its explosion in
knowledge and in demand for intellectuals. Marxism was the
great comprehensive plan for all of society, but its very comprehensiveness frightened many people. A close cousin was fascism, a system that proposed to leave productive resources in
private hands but to coordinate them according to a central
plan. In his book \textit{Fascism: Doctrine and Institutions}, Benito Mussolini, who ruled Italy from 1922 until 1943, presented fascism
as a direct response to individualist liberalism:
\switchcolumn
特别是在20世纪,随着知识爆炸和对知识分子需求的猛
增,知识分子为国家发明了所有形式的计划体制。法西斯主义
主张生产性资源留在个人手里,但是需要通过中央计划来协
调。1922$\sim$1943年间,统治意大利的墨索里尼在他的一本书
《法西斯主义:宗旨与制度》(\textit{Fascism: Doctrine and Institu­tions}) 中展示了法西斯主义的内容,主要针对个人主义、自由主义:
\switchcolumn*
\begin{quote}
It is opposed to classical liberalism, which arose as a reaction to
absolutism and exhausted its historical function when the State
became the expression of the conscience and will of the people.
Liberalism denied the State in the name of the individual; Fascism reasserts the rights of the State as expressing the real essence
of the individual.
\end{quote}
\switchcolumn
\begin{quote}
法西斯主义是古典的自由主义的对立面。 自由主义崛
起于对专制主义的反抗,但是当国家成为良知象征和人民
意愿的时候,自由主义就完成了它的历史使命。 自由主义
以个人的名义否定国家;法西斯主义则再度重申作为反映
了个人真正本质的国家的权力。
\end{quote}
\switchcolumn*
In the 1930s fascism was much admired by some American
intellectuals, who despaired of bringing such a rational system
to the still individualist United States. The Nation, by then a socialist magazine, found ``the New Deal in the United States, the
new forms of economic organization in Germany and Italy, and
the planned economy of the Soviet Union'' all signs of a ten-
dency ``for nations and groups, capital as well as labor, [to] demand a larger measure of security than can be provided by a
system of free competition.'' After fascism was discredited by its
association with Hitler and Mussolini, statist intellectuals came
up with new names for central planning in a system of officially
private property: the French ``indicative planning'' of the 1960s,
the ``national economic planning'' proposed by economist
Wassily Leontief and labor leader Leonard Woodcock in the
1970s, the ``economic democracy'' of Tom Hayden and Derek
Shearer, the reindustrialization policy of Felix Rohatyn and
Robert Reich, and the ``competitiveness'' policy also touted by
Reich. As each variant was discredited, intellectuals moved on
to another name and a superficially different plan. But each one
involved the state hiring intellectuals, who would rationally determine what society needed and direct everyone's economic activities accordingly.
\switchcolumn
20世纪30年代,法西斯主义被一些美国知识分子所推
崇,但他们对不能把这样一个理性制度带到仍然盛行个人主义
的美国感到失望。当时还是社会主义倾向的《国家》杂志发
现,“美国的新政、德国和意大利的新型经济组织形式以及苏
联的计划经济” 是一种趋势的信号,这 种 趋 势 就 是 “国家和
各种社会利益集团、资本家和劳动者要求自由竞争制度所不能
提供的更大程度的社会保障”。法西斯主义因为与希特勒和墨
索里尼相联系而破产之后,国家主义知识分子就为在法定体制
仍为私人财产权制度的情况下为中央计划寻找新名字:如法国I960年 代 提 出 的 “指导性计划” 、1970年代经济学家列昂惕夫\footnote{瓦西里$\cdot$列 昂 惕 夫(Wassily Leontief, 1905$\sim$1999) , 俄罗斯裔经济学家,后移居美国,任教于哈佛大学,1973年 因 “投人产出理论”而获得诺贝尔经济学奖。投人产出分析为研究社会生产各部门之间的相互依赖关系,特别是系统地分析经济内部各产业之间错综复杂的交易提供了一种方法。}和工会领导人伍德科克\footnote{伦纳德$\cdot$伍德科克(Leonaixl Woodcock, 1911$\sim$2001) , 美国工会领导人和外交家,曾任美国驻中华人民共和国第一任大使。}提 出 的 “国家经济计划” 、汤 姆 $\cdot$海登和德勒克$\cdot$ 希尔提出的“经济民主” 、费利克斯$\cdot$罗哈廷和罗伯特$\cdot$ 赖克提出的“再工业化政策” ; “竞争性”政策也被赖克所扭曲。在每一种改头换面的国家主义理论破产之后,
知识分子们都会炮制出下一种,或者炮制出各种表面上看上去
互不相同的计划经济。但每一种理论都和国家雇用的知识分子
有关。他们试图理性地判定社会需要的东西,并据此来指导每
个人的经济活动。
\switchcolumn*
Despite the growing disillusionment with big government,
the Holy Grail of planning dies hard among intellectuals. What
was the Clinton health-care proposal but a central plan for one-
seventh of the American economy? And that wasn't the only
example of President Clinton's fascination with planning. In a
little-noted comment during the 1992 campaign, Clinton offered a breathtaking view of the ability and obligation of government to plan the economy:
\switchcolumn
尽管大政府的幻想不断破灭,但 计划 经济的圣 杯(Holy Grail)很难在知识分子当中泯灭。克林顿的医保计划难道不
是在对美国经济的七分之一进行计划吗?这还不是克林顿总统
迷恋计划经济的唯一例子。在 1992年选举的一次很少被人提
及的讲话中,克林顿对政府计划经济的能力和责任的观点让人
不禁倒吸一口凉气:
\switchcolumn*
\begin{quotation}
We ought to say right now, we ought to have a national inventory of the capacity of every $\ldots$ manufacturing plant in the
United States: every airplane plant, every small business subcontractor, everybody working in defense.

We ought to know what the inventory is, what the skills of the
work force are and match it against the kind of things we have to produce in the next 20 years and then we have to decide how to
get from here to there. From what we have to what we need to do.
\end{quotation}
\switchcolumn
\begin{quotation}
现在我们不得不说,必须有一个全国性的目录清单,
包括美国所有工厂的生产能力:所有的飞机制造厂、所有
的小分包商以及所有在国防工业中工作的人的名单。

我们应该了解这个清单的内容,了解工人的技术水
平,将其与我们在未来二十年当中所需要产品的相应生产
能力进行比较,然后我们必须决定如何达到这种能力,从现有的能力达到应有的能力。
\end{quotation}
\switchcolumn*
After the election, a White House aide named Ira Magaziner
fleshed out this sweeping vision: Defense conversion would require a twenty-year plan developed by government commit-
tees, ``a detailed organizational plan $\ldots$ to lay out how, in
specific, a proposal like this could be implemented.'' Five-year
plans, you see, had failed in the Soviet Union; maybe a twenty-year plan would be sufficient to the task.
\switchcolumn
大选结束之后,一 名 叫 做 艾 拉 $\cdot$ 马格辛那(Ira Magaziner)的白宫髙级助理拋出了他那横扫一切的伟大蓝图:需要
一个国防装备改装的二十年计划,这个计划应由政府委员会来
制定,“同时需要一份详细的成立相关机构的计划,以确保这
个规划的实施。” 五年计划在苏联遭到了失败,这个大家都看
到了,不过没准二十年计划会成功吧。
\switchcolumn*
A second reason that intellectuals are attracted to state
power is what Thomas Sowell calls their unconstrained view of
man, the view that there are no natural limits to building a
Utopia on earth. This perspective is understandable in the late
twentieth century, after two centuries of the most rapid advances in knowledge, life expectancy, and standard of living
ever witnessed on earth. The attitude is summed up in the popular catchphrase, ``If we can put a man on the moon, why can't
we $\ldots$ cure cancer, end racism, pay teachers more than movie
stars, stop pollution?'' After all, human ingenuity over the past
200 years has moved us from a life that was ``nasty, brutish, and
short'' to a society that has conquered many age-old diseases,
dramatically reduced the barriers to travel, and vastly increased
the store of knowledge. But these achievements were not just
willed into being; they took effort, physical and intellectual,
and they occurred in a social system largely based on the rule of
law, private property, and individual freedom.
\switchcolumn
知识分子被国家权力所吸引的另一个原因是托马斯$\cdot$索维
尔所说的人类的那种“心想事成” 心理:认为在地球上建立
乌托邦不会受到任何自然条件的限制。20世纪后期,在知识、
预期寿命和生活水平经历了长达两个世纪的巳知最快速的增长
之后,这种观点完全是可以理解的。这种观点在下面这句流行
广告语中被集中体现:“如果我们能把人类送上月球,为什么
我们就不能治愈癌症、终止种族主义、让教师的收人超过电影
明星、停止环境污染呢?” 毕竟,在过去二百年中,人类的智慧
让我们从“肪脏、粗鄙和寿命短暂”的生活状态进步到征服了
无数顽固的古老疾病,大幅降低了旅行门檻,极大提高了知识
储备的社会。但是这些成就都不是靠心想事成的许愿得来的,
而是经过努力求知和辛勤劳动才成为现实的,而且这些成就依
赖于建立在法治、私人产权和个人自由基础之上的社会制度。
\switchcolumn*
The vulgar version of the unconstrained view of man can be
seen in a bumper sticker I spied in my Washington neighbor-
hood: ``Demand a cure for AIDS.'' Well, of course; how cruel of
$\ldots$ corporations or society or the government or whomever $\ldots$
not to give us the cure for AIDS. Let's demand it. If we can put
a man on the moon, we can find a cure for AIDS.
\switchcolumn
在华盛顿我居住的社区流行的汽车贴纸\footnote{美国人喜欢在汽车尾箱或者保险杠上面贴几张花花绿绿的小纸片,或表明立场,或耍耍小幽默,这种小贴纸叫汽车贴纸(bumper sticker),是美国汽车文化里面比较有意思的东西。}中就有这种“心想事成” 观的民间版: “要求治愈艾滋病”。当然,企业、社会或者政府,或者其他甭管什么人不给我们治愈艾滋病,这多残酷啊。让我们来要求吧。如果我们能把一个人送上月球,我们就应该有治愈艾滋病的办法。
\switchcolumn*
The more sophisticated exponents of the unconstrained view
would laugh at such a naive version; they are intellectuals, after
all. But they, too, fail to understand the limits on human
knowledge that prevent us from solving all problems at once,
the trade-offs that are ignored in the sweeping plans they promulgate.
\switchcolumn
“心想事成” 更高级的拥护者们肯定会对天真版的“心想
事成” 大笑不已,因为前者毕竟是知识分子。但是,在鼓吹
彻底的社会解决计划时,他们也没有明白,人类由于知识不足
不可能在一夜之间解决所有问题,并忽视了无可避免的权衡
取舍。
\switchcolumn*
Finally, the libertarian vision of a free society seems, to many
people, essentially irrational because society is supposed to be
left to its own devices. Karl Marx, a brilliant if profoundly
wrong scholar, complained about ``the anarchy of capitalist production.'' Indeed, it seems that way. In a great society, millions
of people go about their daily routines according to no central
plan. Every day some businesses start and others fail, people are
hired and others are fired. At this very moment several different
companies are developing similar or even identical products to
offer to consumers: Internet web-browsers, perhaps, or roast-chicken restaurants, or drugs to relieve stress on the heart.
Wouldn't it make more sense to have a central authority pick
one company to do each project, and to make sure all companies are putting resources into truly important tasks rather than
Rap Star Barbie or new colors for Chevrolets? No, it wouldn't---
and that's what is so hard for intellectuals to see. The market
process coordinates economic activity much better than any
plan ever could. In fact, that sentence dramatically understates
the comparison. No plan could give us the standard of living we
have today. \textit{Only} the apparently chaotic market process can coordinate the desires and abilities of thousands, millions, billions
of people in order to produce a continually higher living standard for the whole society.
\switchcolumn
最后,在很多人看来,古典自由主义对自由社会的构想基
本上是荒谬的,因为社会应当按照其自己的意志来运行。马克
思 对 “资本主义生产的无政府状态” 进行了谴责。事实上,
他说得没错。在一个庞大社会里,数以百万计的人们确实是在
没有中央计划的情况下进行日常工作的。每天都有一些企业开
张,一些企业关张,一些人找到工作,一些人丢掉工作。就在
这一刻,每一种近似的甚至相同的产品都有几家不同的公司在
开发以提供给消费者:几种网页浏览器、几家烤鸡肉餐馆或者
几种缓解心脏紧张的药物。那么同一种产品由中央权威选择一
家公司来开发难道不是更合适一些吗?或者让所有公司都把资
源投入到真正重要的项目当中,而不是搞一些诸如打扮得像说
唱歌星一样的芭比娃娃或者雪佛兰汽车的新颜色之类的项目不
是更好吗?不,当然不是 --- 这一点恰恰是那些知识分子很难
理解的。市场经济协调经济行为,比任何类型的计划所要达到
的程度都要好得多。实际上这样说巳经是打了很大的折扣了。
没有任何计划能够提供今天这种水准的生活。\textbf{只有}看上去混乱
的市场经济才能够协调数以千计、百万计乃至数以十亿计的人
的需要和能力,从而为整个社会创造持续的高生活水平。
\switchcolumn*
The inability to see this results in what F. A. Hayek called the
Fatal Conceit---the idea that smart people could plan an economic system that would be better than the unplanned, anarchic market. It is a remarkably persistent notion.
\switchcolumn
看不到这个结果的人,就是犯了哈耶克所说的“致命的
自负”。所 谓 “致命的自负” 指的是一种理念,认为聪明人能够设计出一种比无计划、无政府的市场经济更好的经济体制。
这是一种相当顽固的理念。

\switchcolumn*[\section{The State and War\\国家与战争}]
The apotheosis of state power is war. In war the state's force is
not hidden or implicit; it is vividly on display. War creates a hell
on earth, a nightmare of destruction on an otherwise unimaginable scale. No matter how much hatred people may sometimes
feel for other groups of people, it's difficult to conceive why nations have chosen so often to go to war. The calculation of the
ruling class may be different from that of the people, however.
War often brings the state more power, by drawing more people under its control. But war can enhance state power even in
the absence of conquest. (Losing a war, of course, can topple a
ruling class, so making war is a gamble, but the payoff is good
enough to attract gamblers.)
\switchcolumn
国家权力的顶峰就是战争。在战争中,国家暴力不再被隐
藏和掩饰,而是清清楚楚地显示了出来。战争以其他时候难以
想像的规模创造出一种人间地狱和毁灭的梦魇。不论人们有时
候会对另一群人感到如何愤怒,我们都很难想像为什么国家会
这样频繁地选择战争。无论如何,统治阶层的想法和人民是不
一样的。战争常常通过把更多的人绑上战车而给国家带来更大
的权力。甚至在没有征服和被征服发生的时候,战争也会加强
国家权力。当然,战争失败也会动摇或者颠覆统治阶层,因此
战争也是一种赌博,但是这种赌博的高额回报已经足以吸引那
些赌徒了。
\switchcolumn*
Classical liberals have long understood the connection between war and state power. Thomas Paine wrote that an observer of the British government would conclude ``that taxes
were not raised to carry on wars, but that wars were raised to
carry on taxes.'' That is, the English and other European governments gave the impression of quarreling in order ``to fleece
their countries by taxes.'' The early twentieth-century liberal
Randolph Bourne wrote simply, ``War is the health of the
State''---the only way to create a herd instinct in a free people
and the best way to extend the powers of government.
\switchcolumn
古典自由主义者很久以前就知道战争和国家权力之间的关
系。潘恩写道:一个对英国政府进行观察的人会得出结论,
“不是税收为战争而征,而是战争为征税而打”。也就是说,
英国和其他欧洲政府给人的印象是他们之间的互相争斗只是为
了 “骗取他们国家加税”。20世纪早期的自由主义者伦道
夫 $\cdot$ 伯恩\footnote{伦道夫$\cdot$伯恩(Randolph  Silliman Bourne, 1886$\sim$1918) , 美国作家和公共知识分子。20世纪初期被称为“青年反叛第一文化英雄”。}说得很简洁: “战争就是国家的生命” --- 这是唯一能够在自由人当中制造出羊群心理\footnote{羊 群 心 理(herd instinct),有时也称为羊群效应,指追随大众的想法及行为 ,缺乏自己的个性和主见的行为状态。}的方式,也是扩张国家权力的最佳途径。
\switchcolumn*
U.S. history provides ample evidence of that. The great
leaps in federal spending, taxation, and regulation have occurred during wartime---first, notably, the Civil War, then
World War I and World War II. War threatens the survival of
the society, so even naturally libertarian Americans are more
willing to put up with state demands at such a time---and
courts agree to sanction unconstitutional extensions of federal
power. Then, after the emergency passes, the government neglects to give up the power it has seized, the courts agree that
a precedent has been set, and the state settles comfortably into
its new, larger domain. During major American wars, the federal budget has gone up ten- or twenty-fold, then fallen after
the war, but never to as low a level as it was before. Take World
War I, for example: Federal spending was \$713 million in
1916 but rose to nearly \$19 billion in 1919- It never again fell
below \$2.9 billion.
\switchcolumn
对此,美国的历史给我们提供了丰富的案例。联邦政府在
政府开支、税收和管制方面的大跃进都发生在战争时期 --- 首
先就是南北战争,然后是第一次世界大战和第二次世界大战。
战争威胁着市民社会的生存,因此,即使天生是古典自由主义
者的美国人在战争中也能够忍受国家的各种要求,最高法院也
会批准违宪的联邦权力的扩张。而在紧急状态结束之后,政府
却不放弃抓到手的权力,最高法院也会承认之前的判例成为政
府的正式权力,于是国家就舒舒服服地掌控了新的更大的领
域。美国历史上的历次重大战争中,联邦政府的预算都会上升
到战前的10$\sim$20倍 ,战争结束之后会降下来,但是绝不会降
到战前的水平。以第一次世界大战为例:1916年联邦政府支
出是7. 13亿美元,1919年上升到将近190亿美元。此后则再
也没有低于29亿美元。
\switchcolumn*
It isn't just money, of course. Wartime has occasioned such
extensions of state power as conscription, the income tax, tax
withholding, wage and price controls, rent control, censorship,
crackdowns on dissent, and Prohibition, which really began
with a 1917 statute. World War I was one of the great disasters
of history: In Europe it ended ninety-nine years of relative peace
and unprecedented economic progress and led to the rise of
communism in Russia and Nazism in Germany and to the even
greater destruction of World War II. In the United States the consequences were far less dramatic but still noteworthy; in two
short years President Woodrow Wilson and Congress created
the Council of National Defense, the United States Food Ad-
ministration, the United States Fuel Administration, the War
Industries Board, the Emergency Fleet Corporation, the United
States Grain Corporation, the United States Housing Corporation, and the War Finance Corporation. Wilson also nationalized the railroads. It was a dramatic leap toward the megastate
we now struggle under, and it could not have been done in the
absence of the war.
\switchcolumn
当然,被拿走的不仅仅是钱。战争还引起了国家权力的扩
张。征兵、征收所得税、税收扣押、工资和价格管制、租金管
制、新闻出版审查、打击持不同政见者以及禁酒令等都开始于
1917年的一系列法令。第一次世界大战是人类历史上最大的
灾难之一:在欧洲它结束了长达99年的相对和平和史无前例
的经济增长,导致了共产主义在俄国、纳粹主义在德国的崛
起,并且还导致了更大规模的毁灭性战争 --- 第二次世界大
战。在美国,战争的影响远没有那么戏剧性,但是仍然值得一
提;在短短的两年时间里,伍德罗$\cdot$威尔逊总统和国会建立了
国防委员会、美国食品管理局、美国石油管理局、军事工业委
员会、紧急舰队公司(Emergency Fleet Corporation)、美国粮
食公司、美国房屋公司、战争信贷公司等机构。威尔逊还把铁
路国有化。这是朝向我们现在正与之斗争的这个庞大政府迈出
的戏剧性的一大步,而且这种权力扩张也不可能在战争结束之后就被降下来。
\switchcolumn*
Statists have always been fascinated by war and its possibilities, even if they sometimes shrink from the implications. The
rulers and the court intellectuals understand that free people
have their own concerns---family and work and recreation---
and it's not easy to get them enrolled voluntarily in the rulers'
crusades and schemes. Court intellectuals are constantly calling
for a ``national effort'' to undertake some task or other, and
most people blithely ignore them and go on about the business
of providing for their families and trying to build a better
mousetrap. But in time of war---\textit{then} you can organize society
and get everyone dancing to the same tune. As early as 1910,
William James came up with the idea of ``The Moral Equivalent
of War,'' in an essay proposing that young Americans be conscripted into ``an army enlisted against Nature'' that would
cause them to ``get the childishness knocked out of them, and
to come back into society with healthier sympathies and soberer
ideas.''
\switchcolumn
国家主义者常常醉心于战争以及战争带来的各种可能性,
尽管他们有时会避免卷入进去。统治者和御用知识分子们知道
自由的人们都有自己的考虑 --- 家庭、工作和娱乐,很难让他
们自愿加入到统治者的十字军和战争计划当中。御用知识分子
们不断呼吁“全国人民团结起来”,共同承担什么任务或者做
什么事情,但是大多数人都不会听他们的,依然继续为他们的
家人而工作,为自己和家人的明天而做准备。但当战争来临的
时候,就可以把整个社会组织起来,\textbf{然后}让每个人按照同样的节拍跳舞了。早在 1910年,威廉 $\cdot$ 詹姆斯\footnote{威廉$\cdot$詹姆斯(William  James, 1842$\sim$1910) , 美国实用主义哲学家及机能心理学的先驱,其意识流说为批判心理学元素主义的先声,情绪说则预示20世纪行为主义的诞生,在美国心理学史中,特别是在理论上有重要贡献。主要著作有 《心理学原理》《多元的宇宙> 《真理的意义》 《战争的道德等效物》。}提出了 “战争的道德等效物”(The  Moral Equivalent of War)概念。他在一篇文章中建议年轻的美国人应征入伍, “军队会让人们摆脱自然状态” ,那将会让他们“把身上的幼稚习气赶走,让他们回到社会时有着更健康的同情心和更冷静的思想”。
\switchcolumn*
The fascination of collectivists with war and its ``moral equivalent'' is undying. In 1977 President Carter revived James's
phrase to describe his energy policy, with its emphasis on government direction and reduced living standards. It was to be his
peacetime substitute for the sacrifice and despotism of war. In
1988 the Democratic Leadership Council proposed an almost-
compulsory national service program, which would entail ``sacrifice'' and ``self-denial'' and revive ``the American tradition of
civic obligation.'' Nowhere in the DLC paper on the subject was
there any mention of the American tradition of individual
rights. The proposal was described as a way to ``broaden the political base of support for new public initiatives that otherwise would not be possible in the current era of budgetary restraint.''
In other words, it would be a way for government to hand out
benefits by enlisting cheap, quasi-conscript labor. The last
chapter of the paper was, inevitably, titled ``The Moral Equivalent of War.''
\switchcolumn
集体主义者们对战争及其“道德等效物” 的迷恋一直都
没有消失。1977年 ,卡特总统引用了詹姆斯的这段话来描述
他的能源政策,强调遵从政府指令和要求大家降低生活标准。
这是战时的牺牲奉献和专制在和平时期的替代品。1988年,
民主党领导委员会提出了一个近似强制义务的国家服务计划,
它要求人们进行“奉献” 和 “牺牲” ,恢 复 “美国的公民责任
传统” ,对作为美国传统的个人权利却只字未提。而这份提案
被描述为一种方法,目 的 在 于“扩大支持新的公共事业的政
治基础,做一些在当今时代财政预算有限的条件下不可能做的事情”。换句话说,就是政府通过使用廉价的类似于征兵式的
劳动力来获得利益。这个文件最后一章的题目恰恰就是“战
争的道德等效物”。
\switchcolumn*
Then, in 1993 DLC chairman Bill Clinton became president
and proposed his own national service plan, and darned if it
didn't sound a lot like ``the moral equivalent of war.'' He
wanted to ``rekindle the excitement of being Americans'' and
``bring together men and women of every age and race and lift
up our nation's spirit'' to ``attack the problems of our time.''
Eventually, perhaps, every young person would be enlisted. For
the moment, however, the president envisioned ``an army of
100,000 young people $\ldots$ to serve here at home $\ldots$ to serve
our country.''
\switchcolumn
然后,1993年 ,民主党主席比尔$\cdot$ 克林顿成了美国总统。
他提出了自己的国家服务计划,但进行了一些修饰,让它看上
去不那么像一个“战争的道德等效物”。他 想 “重新燃起作为
美国人的振奋之情”,以 及 “不同年龄、不同种族的男人和女
人团结在一起,提升我们民族的精神” ,来 “与我们这个时代
的各种问题作战”。也许,每个年轻人都将逐步被征召入这个
服务计划。而在当前,总统展望道: “10万青年 10万军……
就在这里为我们的家乡,为我们的国家服务。”
\switchcolumn*
In 1982 British Labour Party leader Michael Foot, a distinguished leftist intellectual, was asked for an example of socialism
in practice that could ``serve as a model of the Britain you envision,'' and he replied, ``The best example that I've seen of democratic socialism operating in this country was during the second
world war. Then we ran Britain highly efficiently, got everybody
a job$\ldots$ The conscription of labor was only a very small element of it. It was a democratic society with a common aim.''
\switchcolumn
英国工党领袖弗特(Michael Foot), ---个典型的左派知识
分子,1982年有人问他能不能举一个“你认为可以作为英国
社会主义实践典范” 的例子。他回答道:“这个国家关于民主
社会主义的运行我所看到的最好例子就是第二次世界大战时
期。那时候英国整个国家运转高效,人人都有工作……对劳动
力的征用只是其中非常小的一个因素。那是一个有着共同目标
的民主社会。”
\switchcolumn*
The American socialist Michael Harrington wrote, ``World
War I showed that, despite the claims of free-enterprise ideologues, government could organize the economy effectively.''
He hailed World War II for having ``justified a truly massive
mobilization of otherwise wasted human and material resources'' and complained that the War Production Board was
``a success the United States was determined to forget as
quickly as possible.'' He went on, ``During World War II, there
was probably more of an increase in social justice than at any
(other) time in American history. Wage and price controls were
used to try to cut the differentials between the social
classes$\ldots$ There was also a powerful moral incentive to spur
workers on: patriotism.''
\switchcolumn
美国社会主义者哈林顿(Michael Harrington)则说:“第
一次世界大战表明,和那些自由企业理论家的断言相反,政府
能够高效地组织起整个经济。” 他为第二次世界大战欢呼,认
为第二次世界大战“证明了真正的大规模动员的正确性,正
是这种动员把本来会浪费掉的人员和物质资源利用了起来”。
他抱怨战时生产委员会“这样一个成功模式竟然被美国决定
尽快抛弃”。他还说: “第二次世界大战期间,社会公平的增
长也许比美国历史上任何时候都要多。颁布了工资和价格的管
制政策以削平社会各阶层之间的差别……还有一个强烈的道德动机激励着工人的工作----爱国主义。”
\switchcolumn*
Collectivists such as Foot and Harrington don't like the
killing involved in war, but they love its domestic effects: centralization, the growth of government power, and, not coincidentally, an enhanced role for court intellectuals and planners
with Ph.D.'s. The dangers of war in the modern era have encouraged the state and its intellectual allies to look for more
trumped-up emergencies and ``moral equivalents of war'' to
rally the citizenry and persuade them to give up more of their
liberty and their property to the state's plans. Thus we've had
the War on Poverty, and the War on Drugs, and more crises and
national emergencies than a planner could count on a supercomputer. One advantage of these ``moral equivalents of war'' is
that real wars eventually end, while the War on Poverty and the
War on Drugs can go on for generations. And thus does the alliance between the state and its compliant intellectuals reach its
zenith in war or its moral equivalent.
\switchcolumn
弗特和哈林顿这样的集体主义者并不喜欢战争带来的杀
戮,但是他们喜欢战争给国家带来的影响:中央集权、政府权
力的扩张。还有一个不能算是巧合的影响是,.御用知识分子和
拥有博士学位的计划专家的角色变得更重要了。现代战争的危
险鼓励国家及其知识分子盟友寻找更多虚构的非常时期以及
“战争的道德等效物” 来把公民们召集起来,说服他们把更多
的自由和财产交给国家来支配。因此,就 抛出 了 所 谓 的 “反
贫困战争” “反毒品战争” 以及其他各种危机和国家非常时
期。这些被抛出来的危机甚至比计划专家在超级计算机上所能
计算的还要多。这 些 “战争的道德等效物” 的一个好处就是,
真正的战争会结束,但 “反贫困战争” 和 “反毒品战争”之
类的可以一代一代搞下去。这样,国家及其顺从的知识分子之
间的联盟就在战争及其“道德等效物” 中达到了顶点。
\switchcolumn*
War, then, is Public Choice theory writ large: bad for the
people but good for the governing class. No wonder everyone
wishes it would stop but no one can stop it.
\switchcolumn
因此,战争,就像公共选择理论早就深刻指出的那样:对
人民是坏事,对统治阶层是好事。毫无疑问,每个人都希望它
能够结束,但每个人都无法阻止它。

\end{paracol}