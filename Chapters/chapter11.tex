\chapter{THE OBSOLETE STATE\\过时的国家}
\begin{paracol}{2}
\hbadness5000

As public services decline and markets get more sophisticated in the Information Age, people are
turning to private provision of everything from education to
first-class mail to disaster insurance. Even people who once saw
a need for government to provide such services now view the
state as an increasingly clumsy and obsolete way to supply
most goods and services.
\switchcolumn
随着公共服务的衰落以及市场在信息时代变得越来越复
杂 ,人们开始转而从私营供应商那里寻求从教育、快递到意外
灾害保险等所有东西。甚至那些曾经认为政府应当提供这些服
务的人,现在也意识到,国家在提供绝大多数商品和服务方面
越来越笨拙和过时。
\switchcolumn*
Why does government provide so many goods that could be
better provided privately? Some answers, mostly having to do
with political imperialism and the dysfunctional nature of politics, were suggested in chapter 9. But there are certainly less
sinister reasons, such as the argument that government is
needed to supply public goods. Scholars have recently subjected
the ``public goods'' argument to withering examination. Entrepreneurs, however, didn't wait for the scholars to show them the
way; from lighthouses and schools to postal service and flood insurance, markets have produced what consumers needed while
scholars argued about whether markets \textit{could} work.
\switchcolumn
很多商品本来可以由私营企业和个人更好地提供,那么为
什么要让政府来提供呢?很大程度上是和政治帝国主义以及政
治天然的机能障碍有关,这一点我们在第九章中进行了论述。
当然,也有一些不那么邪恶的原因,比如认为必须由政府来提
供公共物品。很多学者最近常常用“公共物品” 的概念来消
弭质疑。然而,企业家们并没有等待学者来指出道路;就在学
者们争论市场在这些领域\textbf{是否有效}的时候,市场已经创造出了
消费者需要的产品,从灯塔、学校一直到邮递服务、洪灾保险。

\switchcolumn*[\section{Market Failure and Public Goods\\市场失败和公共物品}]
The claim of ``market failure'' is probably the most important
intellectual argument for state intervention in the market. There is a serious argument developed by economists that in
some circumstances, markets fail to supply something that
many people want and would be willing to pay for. Outside the
economics journals, however, the person claiming a market failure usually means that the market has failed to supply something that \textit{he} wants. A friend of mine likes to poke fun at my
dogged faith in the market process by declaring ``market failure'' every time I complain about my inability to find a particular product or service: There's no good pizzeria in my
neighborhood? ``Market failure!''
\switchcolumn
宣 告 “市场失败”也许是支持国家干预市场的最重要的
知识理由。一种由经济学家提出来的严肃观点认为,在某些情
况下,市场不能提供很多人需要并且愿意付费的某些商品。然
而,在经济学杂志之外,人们所说的市场失败常常是指市场不能提供\textbf{他想要}的东西。一位朋友喜欢拿我对市场经济的顽固信仰开玩笑,每次当我抱怨不能找到某种商品或服务的时候,他就宣布 “市场失败了”。小区没有像样的比萨店, “这是市场
失败啊!”
\switchcolumn*
In most cases, of course, if we can't find a good or service that
we want, it's for one of two reasons: entrepreneurs are missing
an opportunity, in which case we ought to consider supplying it,
or there's some good reason that no one is supplying it. These
days, it seems that many people would like to go to nonsmoking bars. So why are there almost none in existence? It's possible that this is a great entrepreneurial opportunity. It's more
likely that smokers tend to drink more and tip better, so that
it's extremely difficult to make a profit on a nonsmoking bar
(though this may well change in the next few years).
\switchcolumn
当然,在大多数情况下,如果我们不能找到想要的产品或
服务,通常是因为下面两种原因中的一个:企业家错过了一个
机会,在这种情况下我们就应该考虑去提供这种产品或服务;
或者是因为某种很好的原因而没有人提供它。看起来很多人愿
意去禁烟的酒吧。但是为什么几乎没有一家酒吧禁烟呢?可能
这是一个重大的商业机会,但更可能是因为吸烟者往往会喝更
多的酒,给的小费也更多,因此一家禁烟酒吧很难赚到钱
(尽管这种情况也许几年之后会改变)。
\switchcolumn*
Most serious claims of market failure are based on the theory
of public goods. A ``public good'' is defined by economists as an
economic good with two characteristics: nonexcludability and
nonrivalrous consumption. That is, first, it's impossible to exclude nonpaying individuals from enjoying the good. The classic example is the lighthouse, whose beam can be seen by all
ships. And second, individuals' ability to enjoy a good or service
is not diminished by allowing other individuals to consume it as
well. For instance, a broadcast signal or a movie, unlike an automobile or a haircut, can be enjoyed by many people simultaneously.
\switchcolumn
大多数真诚地认为市场失败的观点都是建立在公共物品理
论的基础之上的。按照经济学家的定义,“公共物品”有两个
特征:非排他性和非竞争性的消费。也就是说,首先,不可能
排除不付费的个人使用这个产品。一个经典例子是灯塔,灯塔
的光束能够被所有的船只看到。其次,个人使用一种产品或服
务的能力并不因为允许其他人一同使用而减少。例如,和汽车
或者理发不同,一段广播信号或者一部电影能够被很多人同时
欣赏。
\switchcolumn*
Economists have argued that people will ``free-ride'' on the
provision of nonexcludable goods; that is, ships won't contribute to the upkeep of a lighthouse because they can enjoy its
services as long as other ships contribute. Of course, if many
people seek to free-ride, the service may not be provided at all.
Some economists therefore argue that government should tax
people and provide the service itself to overcome market failure.
\switchcolumn
经济学家证明,在非排他性的产品上人们会“搭便车”,
也就是说,过往船只不会为维护灯塔而付费,因为只要别的船
只付费,他们也可以使用灯塔的服务。当然,如果很多人搭便
车的话,这种服务根本就不会存在。有的经济学家因此就认为
在市场失败时政府应该向人们征税来提供这种服务。
\switchcolumn*
There are several problems with this analysis. Goods can be produced and distributed in many ways, some of which allow
for exclusion of nonpayers while others don't. Almost all goods
could be produced ``publicly,'' that is, in a way that would make
it difficult to exclude nonpayers, or privately. It may often be
the case that a good's ``publicness'' reflects the fact that government has produced it without regard to excludability. As Tom
G. Palmer wrote in \textit{Cato Policy Report} in 1983,
\switchcolumn
这种分析存在几个问题。商品可以以很多种方式来生产和
分配,其中有的能够排除不付费者,而有的则不能。几乎所有的商品都既能“公共地” 生产,也就是说使得它很难排除不付
费者,也能私下生产。也许事情常常是这样:一种商品的“公
共性”恰恰是因为政府在生产它的时候根本就没有考虑过排他性。正如汤姆$\cdot$帕尔马(Tom G. Palmer)在 1983年 《Cato政策报告》 中所说的:
\switchcolumn*
\begin{quote}
The argument for state provision is framed in purely static,
rather than dynamic, terms: \textit{given} a good, for which the marginal
cost of making it available to one more person is zero (or less than
the cost of exclusion), it is inefficient to expend resources to exclude nonpurchasers. But this begs the question. Since we live in a world where goods are not a \textit{given}, but have to be produced, the
problem is how best to produce these goods. An argument for
state \textit{provision} that assumes the goods are already produced is no
argument at all.
\end{quote}
\switchcolumn
\begin{quote}
对政府应当提供某种商品的论证是在假定一种完全静
止而不是动态的条件:\textbf{给定}一种商品,每多提供给一个人
的边际成本为零(或者低于排除一个人的成本),那么耗
费资源来排除不付费者是无效率的。但这是在回避问题。
因为我们生活的世界里商品并没有\textbf{给定}、而是必须生产出
来,问題就在于如何才能最好地生产这些商品。假设商品
已经生产出来,并以此论证国家应当\textbf{提供}某种商品,这根
本就不是论证。
\end{quote}
\switchcolumn*
The question then becomes whether it is more efficient to let
entrepreneurs find ways to supply goods at a profit on the market or to turn the provision of important goods over to government, where we will encounter such problems as a lack of real
market signals, an absence of incentives, and a decision-making
process dominated by special interests and political influence.
The basic argument of this book has been that valuable goods
and services are best provided in the competitive marketplace.
In this chapter, we'll look at some specific examples of goods
and services that people thought the market couldn't provide
but found that it not only could but did.
\switchcolumn
于是问题就变成了:企业家寻找提供商品及其在市场中获
利的方法更有效率,还是让政府来提供重要的商品更有效率?
分析到这里,我们就会发现诸如缺乏真正的市场信号、缺乏激
励以及决策过程被特殊利益和政治影响所支配的问题。本书的
基本观点是,有价值的商品和服务由竞争性市场来提供是最佳
途径。在本章中,我们将讨论一些人们认为市场不能提供的产
品和服务的例子,读者会发现这些产品和服务市场不但能够提
供,而且是已经提供了。
\switchcolumn*[\section{Some Classic Examples That Weren't Public Goods\\并非公共物品的一些经典案例}]
The traditional example of a public good was the lighthouse.
Obviously, economists told generations of students, lighthouses
couldn't be supplied privately because it would be impossible to
charge everyone who would benefit from the lighthouse. From
John Stuart Mill's \textit{Principles of Political Economy} in 1848 to Nobel
laureate Paul A. Samuelson's Economics, read by millions of
modern American college students, textbooks pointed to the lighthouse to show the need for government provision of public
goods.
\switchcolumn
传统的公共物品案例是灯塔。经济学家们告诉一代又一代的学生,灯塔不能由私人提供是因为不可能向所有因灯塔而获益的人收费。从密尔在1848年 写 的 《政治经济学原理》 到被数百万现代美国大学生所阅读的诺贝尔奖获得者萨缪尔森的著作 《经济学》,这些教科书都指出灯塔的例子证明了需要政府来提供公共物品。
\switchcolumn*
Then in 1974 an economist decided to find out how light-houses had actually been provided. Ronald H. Coase of the University of Chicago, who would also go on to win a Nobel Prize,
investigated the history of lighthouses in Britain and found that
they had not been built or financed by government:
\switchcolumn
其后,1974年,一名经济学家决心找出灯塔究竟是如何
提供服务的答案。芝 加 哥 大 学 的 罗 纳 德 $\cdot$ 科 斯 (Ronald
H. Coase)(他后来获得了诺贝尔奖)调查了英国灯塔的历史,
发现它们并没有由政府来建造或者提供依靠政府的财政支持:
\switchcolumn*
\begin{quote}
The early history shows that, contrary to the beliefs of many
economists, a lighthouse service can be provided by private enterprise$\ldots$ The lighthouses were built, operated, financed and
owned by private individuals$\ldots$ The role of the government
was limited to the establishment and enforcement of property
rights in the lighthouse.
\end{quote}
\switchcolumn
\begin{quote}
早期的历史显示,与许多经济学家所相信的理论相
反 ,灯塔服务能够由私人企业来提供……灯塔是由私人出
资、建造、运营和拥有……政府的作用仅限于确认和保护
灯塔的产权。
\end{quote}
\switchcolumn*
Tolls were collected at ports; recognizing the value of the lighthouses, shipowners were glad to pay. In the nineteenth century,
all British lighthouses became the property of Trinity House, an
ancient organization that had apparently evolved out of a medieval seamen's guild, but the service was still financed out of
tolls paid by ships.
\switchcolumn
费用是在港口收取的;由于承认灯塔的作用,船主们很乐
意为此付费。19世纪,英国所有灯塔都变成了引航公会的财
产,引航公会是一个古老的组织,显然是由中世纪的船员行会
演变而来,但是灯塔服务仍然由船只付费来维持。
\switchcolumn*
After Coase's article appeared, the economist Kenneth
Goldin wrote, ``Lighthouses are a favorite example of public
goods, because most economists cannot imagine a method of
exclusion. (All this proves is that economists are less imaginative than lighthouse keepers.)''
\switchcolumn
科斯的文章发表之后,经济学家肯尼思$\cdot$ 高 定 (Kenneth
G oldin)在文章中写道:“灯塔是人们谈到公共物品时最爱用
的案例,因为大多数经济学家想像不出一种排他使用的办法。
这充分证明了经济学家比灯塔所有者更缺乏想像力。”
\switchcolumn*
Another classic example of a public good, though much
newer than the lighthouse case, was beekeeping. Several distinguished twentieth-century economists argued that apple growers benefit from the presence of bees because they pollinate the
apple blossoms; but the beekeepers have no incentive to help
the apple growers, and bees can't be confined to particular
farms, so there will be less investment in beekeeping than
would be good for the economy. Again, it seemed plausible and
even obvious---so obvious in theory that no one bothered to
check the facts.
\switchcolumn
另一个关于公共物品的经典案例(比灯塔案例更新一些)
是养蜂业。20世纪的几位知名经济学家论证道:苹果因蜜蜂
的出现而受益,因为它们给苹果花授粉;但是养蜂人并没有帮
助苹果种植者的动机,而蜜蜂不可能被限制在特定的农场,因
此对养蜂业的投资将会少于对经济带来的好处。又一次,这个
论证看起来很有道理甚至显而易见 --- 理论上是如此显而易见 ,以至于没有人想到去进行事实验证。
\switchcolumn*
When economist Stephen Cheung of the University of
Washington went out to examine the Washington apple-growing business, he found once again that businesspeople were already doing what economists said couldn't be done. There was a long history of contractual arrangements between apple
growers and beekeepers. Those contracts ensured that beekeepers would have an incentive to supply the bees that apple growers profited from. Informal agreements among the apple
growers ensured that all of them paid similar amounts to the
beekeepers instead of trying to free-ride off the other growers.
Those informal agreements, like the written contracts, are part
of the vast network of cooperation that we call the market
process or civil society. Economists who wanted to point out examples of market failure were running out of cases, as other
economists actually examined the working of the market.
\switchcolumn
而当华盛顿大学经济学家张五常去调查华盛顿苹果种植业
的时候,又一次发现商人已经做了经济学家认为不可能做的事
情。苹果种植者和养蜂人之间的合约安排已经有了很长历史。
这些合约确保养蜂人有足够的激励提供蜜蜂让苹果种植者受
益。苹果种植者之间的非正式协议确保所有人都向养蜂人支付
数量大体相近的一笔钱,并没有人试图搭其他种植者的便车。
这些非正式协议就像书面合同一样,是庞大的合作网络的一部
分,这就是我们所称的市场经济或者市民社会。由于其他经济
学家不断地去实际检验市场的有效性,那些主张市场失败的经
济学家已经没有多少案例可举了。

\switchcolumn*[\section{When Does Government Provide Services?\\政府什么时候提供服务}]
It's usually assumed that government steps in to provide a service when the private sector fails to supply it. Even if that were
true, it would raise the question of why people should be taxed
to supply a service that they weren't willing to pay for. Unless a
good case can be made that the particular good or service is a
public good---and as we've seen, that's difficult to do---then
the argument for government provision is simply that some
person's preferences should be substituted for the decisions that
millions of consumers make by spending their own money.
\switchcolumn
们常常认为当私人部门不能提供某种服务的时候政府就
应该进入了。即便这是正确的,也会带来一个问题:为什么人
们应该为他们不愿意付钱的服务交税?除非有很好的例子证明
某种产品或服务是公共物品 --- 我们前面看到了,这很难做到
--- 否则,要求政府提供某种服务就只不过是某人的偏好,不
应该代替数以百万计的消费者通过花自己的钱而做出的选择。
\switchcolumn*
In fact, however, government usually doesn't supply a service
that isn't being provided in the market. Rather, politicians
promise to give people something at public expense that people
don't like paying for. Provision by a bureaucratic monopoly
doesn't actually make the service cheaper, but it does conceal
the cost. People no longer connect a specific payment with the
service, so they appreciate getting a formerly expensive service
apparently for free, even though they wish their taxes wouldn't
keep going up. The political opportunity to make gains by offering a new government service seems to come when enough
people are paying for the service that many voters would prefer
to have the expense taken off their hands.
\switchcolumn
然而,事实上,政府提供的常常并不是市场上没有的服
务。相反,政治家们常常承诺用公共开支提供某些人们并不愿
意为之付钱的东西。通过官僚垄断机构来提供服务并不会让服
务真正变得更便宜,而是把某些费用隐藏了起来。人们看不到
付费与服务的直接联系,他们很高兴得到以前很贵但现在表面
上是免费的服务,虽然他们也希望税收不要不断上升。当足够
多的人为这项服务付费,而很多选民希望他们能不付费时,通过提供一项新的政府服务获得利益的政治机会就来了。
\switchcolumn*
The economist W. Allen Wallis argues that education in
Britain and the United States is a good example of this. He
writes, ``In 1833, when the government of England first began to subsidize schools, at least two-thirds of the youth of the
working class were literate, and the school population had doubled in a decade---although until then the government had deliberately \textit{hindered} the spread of literacy to the lower orders'
because it feared the consequences of printed propaganda.''
(Emphasis added.) By 1870, when government education was
made free and compulsory, nearly all young people were literate. Their literacy had been achieved in schools that charged
fees, including the inexpensive ``dames' schools'' set up by
working-class families. The philosopher James Mill had noted
as early as 1813 ``the rapid progress which the love of education
is making among the lower orders in England.''
\switchcolumn
经济学家艾伦$\cdot$ 华莱士( W. Allen Wallis) 认为英国和美国的教育就是一个很好的例子。他写道:“1833年,当英国政
府开始资助学校的时候,至少三分之二的工人阶级的年轻人是受过教育的,而在校人数在10年当中翻了一番 --- 尽管那时
政府在有意\textbf{阻挠}教育向较低阶层扩散,因为它害怕印刷宣传品
所带来的后果。” 到1870年 ,政府教育成为免费和强制义务,这时几乎所有的年轻人都受过教育。但他们的教育是通过收费
的学校达到的,包括工人阶层家庭建立的学费便宜的“太太学校”。哲学家詹姆斯$\cdot$密尔(Jamer Mill)早在1813年就注
意到,“英国的低等阶层中对教育的热衷进展很快”。
\switchcolumn*
In the United States, too, Wallis writes, ``the government
began to provide 'free' schooling only after schooling had become nearly universal.'' State governments may have decided to
make education free, compulsory, and government-run in the
late nineteenth century in order to win favor with voters who
would no longer pay directly for schools, or in order to impose a
particular religious and political agenda on the schools, but it is
clear that state action was not needed to make schooling widely
available.
\switchcolumn
在美国也是如华莱士所写的那样:“就在学校教育变得相
当普遍的时候,政府开始提供‘ 免费’ 的教育了。” 国家政府
也许决定让教育变为免费和强制义务,19世纪晚期开始由政
府来办教育,以获得那些不再为学校教育直接付钱的选民的好
感 ,或者为了在学校教育中加人某种宗教和政治的课程,但是
很清楚,政府的行为对学校教育的普及是没有必要的。
\switchcolumn*
Medicare was another example of a service that was being
provided privately, at individual expense, until the federal government took it over. A 1957 survey by the National Opinion
Research Center found that ``about one person in twenty in the
older population [aged sixty-five years or older] reported that
he was doing without needed medical care because he lacked
money for such care.'' If more than 90 percent of the elderly
could afford the medical care they needed, why was a government program needed to provide medical care for all the elderly? Wallis sums up the lessons this way:
\switchcolumn
在被联邦政府代替之前,医疗是另一个由私人提供、由个
人付费的服务的例子。一份由全国民意研究中心(National Opinion Research Center)做的调查显示, “在老年人口(65岁以上)当中,大约5\%的人认为他们不需要医保,因为没钱给
这样的医保付费。”如果超过90\%的老年人能够负担他们需要
的医疗服务,为什么需要一个为所有老人的医疗付费的政府计
划呢?华莱士这样对这些教训进行总结:
\switchcolumn*
\begin{quote}
The task of the political entrepreneur, then, is to identify services
which are being purchased by substantial and identifiable blocs
of his electorate and to devise means by which the cost of these
services will be transferred to the public. Successful innovation
lies not in getting something done that was not being done before, but in transferring the costs to the public at large. Only if
fairly large numbers of voters are already paying for the service will the offer to relieve them of the cost be likely to influence
their votes.
\end{quote}
\switchcolumn
\begin{quote}
于是,政治企业家的工作就是找出选民当中由可靠的
清晰可见的集团正在购买的服务,并且设计一些方法把这
些服务的成本转移给公众。成功的创新并不是建立在做某件以前没人做的事情之上,而是建立在把成本转移给广大
公众之上。只有相当大数量的选民已经为这种服务付费之
后,提供降低这些服务成本的方案才可能影响到他们的
选票。
\end{quote}
\switchcolumn*
A more recent example might be government subsidies for
child care. As more and more parents pay for child care outside
the home, there is a larger constituency of people who would
like to be relieved of the expense. Thus politicians begin to declare that child care is a national responsibility or that parents
``can't afford'' child-care expenses. Actually, they \textit{can} afford it---
they \textit{are} affording it---but they don't especially like paying for
it. The politicians never address exactly why childless people
and stay-at-home mothers should be taxed to pay for the care of
other people's children, but taxes have become so large and so
seemingly inevitable that voters don't seem to connect rising
tax burdens with new services from government.
\switchcolumn
一个更近的例子也许是政府为照看儿童提供补贴。随着越
来越多父母为儿童在家庭外被照看而付费,更多选民将会要求
降低这方面的开支。于是政治家们就开始宣布,照看儿童是国
家的责任,或者说父母 “不能负担” 照看儿童的开支。而实
际上他们\textbf{能够负担} --- 他们\textbf{正在}负担着---但并不是特别喜欢
为此付费。政治家们从来不解释,为什么没孩子的人以及呆在
家里的母亲们应该被征税,用来为照顾别人的孩子而付账。税
收已经变得如此多,并且看上去是不可避免的,因此选民们并
没有把不断升高的税收和政府的新服务联系起来。
\switchcolumn*
When a service is transferred from the market to the government, of course, its provision is no longer directly responsive to
the consumers but will increasingly reflect the preferences of
the providers rather than the customers. Recipients of government services can influence them only through the cumbersome political process rather than by the much more efficient
process of choosing among competing providers.
\switchcolumn
当一种服务从市场转移给政府的时候,它提供的服务当然
就不再直接对消费者负责,将更多地反映服务提供者的偏好而
不是顾客的偏好。政府服务的接受者对其产生影响的办法只能
是通过繁芜的政治程序,而不是通过更有效的在互相竞争的供
应商中进行选择。

\switchcolumn*[\section{The Contemporary Flight from Government Services\\来自政府服务的当代航班}]
These days government endeavors to supply more goods and
services than anyone could count, and people are increasingly
disillusioned with the quality of government services. The
world is moving rapidly into the Information Age, except for
the schools and the post office. Giant financial-services
providers offer an array of products designed to meet each customer's needs, with twenty-four-hour customer service, except
for Social Security and other government-run systems. Government parks, streets, housing projects, and schools are increasingly dirty and dangerous. That's why more and more
Americans seek to flee government services, often to pay
extra for products and services that they've already paid for in
taxes.
\switchcolumn
今天,政府提供数不胜数的产品和服务,而人们对政府服
务的质量越来越失望。整个世界在迅速进入信息时代,而学校
和邮局却是例外。庞大的金融服务供应商们提供了一系列产
品,来满足消费者的需求,有24小时的客户服务,只有社保
和其他由政府运营的系统例外。政府的公园、街道、住房项目
以及学校越来越肮脏和危险。这就是为什么越来越多的美国人在摆脱政府服务,常常为他们已经通过税收付过费的产品和服
务再额外付款。
\switchcolumn*
Robert Reich, secretary of labor in the Clinton administration and author of several best-selling books on economic
change, has complained about what he calls ``the secession of
the successful''; in 1995 he told University of Maryland graduates that the richest Americans are walling themselves off from
the rest of society---working in the suburbs, shopping in secure
suburban malls, and even living in private communities. Worse,
he said, they are resisting government efforts to spend their tax
dollars outside their own communities. Social democrats like
Reich concerned about community values ought to reflect on
what their policies have done to divide Americans. They've
given government so many tasks, and so undermined the old
notions of personal responsibility and morality, that government can no longer perform its basic function of protecting us
from physical harm. They have centralized and bureaucratized
the schools so that little learning goes on there. They have nationalized and bureaucratized charity. Is it any wonder that people flee the institutions thus created?
\switchcolumn
罗伯特$\cdot$ 赖 克 (Robert  Reich) 是克林顿政府的劳工秘书
以及多本关于经济变化的畅销书的作者,他 抱 怨 “成功人士
的分离”。1995年,他对马里兰大学的研究生说,最富裕的美
国人正在把自己和社会的其他部分隔离开来一 在郊区工作,
在安全的郊区购物中心购物,甚至居住在私人社区当中。他
说 ,更糟糕的是,这些富人们反对政府将他们的税收花在他们
社区之外。其实,像赖克这样关心社区价值的社会民主派应当
思考的是,到底是什么造成了美国人的分裂。他们给了政府如
此多的任务,因此就破坏了个人责任和道德的古老原则。政府
不再行使保护我们不受身体伤害的基本功能。他们对学校进行
集中管理和官僚化,结果很少有人再真正学习。他们对慈善事
业进行了国有化和官僚化。人们从他们创建的组织中逃离,这
有什么可奇怪的呢?

\switchcolumn*[\subsection{Communications\\通信}]
The U.S. Postal Service is one of the world's largest monopolies,
and it displays all the sluggishness we expect from a government-run monopoly. Every other form of information transfer
has been changed beyond recognition in the past generation,
but the Postal Service still chugs along with 800,000 employees
delivering letters the old-fashioned way, just a little bit slower
each year. The price of a megabit of memory in a personal computer has fallen from \$46,000 to \$1 in fifteen years, but the
price of stamps keeps rising. We've heard all sorts of horror stories about the post office---200 pounds of mail found under a
viaduct in Chicago, 800,000 pieces of first-class mail stashed in
tractor-trailer trucks near a Maryland postal facility because
mail isn't counted as ``delayed'' unless it's inside a facility---but
the big issue is speed and reliability of communications.
\switchcolumn
美国邮政服务局(USPS)是世界上最大的垄断机构之一。
它显示了一个政府运营的垄断机构的所有惰性。其他所有信息
传递方式都在过去一代人的时间里彻底改变,但是邮政服务仍
然如老牛拉破车一样嘎吱嘎吱地运转,它的80万名雇员仍然在用古老的方式传递信件,唯一的不同是送信速度更慢一些。
个人计算机上每1M数据的存储费用在15年当中从4.6 万美元降到了1美元,但是邮费的价格却在不断上涨。我们曾经听
说过各种各样关于邮局的恐怖故事200磅邮件在芝加哥的一个高架桥下被发现;在马里兰州的一个邮局附近,80万份普通邮件被藏在带有拖车的卡车里,因为在邮件进入邮局之前是不会被算作“延误” 的 --- 但是最大的问题是通信的速度和可靠性。
\switchcolumn*
In areas where competition is allowed, the U.S. Postal Service
has lost almost all its market share. Its share of the parcel post
market fell from 65 percent twenty-five years ago to 6 percent
in 1990, and its share of overnight deliveries fell from 100 percent to 12 percent or less (industry estimates vary). Even postal
boxes and counter service are increasingly provided by firms like Mail Boxes Etc., which was described by one customer as
``just what you'd like the post office to be''---friendly, efficient
service with helpful accessories like boxes and packing tape.
Given a choice, businesses and individuals overwhelmingly opt
to have their letters and packages delivered by competitive private firms.
\switchcolumn
在被允许竞争的领域里,美国邮政服务局几乎丧失了所有
市场份额。它在包裹投递市场上的份额从25年前的65\% 降到
1990年的6\% ,它的特快专递服务市场份额从100\% 降到了
12\%或更低(行业统计数据有所不同)。那些私营公司开始越
来越多地提供邮政信箱和柜台服务。邮箱易特斯公司\footnote{邮箱易特斯公司(Mail  Boxes Etc.,Inc.),UPS旗下的一个公司,主要从事邮政快递和商务服务。目前在全球有1500家邮箱易特斯连锁店。}被客户描述为“你理想当中邮局应该是什么样,它 就是什么样” ---亲切友好、高效的服务,还有一些贴心的附加服务,如私人邮箱和打包带。如果有选择的话,公司和个人几乎会亳无例外地将他们的信件、包裹交给竞争性的私营公司来投递。
\switchcolumn*
With first-class mail, however, there is no choice. The U.S.
Postal Service has a legal monopoly, which means it is illegal for
a private firm to offer to carry a letter to its recipient, except for
``urgent'' communications, for which private firms must charge
at least \$3. The USPS takes that ``urgent'' exception seriously;
it conducts surveillance, with binoculars and telescopes, of shipments and delivery trucks, and sends agents into private firms
to audit what they're sending out by Federal Express or United
Parcel Service. It imposes hundreds of thousands of dollars in
fines each year on firms whose privately delivered packages are
deemed not to be urgent. One would think a firm's willingness
to pay several dollars to get a piece of mail delivered the next
day would be adequate evidence of its urgency, but the Postal
Service believes it is the best judge of what's urgent for private
businesses.
\switchcolumn
然而寄送普通邮件是没有选择的。美国邮政服务局对此拥
有法定垄断权,也就是说,私营公司把普通信件投递给接收人
是非法的,除 非 是 “紧急” 通信,但紧急通信私营公司的收
费不能低于3 美元。美 国 邮 政 对 “紧急通信” 的例外条款非
常重视;使用望远镜对装卸和投递的卡车进行监视,还派员到
私营公司检查他们使用联邦快递公司(FedEx) 或联合包裹服
务 公 司(UPS)的服务时在邮件里装了什么东西。每年,它对
被指控通过私营快递公司投递非紧急邮件的公司的罚款都达到
数十万美元。人们很自然地认为,一个公司为了让一封邮件第
二天送到,愿意支付几美元,这足以证明是紧急邮件了,但是
邮政服务局却认为,只有它才有权判断什么是紧急邮件。
\switchcolumn*
Meanwhile, private firms and individuals are increasingly
looking for ways to get around the postal monopoly. In a sense,
faxes and electronic mail are eroding the Postal Service's share
even of a market where it has a legal monopoly. Already, it is estimated that 50 percent of telephone traffic across the Atlantic
and 30 percent of U.S.-Pacific traffic is fax messages. Electronic
mail will be even more revolutionary. Steve Gibson of the Bionomics Institute points out that Gutenberg's invention of movable type cut the cost of copying written information a
thousandfold in just forty years. By contrast, he says, in the first
twenty-five years after the invention of the microprocessor in
1971, the cost of copying information dropped ten-million-fold. During the next decade, computing power is expected to
rise 100 times; and bandwidth, the size of the ``pipe'' that carries digital information like e-mail, will increase 1,000 times.
Letter mail will soon be left in the dustbin of history.
\switchcolumn
与此同时,越来越多的私营公司和个人开始寻求绕开垄断
的美国邮政的办法。某种程度上,传真和电子邮件侵蚀着美国邮政的市场,尽管它仍然拥有法定垄断权。据估计,穿越大西
洋的电话往来中的50\%以及穿越太平洋的电话往来中的30\%
是传真。电子邮件更加具有革命性。生 态 研 究 所 (Bionomics
Institute)的 吉 布 森 (Steve Gilbson)指出,古登堡发明的活
字印刷技术让文字信息的复制成本在40年中降低了数千倍。
相比之下,他说,1971年发明计算机微处理器之后的25年当
中,信息的复制成本降低了数千万倍。下 一 个 10年 ,计算机
的处理能力有望提升100倍 ;而带宽将会提升1000倍。纸质
邮件将会很快被扔进历史的垃圾堆。
\switchcolumn*
In the late 1970s the Postal Service tried to protect its monopoly by moving to monopolize electronic mail. That's the natural reaction of a monopolist to potential competition, and
we can all be glad the plan failed. Now the question is why a
clunky bureaucracy should have a monopoly on letter mail.
Maybe if the postal monopoly were eliminated, private firms
could find an efficient way to go on delivering mail house-to-house for a few more years. Otherwise, the economy will treat
the Postal Service like a disruption in a telephone line and route
important traffic around it.
\switchcolumn
1970年代后期,美国邮政曾试图通过垄断电子邮件来保
护它的垄断权。这当然是垄断机构对潜在竞争的自然反应,而
我们每个人都应该庆幸这个计划失败了。现在的问题是为什么
一个笨拙沉重的官僚机构要对邮件递送拥有垄断权。如果邮政
垄断被消除的话,私营公司也许将会找到一种让门对门的邮件
投递再继续存在很多年的有效方式。否则,整个经济对待美国
邮政就会像对待电话线中断一样,让重要的通信避开它。

\switchcolumn*[\subsection{Education\\教育}]
We spend more money every year on the public schools---three
times as much in real terms as we spent in 1960---yet test
scores decline and many urban schools are actually dangerous.
According to Keith Geiger, president of the National Education
Association, about 40 percent of big-city public school teachers
send \textit{their} children to private schools. They must know something. Yet the NEA bitterly resists making such choices easy for
other families; it spent \$16 million to defeat just one school-choice initiative, in California in 1993.
\switchcolumn
我们把越来越多的钱花在公立学校上 --- 按可比价格计算
是 1960年的3倍 --- 但是学生的考试成绩却在下降,并且很
多城区学校实际上变得很危险。根据全国教育协会主席凯斯$\cdot$
盖 格(Keith Geiger)的统计,大约40\% 的大城市公立学校教
师把\textbf{自己的孩子}送到私立学校。他们应该很清楚公立学校的状况。而全国教育协会却对其他家庭自主择校设置障碍;1993年,他们花1600万美元在加州阻止了一次家长自由选择学校的努力。
\switchcolumn*
Many Americans have chosen to take their children out of
government schools and send them to private schools, in effect
paying twice for education. Among those parents are President
Clinton, Vice President Gore, Senator Edward M. Kennedy, the
Reverend Jesse Jackson, and Children's Defense Fund founder
Marian Wright Edelman, all staunch opponents of school
choice. Less wealthy families find it difficult to pay high taxes
and then pay again for private education. Nevertheless, some
families think private education is worth whatever sacrifice it
takes. The Institute for Independent Education has identified
390 small black-run schools across the country and finds that
22 percent of their students come from families making less
than \$15,000 a year, while another 35 percent of families earn
\$15,000 to \$35,000.
\switchcolumn
很多美国人选择让自己的孩子退出政府办的学校,进入私立学校学习,尽管这实际上等于在为孩子的教育重复付费。在
这些家长当中包括总统克林顿、副总统戈尔、参议员爱德华$\cdot$
肯尼迪、牧师 杰 西$\cdot$杰克逊以及保卫孩子基金会(Children’s
Defense Fund)创始人埃德曼(Marian Wright Edelman) 。 他们
都是反对家长自由选择学校的坚定分子。一些不那么富裕的家
庭发现自己很难在支付高额税收的同时再为孩子接受私立教育
付钱。不过,也有很多家庭认为私立教育很值得,哪怕付出再
大代价也在所不惜。独立教育研究所(Institute for Independent
Education)统计全国390家小型黑人学校,它发现其中22\%
的学生来自年收入低于15000美元的家庭,还 有 35\%的家庭
年收入在15000到 35000美元之间。
\switchcolumn*
Other families, often those with more political skills, try to
game the system, sneaking their children into better schools in
another part of town or a nearby town. Families use friends' and
relatives' addresses to enroll their children in other school districts, establish mail drops, or get school officials to grant them
waivers so their children can go to better schools. In response, school officials have taken to videotaping children coming off
the subway to find out-of-district students, and they've asked
legislatures to stiffen the penalties for ``school enrollment
fraud.''
\switchcolumn
而其他的一些家庭,常常是那些掌握更多政治技巧的人,
则通过钻制度的空子把自己的孩子送到同一城市的其他城区或
者邻近城镇的好学校。这些家庭常常使用朋友或者亲戚的住址
来让他们的孩子进入别的学区,然后竖立新的邮筒,或者让学
校官员给他们提供一份学区转移证明,以让他们的孩子进入更
好的学校。为了防止这些情况,很多学校官员常常在孩子们从
地铁出来的时候拍摄录像,以发现那些住在本学区外的学生,
然后要求立法机构对“学生入学欺诈” 进行严厉惩罚。
\switchcolumn*
Many families have given up on organized schooling altogether and begun teaching their children at home. Families
choose home schooling for a variety of reasons. Many object to
what they see as aggressive secularism in the government
schools and want to give their children a religiously based education. Others dislike the conformity and authoritarianism that
are probably inherent in the process of grouping small children
in classes of twenty to thirty and trying to teach all of them the
same thing at the same time. ``Public schools as we know them
are an aberrant bureaucracy,'' says David Colfax, who has sent
three home-schooled sons to Harvard University. Mothers who
want to stay home with their children may find home schooling
a less expensive alternative than private education. And some
families just think the schools don't do a good job of teaching
the basics.
\switchcolumn
还有很多家庭放弃了让孩子接受有组织的学校教育的机
会,在家里对孩子进行教育。这些家长选择在家教育(home
schooling)有各种各样的原因。很多家长认为公立学校在进行
激进的世俗化教育,而他们希望孩子接受以宗教为基础的教
育。其他一些家长则是不喜欢在二三十个小孩子组成的班级中
必然会存在的整齐划一和命令式教育,而尝试在自己家里以同
样的时间对孩子教授同样的课程。通过在家教育将3 个儿子送
进哈佛大学的柯尔法克斯(David Colfax) 对公立学校是这样评价的: “我们都知道公立学校是一个畸形的官僚机构。”那
些希望和自己的孩子呆在家里的母亲们发现,在家教育比私立
教育更加省钱。而另一些选择在家教育的家长只不过是认为学
校在基础教育上做得并不好。
\switchcolumn*
Estimates on the number of children being home-schooled
vary widely, from about 500,000 to as many as 1.5 million,
but all observers agree that the number has grown rapidly in
the past twenty years. There are newsletters for Christian,
Jewish, black, and sixties-secular home-schooling families.
There are on-line services for home schoolers and sports
leagues to bring them together for physical activity and social
interaction. Home schooling means opting out of government, not civil society.
\switchcolumn
对在家教育的孩子数量的估计有很大差别,从 50万到
150万都有,但是观察家们都同意,过 去 20年中在家教育的
数量在迅速增加。专为基督徒、犹太教徒、黑人和I960年代
的世俗家长提供信息的在家教育杂志纷纷出现。为在家教育者
提供在线服务的网站和把孩子们集中起来进行体育和社交活动
的体育联合会也开始出现。在家教育意味着选择脱离政府,而
不是脱离社会。
\switchcolumn*
Despite the good test scores earned by home schoolers,
school systems have bitterly resisted letting parents educate
their own children. A Michigan education official defended the
state's arrest of a mother who wasn't a certified teacher by saying, ``The state has an interest in the future of the state, and the
children are the future of the state.'' School officials seem to regard home schooling as a rejection of their schools, which of
course it is. In addition, school districts receive an average of
\$4,000 to \$7,000 per student in state and federal aid, so each
home-schooled child means less money for school administrators. Most states have liberalized their laws, but some 2,500 home-schooling families a year seek legal advice from the
Home School Legal Defense Association (there were seventy-
five cases contested in court in 1991, up from fifty-five in
1987).
\switchcolumn
尽管很多接受在家教育的学生都取得了良好的学习成绩,
但是公立学校系统激烈地反对在家教育。密歇根州政府曾经逮
捕过一位没有教师资格证的母亲。该州的一名教育官员在为政
府行为进行辩护时说:“国家关心的是国家的未来,而孩子就
是国家的未来。” 看来,公立学校的官员们是把在家教育看作
对他们的学校的抵制(事实上也是如此)。而且,学区每接受
一个学生,就会从州政府和联邦政府那里得到4000 - 7000美
元的资助,因此,每多一个在家教育的孩子,就意味着学校管
理方少了一份收入。大多数州都放开了有关法律,但是每年仍
然有2500个 在 家 教 育 的 家 庭 到 “在家教育法律保护协会”
(Home School Legal Defense Association)寻求法律帮助。1991
年有75个案件被提交到法庭辩论,而 1987年只有55件。
\switchcolumn*
The next big challenge to the education establishment will
be the entry of for-profit firms into the education business.
Americans spend about \$600 billion a year on education, half of
that for kindergarten through twelfth-grade schools. If that
money were all spent by families, it seems likely that for-profit
companies could provide education that would be far superior
to stultified monopoly school systems. But the money is spent
collectively, of course, which means that for-profit firms have
largely been kept out of the field, so educational technology has
remained at eighteenth-century levels. But schools are becoming so inefficient that 60 percent of school boards have considered hiring firms to run some part of the school operation. The
First Annual Education Industry Conference was held in 1996,
and a new newsletter, the \textit{Education Industry Report}, has compiled a list of twenty-five education companies in an Education
Industry Index like the Dow-Jones Index; it's soaring. Companies like Sylvan Learning Systems and Huntington Learning
Centers are making a profit teaching children what the schools
have failed to teach them. Hooked on Phonics advertises, ``We
have a money-back guarantee. Don't you wish the schools did?''
\switchcolumn
对教育系统的下一次重大挑战将是以营利为目的的公司进
入教育行业。美国人每年的教育支出是6000亿美元,其中一
半花在了从幼儿园到12年级的学校教育上。如果这笔钱都由
家庭自己来支配,以营利为目的的公司很可能会提供远远优于目前效率低下的垄断学校教育系统的教育。但是这笔钱当然是
被集中支配的,这就意味着以营利为目的的公司很大程度上被
排斥在这个领域之外,正因为如此,教育技术才至今仍然保持
18世纪的水平。但是公立学校的效率是如此低下,以至于
60\% 的学校董事会已经考虑把学校的部分运作交给私营公司来
经营。美国第一次教育行业年会在1996年召开,这次会议出
的新简报《教育行业报告》编制了一个包括25家教育公司的
教育行业指数,就像道琼斯指数一样;这个指数一直在迅速上
涨。这些公司包括 “森林女妖学习系统”(Sylvan  Learning Systems)、“亨廷顿学习中心”(Huntington  Learning Centers),他们致力于向孩子们教授在学校里学不到的东西,并从中获得了不错的利润。《自然发音法教材》\footnote{《自然发音法》(Hooked on Phonics) , 是美国的一种非常流行的英语发音教材,也是美国最著名的教育品牌之一。推出20年来,已有超过200万笑国家庭及数千家学校使用。“自然发音法” 的学习内容充满趣味性及互动性、应用容易、效果显著。儿童在学习中不觉得沉闷,能在轻松愉快的气氛下获得知识。}的广告说 “如果达不到学习效果, 我们保证退款。你能指望公立学校这么做吗?”
\switchcolumn*
The problem is not that civil society and the market \textit{can't}
supply education. The problem is that special interests that
benefit from the current tax-supported system won't let parents
keep their own money and purchase education where they find
the best product. But in the next few years, as government
schools continue to deteriorate and new learning technologies
become available even in a severely stunted market, families
will increasingly bypass state schooling to get the education
they need.
\switchcolumn
问题并不是市民社会和市场\textbf{不能提供}教育,而在于那些从现有的靠税收支持的教育体系中获益的特殊利益集团不让家长
支配自己的钱去购买他们认为最好的教育产品。但是,在未来
几年,随着公立学校的品质继续恶化,尽管市场仍然遭受严重
阻 碍 , 新的学习技术将不断出现在市场上,越来越多的家长将
会抛弃国家的学校教育,而得到他们真正需要的教育。

\switchcolumn*[\subsection{Private Communities\\私人社区}]
Despite Robert Reich's advice, 4 million Americans have chosen
to live in some 30,000 private communities. Another 24 million live in locked condominiums, cooperatives, or apartment houses, which are small gated communities. Why do people choose to live in private communities? First, to protect themselves from crime and the dramatic deterioration of public services in many large cities. A college professor complains about
``the new Middle Ages ... a kind of medieval landscape in
which defensible, walled and gated towns dot the countryside.''
People built walls around their cities in the Middle Ages to protect themselves from bandits and marauders, and many Americans are making the same choice.
\switchcolumn
有 400万美国人并没有听从罗伯特$\cdot$ 赖克的劝告,他们选
择住进了大约3 万个私人社区。还有大约2400万美国人住在封闭的公寓楼、合作社区或者公寓小区,这些都是有大门的小
型社区。人们为什么选择居住在私人社区呢?首先是为了保证
他们不受犯罪以及在很多大城市中公共服务质量急剧恶化的侵
害。一个大学教授抱怨说:“这是一个新的中世纪……那种中
世纪的景观,如设置防卫设施、围墙和大门的城镇在乡村重新
出现。” 中世纪,人们在城市周围建造城墙是为了防止匪徒和
抢劫者,而现在很多美国人正在做出同样的选择。
\switchcolumn*
Private communities are a peaceful but comprehensive response to the failure of big government. Like their federal counterpart, local governments today tax us more heavily than ever
but offer deteriorating services in return. Not only do police
seem unable to combat rising crime, but the schools get worse
and worse, garbage and litter don't get picked up, potholes
aren't fixed, panhandlers confront us on every corner. Private
communities can provide physical safety for their residents,
partly by excluding from the community people who are neither residents nor guests.
\switchcolumn
私人社区是和平的,但却是对大政府失败的全面回应。和
联邦政府一样,地方政府现在也向我们抽取比以前任何时候都
要多的税,而回报给我们的是越来越糟糕的服务。不但警察在
与不断增加的犯罪作斗争时显得越来越无能,而且学校也越来
越糟糕。垃圾和落叶得不到清理,路面坑坑洼洼得不到维护,
在每个角落都会碰到乞丐。私人社区能够保障居民的安全,部
分原因是禁止既不是居民也不是客人的人进入社区。
\switchcolumn*
But there's a broader reason for choosing to live in a private
community. Local governments can't satisfy the needs and preferences of all their residents. People have different requirements
in terms of population density, type of housing, presence of children, and so on. Rules that might cater to some citizens' preferences  would  be  unconstitutional  or  offensive  to  the
free-wheeling spirit of other citizens.
\switchcolumn
但是人们选择居住在私立小区还有一个更普遍的理由,即
地方政府无法满足所有居民的偏好。人们对居住密度、住房类
型、是否有小孩等都有着不同的要求。为满足一部分市民的要
求而制定的规定也许是违宪的,或者是对其他市民自由的
侵犯。
\switchcolumn*
Private communities can solve some of these public goods
problems. In the larger developments, the homes, the streets,
the sewers, the parklands are all private. After buying a house
or condominium there, residents pay a monthly fee that covers
security, maintenance, and management. Many of the communities are both gated and guarded.
\switchcolumn
私人社区就能够解决这些公共物品问题。在大型的新开发
小区当中,住房、街道、裁缝店以及停车区都是私人的。在那
里购买独栋小楼或者公寓之后,居民只需每月支付一笔费用,
就涵盖了保安、维修和管理的费用。很多这样的社区有大门,
门口设有保安。
\switchcolumn*
Many have rules that would range from annoying to infuriating to unconstitutional if imposed by a government: regulations on house colors, shrubbery heights, on-street parking,
even gun ownership. People choose such communities partly
because they find the rules---even strict rules---congenial.
\switchcolumn
这样的社区往往有各种各样的规则,如果这些规则是由政
府制定出来的话,很可能会引起公众的不满和愤怒,甚至是违
宪的,例如规定房屋的颜色、灌木丛的高度、街上停车的位
置,甚至对能否拥有枪支都会有规定。人们选择这样的社区,
很大程度上是因为他们认为这些规则 --- 甚至是严厉的规则
--- 很合自己的心意。
\switchcolumn*
In a 1989 issue of \textit{Public Finance Quarterly}, economists Donald
J. Boudreaux and Randall G. Holcombe offer a theoretical explanation for the growing popularity of private communities, which they call contractual governments. Having constitutional rules drawn up by a single developer, who then offers the
property and the rules as a package to buyers, reduces the decision-making costs of developing appropriate rules and allows
people to choose communities on the basis of the kind of rules
they offer. The desire to make money is a strong incentive for
the developer to draw up good rules.
\switchcolumn
在 1989年发行的《公共财政季刊》当中,经济学家布德罗(Donald J. Boudreaux)和霍尔康比(Randall G. Holcombe)
对私人社区越来越受欢迎的现象提出了理论上的解释。他们把
私人社区称作契约政府,意思是,这些社区由某一个开发商来
制定宪法性规则,然后把房产和规则打包提供给购房者,这种
做法降低了形成合适规则的决策成本。人们可以比较不同社区
并作出选择。追求利润的欲望让幵发商有足够的激励来制定良好的规则。
\switchcolumn*
Boudreaux and Holcombe write, ``The establishment of a
contractual government appears to be the closest thing to a
real-world social contract that can be found because it is created
behind something analogous to a veil [of ignorance], and because everyone unanimously agrees to move into the contractual government's jurisdiction.''
\switchcolumn
布德罗和霍尔康比写道:“契约政府制度是目前所发现的一种最接近于真实世界的社会契约的东西,因为它是在一种类似于“无知之幕”\footnote{“无知之幕”(veil  of ignorance), 政治哲学家罗尔斯在论证理论时所做的一个前提假设,指人们在参与决定制度安排时被一重厚厚的幕布给遮掩住了,不知道有关他个人及其社会的任何特殊事实,将所有能够影响其公正选择的功利性信息都给过滤掉了。罗尔斯认为,只有当所有人都处在“无知之幕” 的情况下,他们所一致公认的社会契约才是正义的。}的情况下制订出来的,还因为所有人都一致同意接受这个契约政府的裁判权。”
\switchcolumn*
Fred Foldvary points out that most ``public goods'' exist
within a particular space, so the goods can be provided only to
people who rent or purchase access to the space. That allows entrepreneurs to overcome the problem of people trying to ``freeride'' off others' payments for public goods. Entrepreneurs try
to make their space attractive to customers by supplying the
best possible combination of characteristics, which will vary
from space to space. Foldvary points out that private communities, shopping centers, industrial parks, theme parks, and hotel
interiors are all private spaces created by entrepreneurs, who
have a much better incentive than governments to discover and
respond to consumer demand. And many private entrepreneurs
competing for business can supply a much wider array of
choices than governments will.
\switchcolumn
弗尔德瓦里\footnote{雷德$\cdot$弗尔德瓦里(Fred Foldvary),圣克拉拉大学公民社会研究所研究员、经济系教授。本书所提到的他关于公共物品和私人社区的观点来自他的著作 《公共物品与私人社区 --- 社会服务的市场供给》,这本书的中译本已由经济管理出版社在国内出版。弗尔德瓦里创造性地将公共选择和社会选择理论应用于城市经济学之中,提出在市场过程中私人是可以提供集体物品的,对于公共物品尤其是区域性的公共物品来说,“市场失灵论” 是一个谬误。}指出,大多数 “公共物品”都存在于一个特
定的空间里,因此这些物品就只能提供给那些租赁或购买了进人这个空间的权利的人使用。也就是说,企业家们完全可以克服公共物品的“搭便车” 问题。企业家们为了使得他们的空间吸引客户,就会把各种特征组合起来尽可能提供最好的产
品,这些特征各个空间之间会互不相同。弗尔德瓦里指出,私
人社区、购物中心、工业园、主题公园和酒店内部都是这样由
企业家们创造出来的私人空间,与政府相比,他们有更强的动
机去发现和满足消费者的需求。而无数私营企业家之间的竞争
能够向人们提供比政府服务种类更多、更广泛的选择。
\switchcolumn*
Private communities---including condominiums and apartment buildings---come in virtually unlimited variety. Prices vary
widely, as does the general level of amenities. Some have policies
banning children, pets, guns, garish colors, rentals, or whatever
else might be perceived to reduce residents' enjoyment of the
space. The growing ``cohousing'' movement responds to the
need many people feel for a closer sense of community by offering living spaces centered around a common house for group
meals and activities. Some people create cohousing arrangements based on a shared religious commitment.
\switchcolumn
私人社区 --- 包括共管公寓和公寓小区 --- 实际上有无限
种形式。价格和社区的舒适程度也相差很大。有的禁止居民有
小孩、养宠物、拥有枪支,禁止房屋粉刷成艳丽的颜色,禁止
房屋出租,或者禁止其他任何可能降低居民使用这个空间时舒
适度的东西。正在蓬勃兴起的“共居小区”\footnote{共居小区(cohousing),是在欧美兴起的一种住房合作和社区组建的形式。主要是由数量不等的一些有共同理想和爱好的家庭共同购买一块土地,小区成员共同参与小区的房屋建筑风格和社会功能的设计,以及小区的发展与维护。小区居民既可享有隐私权,对小区也有强烈的认同感。共居小区以个人的自由生活为前提,每家人都有自己的独立的小楼,但是把生活的一部分共同化,有共同的食堂,下班后共同用餐。在这里,独居老人和小区学童都有轮值人员照料。}运动则是为满足人们对更亲密的社区关系的需要而发展起来的。共居小区里有一栋公共房屋用来集体用餐和活动。居民的生活空间和住房围绕在这栋公共房屋周围。也有一部分人建立共居小区是为了进行共同的宗教活动。
\switchcolumn*
Private communities are a vital part of civil society. They give
more people an opportunity to find the kinds of living (or working, or shopping, or entertainment) arrangements they want.
They reflect the understanding of a free society as not one large
community but a community of communities.
\switchcolumn
私人社区是市民社会的一个重要组成部分。它们给更多的
人以机会找到他们想要的生活(或工作、购物、娱乐)方式。
它们真正地体现了自由社会的实质,即自由社会不是一个大的
社区,而是一个由无数小社区组成的共同体。

\switchcolumn*[\subsection{Law and Justice\\法律与司法}]
Libertarians believe that the one proper function of government
is to protect our rights. To that end, governments hire police to
protect us from aggression by our neighbors and establish
courts to settle legal disputes. Yet, perhaps because they are distracted by all the additional tasks they have taken on, governments aren't doing even these basic functions well, and people
are forced to find alternatives in the marketplace.
\switchcolumn
古典自由主义者认为政府的一个正当职能是保护我们的权
利。为了这个目标,政府雇用警察来保护我们不受邻居的侵
犯 ,建立法庭来解决法律纠纷。然而,政府在这些基本职能上
做得并不很好,也许是因为被它们自己强行揽过来的其他附加
任务分散了精力吧,而由于政府做得不到位,人们就不得不到
市场上去寻找这些基本功能的替代产品。
\switchcolumn*
As courts become backlogged and people find litigation both
costly and unpleasant, more people are taking disputes to private arbitrators. Decisions by arbitrators are legally binding
and, if necessary, can be enforced in the public courts, although
the whole point of private arbitration is to avoid the costs and
delays of going to court. The next wave in alternative dispute
resolution (ADR) is likely to be mediation, a nonbinding, less
formal process in which a neutral party helps disputants to
reach a settlement among themselves. Many people prefer mediation because it helps to avoid the adversarial atmosphere and
lingering bad feelings of both courts and binding arbitration.
Since most disputes are among people who will go on dealing
with one another---family members, neighbors, businesses that
have ongoing relationships---it makes sense to try to work
problems out without having a third party impose a solution.
\switchcolumn
由于法院积压案件严重,人们发现诉讼变得既费钱又让人
不快。越来越多的人开始通过私人仲裁来解决纠纷。仲裁人作
出的仲裁都是依据法律来进行的,如果有必要的话,还会通过
公共法院来强制执行,尽管选择私人仲裁完全是为了回避去法
院打官司所耗费的成本和时间。下一波兴起的纠纷替代解决方
案(ADR)很可能是调解,这是一种不具约束性的、不那么
正式的程序,这种方式可以让第三方帮助纠纷双方自行达成解
决方案。很多人喜欢通过调解,因为它有助于避免在法庭和强
制性仲裁中常见的对立气氛和长时间的糟糕心情。由于大多数
纠纷都发生在那些今后还必须长期相处的人之间,如家庭成
员、邻居、有长期合作关系的公司等,那么双方在没有第三方
强加一个解决方案的情况下解决问题就很有必要了。
\switchcolumn*
There are about 200,000 cases filed in the federal courts each
year, while the private, nonprofit American Arbitration Association handles about 60,000 arbitrations and mediations.
JAMS/Endispute, a for-profit firm, handled about 20,000 cases
in 1995, double the number three years earlier. AAA,
JAMS/Endispute, and other arbitration firms have large net-
works of ``neutrals''---impartial third parties available to settle
disputes for customers. Those employed by JAMS/Endispute
are all lawyers, many of them retired judges, while AAA offers
both lawyers and business professionals. Providers argue that,
compared to the government courts, ADR saves time and
money, allows procedural flexibility, gives disputants more control over the arbitration process, preserves relationships, offers
confidentiality, and provides closure, because arbitration and
mediation agreements cannot be appealed except in extraordinary circumstances. Many business contracts provide that any
dispute arising from the contract will be settled by a representative of a particular ADR firm. Arbitrators make decisions
based on the terms of the contract and on common law, which
was itself originally a private institution and is still a process of
case-by-case lawmaking rather than legislative edict.
\switchcolumn
每年大约有20万个案件递交到联邦法院,而私营的非营
利的美国仲裁协会(American Arbitration Association) 每年处
理的仲裁和调解案件是6 万件。杰姆斯纠纷调停公司(JAMS/
Endispute)则是一个以营利为目的的公司,他 们 1995年处理
了大约两万件案子,是三年前的两倍。美国仲裁协会、杰姆斯
纠纷调停公司以及其他仲裁机构都拥有庞大的中间人网络,为
客户提供公正的第三方来解决纠纷。杰姆斯纠纷调停公司所雇用的中间人都是律师或者退休的法官,而美国仲裁协会所提供
的中间人都是律师或者商务专业人士。纠纷调停供应商们提
出,与政府的法院相比,ADR节约了时间和金钱,在程序上
具有灵活性,让更多纠纷被控制在仲裁程序当中,维护了各方
的关系,保护了各方的机密,让各方停止了争议,因为除非很
特殊的情况,仲裁和调解是没有上诉的。很多商业合同都约定
合同争议都应当通过某个指定的ADR公司的代表来解决。仲
裁人依据合同条款和普通法来进行仲裁,而普通法本身最开始
也是一种民间制度,现在仍然也是一种通过一个个案例来创制
法律的过程,而不是由立法机构颁布的法令。
\switchcolumn*
Meanwhile, concerns over crime have also spurred Americans to rely more on private police officers. There are about
550,000 officers serving on state and local police forces; there
are about 1.5 million private police officers. Many of those are
employed by businesses to guard the firm's property, shipments, and so on. Others work for security firms such as
Brink's, which contract their services out to banks, businesses,
housing developments, and event organizers. There would be
fewer private police officers if the government did a better job
of preventing crime and punishing criminals, but private
guards also provide services that would not be appropriately
provided by government, such as round-the-clock protection
for factories, offices, and housing developments.
\switchcolumn
与此同时,对犯罪的担忧也迫使美国人更多地依赖私人警察。在美国,州和地方政府有55万警察;而私人警察则有大约 150万名。其中很多人是被公司雇用来保卫公司财产、押运货物等。另一些则是为保安公司工作,如为银行、企业、建筑工地和社会活动提供保安服务的布林克集团\footnote{林克集团 (Brink’s ) 是世界上最大、历史最长的保安运输公司,成立于 1859年 ,在世界各地有200个分公司和办事处。现在除了保安运输业务之外,还全面进入物流行业,旗下拥有伯林顿快递公司等,是世界500强企业之一。}。如果政府制止犯罪和惩罚罪犯的工作做得好的话,私人警察的数量就会减少,但是私人保安所能提供的服务如果由政府来提供则是不合适的,例如为工厂、写字楼和建筑工地提供24小时的保护。
\switchcolumn*
In some areas, businesses and individuals have paid for extra
police protection in a sort of public-private partnership. Merchants and residents in the Koreatown-West Adams neighborhood of Los Angeles raised about \$400,000 and obtained a
building for a neighborhood police station. Some people complained that taxpayers shouldn't have to pay extra to get basic
services, others that not every neighborhood could afford to pay
for police service. But at least such privately funded efforts
avoid the problem of agreeing to pay higher taxes to a vast jurisdiction like Los Angeles in the hope that \textit{your} neighborhood
\textit{might} get some additional services.
\switchcolumn
在某些地区,企业和个人也在付费使用以某种形式的公私
合作方式进行的额外警察保护服务。在洛杉矶的韩国城 --- 西
亚当斯社区,当地商户和居民筹集了 40万美元买下了一座建
筑作为社区警察局的办公地点。有人抱怨说,纳税人不应该为
得到基本服务而额外付费,不是每一个社区的人都能够承担警察服务的费用的。但是这种私人出资的努力,至少比为使\textbf{本}社
区\textbf{获得}额外服务而同意缴纳更多税款给一个庞大的司法体系( 如洛杉矶市)要好。

\switchcolumn*[\section{Insurance and Futures\\保险与未来}]
People have often thought that insurance is a valuable service
for government to provide. Many of the largest federal programs are intended to insure Americans against economic and
other risks: Social Security, Medicare and Medicaid, deposit insurance, flood insurance, and more. The general argument for
insurance is to spread risks; a loss that would be disastrous for a
single individual can be absorbed by a large group of individuals. We pool our money in an insurance plan to guard against
the small possibility of a catastrophic event.
\switchcolumn
人们常常认为保险是政府所提供的有价值的服务。很多最
大的联邦项目就是为了给美国人提供对经济和其他风险的保
护,如社会保障、医疗保障、公共医疗补助、储蓄保险、水灾
保险等。关于保险的最常见论证是指出保险能够分散风险;个
人的灾难性损失可以被一大群人所吸收。我们把钱投入到一个
保险计划是为了预防一个小概率灾难性事件的发生。
\switchcolumn*
The argument for government insurance, as opposed to competitive private insurers, is that you can spread the risk over a
larger number of people. But as George L. Priest of the Yale
Law School points out, government insurance has had many
unfortunate results. There's no economic advantage to creating
an insurance pool larger than necessary, and there are definite
disadvantages to large monopolies. Government is very bad at
charging risk-appropriate premiums, so its insurance tends to
be too expensive for risk-averse people and too cheap for those
who engage in high-risk activities. And government dramatically compounds the ``moral hazard'' problem---that is, the tendency of people who have insurance to take more risks.
Insurance companies try to control this by having deductibles
and copayments, so the insured will still face some loss beyond
what insurance covers, and by excluding certain kinds of activities from coverage (like suicide or behavior that is more risky
than the insurance pool is designed for). For both economic and
political reasons, government usually doesn't employ such
tools, so it actually encourages more risk taking.
\switchcolumn
对政府保险的论证,与竞争性的私营保险公司不同。政府
保险被认为可以把风险分散到数量更大的人群当中。但是正如
耶鲁大学法学院的普列斯特(George L. Priest)所指出的,政
府保险有很多不幸的后果。建立一个超过需要的保险基金在经
济上并没有任何好处,而庞大的垄断机构只会带来无尽的坏
处。政府很不擅长根据风险评估确定保险费率,因此它提供的
保险很容易让低风险的人感到太贵,而对髙风险的人来说则又
太低。同时,政府极大地助长了 “道德风险” 问题 --- 即拥
有保险的人倾向于更加冒险。保险公司一方面是通过设置免赔
率和自付费率来控制这种风险,这样被保险人除了保险赔付所
覆盖的部分之外还要承担一些损失,另一方面则是通过把某种
行 为 (如自杀或者风险超过保险基金覆盖范围的行为)排除
在保险覆盖范围之外来控制风险。因为经济的和政治的原因,
政府常常不会采用这些工具,因此它就是在事实上鼓励更多的
冒险行为。
\switchcolumn*
Priest cites several specific examples: Federal savings-and-loan insurance increased the risk level of investments; the savings-and-loan companies would reap the profits from high-risk
ventures, but the taxpayers would make up the losses, so why
not go for the big return? Government-provided unemployment insurance increases both the extent and the duration of
unemployment; people would find new jobs sooner if they
didn't have unemployment insurance, or if their own insurance
rates were affected by how much they used, as car insurance
rates are. Priest writes, ``I will not go so far as to claim that government-provided insurance increases the frequency of natural
disasters. On the other hand, I have no doubt whatsoever that
the government provision of insurance increases the magnitude of \textit{losses} from natural disasters.'' Flood insurance, for instance,
provided by the government at less than the market price, encourages more building on flood plains and on the fragile barrier islands off the East Coast.
\switchcolumn
普列斯特引用了几个例子:联邦的存贷款保险提髙了投资
的风险水平;存贷款公司从高风险投资当中获得利润,却可以
让纳税人承担损失,那他们有什么理由不去追求高额回报呢?
政府提供的失业保险提高了失业率和失业时间;如果没有失业
保险,或者保险费率像汽车保险一样与已赔付的额度挂钩,人
们就会更快地找工作。普列斯特写道:“我不至于宣称政府提
供的保险提髙了自然灾害的频率;但另一方面,无论如何我都
毫不怀疑政府提供保险会提髙自然灾害所造成的\textbf{损失程度}。”
例如,政府提供的水灾保险低于市场价格,这就鼓励人们把房
子建在易受洪水侵袭的平地上或者东海岸外防护脆弱的小
岛上。
\switchcolumn*
The desire to reduce one's exposure to risk is natural, and
markets provide people with means to that end. But when people sought to reduce risk through government insurance programs, the result was to channel resources toward \textit{more} risky
activities and thus to increase the level of risk and the level of
losses suffered by the whole society.
\switchcolumn
人们希望降低自己面对的风险是很自然的,而市场给人们
提供了达到这个目的的手段。但是当人们通过政府保险计划来
寻求降低风险时,结果却是资源被引向\textbf{风险更大}的行为,并且
因此提高了整个社会的风险程度和遭受到的损失的程度。
\switchcolumn*
Still, the market has provided many opportunities for people
to choose the level of risk with which they're comfortable.
Many kinds of insurance are available. Different investments---
stocks, bonds, mutual funds, certificates of deposit---allow people to balance risk versus return in the way they prefer. Farmers
can reduce their risks by selling their expected harvest before it
comes in, locking in a price. They're protected against falling
prices, but they lose the opportunity to make big profits from
rising prices. Commodities markets give farmers and others the
opportunity to hedge against price shifts. Many people don't
understand commodities and futures markets, or even the simpler securities markets; in Tom Wolfe's novel \textit{The Bonfire of the Vanities}, the bond trader Sherman McCoy thought of himself as
Master of the Universe but couldn't explain to his daughter the
value of what he did. Politicians and popular writers rail against
``paper entrepreneurs'' or ``money changers,'' but those mysterious markets not only guide capital to projects where it will best
serve consumer demand, they also help millions of Americans
to regulate their risks.
\switchcolumn
而市场提供了很多机会,让人们选择对自己合适的风险等
级。市场上有很多种类的保险可供选择。多种形式的投资如股
票、债券、互助基金、储蓄,让人们可以在权衡风险和回报之
后按照个人偏好来进行选择。农场主可以通过在粮食收获之前
按照锁定价格把未来出产的粮食卖掉来降低风险。他们避免了
价格下跌可能造成的损失,却失去了从价格上涨中获得巨大收
益的机会。期货市场给农场主和其他人规避价格波动风险的机
会。很多人无法理解商品期货市场,也不理解更简单的证券市
场 ;在 汤 姆 $\cdot$ 沃 尔 夫 (Tom  Wolfe)的 小 说 《虚荣的篝火》(\textit{The Bonfire of the Vanities})中,债券交易商谢尔曼$\cdot$麦科伊把自己看作宇宙的主人,却无法对女儿解释他所做的事情的价值。政治家和畅销书作家们讽刺他们是“纸上企业家”或者“换钱的人”,但是这些神秘的市场不仅把资本引导到能够最好满足消费者需要的项目上去,而且还帮助数以百万计的美国人控制风险。
\switchcolumn*
A new twist for farmers is the opportunity to contract with
food processors to grow specific crops. More than 90 percent of
vegetables are now grown under production contracts, along
with smaller percentages of other crops. The contracts give
farmers less independence but also less risk, which many of
them prefer.
\switchcolumn
对农场主来说,一个新的窍门是和食品加工厂签订合同种
植专门的作物。现在超过90\% 的蔬菜是通过生产合同来种植
的,其他一些作物也是如此,只不过签约生产的比例不像蔬菜
那么高。生产合同降低了农场主的独立性,但同时也降低了他
们的风险,这是他们当中很多人所希望的。
\switchcolumn*
Meanwhile, major commodities markets like the Chicago
Board of Trade, the Chicago Mercantile Exchange, and the New
York Mercantile Exchange (Nymex) are looking for new investment options to offer to customers. The Chicago Merc recently began offering milk price futures---allowing people to lock in
milk prices or bet on price shifts---in response to deregulation,
which will likely mean lower but fluctuating prices. The
Nymex established a market in electricity futures, which will
come in handy as electric utilities are deregulated.
\switchcolumn
与此同时,大型期货交易市场如芝加哥期货交易所(CBOT)、芝加哥商品交易所(CME)\footnote{经将近一个半世纪的激烈竞争后,CME和CBOT于2006年合并为新的 CME,合并后的公司是世界上规模最大、最活跃的期货交易所。}以及纽约商品交易所(NYMEX)正在寻找新的投资选择以提供给客户。为应对解除牛奶价格管制(这很可能意味着价格更低但是起伏更大),芝加哥商品交易所最近开始提供牛奶期货价格,允许人们锁定牛奶的价格,或者把赌注押在价格波动上。而纽约商品交易所则建立了一个电价期货市场,随着对电力设施解除管制,这迟早会进入实际操作。
\switchcolumn*
The Board of Trade is one of the players looking for new ways
to protect insurance companies---and by extension, everyone
who buys insurance or invests in insurance companies---from
the threat posed by megadisasters. According to the \textit{New York Times}, ``Two of the most destructive natural disasters in American history have occurred'' in the past few years: Hurricane Andrew in 1992, which cost insurers \$16 billion in South Florida,
and the 1994 Los Angeles earthquake, which cost \$11 billion.
(Note that the reason these were the ``most destructive'' disasters ever is that Americans own more wealth than ever, so financial losses are greater.) Insurers fear a disaster of \$50 billion
magnitude, which could put insurance companies out of busi-
ness and even be too much for the reinsurance business, which
sells policies to protect insurers from large losses. They are looking for new ways to spread the risk, including catastrophe futures on the Board of Trade, with which insurers could hedge
against the possibility of large losses. Investors would make
money by, in effect, betting that there would be no such catastrophe.
\switchcolumn
芝加哥期货交易所在寻求新的方式来帮助保险公司抵御大灾难 --- 从而保护了投保人和保险公司投资者的利益 --- 的过
程中扮演了重要角色。据 《纽约时报》报道,在过去的几年,“美国发生了历史上破坏性最大的两场自然灾害” :其 中 1992
年的安德鲁飓风给佛罗里达州南部带来的损失让保险公司赔付了 160亿美元,而 1994年的洛杉矶大地震则让保险公司付出
了 110亿美元的代价(请注意,这两次灾害之所以是美国历史上 “最具破坏性” 的,是因为美国人拥有了更多的财富,因此财务上的损失比以前任何时候都大)。保险公司担心,一场灾难如果达到500亿美元的量级,就将导致保险公司破产,甚
至对再保险公司来说这也超出了其承受能力。他们在寻找新的方式来分散风险,包括把巨灾期货\footnote{巨灾期货(Catastrophe Future), 1992年 12月 11日,美国芝加哥期货交易所推出巨灾保险期货交易。巨灾期货和商品期货的运作机制基本相同,如果保险公司预测到巨灾损失会发生,它就会购买巨灾期货,如果该损失未发生,或低于某一界限,那么保险公司将在期货市场上受到损失,但由于保险损失较小,保险人可用保险方面获得的收益弥补其期货方面的损失,这样,就可以使保险业免受巨灾带来的损失。巨灾期货作为一种全新的风险转移方式,在巨灾风险证券化理论上和实践上的重要意义都是不可替代的。后来由于指数的准确性不髙、严重的道德风险及其他因素,该项交易夭折了,但是,它毕竟是人们在巨灾风险证券化领域迈出的第一步,为今后其他品种产品的成功推出提供了重要的借鉴意义。}放到芝加哥期货交易所进行交易,这就使得保险公司可以用期货来避免重大损失的可
能。而实际上,投资巨灾期货的人可以通过把赌注押在不会有这样的巨大灾难的预期上而获利。
\switchcolumn*
Reinsurers are also offering ``act of God'' bonds that would
pay very high interest but would require bondholders to forgo
repayment in the event of disaster. Catastrophe futures and ``act
of God'' bonds will help keep insurance coverage available and
reasonably priced. They also raise the question: If the market
can adequately deal with even the prospect of multi-billion-dollar financial disasters, why does government need to intervene
in the economic system at all?
\switchcolumn
再保险公司也在发行“天灾债券”\footnote{天灾债券(Act of God Bond),保险公司发行的债券,旨在将债券的本金及利息与天然灾害造成的公司损失联系起来。}这种债券会支付很高的利息,但是要求债券持有者在发生大灾难的时候放弃偿还
资格。巨灾期货和天灾债券将帮助保险公司继续提供保险覆盖
和让保险保持合理的价格。这也给我们提出了一个问题:既然
市场对甚至是以十亿美元为单位计算的灾害前景都能够有效应
对 ,那么为什么还需要政府来干预经济体系呢?

\switchcolumn*[\section{Bypassing the State\\绕过国家}]
The twentieth century has been a failed experiment in big government. Every day more people see more ways that problems
could be better solved by profit-seeking companies, mutual-aid associations, or charities than by government. Private capital
markets can provide actuarially sound insurance and offer better retirement benefits than Social Security. One of the world's
largest engineering projects, the \$12 billion tunnel under the
English Channel, was designed, financed, built, owned, and operated by a private consortium. A company called Human Capital Resources wants to sell equity investments in the future
earning power of college students as an alternative to student
loans---better return for investors, less postgraduation burden
on students, and no cost to the taxpayers.
\switchcolumn
20世纪是大政府实验失败的一个世纪。每夭都有更多的人看到问题通过以营利为目的的公司、互助组织或者慈善机构
能够得到更好的解决,而不是通过政府。私人资本市场能够提
供精算良好的保险和比政府社保更好的退休福利。世界上最大
的工程计划之一,耗 资 120亿美元的英吉利海底隧道完全由一
家私营财团设计、出资、建造、拥有并运行。一 家 叫 做 “人
力资本资源” 的公司计划发行股票,将筹集到的资金投资于
大学生未来的赚钱能力,作为学生贷款的一种替代选择 --- 这
项投资将给投资人带来很好的回报,减轻学生毕业后的负担,
而且不增加纳税人的负担。
\switchcolumn*
Private communities, based on governance by consent, can
be better tailored to the needs and preferences of 250 million
diverse Americans than can local governments. Private schools
provide a better education at lower cost than government
schools, and in the next few years information technology and
for-profit companies will revolutionize learning. Private charities get people \textit{off} welfare rather than snaring them in it.
\switchcolumn
建立在共识统治基础之上的私人社区,为2.5亿形形色色
的美国人量身定做,比地方政府更好地满足了他们的个性化需
要。与政府公立学校相比,私立学校能够以更低的成本提供更
好的教育,而在未来的几年里,信息技术的进步和以追求利润
为目标的公司进入市场将给学习带来革命性的变化。私立慈善
机构让人们\textbf{摆脱}福利,而不是陷入福利不能自拔。
\switchcolumn*
Some day soon we may be able to bypass governments to get
all the goods and services we need. But in the meantime, our
\$2.5 trillion federal-state-local governments are not going to
give up their power without a fight. The U.S. Postal Service
tenaciously clings to its legal monopoly. School boards and
teachers' unions declare that they won't let children ``escape''
from their schools, and they spend millions to prevent the implementation of school-choice plans. The people who benefit
from the existing system won't willingly downsize government
even if all the customers desert it. As school enrollment in the
District of Columbia fell by 33,000---about 25 percent---the
system actually \textit{added} 516 administrators. The 800,000 postal
employees are not going to quietly accept layoffs even if we
send all our communications electronically.
\switchcolumn
很快会有一天,我们能够不需要政府就得到所有所需要的
商品和服务。但与此同时,2.5万亿美元规模的联邦一州一地
方三级政府不会不战而降地放弃他们的权力。美国邮政也会顽
固地抓住法定垄断权不放。学校董事会和教师工会宣布不会让
学 生 们 “逃离” 他们的学校,他们会花数百万美元来阻止自
由择校计划的实施。那些从现行体制中获益的人不会希望裁减
政府,即使所有的客户都渴望如此。哥伦比亚特区学校的人学
人数降到了 33000人 ,下降了大约25\%,而与此同时,该特
区教育机构的管理人员却\textbf{增加了}516名。80万美国邮政雇员
不会默默接受下岗的结果,尽管通信已经几乎完全使用电子方
式了。
\switchcolumn*
We cannot simply wait for ``social forces'' or technology to
automatically replace bloated government. To ensure that such
changes happen, individuals will have to demand their \textit{right} to
choose schools for their children, to compete with the U.S.
Postal Service, to invest their money in a secure private retirement fund. And then taxpayers will have to work to ensure
that government stops producing services no one uses anymore.
\switchcolumn
我们不可能仅仅是等待“社会力量”或者技术来自动取代臃肿的政府。为了确保这个变革发生,个人应当要求为自己
的孩子自由选择学校的\textbf{权利},要求与美国邮政进行竞争,并把钱投入安全的私营退休基金当中。而纳税人则需要努力,争取让政府不再进行那些没有任何人使用的服务 。

\end{paracol}